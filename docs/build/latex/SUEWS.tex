%% Generated by Sphinx.
\def\sphinxdocclass{report}
\documentclass[letterpaper,10pt,english]{sphinxmanual}
\ifdefined\pdfpxdimen
   \let\sphinxpxdimen\pdfpxdimen\else\newdimen\sphinxpxdimen
\fi \sphinxpxdimen=.75bp\relax

\PassOptionsToPackage{warn}{textcomp}

\catcode`^^^^00a0\active\protected\def^^^^00a0{\leavevmode\nobreak\ }
\usepackage{cmap}
\usepackage{fontspec}
\usepackage{amsmath,amssymb,amstext}
\usepackage{polyglossia}
\setmainlanguage{english}

\usepackage[Bjarne]{fncychap}
\usepackage[,numfigreset=1,mathnumfig]{sphinx}

\usepackage{geometry}

% Include hyperref last.
\usepackage{hyperref}
% Fix anchor placement for figures with captions.
\usepackage{hypcap}% it must be loaded after hyperref.
% Set up styles of URL: it should be placed after hyperref.
\urlstyle{same}

\addto\captionsenglish{\renewcommand{\figurename}{Fig.}}
\addto\captionsenglish{\renewcommand{\tablename}{Table}}
\addto\captionsenglish{\renewcommand{\literalblockname}{Listing}}

\addto\captionsenglish{\renewcommand{\literalblockcontinuedname}{continued from previous page}}
\addto\captionsenglish{\renewcommand{\literalblockcontinuesname}{continues on next page}}

\def\pageautorefname{page}

\setcounter{tocdepth}{1}


\usepackage[titles]{tocloft}
\addto\captionsenglish{\renewcommand{\bibname}{References}}
\cftsetpnumwidth {1.25cm}\cftsetrmarg{1.5cm}
\setlength{\cftchapnumwidth}{0.75cm}
\setlength{\cftsecindent}{\cftchapnumwidth}
\setlength{\cftsecnumwidth}{1.25cm}


\title{SUEWS Documentation}
\date{Mar 29, 2018}
\release{2018a}
\author{Sue Grimmond, Ting Sun}
\newcommand{\sphinxlogo}{\vbox{}}
\renewcommand{\releasename}{Release}
\makeindex

\begin{document}

\maketitle
\sphinxtableofcontents
\phantomsection\label{\detokenize{index::doc}}


The current version of SUEWS is v2017b. The software can be downloaded
by completing \sphinxhref{http://micromet.reading.ac.uk/software/}{this form}.

{\hyperref[\detokenize{index:index-page}]{\sphinxcrossref{\DUrole{std,std-ref}{This documentation site}}}} (\autopageref*{\detokenize{index:index-page}}) is regularly
updated with new developments. For what’s new in this version, see {\hyperref[\detokenize{version-history:new-latest}]{\sphinxcrossref{\DUrole{std,std-ref}{New in SUEWS Version 2018a}}}} (\autopageref*{\detokenize{version-history:new-latest}}).

The \sphinxstylestrong{latest formal} release of SUEWS is \sphinxstylestrong{v2017b} (released 1 August
2017).

The manual for SUEWS v2017b can be accessed
\sphinxhref{:File:SUEWS\_V2017b\_Manual.pdf}{here} and should be referenced as
follows:
\begin{quote}

Ward HC, L Järvi, T Sun, S Onomura, F Lindberg, F Olofson, A Gabey, CSB Grimmond (2017).
SUEWS Manual V2017b Department of Meteorology, University of Reading, Reading, UK
\end{quote}

Please refer to \sphinxhref{http://onlinelibrary.wiley.com/doi/10.1002/joc.5200/full}{Ward et al. (2017)} for
further details v2017a:
\begin{quote}

Ward HC, Yin San Tan, AM Gabey, S Kotthaus, WTJ Morrison, CSB Grimmond.
Impact of temporal resolution of precipitation forcing data on modelled urban-atmosphere exchanges and surface conditions.
International Journal of Climatology.
\sphinxhref{http://doi.org/10.1002/joc.5200/}{doi: 10.1002/joc.5200}
\end{quote}

\begin{sphinxadmonition}{note}{Note:}
See other publications in the next section (if you have papers that could be added, please send them through)
\end{sphinxadmonition}


\chapter{Recent publications}
\label{\detokenize{recent-publications:recent-publications}}\label{\detokenize{recent-publications:index-page}}\label{\detokenize{recent-publications::doc}}\label{\detokenize{recent-publications:suews-surface-urban-energy-and-water-balance-scheme}}
\begin{sphinxadmonition}{note}{Note:}
If you have papers to add to this list please let us and others know
via the \sphinxhref{http://urban-climate.net/umep/SUEWS\#Development.2C\_Suggestions\_and\_Support}{email
list.}
\end{sphinxadmonition}
\begin{itemize}
\item {} 
\sphinxhref{https://www.nature.com/articles/s41598-017-05733-y}{Järvi et al. (2017)}

\end{itemize}
\begin{quote}\begin{description}
\item[{topic}] \leavevmode
Application and evalution in cold climates. Implications of warming

\item[{citation}] \leavevmode
Järvi L, S Grimmond, JP McFadden, A Christen, I Strachan, M Taka, L Warsta, M Heimann 2017:
Warming effects on the urban hydrology in cold climate regions
Scientific Reports 7: 5833

\end{description}\end{quote}
\begin{itemize}
\item {} 
\sphinxhref{https://doi.org/10.1016/j.uclim.2017.05.001}{Kokkonen et al. (2017)}

\end{itemize}
\begin{quote}\begin{description}
\item[{topic}] \leavevmode
Downscaling climate (rainfall) data to 1 h

\item[{citation}] \leavevmode
Kokkonen T, CSB Grimmond, O Räty, HC Ward, A Christen, T Oke, S Kotthaus, L Järvi 2017:
Sensitivity of Surface Urban Energy and Water Balance Scheme (SUEWS)

\end{description}\end{quote}
\begin{itemize}
\item {} 
\sphinxhref{http://dx.doi.org/10.1016/j.landurbplan.2017.04.001}{Ward and Grimmond (2017)}

\end{itemize}
\begin{quote}\begin{description}
\item[{topic}] \leavevmode
for example applications:

\item[{citation}] \leavevmode
Ward HC, S Grimmond 2017:
Using biophysical modelling to assess the impact of various scenarios
on summertime urban climate across Greater London
Landscape and Urban Planning 165, 142\textendash{}161

\end{description}\end{quote}
\begin{itemize}
\item {} 
\sphinxhref{http://onlinelibrary.wiley.com/doi/10.1002/qj.3028/full}{Demuzere et al. 2017}

\end{itemize}
\begin{quote}\begin{description}
\item[{topic}] \leavevmode
evaluation in Singapore and comparison with other urban land surface models

\item[{citation}] \leavevmode
Demuzere M, S Harshan, L Järvi, M Roth, CSB Grimmond, V Masson, KW Oleson, E Velasco H Wouters 2017:
Impact of urban canopy models and external parameters on the modelled urban energy balance
QJRMS, 143, Issue 704, Part A, 1581\textendash{}1596

\end{description}\end{quote}
\begin{itemize}
\item {} 
\sphinxhref{http://www.sciencedirect.com/science/article/pii/S2212095516300256}{Ward et al.(2016)}

\end{itemize}
\begin{quote}\begin{description}
\item[{topic}] \leavevmode
Evaluation of SUEWS model

\item[{citation}] \leavevmode
Ward HC, Kotthaus S, Järvi L and Grimmond CSB (2016)
Surface Urban Energy and Water Balance Scheme (SUEWS): Development and evaluation at two UK sites.
Urban Climate

\end{description}\end{quote}
\begin{itemize}
\item {} 
\sphinxhref{http://dx.doi.org/10.1175/JAMC-D-16-0082.1}{Ao et al. (2016)}

\end{itemize}
\begin{quote}\begin{description}
\item[{topic}] \leavevmode
Evaluation of radiation in Shanghai

\item[{citation}] \leavevmode
Ao XY, CSB Grimmond, DW Liu, ZH Han, P Hu, YD Wang, XR Zhen, JG Tan 2016:
Radiation fluxes in a business district of Shanghai
JAMC, 55, 2451-2468

\end{description}\end{quote}
\begin{itemize}
\item {} 
\sphinxhref{http://dx.doi.org/10.1016/j.uclim.2014.11.001}{Onomura et al. (2015)}

\end{itemize}
\begin{quote}\begin{description}
\item[{topic}] \leavevmode
Boundary layer modelling

\item[{citation}] \leavevmode
Onomura S, Grimmond CSB, Lindberg F, Holmer B \& Thorsson S (2015)
Meteorological forcing data for urban outdoor thermal comfort models
from a coupled convective boundary layer and surface energy balance scheme
Urban Climate, 11, 1-23

\end{description}\end{quote}
\begin{itemize}
\item {} 
\sphinxhref{https://www.geosci-model-dev.net/7/1691/2014/gmd-7-1691-2014.pdf}{Järvi et al. (2014)}

\end{itemize}
\begin{quote}\begin{description}
\item[{topic}] \leavevmode
Snow melt model development

\item[{citation}] \leavevmode
Järvi L, Grimmond CSB, Taka M, Nordbo A, Setälä H \& Strachan IB 2014:
Development of the Surface Urban Energy and Water balance Scheme (SUEWS) for cold climate cities
Geosci. Model Dev. 7, 1691-1711

\end{description}\end{quote}

\sphinxhref{http://urban-climate.net/umep/UMEP\_Manual\#Evaluation\_and\_application\_studies}{Other papers}


\chapter{Introduction}
\label{\detokenize{introduction:introduction}}\label{\detokenize{introduction::doc}}
\begin{figure}[htbp]
\centering
\capstart

\noindent\sphinxincludegraphics{{SUEWS_Overview_s}.png}
\caption{Overview of SUEWS}\label{\detokenize{introduction:id4}}\end{figure}

Surface Urban Energy and Water Balance Scheme (\sphinxstylestrong{SUEWS}) (Järvi et al.
2011 \phantomsection\label{\detokenize{introduction:id1}}{\hyperref[\detokenize{references:j11}]{\sphinxcrossref{{[}J11{]}}}} (\autopageref*{\detokenize{references:j11}}), Ward et al. 2016 \phantomsection\label{\detokenize{introduction:id2}}{\hyperref[\detokenize{references:w16}]{\sphinxcrossref{{[}W16{]}}}} (\autopageref*{\detokenize{references:w16}})) is able to simulate the urban
radiation, energy and water balances using only commonly measured
meteorological variables and information about the surface cover. SUEWS
utilizes an evaporation-interception approach (Grimmond et al.
1991 \phantomsection\label{\detokenize{introduction:id3}}{\hyperref[\detokenize{references:g91}]{\sphinxcrossref{{[}G91{]}}}} (\autopageref*{\detokenize{references:g91}})), similar to that used in forests, to model evaporation from
urban surfaces.

\begin{figure}[htbp]
\centering
\capstart

\noindent\sphinxincludegraphics{{SUEWS_SurfaceWaterBalance_v2_xxs}.jpg}
\caption{The seven surface types considered in SUEWS}\label{\detokenize{introduction:id5}}\end{figure}

The model uses seven surface types: paved, buildings, evergreen
trees/shrubs, deciduous trees/shrubs, grass, bare soil and water. The
surface state for each surface type at each time step is calculated from
the running water balance of the canopy where the evaporation is
calculated from the Penman-Monteith equation. The soil moisture below
each surface type (excluding water) is taken into account.

Horizontal movement of water above and below ground level is allowed.
The user can specify the model time-step, but 5 min is strongly
recommended. The main output file is provided at a resolution of 60 min
by default. The model provides the radiation and energy balance
components, surface and soil wetness, surface and soil runoff and the
drainage for each surface. Timestamps refer to the end of the averaging
period.

Model applicability: SUEWS is a neighbourhood-scale or local-scale
model.


\chapter{SUEWS and UMEP}
\label{\detokenize{suews-and-umep:suews-and-umep}}\label{\detokenize{suews-and-umep::doc}}
SUEWS can be run as a standalone model but also can be used within
\sphinxhref{http://urban-climate.net/umep/UMEP\_Manual}{UMEP}. There are numerous
tools included within UMEP to help a user get started. The \sphinxhref{http://urban-climate.net/umep/UMEP\_Manual\#Urban\_Energy\_Balance:\_Urban\_Energy\_Balance\_.28SUEWS.2C\_simple.29}{SUEWS
simple}
within UMEP is a fast way to start using SUEWS.

The version of SUEWS within UMEP is the complete model. Thus all options
that are listed in this manual are available to the user. In the UMEP
\sphinxhref{http://urban-climate.net/umep/UMEP\_Manual\#Urban\_Energy\_Balance:\_Urban\_Energy\_Balance\_.28SUEWS.2C\_simple.29}{SUEWS
simple}
runs all options are set to values to allow intial exploration of the
model behaviour.

The version of SUEWS within UMEP is a more recent release of the model
than the independent SUEWS release.


\begin{savenotes}\sphinxattablestart
\centering
\begin{tabulary}{\linewidth}[t]{|T|T|T|T|}
\hline
\sphinxstartmulticolumn{3}%
\begin{varwidth}[t]{\sphinxcolwidth{3}{4}}
\sphinxstyletheadfamily UMEP
\par
\vskip-\baselineskip\vbox{\hbox{\strut}}\end{varwidth}%
\sphinxstopmulticolumn
&\sphinxstyletheadfamily 
Description
\\
\hline\sphinxmultirow{11}{3}{%
\begin{varwidth}[t]{\sphinxcolwidth{1}{4}}
\sphinxstylestrong{Pre-Processor}
\par
\vskip-\baselineskip\vbox{\hbox{\strut}}\end{varwidth}%
}%
&\sphinxmultirow{2}{4}{%
\begin{varwidth}[t]{\sphinxcolwidth{1}{4}}
Meteorological
Data
\par
\vskip-\baselineskip\vbox{\hbox{\strut}}\end{varwidth}%
}%
&
\sphinxhref{http://urban-climate.net/umep/UMEP\_Manual\#Meteorological\_Data:\_MetPreprocessor}{Prepare
Existing
Data}
&
Transforms
meteorological
data into UMEP
format
\\
\cline{3-4}\sphinxtablestrut{3}&\sphinxtablestrut{4}&
\sphinxhref{http://www.urban-climate.net/umep/UMEP\_Manual\#Meteorological\_Data:\_Download\_data\_.28WATCH.29}{Download data
(WATCH)}
&
Prepare
meteorological
dataset from
\sphinxhref{http://www.eu-watch.org/data\_availability}{WATCH}
\\
\cline{2-4}\sphinxtablestrut{3}&\sphinxmultirow{2}{9}{%
\begin{varwidth}[t]{\sphinxcolwidth{1}{4}}
Spatial Data
\par
\vskip-\baselineskip\vbox{\hbox{\strut}}\end{varwidth}%
}%
&
\sphinxhref{http://www.urban-climate.net/umep/UMEP\_Manual\#Spatial\_Data:\_Spatial\_Data\_Downloader}{Spatial Data
Downloader}
&
Plugin for
retrieving
geodata from
online services
suitable for
various UMEP
related tools
\\
\cline{3-4}\sphinxtablestrut{3}&\sphinxtablestrut{9}&
\sphinxhref{http://www.urban-climate.net/umep/UMEP\_Manual\#Spatial\_Data:\_LCZ\_Converter}{LCZ
Converter}
&
Conversion from
Local Climate
Zones (LCZs) in
the WUDAPT
database into
SUEWS input
data
\\
\cline{2-4}\sphinxtablestrut{3}&\sphinxmultirow{3}{14}{%
\begin{varwidth}[t]{\sphinxcolwidth{1}{4}}
Urban land
cover
\par
\vskip-\baselineskip\vbox{\hbox{\strut}}\end{varwidth}%
}%
&
\sphinxhref{http://urban-climate.net/umep/UMEP\_Manual\#Urban\_Land\_Cover:\_Land\_Cover\_Reclassifier}{Land Cover
Reclassifier}
&
Reclassifies a
grid into UMEP
format land
cover grid.
\sphinxstyleemphasis{Land surface
models}
\\
\cline{3-4}\sphinxtablestrut{3}&\sphinxtablestrut{14}&
\sphinxhref{http://urban-climate.net/umep/UMEP\_Manual\#Urban\_Land\_Cover:\_Land\_Cover\_Reclassifier}{Land Cover
Fraction
(Point)}
&
Land cover
fractions
estimates from
a land cover
grid based on a
specific point
in space
\\
\cline{3-4}\sphinxtablestrut{3}&\sphinxtablestrut{14}&
\sphinxhref{http://urban-climate.net/umep/UMEP\_Mnual\#Urban\_Land\_Cover:\_Land\_Cover\_Fraction\_.28Grid.29}{Land Cover
Fraction
(Grid)}
&
Land cover
fractions
estimates from
a land cover
grid based on a
polygon grid
\\
\cline{2-4}\sphinxtablestrut{3}&\sphinxmultirow{3}{21}{%
\begin{varwidth}[t]{\sphinxcolwidth{1}{4}}
Urban
Morphology
\par
\vskip-\baselineskip\vbox{\hbox{\strut}}\end{varwidth}%
}%
&
\sphinxhref{http://urban-climate.net/umep/UMEP\_Manual\#Urban\_Morphology:\_Morphometric\_Calculator\_.28Point}{Morphometric
Calculator (Poi
nt)}
&
Morphometric
parameters from
a DSM based on
a specific
point in space
\\
\cline{3-4}\sphinxtablestrut{3}&\sphinxtablestrut{21}&
\sphinxhref{http://urban-climate.net/umep/UMEP\_Manual\#Urban\_Morphology:\_Morphometric\_Calculator\_.28Grid.29}{Morphometric
Calculator
(Grid)}
&
Morphometric
parameters
estimated from
a DSM based on
a polygon grid
\\
\cline{3-4}\sphinxtablestrut{3}&\sphinxtablestrut{21}&
\sphinxhref{http://urban-climate.net/umep/UMEP\_Manual\#Urban\_Morphology:\_Source\_Area\_.28Point.29}{Source Area
Model
(Point)}
&
Source area
calculated from
a DSM based on
a specific
point in space.
\\
\cline{2-4}\sphinxtablestrut{3}&\sphinxstartmulticolumn{2}%
\begin{varwidth}[t]{\sphinxcolwidth{2}{4}}
\sphinxhref{http://urban-climate.net/umep/UMEP\_Manual\#Pre-Processor:\_SUEWS\_Prepare}{SUEWS
Prepare}
\par
\vskip-\baselineskip\vbox{\hbox{\strut}}\end{varwidth}%
\sphinxstopmulticolumn
&
Preprocessing
and preparing
input data for
the SUEWS model
\\
\hline\sphinxmultirow{4}{30}{%
\begin{varwidth}[t]{\sphinxcolwidth{1}{4}}
\sphinxstylestrong{Processor}
\par
\vskip-\baselineskip\vbox{\hbox{\strut}}\end{varwidth}%
}%
&\sphinxmultirow{4}{31}{%
\begin{varwidth}[t]{\sphinxcolwidth{1}{4}}
Urban Energy
Balance
\par
\vskip-\baselineskip\vbox{\hbox{\strut}}\end{varwidth}%
}%
&
Anthropogenic
Heat
(Q:sub:\sphinxcode{\sphinxupquote{F}})
(LQF)
&
Spatial
variations
anthropogenic
heat release
for urban areas
\\
\cline{3-4}\sphinxtablestrut{30}&\sphinxtablestrut{31}&
\sphinxhref{http://www.urban-climate.net/umep/UMEP\_Manual\#Urban\_Energy\_Balance:\_GQF}{GQF}
&
Anthropogenic
Heat
(Q :sub:\sphinxcode{\sphinxupquote{F}}).
\\
\cline{3-4}\sphinxtablestrut{30}&\sphinxtablestrut{31}&
\sphinxhref{http://urban-climate.net/umep/UMEP\_Manual\#Urban\_Energy\_Balance:\_Urban\_Energy\_Balance\_.28SUEWS.2C\_simple.29}{SUEWS
(Simple)}
&
Urban Energy
and Water
Balance.
\\
\cline{3-4}\sphinxtablestrut{30}&\sphinxtablestrut{31}&
\sphinxhref{http://urban-climate.net/umep/UMEP\_Manual\#Urban\_Energy\_Balance:\_Urban\_Energy\_Balance\_.28SUEWS.2FBLUEWS.2C\_advanced.29}{SUEWS
(Advanced)}
&
Urban Energy
and Water
Balance.
\\
\hline
\sphinxstylestrong{Post-Processo
r}
&
Urban Energy
Balance
&
\sphinxhref{http://urban-climate.net/umep/UMEP\_Manual\#Urban\_Energy\_Balance:\_SUEWS\_Analyser}{SUEWS analyser}
&
Plugin for
plotting and
statistical
analysis of
model results
from SUEWS
simple and
SUEWS advanced
\\
\hline&
Benchmark
&
\sphinxhref{http://urban-climate.net/umep/UMEP\_Manual\#Benchmark\_System}{Benchmark
System}
&
For statistical
analysis of
model results,
such as SUEWS
\\
\hline
\end{tabulary}
\par
\sphinxattableend\end{savenotes}


\chapter{Parameterisations and sub-models within SUEWS}
\label{\detokenize{parameterisations-and-sub-models::doc}}\label{\detokenize{parameterisations-and-sub-models:parameterisations-and-sub-models-within-suews}}

\section{Net all-wave radiation, Q*}
\label{\detokenize{parameterisations-and-sub-models:net-all-wave-radiation-q}}
There are several options for modelling or using observed radiation
components depending on the data available. As a minimum, SUEWS requires
incoming shortwave radiation to be provided.
\begin{enumerate}
\item {} 
Observed net all-wave radiation can be provided as input instead of
being calculated by the model.

\item {} 
Observed incoming shortwave and incoming longwave components can be
provided as input, instead of incoming longwave being calculated by
the model.

\item {} 
Other data can be provided as input, such as cloud fraction (see
options in {\hyperref[\detokenize{parameterisations-and-sub-models:RunControl.nml}]{\emph{RunControl}}} (\autopageref*{\detokenize{parameterisations-and-sub-models:RunControl.nml}})).

\item {} 
\sphinxstylestrong{NARP} (Net All-wave Radiation Parameterization, Offerle et al.
2003 \phantomsection\label{\detokenize{parameterisations-and-sub-models:id1}}{\hyperref[\detokenize{references:o2003}]{\sphinxcrossref{{[}O2003{]}}}} (\autopageref*{\detokenize{references:o2003}}) , Loridan et al. 2011 \phantomsection\label{\detokenize{parameterisations-and-sub-models:id2}}{\hyperref[\detokenize{references:l2011}]{\sphinxcrossref{{[}L2011{]}}}} (\autopageref*{\detokenize{references:l2011}}) ) scheme calculates outgoing
shortwave and incoming and outgoing longwave radiation components
based on incoming shortwave radiation, temperature, relative humidity
and surface characteristics (albedo, emissivity).

\end{enumerate}


\section{Anthropogenic heat flux, Q$_{\text{F}}$}
\label{\detokenize{parameterisations-and-sub-models:anthropogenic-heat-flux-qf}}\begin{enumerate}
\item {} 
Two simple anthropogenic heat flux sub-models exist within SUEWS:
\begin{itemize}
\item {} 
Järvi et al. (2011) \phantomsection\label{\detokenize{parameterisations-and-sub-models:id3}}{\hyperref[\detokenize{references:j11}]{\sphinxcrossref{{[}J11{]}}}} (\autopageref*{\detokenize{references:j11}}) approach, based on heating and cooling
degree days and population density (allows distinction between
weekdays and weekends).

\item {} 
Loridan et al. (2011) \phantomsection\label{\detokenize{parameterisations-and-sub-models:id4}}{\hyperref[\detokenize{references:l2011}]{\sphinxcrossref{{[}L2011{]}}}} (\autopageref*{\detokenize{references:l2011}}) approach, based on a linear piece-wise
relation with air temperature.

\end{itemize}

\item {} 
Pre-calculated values can be supplied with the meteorological forcing
data, either derived from knowledge of the study site, or obtained
from other models, for example:
\begin{itemize}
\item {} 
\sphinxstylestrong{LUCY} (Allen et al. 2011 \phantomsection\label{\detokenize{parameterisations-and-sub-models:id5}}{\hyperref[\detokenize{references:lucy}]{\sphinxcrossref{{[}lucy{]}}}} (\autopageref*{\detokenize{references:lucy}}), Lindberg et al. 2013 \phantomsection\label{\detokenize{parameterisations-and-sub-models:id6}}{\hyperref[\detokenize{references:lucy2}]{\sphinxcrossref{{[}lucy2{]}}}} (\autopageref*{\detokenize{references:lucy2}})). A
new version has been now included in UMEP. To distinguish it is
referred to as
\sphinxhref{http://urban-climate.net/umep/LQF\_Manual}{**LQF**}

\item {} 
\sphinxstylestrong{GreaterQF} (Iamarino et al. 2011 \phantomsection\label{\detokenize{parameterisations-and-sub-models:id7}}{\hyperref[\detokenize{references:i11}]{\sphinxcrossref{{[}I11{]}}}} (\autopageref*{\detokenize{references:i11}})). A new version has been
now included in UMEP. To distinguish it is referred to as
\sphinxhref{http://urban-climate.net/umep/GQF\_Manual}{**GQF**}

\end{itemize}

\end{enumerate}


\section{Storage heat flux, \(\Delta\)Q$_{\text{S}}$}
\label{\detokenize{parameterisations-and-sub-models:storage-heat-flux-qs}}\begin{enumerate}
\item {} 
Three sub-models are available to estimate the storage heat flux:
\begin{itemize}
\item {} 
\sphinxstylestrong{OHM} (Objective Hysteresis Model, Grimmond et al. 1991 \phantomsection\label{\detokenize{parameterisations-and-sub-models:id8}}{\hyperref[\detokenize{references:g91ohm}]{\sphinxcrossref{{[}G91OHM{]}}}} (\autopageref*{\detokenize{references:g91ohm}}),
Grimmond \& Oke 1999a \phantomsection\label{\detokenize{parameterisations-and-sub-models:id9}}{\hyperref[\detokenize{references:go99qs}]{\sphinxcrossref{{[}GO99QS{]}}}} (\autopageref*{\detokenize{references:go99qs}}), 2002 \phantomsection\label{\detokenize{parameterisations-and-sub-models:id10}}{\hyperref[\detokenize{references:go2002}]{\sphinxcrossref{{[}GO2002{]}}}} (\autopageref*{\detokenize{references:go2002}})). Storage heat heat flux is
calculated using empirically-fitted relations with net all-wave
radiation and the rate of change in net all-wave radiation.

\item {} 
\sphinxstylestrong{AnOHM} (Analytical Objective Hysteresis Model, Sun et al.
2017 \phantomsection\label{\detokenize{parameterisations-and-sub-models:id11}}{\hyperref[\detokenize{references:anohm17}]{\sphinxcrossref{{[}AnOHM17{]}}}} (\autopageref*{\detokenize{references:anohm17}})). OHM approach using analytically-derived coefficients.
(\sphinxstylestrong{Not recommended in v2017b})

\item {} 
\sphinxstylestrong{ESTM} (Element Surface Temperature Method, Offerle et al.
2005 \phantomsection\label{\detokenize{parameterisations-and-sub-models:id12}}{\hyperref[\detokenize{references:oaf2005}]{\sphinxcrossref{{[}Oaf2005{]}}}} (\autopageref*{\detokenize{references:oaf2005}})). Heat transfer through urban facets (roof, wall, road,
interior) is calculated from surface temperature measurements and
knowledge of material properties. (\sphinxstylestrong{Not recommended in v2017b})

\end{itemize}

\item {} 
Alternatively, ‘observed’ storage heat flux can be supplied with the
meteorological forcing data.

\end{enumerate}


\section{Turbulent heat fluxes, Q$_{\text{H}}$ and Q$_{\text{E}}$}
\label{\detokenize{parameterisations-and-sub-models:turbulent-heat-fluxes-qh-and-qe}}\begin{enumerate}
\item {} 
\sphinxstylestrong{LUMPS} (Local-scale Urban Meteorological Parameterization Scheme,
Grimmond \& Oke 2002 \phantomsection\label{\detokenize{parameterisations-and-sub-models:id13}}{\hyperref[\detokenize{references:go2002}]{\sphinxcrossref{{[}GO2002{]}}}} (\autopageref*{\detokenize{references:go2002}})) provides a simple means of estimating
sensible and latent heat fluxes based on the proportion of vegetation
in the study area.

\item {} 
\sphinxstylestrong{SUEWS} adopts a more biophysical approach to calculate the latent
heat flux; the sensible heat flux is then calculated as the residual
of the energy balance. The initial estimate of stability is based on
the LUMPS calculations of sensible and latent heat flux. Future
versions will have alternative sensible heat and storage heat flux
options.

\end{enumerate}

Sensible and latent heat fluxes from both LUMPS and SUEWS are provided
in the {\hyperref[\detokenize{output_files/output_files:output-files}]{\sphinxcrossref{\DUrole{std,std-ref}{Output files}}}} (\autopageref*{\detokenize{output_files/output_files:output-files}}). Whether the turbulent heat
fluxes are calculated using LUMPS or SUEWS can have a major impact on
the results. For SUEWS, an appropriate surface conductance
parameterisation is also critical \phantomsection\label{\detokenize{parameterisations-and-sub-models:id14}}{\hyperref[\detokenize{references:j11}]{\sphinxcrossref{{[}J11{]}}}} (\autopageref*{\detokenize{references:j11}}) \phantomsection\label{\detokenize{parameterisations-and-sub-models:id15}}{\hyperref[\detokenize{references:w16}]{\sphinxcrossref{{[}W16{]}}}} (\autopageref*{\detokenize{references:w16}}). For more details see
{\hyperref[\detokenize{parameterisations-and-sub-models:Differences_between_SUEWS,_LUMPS_and_FRAISE}]{\emph{Differences between SUEWS, LUMPS and
FRAISE}}} (\autopageref*{\detokenize{parameterisations-and-sub-models:Differences_between_SUEWS,_LUMPS_and_FRAISE}}).


\section{Water balance}
\label{\detokenize{parameterisations-and-sub-models:water-balance}}
The running water balance at each time step is based on the urban water
balance model of Grimmond et al. (1986) \phantomsection\label{\detokenize{parameterisations-and-sub-models:id16}}{\hyperref[\detokenize{references:g86}]{\sphinxcrossref{{[}G86{]}}}} (\autopageref*{\detokenize{references:g86}}) and urban
evaporation-interception scheme of Grimmond and Oke (1991) \phantomsection\label{\detokenize{parameterisations-and-sub-models:id17}}{\hyperref[\detokenize{references:g91}]{\sphinxcrossref{{[}G91{]}}}} (\autopageref*{\detokenize{references:g91}}).
\begin{itemize}
\item {} 
Precipitation is a required variable in the meteorological forcing
file.

\item {} 
Irrigation can be modelled \phantomsection\label{\detokenize{parameterisations-and-sub-models:id18}}{\hyperref[\detokenize{references:j11}]{\sphinxcrossref{{[}J11{]}}}} (\autopageref*{\detokenize{references:j11}}) or observed values can be provided
if data are available.

\item {} 
Drainage equations and coefficients to use must be specified in the
input files.

\item {} 
Soil moisture can be calculated by the model (\sphinxstylestrong{Use of observed soil
moisture is not possible in v2017b}).

\item {} 
Runoff is permitted:
\begin{itemize}
\item {} 
between surface types within each model grid

\item {} 
between model grids (\sphinxstylestrong{Not implemented in v2017b})

\item {} 
to deep soil

\item {} 
to pipes.

\end{itemize}

\end{itemize}


\section{Snowmelt}
\label{\detokenize{parameterisations-and-sub-models:snowmelt}}
The snowmelt model within SUEWS is described in Järvi et al.
(2014) \phantomsection\label{\detokenize{parameterisations-and-sub-models:id19}}{\hyperref[\detokenize{references:leena2014}]{\sphinxcrossref{{[}Leena2014{]}}}} (\autopageref*{\detokenize{references:leena2014}}). Due to changes in the new model version (since v2016a)
when compared to the older versions, the snow calculation has slightly
changed. The main difference is that previously all surface state could
freeze in 1-h time step but now the amount of freezing surface state is
calculated similar way as melt water can freeze within the snow pack.
Also the snowmelt-related coefficients have slightly changed (see
{\hyperref[\detokenize{parameterisations-and-sub-models:SUEWS_Snow.txt}]{\emph{SUEWS\_Snow.txt}}} (\autopageref*{\detokenize{parameterisations-and-sub-models:SUEWS_Snow.txt}})).


\section{Convective boundary layer}
\label{\detokenize{parameterisations-and-sub-models:convective-boundary-layer}}
A convective boundary layer (CBL) slab model (Cleugh and Grimmond
2001 \phantomsection\label{\detokenize{parameterisations-and-sub-models:id20}}{\hyperref[\detokenize{references:cg2001}]{\sphinxcrossref{{[}CG2001{]}}}} (\autopageref*{\detokenize{references:cg2001}})) calculates the CBL height, temperature and humidity during
daytime (Onomura et al. 2015 \phantomsection\label{\detokenize{parameterisations-and-sub-models:id21}}{\hyperref[\detokenize{references:shiho2015}]{\sphinxcrossref{{[}Shiho2015{]}}}} (\autopageref*{\detokenize{references:shiho2015}})).


\section{Thermal comfort}
\label{\detokenize{parameterisations-and-sub-models:thermal-comfort}}
\sphinxstylestrong{SOLWEIG} (Solar and longwave environmental irradiance geometry model,
Lindberg et al. 2008 \phantomsection\label{\detokenize{parameterisations-and-sub-models:id22}}{\hyperref[\detokenize{references:fl2008}]{\sphinxcrossref{{[}FL2008{]}}}} (\autopageref*{\detokenize{references:fl2008}}), Lindberg and Grimmond 2011 \phantomsection\label{\detokenize{parameterisations-and-sub-models:id23}}{\hyperref[\detokenize{references:fl2011}]{\sphinxcrossref{{[}FL2011{]}}}} (\autopageref*{\detokenize{references:fl2011}})) is a 2D
radiation model to estimate mean radiant temperature.

\begin{figure}[htbp]
\centering

\noindent\sphinxincludegraphics{{Bluews_2}.jpg}
\end{figure}


\chapter{Preparing to run the model}
\label{\detokenize{prepare-to-run-the-model::doc}}\label{\detokenize{prepare-to-run-the-model:preparing-to-run-the-model}}
The following is to help with the model setup. Note that there is a
version of SUEWS in \sphinxhref{http://urban-climate.net/umep/UMEP\_Manual}{UMEP}
and there are some starting
\sphinxhref{http://urban-climate.net/umep/UMEP\_Manual\#Tutorials}{tutorials} for
that. The version there is the same (i.e. the executable) as the
standalone version so you can swap to that later once you have some
familiarity.


\section{Preparatory reading}
\label{\detokenize{prepare-to-run-the-model:preparatory-reading}}
Read the manual and relevant papers (and references therein):
\begin{itemize}
\item {} 
Järvi L, Grimmond CSB \& Christen A (2011) The Surface Urban Energy
and Water Balance Scheme (SUEWS): Evaluation in Los Angeles and
Vancouver. J. Hydrol. 411, 219-237.
\sphinxhref{http://www.sciencedirect.com/science/article/pii/S0022169411006937}{doi:10.1016/j.jhydrol.2011.10.00}

\item {} 
Järvi L, Grimmond CSB, Taka M, Nordbo A, Setälä H \& Strachan IB
(2014) Development of the Surface Urban Energy and Water balance
Scheme (SUEWS) for cold climate cities. Geosci. Model Dev. 7,
1691-1711.
\sphinxhref{http://www.geosci-model-dev.net/7/1691/2014/}{doi:10.5194/gmd-7-1691-2014}

\item {} 
Ward HC, Kotthaus S, Järvi L and Grimmond CSB (2016) Surface Urban
Energy and Water Balance Scheme (SUEWS): development and evaluation
at two UK sites. Urban Climate 18, 1-32.
\sphinxhref{http://www.sciencedirect.com/science/article/pii/S2212095516300256/}{doi:10.1016/j.uclim.2016.05.001}

\end{itemize}

\sphinxhref{http://urban-climate.net/umep/SUEWS\#Recent\_publications}{See other publications with example
applications}


\section{Decide what type of model run you are interested in}
\label{\detokenize{prepare-to-run-the-model:decide-what-type-of-model-run-you-are-interested-in}}

\begin{savenotes}\sphinxattablestart
\centering
\begin{tabular}[t]{|\X{50}{100}|\X{50}{100}|}
\hline
\sphinxstyletheadfamily &\sphinxstyletheadfamily 
Available in this release
\\
\hline
LUMPS
&
Yes \textendash{} not standalone
\\
\hline
SUEWS at a point or for an individual area
&
Yes
\\
\hline
SUEWS for multiple grids or areas
&
Yes
\\
\hline
SUEWS with Boundary Layer (BL)
&
Yes
\\
\hline
SUEWS with snow
&
Yes
\\
\hline
SUEWS with SOLWEIG
&
No
\\
\hline
SUEWS with SOLWEIG and BL
&
No
\\
\hline
\end{tabular}
\par
\sphinxattableend\end{savenotes}


\section{Download the program and example data files}
\label{\detokenize{prepare-to-run-the-model:download-the-program-and-example-data-files}}
Visit the website to receive a link to download the program and example
data files. Select the appropriate compiled version of the model to
download. For windows there is an installation version which will put
the programs and all the files into the appropriate place. There is also
a version linked to QGIS:
\sphinxhref{http://urban-climate.net/umep/UMEP}{**UMEP**}.

Note, as the definition of long double precision varies between
computers (e.g. Mac vs Windows) slightly different results may occur in
the output files.

Test/example files are given for the London KCL site, 2011 data (denoted
Kc11)

In the following SS is the site code (e.g. Kc), ss the grid ID, YYYY the
year and tt the time interval.


\begin{savenotes}\sphinxattablestart
\centering
\begin{tabular}[t]{|\X{33}{99}|\X{33}{99}|\X{33}{99}|}
\hline
\sphinxstyletheadfamily 
Filename
&\sphinxstyletheadfamily 
Description
&\sphinxstyletheadfamily 
Input/output
\\
\hline
SSss\_data.txt
&
Meteorological input
&
Input file (60-min)
\\
\hline
SSss\_YYYY\_data\_5.txt
&
Meteorological input
&
Input file (5-min)
\\
\hline
InitialConditionsSSss
&
Initial conditions
&
Input - \_YYYY.nml(+) file
\\
\hline
SUEWS\_SiteInfo\_SSss.x
&
Spreadsheet
&
Input lsm containing all other input information
\\
\hline
RunControl.nml
&
Sets model run
&
Input (located in options main directory)
\\
\hline
SS\_Filechoices.txt
&
Summary of model run
&
Output  options
\\
\hline
SSss\_YYYY\_5.txt
&
(Optional) 5-min
&
Output resolution output file
\\
\hline
SSss\_YYYY\_60.txt
&
60-min resolution
&
Output output file
\\
\hline
SSss\_DailyState.txt
&
Daily state variables
&
Output (all years in one file)
\\
\hline
\end{tabular}
\par
\sphinxattableend\end{savenotes}

(+) There is a second file InitialConditionsSSss\_YYYY\_EndOfRun.nml or
InitialConditionsSSss\_YYYY+1.nml in the input directory. At the end of
the run, and at the end of each year of the run, these files are written
out so that this information could be used to initialize further model
runs.


\section{Run the model for example data}
\label{\detokenize{prepare-to-run-the-model:run-the-model-for-example-data}}
Before running the model for your own data it is good to make certain
that you can run the test data and get the same results as in the
example files provided. It is recommended that you make a copy of the
example output files and put them somewhere else so you can compare the
results. When you run the program it will write over the supplied files.

To run the model you can use \sphinxstylestrong{Command Prompt} (in the directory where
the programme is located type the model name) or just double click the
executable file.

Please see {\hyperref[\detokenize{prepare-to-run-the-model:Troubleshooting}]{\emph{Troubleshooting}}} (\autopageref*{\detokenize{prepare-to-run-the-model:Troubleshooting}}) if you have problems
running the model.


\section{Preparation of data}
\label{\detokenize{prepare-to-run-the-model:preparation-of-data}}
This section describes the information required to run SUEWS for your
site. The input data can be summarised as follows:
\begin{enumerate}
\item {} 
Continuous \sphinxstyleemphasis{meteorological forcing data} for the entire period to be
modelled. Note you can not have gaps in the meteorological data. If
you need help with preparing the data you may want to use some of the
tools in
\sphinxhref{http://urban-climate.net/umep/UMEP\_Manual\#Meteorological\_Data:\_MetPreprocessor}{UMEP}.

\item {} 
Knowledge of the \sphinxstyleemphasis{surface and soil conditions immediately before the
start of the run} (if these initial conditions are not known, it is
usually possible to determine suitable values by running the model
and using the output at the end of the run to infer the conditions at
the start of the run).

\item {} 
The \sphinxstyleemphasis{location of the site} (latitude, longitude, altitude).

\item {} 
Information about the \sphinxstyleemphasis{characteristics of the surface}, including
land cover, heights of buildings and trees, radiative characteristics
(e.g. albedo, emissivity), drainage characteristics, soil
characteristics, snow characteristics, phenological characteristics
(e.g. seasonal cycle of LAI).

\item {} 
Information about \sphinxstyleemphasis{human behaviour}, including energy use and water
use (e.g. for irrigation or street cleaning) and snow clearing (if
applicable). The anthropogenic energy use and water use may be
provided as a time series in the meteorological forcing file if these
data are available or modelled based on parameters provided to the
model, including population density, hourly and weekly profiles of
energy and water use, information about the proportion of properties
using irrigation and the type of irrigation (automatic or manual).

\end{enumerate}

It is particularly important to ensure the following input information
is appropriate and representative of the site:
\begin{itemize}
\item {} 
Fractions of different land cover types and (less so) heights of
buildings \phantomsection\label{\detokenize{prepare-to-run-the-model:id1}}{\hyperref[\detokenize{references:w16}]{\sphinxcrossref{{[}W16{]}}}} (\autopageref*{\detokenize{references:w16}})

\item {} 
Accurate meteorological forcing data, particularly precipitation and
incoming shortwave radiation \phantomsection\label{\detokenize{prepare-to-run-the-model:id2}}{\hyperref[\detokenize{references:ko17}]{\sphinxcrossref{{[}Ko17{]}}}} (\autopageref*{\detokenize{references:ko17}})

\item {} 
Initial soil moisture conditions \phantomsection\label{\detokenize{prepare-to-run-the-model:id3}}{\hyperref[\detokenize{references:best2014}]{\sphinxcrossref{{[}Best2014{]}}}} (\autopageref*{\detokenize{references:best2014}})

\item {} 
Anthropogenic heat flux parameters, particularly if there are
considerable energy emissions from transport, buildings, metabolism,
etc \phantomsection\label{\detokenize{prepare-to-run-the-model:id4}}{\hyperref[\detokenize{references:w16}]{\sphinxcrossref{{[}W16{]}}}} (\autopageref*{\detokenize{references:w16}})

\item {} 
External water use (if irrigation or street cleaning occurs)

\item {} 
Snow clearing (if running the snow option)

\item {} 
Surface conductance parameterisation \phantomsection\label{\detokenize{prepare-to-run-the-model:id5}}{\hyperref[\detokenize{references:j11}]{\sphinxcrossref{{[}J11{]}}}} (\autopageref*{\detokenize{references:j11}}) \phantomsection\label{\detokenize{prepare-to-run-the-model:id6}}{\hyperref[\detokenize{references:w16}]{\sphinxcrossref{{[}W16{]}}}} (\autopageref*{\detokenize{references:w16}})

\end{itemize}

SUEWS can be run either for an individual area or for multiple areas.
There is no requirement for the areas to be of any particular shape but
here we refer to them as model ‘grids’.


\subsection{Preparation of site characteristics and model parameters}
\label{\detokenize{prepare-to-run-the-model:preparation-of-site-characteristics-and-model-parameters}}
The area to be modelled is described by a set of characteristics that
are specified in the {\hyperref[\detokenize{prepare-to-run-the-model:SUEWS_SiteSelect.txt}]{\emph{SUEWS\_SiteSelect.txt}}} (\autopageref*{\detokenize{prepare-to-run-the-model:SUEWS_SiteSelect.txt}})
file. Each row corresponds to one model grid for one year (i.e. running
a single grid over three years would require three rows; running two
grids over two years would require four rows). Characteristics are often
selected by a code for a particular set of conditions. For example, a
specific soil type (links to {\hyperref[\detokenize{prepare-to-run-the-model:SUEWS_Soil.txt}]{\emph{SUEWS\_Soil.txt}}} (\autopageref*{\detokenize{prepare-to-run-the-model:SUEWS_Soil.txt}})) or
characteristics of deciduous trees in a particular region (links to
{\hyperref[\detokenize{prepare-to-run-the-model:SUEWS_Veg.txt}]{\emph{SUEWS\_Veg.txt}}} (\autopageref*{\detokenize{prepare-to-run-the-model:SUEWS_Veg.txt}})). The intent is to build a library of
characteristics for different types of urban areas. The codes are
specified by the user, must be integer values and must be unique within
the first column of each input file, otherwise the model will return an
error. (Note in {\hyperref[\detokenize{prepare-to-run-the-model:SUEWS_SiteSelect.txt}]{\emph{SUEWS\_SiteSelect.txt}}} (\autopageref*{\detokenize{prepare-to-run-the-model:SUEWS_SiteSelect.txt}}) the
first column is labelled ‘Grid’ and can contain repeat values for
different years.) See {\hyperref[\detokenize{prepare-to-run-the-model:Input_files}]{\emph{Input files}}} (\autopageref*{\detokenize{prepare-to-run-the-model:Input_files}}) for details. Note
\sphinxhref{http://urban-climate.net/umep/UMEP}{UMEP} maybe helpful for
components of this.


\subsubsection{Land cover}
\label{\detokenize{prepare-to-run-the-model:land-cover}}
For each grid, the land cover must be classified using the following
surface types:


\begin{savenotes}\sphinxattablestart
\centering
\begin{tabular}[t]{|\X{25}{100}|\X{25}{100}|\X{50}{100}|}
\hline
\sphinxstyletheadfamily 
Classification
&\sphinxstyletheadfamily 
Surface type
&\sphinxstyletheadfamily 
File where characteristics are specified
\\
\hline\sphinxstyletheadfamily 
Non-vegetated
&
Paved surfaces
&
{\hyperref[\detokenize{input_files/SUEWS_SiteInfo/SUEWS_NonVeg:suews-nonveg-txt}]{\sphinxcrossref{\DUrole{std,std-ref,std,std-ref}{SUEWS\_NonVeg.txt}}}} (\autopageref*{\detokenize{input_files/SUEWS_SiteInfo/SUEWS_NonVeg:suews-nonveg-txt}})
\\
\hline\sphinxstyletheadfamily &
Building
&
{\hyperref[\detokenize{input_files/SUEWS_SiteInfo/SUEWS_NonVeg:suews-nonveg-txt}]{\sphinxcrossref{\DUrole{std,std-ref,std,std-ref}{SUEWS\_NonVeg.txt}}}} (\autopageref*{\detokenize{input_files/SUEWS_SiteInfo/SUEWS_NonVeg:suews-nonveg-txt}})
\\
\hline\sphinxstyletheadfamily &
Bare soil
&
{\hyperref[\detokenize{input_files/SUEWS_SiteInfo/SUEWS_NonVeg:suews-nonveg-txt}]{\sphinxcrossref{\DUrole{std,std-ref,std,std-ref}{SUEWS\_NonVeg.txt}}}} (\autopageref*{\detokenize{input_files/SUEWS_SiteInfo/SUEWS_NonVeg:suews-nonveg-txt}})
\\
\hline\sphinxstyletheadfamily 
Vegetation
&
Evergreen trees
&
{\hyperref[\detokenize{input_files/SUEWS_SiteInfo/SUEWS_Veg:suews-veg-txt}]{\sphinxcrossref{\DUrole{std,std-ref,std,std-ref}{SUEWS\_Veg.txt}}}} (\autopageref*{\detokenize{input_files/SUEWS_SiteInfo/SUEWS_Veg:suews-veg-txt}})
\\
\hline\sphinxstyletheadfamily &
Deciduous trees
&
{\hyperref[\detokenize{input_files/SUEWS_SiteInfo/SUEWS_Veg:suews-veg-txt}]{\sphinxcrossref{\DUrole{std,std-ref,std,std-ref}{SUEWS\_Veg.txt}}}} (\autopageref*{\detokenize{input_files/SUEWS_SiteInfo/SUEWS_Veg:suews-veg-txt}})
\\
\hline\sphinxstyletheadfamily &
Grass
&
{\hyperref[\detokenize{input_files/SUEWS_SiteInfo/SUEWS_Veg:suews-veg-txt}]{\sphinxcrossref{\DUrole{std,std-ref,std,std-ref}{SUEWS\_Veg.txt}}}} (\autopageref*{\detokenize{input_files/SUEWS_SiteInfo/SUEWS_Veg:suews-veg-txt}})
\\
\hline\sphinxstyletheadfamily 
Water
&
Water
&
{\hyperref[\detokenize{input_files/SUEWS_SiteInfo/SUEWS_Water:suews-water-txt}]{\sphinxcrossref{\DUrole{std,std-ref,std,std-ref}{SUEWS\_Water.txt}}}} (\autopageref*{\detokenize{input_files/SUEWS_SiteInfo/SUEWS_Water:suews-water-txt}})
\\
\hline\sphinxstyletheadfamily 
Snow
&
Snow
&
{\hyperref[\detokenize{input_files/SUEWS_SiteInfo/SUEWS_Snow:suews-snow-txt}]{\sphinxcrossref{\DUrole{std,std-ref,std,std-ref}{SUEWS\_Snow.txt}}}} (\autopageref*{\detokenize{input_files/SUEWS_SiteInfo/SUEWS_Snow:suews-snow-txt}})
\\
\hline
\end{tabular}
\par
\sphinxattableend\end{savenotes}

The surface cover fractions (i.e. proportion of the grid taken up by
each surface) must be specified in
{\hyperref[\detokenize{input_files/SUEWS_SiteInfo/SUEWS_SiteSelect:suews-siteselect-txt}]{\sphinxcrossref{\DUrole{std,std-ref,std,std-ref}{SUEWS\_SiteSelect.txt}}}} (\autopageref*{\detokenize{input_files/SUEWS_SiteInfo/SUEWS_SiteSelect:suews-siteselect-txt}}). The surface cover
fractions are \sphinxstylestrong{critical}, so make certain that the different surface
cover fractions are appropriate for your site.

For some locations, land cover information may be already available
(e.g. from various remote sensing resources). If not, websites like Bing
Maps and Google Maps allow you to see aerial images of your site and can
be used to estimate the relative proportion of each land cover type. If
detailed spatial datasets are available,
\sphinxhref{http://urban-climate.net/umep/UMEP}{UMEP} allows for a direct link
to a GIS environment using QGIS.


\subsubsection{Anthropogenic heat flux (Q $_{\text{F}}$)}
\label{\detokenize{prepare-to-run-the-model:anthropogenic-heat-flux-qf-1}}\label{\detokenize{prepare-to-run-the-model:anthropogenic-heat-flux-q-f}}
You can either model Q$_{\text{F}}$ within SUEWS or provide it as an input.
\begin{itemize}
\item {} 
To model it population density is needed as an input for LUMPS and
SUEWS to calculate Q$_{\text{F}}$.

\item {} 
If you have no information about the population of the site we
recommend that you use the LUCY model \phantomsection\label{\detokenize{prepare-to-run-the-model:id7}}{\hyperref[\detokenize{references:lucy}]{\sphinxcrossref{{[}lucy{]}}}} (\autopageref*{\detokenize{references:lucy}})  \phantomsection\label{\detokenize{prepare-to-run-the-model:id8}}{\hyperref[\detokenize{references:lucy2}]{\sphinxcrossref{{[}lucy2{]}}}} (\autopageref*{\detokenize{references:lucy2}}) to estimate the
anthropogenic heat flux which can then be provided as input SUEWS
along with the meteorological forcing data. The LUCY model can be
downloaded from \sphinxhref{http://micromet.reading.ac.uk/}{here}.

\end{itemize}

Alternatively, you can use the updated version of LUCY called
\sphinxhref{http://urban-climate.net/umep/LQF\_Manual}{LQF}, which is included in
\sphinxhref{http://urban-climate.net/umep/UMEP}{UMEP}.


\subsubsection{Other information}
\label{\detokenize{prepare-to-run-the-model:other-information}}
The surface cover fractions and population density can have a major
impact on the model output. However, it is important to consider the
suitability of all parameters for your site. Using inappropriate
parameters may result in the model returning an error or, worse,
generating output that is simply not representative of your site. Please
read the section on {\hyperref[\detokenize{prepare-to-run-the-model:Input_files}]{\emph{Input files}}} (\autopageref*{\detokenize{prepare-to-run-the-model:Input_files}}). Recommended or
reasonable ranges of values are suggested for some parameters, along
with important considerations for how to select appropriate values for
your site.


\subsubsection{Data Entry}
\label{\detokenize{prepare-to-run-the-model:data-entry}}
To create the series of input text files describing the characteristics
of your site, there are three options:
\begin{enumerate}
\item {} 
Data can be entered directly into the input text files. The example
(.txt) files provide a template to create your own files which can be
edited with a {\hyperref[\detokenize{prepare-to-run-the-model:A_text_editor}]{\emph{text editor}}} (\autopageref*{\detokenize{prepare-to-run-the-model:A_text_editor}}) directly.

\item {} 
Data can be entered into the spreadsheet \sphinxstylestrong{SUEWS\_SiteInfo.xlsm} and
the input text files generated by running the macro.

\item {} 
Use {[}\sphinxurl{http://urban-climate.net/umep/UMEP}\textbar{} UMEP{]}.

\end{enumerate}

\sphinxstylestrong{To run the xlsm macro:} Enter the data for your site into the xlsm
spreadsheet \sphinxstylestrong{SUEWS\_SiteInfo.xlsm} and then use the macro to create the
text files which will appear the same directory.

If there is a problem
\begin{itemize}
\item {} 
Make sure none of the text files to be generated are open.

\item {} 
It is recommended to close the spreadsheet before running the actual
model code.

\end{itemize}

Note that in all txt files:
\begin{itemize}
\item {} 
The first two rows are headers. The first row is the column number;
the second row is the column name.

\item {} 
The names and order of the columns should not be altered from the
templates, as these are checked by the model and errors will be
returned if particular columns cannot be found.

\item {} 
Since v2017a it is no longer necessary for the meteorological forcing
data to have two rows with -9 in column 1 as their last two rows.

\item {} 
“!” indicates a comment, so any text following “!” on the same line
will not be read by the model.

\item {} 
If data are unavailable or not required, enter the value -999 in the
correct place in the input file.

\item {} 
Ensure the units are correct for all input information. See {\hyperref[\detokenize{prepare-to-run-the-model:Input_files}]{\emph{Input
files}}} (\autopageref*{\detokenize{prepare-to-run-the-model:Input_files}}) for a description of parameters.

\end{itemize}

In addition to these text files, the following files are also needed to
run the model.


\subsection{Preparation of the RunControl file}
\label{\detokenize{prepare-to-run-the-model:preparation-of-the-runcontrol-file}}
In the RunControl.nml file the site name ({\color{red}\bfseries{}SS\_}) and directories for the
model input and output are given. This means \sphinxstylestrong{before running} the
model (even the with the example datasets) you must either
\begin{enumerate}
\item {} 
open the RunControl.nml file and edit the input and output file paths
and the site name (with a {\hyperref[\detokenize{prepare-to-run-the-model:A_text_editor}]{\emph{text editor}}} (\autopageref*{\detokenize{prepare-to-run-the-model:A_text_editor}})) so that
they are correct for your setup, or

\item {} 
create the directories specified in the RunControl.nml file

\end{enumerate}

From the given site identification the model identifies the input files
and generates the output files. For example if you specify:

\fvset{hllines={, ,}}%
\begin{sphinxVerbatim}[commandchars=\\\{\}]
FileOutputPath = “C:\PYGZbs{}FolderName\PYGZbs{}SUEWSOutput\PYGZbs{}”
\end{sphinxVerbatim}

and use site code SS the model creates an output file:

\fvset{hllines={, ,}}%
\begin{sphinxVerbatim}[commandchars=\\\{\}]
\PYG{n}{C}\PYG{p}{:}\PYGZbs{}\PYG{n}{FolderName}\PYGZbs{}\PYG{n}{SUEWSOutput}\PYGZbs{}\PYG{n}{SSss\PYGZus{}YYYY\PYGZus{}TT}\PYG{o}{.}\PYG{n}{txt}
\end{sphinxVerbatim}

\begin{sphinxadmonition}{note}{Note:}
remember to add the last backslash in windows and slash in Linux/Mac
\end{sphinxadmonition}

If the file paths are not correct the program will return an error when
run (see {\hyperref[\detokenize{prepare-to-run-the-model:Error_messages:_problems.txt}]{\emph{error messages}}} (\autopageref*{\detokenize{prepare-to-run-the-model:Error_messages:_problems.txt}})) and write
the error to the problems.txt file.


\subsection{Preparation of the Meteorological forcing data}
\label{\detokenize{prepare-to-run-the-model:preparation-of-the-meteorological-forcing-data}}
The model time-step is specified in {\hyperref[\detokenize{prepare-to-run-the-model:RunControl.nml}]{\emph{RunControl.nml}}} (\autopageref*{\detokenize{prepare-to-run-the-model:RunControl.nml}})
(5 min is highly recommended). If meteorological forcing data are not
available at this resolution, SUEWS has the option to downscale (e.g.
hourly) data to the time-step required. See details about the
{\hyperref[\detokenize{prepare-to-run-the-model:SSss_YYYY_data_tt.txt}]{\emph{meteorological forcing data}}} (\autopageref*{\detokenize{prepare-to-run-the-model:SSss_YYYY_data_tt.txt}}) to learn more
about choices of data input. Each grid can have its own meteorological
forcing file, or a single file can be used for all grids. The forcing
data should be representative of the local-scale, i.e. collected (or
derived) above the height of the roughness elements (buildings and
trees).


\subsection{Preparation of the InitialConditions file}
\label{\detokenize{prepare-to-run-the-model:preparation-of-the-initialconditions-file}}
Information about the surface state and meteorological conditions just
before the start of the run are provided in the Initial Conditions file.
At the very start of the run, each grid can have its own Initial
Conditions file, or a single file can be used for all grids. For details
see {\hyperref[\detokenize{prepare-to-run-the-model:InitialConditions}]{\emph{InitialConditions}}} (\autopageref*{\detokenize{prepare-to-run-the-model:InitialConditions}}).


\section{Run the model for your site}
\label{\detokenize{prepare-to-run-the-model:run-the-model-for-your-site}}
To run the model you can use \sphinxstylestrong{Command Prompt} (in the directory where
the programme is located type the model name) or just double click the
executable file.

Please see {\hyperref[\detokenize{prepare-to-run-the-model:Troubleshooting}]{\emph{Troubleshooting}}} (\autopageref*{\detokenize{prepare-to-run-the-model:Troubleshooting}}) if you have problems
running the model.


\section{Analyse the output}
\label{\detokenize{prepare-to-run-the-model:analyse-the-output}}
It is a good idea to perform initial checks that the model output looks
reasonable.


\begin{savenotes}\sphinxattablestart
\centering
\begin{tabular}[t]{|\X{20}{100}|\X{80}{100}|}
\hline
\sphinxstyletheadfamily 
Characteristic
&\sphinxstyletheadfamily 
Things to check
\\
\hline
Leaf area index
&\begin{description}
\item[{Does the phenologylook appropriate  (i.e. what does the seasonal cycle of  \sphinxhref{http://glossary.ametsoc.org/wiki/Leaf\_area\_index}{leaf area index (LAI)} look like?)}] \leavevmode\begin{itemize}
\item {} 
Are the leaves on the trees at approximately the right time of the year?

\end{itemize}

\end{description}
\\
\hline
Kdown
&\begin{description}
\item[{Is the timing of the diurnal cycle correct for the incoming solar radiation?}] \leavevmode\begin{itemize}
\item {} 
Although Kdown is a required input, it is also included in the output file. It is a good idea to check that the timing of Kdown in the output file is appropriate, as problems can indicate errors with the timestamp, incorrect time settings or problems with the disaggregation. In particular, make sure the sign of the longitude is specified correctly in {\hyperref[\detokenize{input_files/SUEWS_SiteInfo/SUEWS_SiteSelect:suews-siteselect-txt}]{\sphinxcrossref{\DUrole{std,std-ref,std,std-ref}{SUEWS\_SiteSelect.txt}}}} (\autopageref*{\detokenize{input_files/SUEWS_SiteInfo/SUEWS_SiteSelect:suews-siteselect-txt}}).

\item {} 
Checking solar angles (zenith and azimuth) can also be a useful check that the timing is correct.

\end{itemize}

\end{description}
\\
\hline
Albedo
&\begin{description}
\item[{Is the bulk albedo correct?}] \leavevmode\begin{itemize}
\item {} 
This is critical because a small error has an impact on all the fluxes (energy and hydrology).

\item {} 
If you have measurements of outgoing shortwave radiation compare these with the modelled values.

\item {} 
How do the values compare to literature values for your area?

\end{itemize}

\end{description}
\\
\hline
\end{tabular}
\par
\sphinxattableend\end{savenotes}


\section{Summary of files}
\label{\detokenize{prepare-to-run-the-model:summary-of-files}}
The table below lists the files required to run SUEWS and the output
files produced. SS is the two-letter code (specified in RunControl)
representing the site name, ss is the grid identification (integer
values between 0 and 2,147,483,647 (largest 4-byte integer)) and YYYY is
the year. TT is the resolution of the input/output file and tt is the
model time-step.

The last column indicates whether the files are needed/produced once per
run (1/run), or once per day (1/day), for each year (1/year) or for each
grid (1/grid):

\fvset{hllines={, ,}}%
\begin{sphinxVerbatim}[commandchars=\\\{\}]
[B] indicates files used with the CBL part of SUEWS (BLUEWS) and therefore are only needed/produced if this option is selected{}`{}`
[E] indicates files associated with ESTM storage heat flux models and therefore are only needed/produced if this option is selected{}`{}`
\end{sphinxVerbatim}


\chapter{Input files}
\label{\detokenize{input_files/input_files::doc}}\label{\detokenize{input_files/input_files:input-files}}\label{\detokenize{input_files/input_files:id1}}
SUEWS allows you to input a large number of parameters to describe the
characteristics of your site. You should not assume that the example
values provided in files or in the tables below are appropriate. Values
marked with ‘MD’ are examples of recommended values (see the suggested
references to help decide how appropriate these are for your site/model
domain); values marked with ‘MU’ need to be set (i.e. changed from the
example) for your site/model domain.
\phantomsection\label{\detokenize{input_files/RunControl/RunControl:runcontrol}}

\section{RunControl.nml}
\label{\detokenize{input_files/RunControl/RunControl:runcontrol}}\label{\detokenize{input_files/RunControl/RunControl:runcontrol-nml}}\label{\detokenize{input_files/RunControl/RunControl::doc}}\label{\detokenize{input_files/RunControl/RunControl:id1}}
The file \sphinxstylestrong{RunControl.nml} is a namelist that specifies the options for
the model run. It must be located in the same directory as the
executable file.

A sample file of \sphinxstylestrong{RunControl.nml} looks like

\fvset{hllines={, ,}}%
\begin{sphinxVerbatim}[commandchars=\\\{\}]
\PYGZam{}RunControl
CBLUse=0
SnowUse=0
SOLWEIGUse=0
NetRadiationMethod=3 
EmissionsMethod=2
StorageHeatMethod=3
OHMIncQF=0
StabilityMethod=2
RoughLenHeatMethod=2
RoughLenMomMethod=2
SMDMethod=0
WaterUseMethod=0
FileCode=\PYGZsq{}Saeve\PYGZsq{}
FileInputPath=\PYGZdq{}./Input/\PYGZdq{}
FileOutputPath=\PYGZdq{}./Output/\PYGZdq{}
MultipleMetFiles=0
MultipleInitFiles=0
MultipleESTMFiles=1
KeepTstepFilesIn=1
KeepTstepFilesOut=1
WriteOutOption=2
ResolutionFilesOut=3600
Tstep=300
ResolutionFilesIn=3600
ResolutionFilesInESTM=3600  !NEW
DisaggMethod=1          !NEW  (1 = default value, so don\PYGZsq{}t actually need here)
RainDisaggMethod=100    !NEW  (100 = default value, so don\PYGZsq{}t actually need here)
DisaggMethodESTM=1      !NEW  (1 = default value, so don\PYGZsq{}t actually need here)
SuppressWarnings=1      !NEW
KdownZen=0
diagnose=0
/
\end{sphinxVerbatim}

\begin{sphinxadmonition}{note}{Note:}\begin{itemize}
\item {} 
In \sphinxstyleemphasis{Linux} and \sphinxstyleemphasis{Mac}, please add an empty line after the end slash.

\item {} 
The file is not case-sensitive.

\item {} 
The parameters and variables can appear in any order.

\end{itemize}
\end{sphinxadmonition}

The parameters and their setting instructions are provided through the links below:
\begin{itemize}
\item {} 
{\hyperref[\detokenize{input_files/RunControl/Model_run_options:model-run-options}]{\sphinxcrossref{\DUrole{std,std-ref}{Model run options}}}} (\autopageref*{\detokenize{input_files/RunControl/Model_run_options:model-run-options}})
\begin{quote}
\begin{itemize}\setlength{\itemsep}{0pt}\setlength{\parskip}{0pt}
\item {} 
{\hyperref[\detokenize{input_files/RunControl/Model_run_options:cmdoption-arg-cbluse}]{\sphinxcrossref{\sphinxcode{\sphinxupquote{CBLuse}}}}} (\autopageref*{\detokenize{input_files/RunControl/Model_run_options:cmdoption-arg-cbluse}})

\item {} 
{\hyperref[\detokenize{input_files/RunControl/Model_run_options:cmdoption-arg-snowuse}]{\sphinxcrossref{\sphinxcode{\sphinxupquote{SnowUse}}}}} (\autopageref*{\detokenize{input_files/RunControl/Model_run_options:cmdoption-arg-snowuse}})

\item {} 
{\hyperref[\detokenize{input_files/RunControl/Model_run_options:cmdoption-arg-solweiguse}]{\sphinxcrossref{\sphinxcode{\sphinxupquote{SOLWEIGUse}}}}} (\autopageref*{\detokenize{input_files/RunControl/Model_run_options:cmdoption-arg-solweiguse}})

\item {} 
{\hyperref[\detokenize{input_files/RunControl/Model_run_options:cmdoption-arg-netradiationmethod}]{\sphinxcrossref{\sphinxcode{\sphinxupquote{NetRadiationMethod}}}}} (\autopageref*{\detokenize{input_files/RunControl/Model_run_options:cmdoption-arg-netradiationmethod}})

\item {} 
{\hyperref[\detokenize{input_files/RunControl/Model_run_options:cmdoption-arg-anthropheatmethod}]{\sphinxcrossref{\sphinxcode{\sphinxupquote{AnthropHeatMethod}}}}} (\autopageref*{\detokenize{input_files/RunControl/Model_run_options:cmdoption-arg-anthropheatmethod}})

\item {} 
{\hyperref[\detokenize{input_files/RunControl/Model_run_options:cmdoption-arg-anthropco2method}]{\sphinxcrossref{\sphinxcode{\sphinxupquote{AnthropCO2Method}}}}} (\autopageref*{\detokenize{input_files/RunControl/Model_run_options:cmdoption-arg-anthropco2method}})

\item {} 
{\hyperref[\detokenize{input_files/RunControl/Model_run_options:cmdoption-arg-storageheatmethod}]{\sphinxcrossref{\sphinxcode{\sphinxupquote{StorageHeatMethod}}}}} (\autopageref*{\detokenize{input_files/RunControl/Model_run_options:cmdoption-arg-storageheatmethod}})

\item {} 
{\hyperref[\detokenize{input_files/RunControl/Model_run_options:cmdoption-arg-ohmincqf}]{\sphinxcrossref{\sphinxcode{\sphinxupquote{OHMIncQF}}}}} (\autopageref*{\detokenize{input_files/RunControl/Model_run_options:cmdoption-arg-ohmincqf}})

\item {} 
{\hyperref[\detokenize{input_files/RunControl/Model_run_options:cmdoption-arg-stabilitymethod}]{\sphinxcrossref{\sphinxcode{\sphinxupquote{StabilityMethod}}}}} (\autopageref*{\detokenize{input_files/RunControl/Model_run_options:cmdoption-arg-stabilitymethod}})

\item {} 
{\hyperref[\detokenize{input_files/RunControl/Model_run_options:cmdoption-arg-roughlenheatmethod}]{\sphinxcrossref{\sphinxcode{\sphinxupquote{RoughLenHeatMethod}}}}} (\autopageref*{\detokenize{input_files/RunControl/Model_run_options:cmdoption-arg-roughlenheatmethod}})

\item {} 
{\hyperref[\detokenize{input_files/RunControl/Model_run_options:cmdoption-arg-roughlenmommethod}]{\sphinxcrossref{\sphinxcode{\sphinxupquote{RoughLenMomMethod}}}}} (\autopageref*{\detokenize{input_files/RunControl/Model_run_options:cmdoption-arg-roughlenmommethod}})

\item {} 
{\hyperref[\detokenize{input_files/RunControl/Model_run_options:cmdoption-arg-smdmethod}]{\sphinxcrossref{\sphinxcode{\sphinxupquote{SMDMethod}}}}} (\autopageref*{\detokenize{input_files/RunControl/Model_run_options:cmdoption-arg-smdmethod}})

\item {} 
{\hyperref[\detokenize{input_files/RunControl/Model_run_options:cmdoption-arg-waterusemethod}]{\sphinxcrossref{\sphinxcode{\sphinxupquote{WaterUseMethod}}}}} (\autopageref*{\detokenize{input_files/RunControl/Model_run_options:cmdoption-arg-waterusemethod}})

\end{itemize}
\end{quote}

\item {} 
{\hyperref[\detokenize{input_files/RunControl/File_related_options:file-related-options}]{\sphinxcrossref{\DUrole{std,std-ref}{File related options}}}} (\autopageref*{\detokenize{input_files/RunControl/File_related_options:file-related-options}})
\begin{quote}
\begin{itemize}\setlength{\itemsep}{0pt}\setlength{\parskip}{0pt}
\item {} 
{\hyperref[\detokenize{input_files/RunControl/File_related_options:cmdoption-arg-filecode}]{\sphinxcrossref{\sphinxcode{\sphinxupquote{FileCode}}}}} (\autopageref*{\detokenize{input_files/RunControl/File_related_options:cmdoption-arg-filecode}})

\item {} 
{\hyperref[\detokenize{input_files/RunControl/File_related_options:cmdoption-arg-fileinputpath}]{\sphinxcrossref{\sphinxcode{\sphinxupquote{FileInputPath}}}}} (\autopageref*{\detokenize{input_files/RunControl/File_related_options:cmdoption-arg-fileinputpath}})

\item {} 
{\hyperref[\detokenize{input_files/RunControl/File_related_options:cmdoption-arg-fileoutputpath}]{\sphinxcrossref{\sphinxcode{\sphinxupquote{FileOutputPath}}}}} (\autopageref*{\detokenize{input_files/RunControl/File_related_options:cmdoption-arg-fileoutputpath}})

\item {} 
{\hyperref[\detokenize{input_files/RunControl/File_related_options:cmdoption-arg-multiplemetfiles}]{\sphinxcrossref{\sphinxcode{\sphinxupquote{MultipleMetFiles}}}}} (\autopageref*{\detokenize{input_files/RunControl/File_related_options:cmdoption-arg-multiplemetfiles}})

\item {} 
{\hyperref[\detokenize{input_files/RunControl/File_related_options:cmdoption-arg-multipleinitfiles}]{\sphinxcrossref{\sphinxcode{\sphinxupquote{MultipleInitFiles}}}}} (\autopageref*{\detokenize{input_files/RunControl/File_related_options:cmdoption-arg-multipleinitfiles}})

\item {} 
{\hyperref[\detokenize{input_files/RunControl/File_related_options:cmdoption-arg-multipleestmfiles}]{\sphinxcrossref{\sphinxcode{\sphinxupquote{MultipleESTMFiles}}}}} (\autopageref*{\detokenize{input_files/RunControl/File_related_options:cmdoption-arg-multipleestmfiles}})

\item {} 
{\hyperref[\detokenize{input_files/RunControl/File_related_options:cmdoption-arg-keeptstepfilesin}]{\sphinxcrossref{\sphinxcode{\sphinxupquote{KeepTstepFilesIn}}}}} (\autopageref*{\detokenize{input_files/RunControl/File_related_options:cmdoption-arg-keeptstepfilesin}})

\item {} 
{\hyperref[\detokenize{input_files/RunControl/File_related_options:cmdoption-arg-keeptstepfilesout}]{\sphinxcrossref{\sphinxcode{\sphinxupquote{KeepTstepFilesOut}}}}} (\autopageref*{\detokenize{input_files/RunControl/File_related_options:cmdoption-arg-keeptstepfilesout}})

\item {} 
{\hyperref[\detokenize{input_files/RunControl/File_related_options:cmdoption-arg-writeoutoption}]{\sphinxcrossref{\sphinxcode{\sphinxupquote{WriteOutOption}}}}} (\autopageref*{\detokenize{input_files/RunControl/File_related_options:cmdoption-arg-writeoutoption}})

\item {} 
{\hyperref[\detokenize{input_files/RunControl/File_related_options:cmdoption-arg-suppresswarnings}]{\sphinxcrossref{\sphinxcode{\sphinxupquote{SuppressWarnings}}}}} (\autopageref*{\detokenize{input_files/RunControl/File_related_options:cmdoption-arg-suppresswarnings}})

\end{itemize}
\end{quote}

\item {} 
{\hyperref[\detokenize{input_files/RunControl/Time_related_options:time-related-options}]{\sphinxcrossref{\DUrole{std,std-ref}{Time related options}}}} (\autopageref*{\detokenize{input_files/RunControl/Time_related_options:time-related-options}})
\begin{quote}
\begin{itemize}\setlength{\itemsep}{0pt}\setlength{\parskip}{0pt}
\item {} 
{\hyperref[\detokenize{input_files/RunControl/Time_related_options:cmdoption-arg-tstep}]{\sphinxcrossref{\sphinxcode{\sphinxupquote{Tstep}}}}} (\autopageref*{\detokenize{input_files/RunControl/Time_related_options:cmdoption-arg-tstep}})

\item {} 
{\hyperref[\detokenize{input_files/RunControl/Time_related_options:cmdoption-arg-resolutionfilesin}]{\sphinxcrossref{\sphinxcode{\sphinxupquote{ResolutionFilesIn}}}}} (\autopageref*{\detokenize{input_files/RunControl/Time_related_options:cmdoption-arg-resolutionfilesin}})

\item {} 
{\hyperref[\detokenize{input_files/RunControl/Time_related_options:cmdoption-arg-resolutionfilesinestm}]{\sphinxcrossref{\sphinxcode{\sphinxupquote{ResolutionFilesInESTM}}}}} (\autopageref*{\detokenize{input_files/RunControl/Time_related_options:cmdoption-arg-resolutionfilesinestm}})

\item {} 
{\hyperref[\detokenize{input_files/RunControl/Time_related_options:cmdoption-arg-resolutionfilesout}]{\sphinxcrossref{\sphinxcode{\sphinxupquote{ResolutionFilesOut}}}}} (\autopageref*{\detokenize{input_files/RunControl/Time_related_options:cmdoption-arg-resolutionfilesout}})

\end{itemize}
\end{quote}

\item {} 
{\hyperref[\detokenize{input_files/RunControl/Options_related_to_disaggregation_of_input_data:options-related-to-disaggregation-of-input-data}]{\sphinxcrossref{\DUrole{std,std-ref}{Options related to disaggregation of input data}}}} (\autopageref*{\detokenize{input_files/RunControl/Options_related_to_disaggregation_of_input_data:options-related-to-disaggregation-of-input-data}})
\begin{quote}
\begin{itemize}\setlength{\itemsep}{0pt}\setlength{\parskip}{0pt}
\item {} 
{\hyperref[\detokenize{input_files/RunControl/Options_related_to_disaggregation_of_input_data:cmdoption-arg-disaggmethod}]{\sphinxcrossref{\sphinxcode{\sphinxupquote{DisaggMethod}}}}} (\autopageref*{\detokenize{input_files/RunControl/Options_related_to_disaggregation_of_input_data:cmdoption-arg-disaggmethod}})

\item {} 
{\hyperref[\detokenize{input_files/RunControl/Options_related_to_disaggregation_of_input_data:cmdoption-arg-kdownzen}]{\sphinxcrossref{\sphinxcode{\sphinxupquote{KdownZen}}}}} (\autopageref*{\detokenize{input_files/RunControl/Options_related_to_disaggregation_of_input_data:cmdoption-arg-kdownzen}})

\item {} 
{\hyperref[\detokenize{input_files/RunControl/Options_related_to_disaggregation_of_input_data:cmdoption-arg-raindisaggmethod}]{\sphinxcrossref{\sphinxcode{\sphinxupquote{RainDisaggMethod}}}}} (\autopageref*{\detokenize{input_files/RunControl/Options_related_to_disaggregation_of_input_data:cmdoption-arg-raindisaggmethod}})

\item {} 
{\hyperref[\detokenize{input_files/RunControl/Options_related_to_disaggregation_of_input_data:cmdoption-arg-rainamongn}]{\sphinxcrossref{\sphinxcode{\sphinxupquote{RainAmongN}}}}} (\autopageref*{\detokenize{input_files/RunControl/Options_related_to_disaggregation_of_input_data:cmdoption-arg-rainamongn}})

\item {} 
{\hyperref[\detokenize{input_files/RunControl/Options_related_to_disaggregation_of_input_data:cmdoption-arg-multrainamongn}]{\sphinxcrossref{\sphinxcode{\sphinxupquote{MultRainAmongN}}}}} (\autopageref*{\detokenize{input_files/RunControl/Options_related_to_disaggregation_of_input_data:cmdoption-arg-multrainamongn}})

\item {} 
{\hyperref[\detokenize{input_files/RunControl/Options_related_to_disaggregation_of_input_data:cmdoption-arg-multrainamongnupperi}]{\sphinxcrossref{\sphinxcode{\sphinxupquote{MultRainAmongNUpperI}}}}} (\autopageref*{\detokenize{input_files/RunControl/Options_related_to_disaggregation_of_input_data:cmdoption-arg-multrainamongnupperi}})

\item {} 
{\hyperref[\detokenize{input_files/RunControl/Options_related_to_disaggregation_of_input_data:cmdoption-arg-disaggmethodestm}]{\sphinxcrossref{\sphinxcode{\sphinxupquote{DisaggMethodESTM}}}}} (\autopageref*{\detokenize{input_files/RunControl/Options_related_to_disaggregation_of_input_data:cmdoption-arg-disaggmethodestm}})

\end{itemize}
\end{quote}

\item {} 
{\hyperref[\detokenize{input_files/RunControl/netCDF_related_options:netcdf-related-options}]{\sphinxcrossref{\DUrole{std,std-ref}{netCDF related options}}}} (\autopageref*{\detokenize{input_files/RunControl/netCDF_related_options:netcdf-related-options}})
\begin{quote}
\begin{itemize}\setlength{\itemsep}{0pt}\setlength{\parskip}{0pt}
\item {} 
{\hyperref[\detokenize{input_files/RunControl/netCDF_related_options:cmdoption-arg-ncmode}]{\sphinxcrossref{\sphinxcode{\sphinxupquote{ncMode}}}}} (\autopageref*{\detokenize{input_files/RunControl/netCDF_related_options:cmdoption-arg-ncmode}})

\item {} 
{\hyperref[\detokenize{input_files/RunControl/netCDF_related_options:cmdoption-arg-nrow}]{\sphinxcrossref{\sphinxcode{\sphinxupquote{nRow}}}}} (\autopageref*{\detokenize{input_files/RunControl/netCDF_related_options:cmdoption-arg-nrow}})

\item {} 
{\hyperref[\detokenize{input_files/RunControl/netCDF_related_options:cmdoption-arg-ncol}]{\sphinxcrossref{\sphinxcode{\sphinxupquote{nCol}}}}} (\autopageref*{\detokenize{input_files/RunControl/netCDF_related_options:cmdoption-arg-ncol}})

\end{itemize}
\end{quote}

\end{itemize}


\subsection{Model run options}
\label{\detokenize{input_files/RunControl/Model_run_options:model-run-options}}\label{\detokenize{input_files/RunControl/Model_run_options::doc}}\label{\detokenize{input_files/RunControl/Model_run_options:id1}}\index{command line option!CBLuse}\index{CBLuse!command line option}

\begin{fulllineitems}
\phantomsection\label{\detokenize{input_files/RunControl/Model_run_options:cmdoption-arg-cbluse}}\pysigline{\sphinxbfcode{\sphinxupquote{CBLuse}}\sphinxcode{\sphinxupquote{}}}~\begin{quote}\begin{description}
\item[{Requirement}] \leavevmode
Required

\item[{Description}] \leavevmode
Determines whether a CBL slab model is used to calculate temperature and humidity.

\item[{Configuration}] \leavevmode

\begin{savenotes}\sphinxattablestart
\centering
\begin{tabular}[t]{|\X{20}{100}|\X{80}{100}|}
\hline
\sphinxstyletheadfamily 
Value
&\sphinxstyletheadfamily 
Comments
\\
\hline
0
&\begin{quote}

CBL model not used. SUEWS and LUMPS use temperature and humidity provided in the meteorological forcing file.
\end{quote}
\\
\hline
1
&\begin{quote}

CBL model is used to calculate temperature and humidity used in SUEWS and LUMPS.
\end{quote}
\\
\hline
\end{tabular}
\par
\sphinxattableend\end{savenotes}

\end{description}\end{quote}

\end{fulllineitems}

\index{command line option!SnowUse}\index{SnowUse!command line option}

\begin{fulllineitems}
\phantomsection\label{\detokenize{input_files/RunControl/Model_run_options:cmdoption-arg-snowuse}}\pysigline{\sphinxbfcode{\sphinxupquote{SnowUse}}\sphinxcode{\sphinxupquote{}}}~\begin{quote}\begin{description}
\item[{Requirement}] \leavevmode
Required

\item[{Description}] \leavevmode
Determines whether the snow part of the model runs.

\item[{Configuration}] \leavevmode

\begin{savenotes}\sphinxattablestart
\centering
\begin{tabular}[t]{|\X{20}{100}|\X{80}{100}|}
\hline
\sphinxstyletheadfamily 
Value
&\sphinxstyletheadfamily 
Comments
\\
\hline
0
&\begin{quote}

Snow calculations are not performed.
\end{quote}
\\
\hline
1
&\begin{quote}

Snow calculations are performed.
\end{quote}
\\
\hline
\end{tabular}
\par
\sphinxattableend\end{savenotes}

\end{description}\end{quote}

\end{fulllineitems}

\index{command line option!SOLWEIGUse}\index{SOLWEIGUse!command line option}

\begin{fulllineitems}
\phantomsection\label{\detokenize{input_files/RunControl/Model_run_options:cmdoption-arg-solweiguse}}\pysigline{\sphinxbfcode{\sphinxupquote{SOLWEIGUse}}\sphinxcode{\sphinxupquote{}}}~\begin{quote}\begin{description}
\item[{Requirement}] \leavevmode
Required

\item[{Description}] \leavevmode
Determines whether a high resolution radiation model to calculate mean radiant temperate should be used (SOLWEIG). NOTE: this option will considerably slow down the model since SOLWEIG is a 2D model.

\item[{Configuration}] \leavevmode

\begin{savenotes}\sphinxattablestart
\centering
\begin{tabular}[t]{|\X{20}{100}|\X{80}{100}|}
\hline
\sphinxstyletheadfamily 
Value
&\sphinxstyletheadfamily 
Comments
\\
\hline
0
&\begin{quote}

SOLWEIG calculations are not performed.
\end{quote}
\\
\hline
1
&\begin{quote}

SOLWEIG calculations are performed. A grid of mean radiant temperature (Tmrt) is calculated based on high resolution digital surface models.
\end{quote}
\\
\hline
\end{tabular}
\par
\sphinxattableend\end{savenotes}

\end{description}\end{quote}

\end{fulllineitems}

\index{command line option!NetRadiationMethod}\index{NetRadiationMethod!command line option}

\begin{fulllineitems}
\phantomsection\label{\detokenize{input_files/RunControl/Model_run_options:cmdoption-arg-netradiationmethod}}\pysigline{\sphinxbfcode{\sphinxupquote{NetRadiationMethod}}\sphinxcode{\sphinxupquote{}}}~\begin{quote}\begin{description}
\item[{Requirement}] \leavevmode
Required

\item[{Description}] \leavevmode
Determines method for calculation of radiation fluxes.

\item[{Configuration}] \leavevmode

\begin{savenotes}\sphinxattablestart
\centering
\begin{tabular}[t]{|\X{20}{100}|\X{80}{100}|}
\hline
\sphinxstyletheadfamily 
Value
&\sphinxstyletheadfamily 
Comments
\\
\hline
0
&\begin{quote}

Uses observed values of Q* supplied in meteorological forcing file.
\end{quote}
\\
\hline
1
&
Q* modelled with L↓ observations supplied in meteorological forcing file.
Zenith angle not accounted for in albedo calculation.
\\
\hline
2
&
Q* modelled with L↓ modelled using cloud cover fraction supplied in meteorological forcing file (Loridan et al. 2011 {[}5{]} ).
Zenith angle not accounted for in albedo calculation.
\\
\hline
3
&
Q* modelled with L↓ modelled using air temperature and relative humidity supplied in meteorological forcing file (Loridan et al. 2011 {[}5{]} ).
Zenith angle not accounted for in albedo calculation.
\\
\hline
100
&
Q* modelled with L↓ observations supplied in meteorological forcing file.
Zenith angle accounted for in albedo calculation.
SSss\_YYYY\_NARPOut.txt file produced.
Not recommended in this release
\\
\hline
200
&
Q* modelled with L↓ modelled using cloud cover fraction supplied in meteorological forcing file (Loridan et al. 2011 {[}5{]} ).
Zenith angle accounted for in albedo calculation.
SSss\_YYYY\_NARPOut.txt file produced.
Not recommended in this release
\\
\hline
300
&
Q* modelled with L↓ modelled using air temperature and relative humidity supplied in meteorological forcing file (Loridan et al. 2011 {[}5{]} ).
Zenith angle accounted for in albedo calculation.
SSss\_YYYY\_NARPOut.txt file produced.
Not recommended in this release
\\
\hline
\end{tabular}
\par
\sphinxattableend\end{savenotes}

\end{description}\end{quote}

\end{fulllineitems}

\index{command line option!AnthropHeatMethod}\index{AnthropHeatMethod!command line option}

\begin{fulllineitems}
\phantomsection\label{\detokenize{input_files/RunControl/Model_run_options:cmdoption-arg-anthropheatmethod}}\pysigline{\sphinxbfcode{\sphinxupquote{AnthropHeatMethod}}\sphinxcode{\sphinxupquote{}}}~\begin{quote}\begin{description}
\item[{Requirement}] \leavevmode
Required

\item[{Description}] \leavevmode
Determines method for QF calculation.

\item[{Configuration}] \leavevmode

\begin{savenotes}\sphinxattablestart
\centering
\begin{tabular}[t]{|\X{20}{100}|\X{80}{100}|}
\hline
\sphinxstyletheadfamily 
Value
&\sphinxstyletheadfamily 
Comments
\\
\hline
0
&
Uses values provided in the meteorological forcing file (SSss\_YYYY\_data\_tt.txt).
If you do not want to include QF to the calculation of surface energy balance, you should set values in the meteorological forcing file to zero to prevent calculation of QF.
UMEP provides two methods to calculate QF
LQF which is simpler
GQF which is more complete but requires more data inputs
\\
\hline
1
&
Currently not recommended!
Calculated according to Loridan et al. (2011) {[}5{]} using coefficients specified in SUEWS\_AnthropogenicHeat.txt.
Modelled values will be used even if QF is provided in the meteorological forcing file.
\\
\hline
2
&
Recommended
Calculated according to Järvi et al. (2011) {[}1{]} using coefficients specified in SUEWS\_AnthropogenicHeat.txt and diurnal patterns specified in SUEWS\_Profiles.txt.
Modelled values will be used even if QF is provided in the meteorological forcing file.
\\
\hline
\end{tabular}
\par
\sphinxattableend\end{savenotes}

\end{description}\end{quote}

\end{fulllineitems}

\index{command line option!AnthropCO2Method}\index{AnthropCO2Method!command line option}

\begin{fulllineitems}
\phantomsection\label{\detokenize{input_files/RunControl/Model_run_options:cmdoption-arg-anthropco2method}}\pysigline{\sphinxbfcode{\sphinxupquote{AnthropCO2Method}}\sphinxcode{\sphinxupquote{}}}~\begin{quote}\begin{description}
\item[{Requirement}] \leavevmode
Required

\item[{Description}] \leavevmode
Determines method for CO2 calculation.

\item[{Configuration}] \leavevmode

\begin{savenotes}\sphinxattablestart
\centering
\begin{tabular}[t]{|\X{20}{100}|\X{80}{100}|}
\hline
\sphinxstyletheadfamily 
Value
&\sphinxstyletheadfamily 
Comments
\\
\hline
1
&\begin{quote}

Not used.
\end{quote}
\\
\hline
2
&
Under development - not recommended in v2017b
Calculate CO2 emissions from traffic based on QF calculation.
\\
\hline
3
&
Under development - not recommended in v2017b
Calculate CO2 emissions from traffic from input data provided.
\\
\hline
\end{tabular}
\par
\sphinxattableend\end{savenotes}

\end{description}\end{quote}

\end{fulllineitems}

\index{command line option!StorageHeatMethod}\index{StorageHeatMethod!command line option}

\begin{fulllineitems}
\phantomsection\label{\detokenize{input_files/RunControl/Model_run_options:cmdoption-arg-storageheatmethod}}\pysigline{\sphinxbfcode{\sphinxupquote{StorageHeatMethod}}\sphinxcode{\sphinxupquote{}}}~\begin{quote}\begin{description}
\item[{Requirement}] \leavevmode
Required

\item[{Description}] \leavevmode
Determines method for calculating storage heat flux \(\Delta\)QS.

\item[{Configuration}] \leavevmode

\begin{savenotes}\sphinxattablestart
\centering
\begin{tabular}[t]{|\X{20}{100}|\X{80}{100}|}
\hline
\sphinxstyletheadfamily 
Value
&\sphinxstyletheadfamily 
Comments
\\
\hline
1
&
\(\Delta\)QS modelled using the objective hysteresis model (OHM) {[}9{]} {[}10{]} {[}11{]} using parameters specified for each surface type.
\\
\hline
2
&
Uses observed values of \(\Delta\)QS supplied in meteorological forcing file.
\\
\hline
3
&
\(\Delta\)QS modelled using AnOHM.
Not available in v2017b
\\
\hline
4
&
\(\Delta\)QS modelled using the Element Surface Temperature Method (ESTM) (Offerle et al. 2005 {[}13{]} ).
Not recommended in v2017b
\\
\hline
\end{tabular}
\par
\sphinxattableend\end{savenotes}

\end{description}\end{quote}

\end{fulllineitems}

\index{command line option!OHMIncQF}\index{OHMIncQF!command line option}

\begin{fulllineitems}
\phantomsection\label{\detokenize{input_files/RunControl/Model_run_options:cmdoption-arg-ohmincqf}}\pysigline{\sphinxbfcode{\sphinxupquote{OHMIncQF}}\sphinxcode{\sphinxupquote{}}}~\begin{quote}\begin{description}
\item[{Requirement}] \leavevmode
Required

\item[{Description}] \leavevmode
Determines whether the storage heat flux calculation uses Q* or (Q*+QF).

\item[{Configuration}] \leavevmode

\begin{savenotes}\sphinxattablestart
\centering
\begin{tabular}[t]{|\X{20}{100}|\X{80}{100}|}
\hline
\sphinxstyletheadfamily 
Value
&\sphinxstyletheadfamily 
Comments
\\
\hline
0
&\begin{quote}

\(\Delta\)QS modelled Q* only.
\end{quote}
\\
\hline
1
&\begin{quote}

\(\Delta\)QS modelled using Q*+QF.
\end{quote}
\\
\hline
\end{tabular}
\par
\sphinxattableend\end{savenotes}

\end{description}\end{quote}

\end{fulllineitems}

\index{command line option!StabilityMethod}\index{StabilityMethod!command line option}

\begin{fulllineitems}
\phantomsection\label{\detokenize{input_files/RunControl/Model_run_options:cmdoption-arg-stabilitymethod}}\pysigline{\sphinxbfcode{\sphinxupquote{StabilityMethod}}\sphinxcode{\sphinxupquote{}}}~\begin{quote}\begin{description}
\item[{Requirement}] \leavevmode
Required

\item[{Description}] \leavevmode
Defines which atmospheric stability functions are used.

\item[{Configuration}] \leavevmode

\begin{savenotes}\sphinxattablestart
\centering
\begin{tabular}[t]{|\X{20}{100}|\X{80}{100}|}
\hline
\sphinxstyletheadfamily 
Value
&\sphinxstyletheadfamily 
Comments
\\
\hline
0
&\begin{quote}

Not used.
\end{quote}
\\
\hline
1
&\begin{quote}

Not used.
\end{quote}
\\
\hline
2
&
Recommended
Momentum - unstable: Dyer (1974) {[}22{]} modified by Högstrom (1988) {[}23{]} ; stable: Van Ulden and Holtslag (1985) {[}24{]}
Heat - Dyer (1974) {[}22{]} modified by Högstrom (1988) {[}23{]}
\\
\hline
3
&
Momentum: Campbell and Norman (Eq 7.27, Pg97) {[}25{]}
Heat - unstable: Campbell and Norman {[}25{]} ; stable: Dyer (1974) {[}22{]} modified by Högstrom (1988) {[}23{]}
\\
\hline
4
&
Momentum: Businger et al. (1971) {[}26{]} modified by Högstrom (1988) {[}23{]}
Heat: Businger et al. (1971) {[}26{]} modified by Högstrom (1988) {[}23{]}
\\
\hline
\end{tabular}
\par
\sphinxattableend\end{savenotes}

\end{description}\end{quote}

\end{fulllineitems}

\index{command line option!RoughLenHeatMethod}\index{RoughLenHeatMethod!command line option}

\begin{fulllineitems}
\phantomsection\label{\detokenize{input_files/RunControl/Model_run_options:cmdoption-arg-roughlenheatmethod}}\pysigline{\sphinxbfcode{\sphinxupquote{RoughLenHeatMethod}}\sphinxcode{\sphinxupquote{}}}~\begin{quote}\begin{description}
\item[{Requirement}] \leavevmode
Required

\item[{Description}] \leavevmode
Determines method for calculating roughness length for heat.

\item[{Configuration}] \leavevmode

\begin{savenotes}\sphinxattablestart
\centering
\begin{tabular}[t]{|\X{20}{100}|\X{80}{100}|}
\hline
\sphinxstyletheadfamily 
Value
&\sphinxstyletheadfamily 
Comments
\\
\hline
1
&
Uses value of 0.1z0m.
\\
\hline
2
&
Recommended
Calculated according to Kawai et al. (2009) {[}27{]} .
\\
\hline
3
&
Calculated according to Voogt and Grimmond (2000) {[}28{]} .
\\
\hline
4
&
Calculated according to Kanda et al. (2007) {[}29{]} .
\\
\hline
\end{tabular}
\par
\sphinxattableend\end{savenotes}

\end{description}\end{quote}

\end{fulllineitems}

\index{command line option!RoughLenMomMethod}\index{RoughLenMomMethod!command line option}

\begin{fulllineitems}
\phantomsection\label{\detokenize{input_files/RunControl/Model_run_options:cmdoption-arg-roughlenmommethod}}\pysigline{\sphinxbfcode{\sphinxupquote{RoughLenMomMethod}}\sphinxcode{\sphinxupquote{}}}~\begin{quote}\begin{description}
\item[{Requirement}] \leavevmode
Required

\item[{Description}] \leavevmode
Determines how aerodynamic roughness length (z0m) and zero displacement height (zdm) are calculated.

\item[{Configuration}] \leavevmode

\begin{savenotes}\sphinxattablestart
\centering
\begin{tabular}[t]{|\X{20}{100}|\X{80}{100}|}
\hline
\sphinxstyletheadfamily 
Value
&\sphinxstyletheadfamily 
Comments
\\
\hline
1
&
Values specified in SUEWS\_SiteSelect.txt are used. Note that UMEP provides tools to calculate these{]}. See Kent et al. (2017a) for recommendations on methods. Kent et al. (2017b) have developed a method to include vegetation which is also avaialble within UMEP.
Kent CW, CSB Grimmond, J Barlow, D Gatey, S Kotthaus, F Lindberg, CH Halios 2017a: Evaluation of urban local-scale aerodynamic parameters: implications for the vertical profile of wind and source areas Boundary Layer Meteorology 164,183\textendash{}213 doi: 10.1007/s10546-017-0248-z
Kent CW, S Grimmond, D Gatey 2017b: Aerodynamic roughness parameters in cities: inclusion of vegetation Journal of Wind Engineering \& Industrial Aerodynamics \sphinxurl{http://dx.doi.org/10.1016/j.jweia.2017.07.016}
\\
\hline
2
&
z0m and zd are calculated using ‘rule of thumb’ (Grimmond and Oke 1999 {[}30{]} ) using mean building and tree height specified in SUEWS\_SiteSelect.txt .
z0m and zd are adjusted with time to account for seasonal variation in porosity of deciduous trees.
\\
\hline
3
&
z0m and zd are calculated based on the MacDonald et al. (1998) {[}31{]} method using mean building and tree heights, plan area fraction and frontal areal index specified in SUEWS\_SiteSelect.txt .
z0m and zd are adjusted with time to account for seasonal variation in porosity of deciduous trees.
\\
\hline
\end{tabular}
\par
\sphinxattableend\end{savenotes}

\end{description}\end{quote}

\end{fulllineitems}

\index{command line option!SMDMethod}\index{SMDMethod!command line option}

\begin{fulllineitems}
\phantomsection\label{\detokenize{input_files/RunControl/Model_run_options:cmdoption-arg-smdmethod}}\pysigline{\sphinxbfcode{\sphinxupquote{SMDMethod}}\sphinxcode{\sphinxupquote{}}}~\begin{quote}\begin{description}
\item[{Requirement}] \leavevmode
Required

\item[{Description}] \leavevmode
Determines method for calculating soil moisture deficit (SMD).

\item[{Configuration}] \leavevmode

\begin{savenotes}\sphinxattablestart
\centering
\begin{tabular}[t]{|\X{20}{100}|\X{80}{100}|}
\hline
\sphinxstyletheadfamily 
Value
&\sphinxstyletheadfamily 
Comments
\\
\hline
0
&
Recommended
SMD modelled using parameters specified in SUEWS\_Soil.txt .
\\
\hline
1
&
Not currently implemented - do not use!
Observed SM provided in the meteorological forcing file is used.
Data are provided as volumetric soil moisture content. Metadata must be provided in SUEWS\_Soil.txt .
\\
\hline
2
&
Not currently implemented - do not use!
Observed SM provided in the meteorological forcing file is used.
Data are provided as gravimetric soil moisture content. Metadata must be provided in SUEWS\_Soil.txt .
\\
\hline
\end{tabular}
\par
\sphinxattableend\end{savenotes}

\end{description}\end{quote}

\end{fulllineitems}

\index{command line option!WaterUseMethod}\index{WaterUseMethod!command line option}

\begin{fulllineitems}
\phantomsection\label{\detokenize{input_files/RunControl/Model_run_options:cmdoption-arg-waterusemethod}}\pysigline{\sphinxbfcode{\sphinxupquote{WaterUseMethod}}\sphinxcode{\sphinxupquote{}}}~\begin{quote}\begin{description}
\item[{Requirement}] \leavevmode
Required

\item[{Description}] \leavevmode
Defines how external water use is calculated.

\item[{Configuration}] \leavevmode

\begin{savenotes}\sphinxattablestart
\centering
\begin{tabular}[t]{|\X{20}{100}|\X{80}{100}|}
\hline
\sphinxstyletheadfamily 
Value
&\sphinxstyletheadfamily 
Comments
\\
\hline
0
&\begin{quote}

External water use modelled using parameters specified in SUEWS\_Irrigation.txt .
\end{quote}
\\
\hline
1
&\begin{quote}

Observations of external water use provided in the meteorological forcing file are used.
\end{quote}
\\
\hline
\end{tabular}
\par
\sphinxattableend\end{savenotes}

\end{description}\end{quote}

\end{fulllineitems}



\subsection{Time related options}
\label{\detokenize{input_files/RunControl/Time_related_options:time-related-options}}\label{\detokenize{input_files/RunControl/Time_related_options::doc}}\label{\detokenize{input_files/RunControl/Time_related_options:id1}}\index{command line option!Tstep}\index{Tstep!command line option}

\begin{fulllineitems}
\phantomsection\label{\detokenize{input_files/RunControl/Time_related_options:cmdoption-arg-tstep}}\pysigline{\sphinxbfcode{\sphinxupquote{Tstep}}\sphinxcode{\sphinxupquote{}}}~\begin{quote}\begin{description}
\item[{Requirement}] \leavevmode
Required

\item[{Description}] \leavevmode
Specifies the model time step {[}s{]}. A value of 300 s (5 min) is strongly recommended. The time step cannot be less than 1 min or greater than 10 min, and must be a whole number of minutes that divide into an hour (i.e. options are 1, 2, 3, 4, 5, 6, 10 min or 60, 120, 180, 240, 300, 360, 600 s).

\item[{Configuration}] \leavevmode
to fill

\end{description}\end{quote}

\end{fulllineitems}

\index{command line option!ResolutionFilesIn}\index{ResolutionFilesIn!command line option}

\begin{fulllineitems}
\phantomsection\label{\detokenize{input_files/RunControl/Time_related_options:cmdoption-arg-resolutionfilesin}}\pysigline{\sphinxbfcode{\sphinxupquote{ResolutionFilesIn}}\sphinxcode{\sphinxupquote{}}}~\begin{quote}\begin{description}
\item[{Requirement}] \leavevmode
Required

\item[{Description}] \leavevmode
Specifies the resolution of the input files {[}s{]} which SUEWS will disaggregate to the model time step. 1800 s for 30 min or 3600 s for 60 min are recommended. (N.B. if ResolutionFilesIn is not provided, SUEWS assumes ResolutionFilesIn = Tstep.)

\item[{Configuration}] \leavevmode
to fill

\end{description}\end{quote}

\end{fulllineitems}

\index{command line option!ResolutionFilesInESTM}\index{ResolutionFilesInESTM!command line option}

\begin{fulllineitems}
\phantomsection\label{\detokenize{input_files/RunControl/Time_related_options:cmdoption-arg-resolutionfilesinestm}}\pysigline{\sphinxbfcode{\sphinxupquote{ResolutionFilesInESTM}}\sphinxcode{\sphinxupquote{}}}~\begin{quote}\begin{description}
\item[{Requirement}] \leavevmode
Optional

\item[{Description}] \leavevmode
Specifies the resolution of the ESTM input files {[}s{]} which SUEWS will disaggregate to the model time step.

\item[{Configuration}] \leavevmode
to fill

\end{description}\end{quote}

\end{fulllineitems}

\index{command line option!ResolutionFilesOut}\index{ResolutionFilesOut!command line option}

\begin{fulllineitems}
\phantomsection\label{\detokenize{input_files/RunControl/Time_related_options:cmdoption-arg-resolutionfilesout}}\pysigline{\sphinxbfcode{\sphinxupquote{ResolutionFilesOut}}\sphinxcode{\sphinxupquote{}}}~\begin{quote}\begin{description}
\item[{Requirement}] \leavevmode
Required

\item[{Description}] \leavevmode
Specifies the resolution of the output files {[}s{]}. 1800 s for 30 min or 3600 s for 60 min are recommended.

\item[{Configuration}] \leavevmode
to fill

\end{description}\end{quote}

\end{fulllineitems}



\subsection{File related options}
\label{\detokenize{input_files/RunControl/File_related_options:file-related-options}}\label{\detokenize{input_files/RunControl/File_related_options::doc}}\label{\detokenize{input_files/RunControl/File_related_options:id1}}\index{command line option!FileCode}\index{FileCode!command line option}

\begin{fulllineitems}
\phantomsection\label{\detokenize{input_files/RunControl/File_related_options:cmdoption-arg-filecode}}\pysigline{\sphinxbfcode{\sphinxupquote{FileCode}}\sphinxcode{\sphinxupquote{}}}~\begin{quote}\begin{description}
\item[{Requirement}] \leavevmode
Required

\item[{Description}] \leavevmode
Two-letter site identification code (e.g. He, Sc, Kc).

\item[{Configuration}] \leavevmode
to fill

\end{description}\end{quote}

\end{fulllineitems}

\index{command line option!FileInputPath}\index{FileInputPath!command line option}

\begin{fulllineitems}
\phantomsection\label{\detokenize{input_files/RunControl/File_related_options:cmdoption-arg-fileinputpath}}\pysigline{\sphinxbfcode{\sphinxupquote{FileInputPath}}\sphinxcode{\sphinxupquote{}}}~\begin{quote}\begin{description}
\item[{Requirement}] \leavevmode
Required

\item[{Description}] \leavevmode
Input directory.

\item[{Configuration}] \leavevmode
to fill

\end{description}\end{quote}

\end{fulllineitems}

\index{command line option!FileOutputPath}\index{FileOutputPath!command line option}

\begin{fulllineitems}
\phantomsection\label{\detokenize{input_files/RunControl/File_related_options:cmdoption-arg-fileoutputpath}}\pysigline{\sphinxbfcode{\sphinxupquote{FileOutputPath}}\sphinxcode{\sphinxupquote{}}}~\begin{quote}\begin{description}
\item[{Requirement}] \leavevmode
Required

\item[{Description}] \leavevmode
Output directory.

\item[{Configuration}] \leavevmode
to fill

\end{description}\end{quote}

\end{fulllineitems}

\index{command line option!MultipleMetFiles}\index{MultipleMetFiles!command line option}

\begin{fulllineitems}
\phantomsection\label{\detokenize{input_files/RunControl/File_related_options:cmdoption-arg-multiplemetfiles}}\pysigline{\sphinxbfcode{\sphinxupquote{MultipleMetFiles}}\sphinxcode{\sphinxupquote{}}}~\begin{quote}\begin{description}
\item[{Requirement}] \leavevmode
Required

\item[{Description}] \leavevmode
Specifies whether one single meteorological forcing file is used for all grids or a separate met file is provided for each grid.

\item[{Configuration}] \leavevmode

\begin{savenotes}\sphinxattablestart
\centering
\begin{tabular}[t]{|\X{20}{100}|\X{80}{100}|}
\hline
\sphinxstyletheadfamily 
Value
&\sphinxstyletheadfamily 
Comments
\\
\hline
0
&
Single meteorological forcing file used for all grids.
No grid number should appear in the file name.
\\
\hline
1
&
Separate meteorological forcing files used for each grid.
The grid number should appear in the file name.
\\
\hline
\end{tabular}
\par
\sphinxattableend\end{savenotes}

\end{description}\end{quote}

\end{fulllineitems}

\index{command line option!MultipleInitFiles}\index{MultipleInitFiles!command line option}

\begin{fulllineitems}
\phantomsection\label{\detokenize{input_files/RunControl/File_related_options:cmdoption-arg-multipleinitfiles}}\pysigline{\sphinxbfcode{\sphinxupquote{MultipleInitFiles}}\sphinxcode{\sphinxupquote{}}}~\begin{quote}\begin{description}
\item[{Requirement}] \leavevmode
Required

\item[{Description}] \leavevmode
Specifies whether one single initial conditions file is used for all grids at the start of the run or a separate initial conditions file is provided for each grid.

\item[{Configuration}] \leavevmode

\begin{savenotes}\sphinxattablestart
\centering
\begin{tabular}[t]{|\X{20}{100}|\X{80}{100}|}
\hline
\sphinxstyletheadfamily 
Value
&\sphinxstyletheadfamily 
Comments
\\
\hline
0
&
Single initial conditions file used for all grids.
No grid number should appear in the file name.
\\
\hline
1
&
Separate initial conditions files used for each grid.
The grid number should appear in the file name.
\\
\hline
\end{tabular}
\par
\sphinxattableend\end{savenotes}

\end{description}\end{quote}

\end{fulllineitems}

\index{command line option!MultipleESTMFiles}\index{MultipleESTMFiles!command line option}

\begin{fulllineitems}
\phantomsection\label{\detokenize{input_files/RunControl/File_related_options:cmdoption-arg-multipleestmfiles}}\pysigline{\sphinxbfcode{\sphinxupquote{MultipleESTMFiles}}\sphinxcode{\sphinxupquote{}}}~\begin{quote}\begin{description}
\item[{Requirement}] \leavevmode
Optional

\item[{Description}] \leavevmode
Specifies whether one single ESTM forcing file is used for all grids or a separate file is provided for each grid.

\item[{Configuration}] \leavevmode

\begin{savenotes}\sphinxattablestart
\centering
\begin{tabular}[t]{|\X{20}{100}|\X{80}{100}|}
\hline
\sphinxstyletheadfamily 
Value
&\sphinxstyletheadfamily 
Comments
\\
\hline
0
&
Single ESTM forcing file used for all grids.
No grid number should appear in the file name.
\\
\hline
1
&
Separate ESTM forcing files used for each grid.
The grid number should appear in the file name.
\\
\hline
\end{tabular}
\par
\sphinxattableend\end{savenotes}

\end{description}\end{quote}

\end{fulllineitems}

\index{command line option!KeepTstepFilesIn}\index{KeepTstepFilesIn!command line option}

\begin{fulllineitems}
\phantomsection\label{\detokenize{input_files/RunControl/File_related_options:cmdoption-arg-keeptstepfilesin}}\pysigline{\sphinxbfcode{\sphinxupquote{KeepTstepFilesIn}}\sphinxcode{\sphinxupquote{}}}~\begin{quote}\begin{description}
\item[{Requirement}] \leavevmode
Optional

\item[{Description}] \leavevmode
Specifies whether input meteorological forcing files at the resolution of the model time step should be saved.

\item[{Configuration}] \leavevmode

\begin{savenotes}\sphinxattablestart
\centering
\begin{tabular}[t]{|\X{20}{100}|\X{80}{100}|}
\hline
\sphinxstyletheadfamily 
Value
&\sphinxstyletheadfamily 
Comments
\\
\hline
0
&
Meteorological forcing files at model time step are not written out. This is the default option
Recommended to reduce processing time and save disk space as (e.g. 5-min) files can be large.
\\
\hline
1
&
Meteorological forcing files at model time step are written out.
\\
\hline
\end{tabular}
\par
\sphinxattableend\end{savenotes}

\end{description}\end{quote}

\end{fulllineitems}

\index{command line option!KeepTstepFilesOut}\index{KeepTstepFilesOut!command line option}

\begin{fulllineitems}
\phantomsection\label{\detokenize{input_files/RunControl/File_related_options:cmdoption-arg-keeptstepfilesout}}\pysigline{\sphinxbfcode{\sphinxupquote{KeepTstepFilesOut}}\sphinxcode{\sphinxupquote{}}}~\begin{quote}\begin{description}
\item[{Requirement}] \leavevmode
Optional

\item[{Description}] \leavevmode
Specifies whether output meteorological forcing files at the resolution of the model time step should be saved.

\item[{Configuration}] \leavevmode

\begin{savenotes}\sphinxattablestart
\centering
\begin{tabular}[t]{|\X{20}{100}|\X{80}{100}|}
\hline
\sphinxstyletheadfamily 
Value
&\sphinxstyletheadfamily 
Comments
\\
\hline
0
&
Output files at model time are not saved. This is the default option.
Recommended to save disk space as (e.g. 5-min) files can be large.
\\
\hline
1
&
Output files at model time step are written out.
\\
\hline
\end{tabular}
\par
\sphinxattableend\end{savenotes}

\end{description}\end{quote}

\end{fulllineitems}

\index{command line option!WriteOutOption}\index{WriteOutOption!command line option}

\begin{fulllineitems}
\phantomsection\label{\detokenize{input_files/RunControl/File_related_options:cmdoption-arg-writeoutoption}}\pysigline{\sphinxbfcode{\sphinxupquote{WriteOutOption}}\sphinxcode{\sphinxupquote{}}}~\begin{quote}\begin{description}
\item[{Requirement}] \leavevmode
Optional

\item[{Description}] \leavevmode
Specifies which variables are written in the output files.

\item[{Configuration}] \leavevmode

\begin{savenotes}\sphinxattablestart
\centering
\begin{tabular}[t]{|\X{20}{100}|\X{80}{100}|}
\hline
\sphinxstyletheadfamily 
Value
&\sphinxstyletheadfamily 
Comments
\\
\hline
0
&
All (except snow-related) output variables written. This is the default option.
\\
\hline
1
&
All (including snow-related) output variables written.
\\
\hline
2
&
Writes out a minimal set of output variables (use this to save space or if information about the different surfaces is not required).
\\
\hline
\end{tabular}
\par
\sphinxattableend\end{savenotes}

\end{description}\end{quote}

\end{fulllineitems}

\index{command line option!SuppressWarnings}\index{SuppressWarnings!command line option}

\begin{fulllineitems}
\phantomsection\label{\detokenize{input_files/RunControl/File_related_options:cmdoption-arg-suppresswarnings}}\pysigline{\sphinxbfcode{\sphinxupquote{SuppressWarnings}}\sphinxcode{\sphinxupquote{}}}~\begin{quote}\begin{description}
\item[{Requirement}] \leavevmode
Optional

\item[{Description}] \leavevmode
Controls whether the warnings.txt file is written or not.

\item[{Configuration}] \leavevmode

\begin{savenotes}\sphinxattablestart
\centering
\begin{tabular}[t]{|\X{20}{100}|\X{80}{100}|}
\hline
\sphinxstyletheadfamily 
Value
&\sphinxstyletheadfamily 
Comments
\\
\hline
0
&
The warnings.txt file is written. This is the default option.
\\
\hline
1
&
No warnings.txt file is written. May be useful for large model runs as this file can grow large.
\\
\hline
\end{tabular}
\par
\sphinxattableend\end{savenotes}

\end{description}\end{quote}

\end{fulllineitems}



\subsection{Options related to disaggregation of input data}
\label{\detokenize{input_files/RunControl/Options_related_to_disaggregation_of_input_data:options-related-to-disaggregation-of-input-data}}\label{\detokenize{input_files/RunControl/Options_related_to_disaggregation_of_input_data::doc}}\label{\detokenize{input_files/RunControl/Options_related_to_disaggregation_of_input_data:id1}}\index{command line option!DisaggMethod}\index{DisaggMethod!command line option}

\begin{fulllineitems}
\phantomsection\label{\detokenize{input_files/RunControl/Options_related_to_disaggregation_of_input_data:cmdoption-arg-disaggmethod}}\pysigline{\sphinxbfcode{\sphinxupquote{DisaggMethod}}\sphinxcode{\sphinxupquote{}}}~\begin{quote}\begin{description}
\item[{Requirement}] \leavevmode
Optional

\item[{Description}] \leavevmode
Specifies how meteorological variables in the input file (except rain and snow) are disaggregated to the model time step. Wind direction is not currently downscaled so non -999 values will cause an error.

\item[{Configuration}] \leavevmode

\begin{savenotes}\sphinxattablestart
\centering
\begin{tabular}[t]{|\X{20}{100}|\X{80}{100}|}
\hline
\sphinxstyletheadfamily 
Value
&\sphinxstyletheadfamily 
Comments
\\
\hline
1
&\begin{quote}

Linear downscaling of averages for all variables, additional zenith check is used for Kdown. This is the default option.
\end{quote}
\\
\hline
2
&\begin{quote}

Linear downscaling of instantaneous values for all variables, additional zenith check is used for Kdown.
\end{quote}
\\
\hline
3
&\begin{quote}

WFDEI setting: average Kdown (with additional zenith check); instantaneous for Tair, RH, pres and U. (N.B. WFDEI actually provides Q not RH)
\end{quote}
\\
\hline
\end{tabular}
\par
\sphinxattableend\end{savenotes}

\end{description}\end{quote}

\end{fulllineitems}

\index{command line option!KdownZen}\index{KdownZen!command line option}

\begin{fulllineitems}
\phantomsection\label{\detokenize{input_files/RunControl/Options_related_to_disaggregation_of_input_data:cmdoption-arg-kdownzen}}\pysigline{\sphinxbfcode{\sphinxupquote{KdownZen}}\sphinxcode{\sphinxupquote{}}}~\begin{quote}\begin{description}
\item[{Requirement}] \leavevmode
Optional

\item[{Description}] \leavevmode
Can be used to switch off zenith checking in Kdown disaggregation. Note that the zenith calculation requires location information obtained from SUEWS\_SiteSelect.txt. If a single met file is used for all grids, the zenith is calculated for the first grid and the disaggregated data is then applied for all grids.

\item[{Configuration}] \leavevmode

\begin{savenotes}\sphinxattablestart
\centering
\begin{tabular}[t]{|\X{20}{100}|\X{80}{100}|}
\hline
\sphinxstyletheadfamily 
Value
&\sphinxstyletheadfamily 
Comments
\\
\hline
0
&\begin{quote}

No zenith angle check is applied.
\end{quote}
\\
\hline
1
&\begin{quote}

Disaggregated Kdown is set to zero when zenith angle exceeds 90 degrees (i.e. sun below horizon) and redistributed over the day. This is the default option.
\end{quote}
\\
\hline
\end{tabular}
\par
\sphinxattableend\end{savenotes}

\end{description}\end{quote}

\end{fulllineitems}

\index{command line option!RainDisaggMethod}\index{RainDisaggMethod!command line option}

\begin{fulllineitems}
\phantomsection\label{\detokenize{input_files/RunControl/Options_related_to_disaggregation_of_input_data:cmdoption-arg-raindisaggmethod}}\pysigline{\sphinxbfcode{\sphinxupquote{RainDisaggMethod}}\sphinxcode{\sphinxupquote{}}}~\begin{quote}\begin{description}
\item[{Requirement}] \leavevmode
Optional

\item[{Description}] \leavevmode
Specifies how rain in the meteorological forcing file are disaggregated to the model time step. If present in the original met forcing file, snow is currently disaggregated in the same way as rainfall.

\item[{Configuration}] \leavevmode

\begin{savenotes}\sphinxattablestart
\centering
\begin{tabular}[t]{|\X{20}{100}|\X{80}{100}|}
\hline
\sphinxstyletheadfamily 
Value
&\sphinxstyletheadfamily 
Comments
\\
\hline
100
&\begin{quote}

Rainfall is evenly distributed among all subintervals in a rainy interval. This is the default option.
\end{quote}
\\
\hline
101
&\begin{quote}

Rainfall is evenly distributed among among RainAmongN subintervals in a rainy interval \textendash{} also requires RainAmongN to be set.
\end{quote}
\\
\hline
102
&\begin{quote}

Rainfall is evenly distributed among among RainAmongN subintervals in a rainy interval for different intensity bins \textendash{} also requires MultRainAmongN and MultRainAmongNUpperI to be set.
\end{quote}
\\
\hline
\end{tabular}
\par
\sphinxattableend\end{savenotes}

\end{description}\end{quote}

\end{fulllineitems}

\index{command line option!RainAmongN}\index{RainAmongN!command line option}

\begin{fulllineitems}
\phantomsection\label{\detokenize{input_files/RunControl/Options_related_to_disaggregation_of_input_data:cmdoption-arg-rainamongn}}\pysigline{\sphinxbfcode{\sphinxupquote{RainAmongN}}\sphinxcode{\sphinxupquote{}}}~\begin{quote}\begin{description}
\item[{Requirement}] \leavevmode
Optional

\item[{Description}] \leavevmode
Specifies the number of subintervals (of length tt) over which to distribute rainfall in each interval (of length TT). Must be an integer value. Use with RainDisaggMethod = 101.

\item[{Configuration}] \leavevmode
to fill

\end{description}\end{quote}

\end{fulllineitems}

\index{command line option!MultRainAmongN}\index{MultRainAmongN!command line option}

\begin{fulllineitems}
\phantomsection\label{\detokenize{input_files/RunControl/Options_related_to_disaggregation_of_input_data:cmdoption-arg-multrainamongn}}\pysigline{\sphinxbfcode{\sphinxupquote{MultRainAmongN}}\sphinxcode{\sphinxupquote{}}}~\begin{quote}\begin{description}
\item[{Requirement}] \leavevmode
Optional

\item[{Description}] \leavevmode
Specifies the number of subintervals (of length tt) over which to distribute rainfall in each interval (of length TT) for up to 5 intensity bins. Must take integer values. Use with RainDisaggMethod = 102. e.g. MultRainAmongN(1) = 5, MultRainAmongN(2) = 8, MultRainAmongN(3) = 12

\item[{Configuration}] \leavevmode
to fill

\end{description}\end{quote}

\end{fulllineitems}

\index{command line option!MultRainAmongNUpperI}\index{MultRainAmongNUpperI!command line option}

\begin{fulllineitems}
\phantomsection\label{\detokenize{input_files/RunControl/Options_related_to_disaggregation_of_input_data:cmdoption-arg-multrainamongnupperi}}\pysigline{\sphinxbfcode{\sphinxupquote{MultRainAmongNUpperI}}\sphinxcode{\sphinxupquote{}}}~\begin{quote}\begin{description}
\item[{Requirement}] \leavevmode
Optional

\item[{Description}] \leavevmode
Specifies upper limit for each intensity bin to apply MultRainAmongN. Any intensities above the highest specified intensity will use the last MultRainAmongN value and write a warning to warnings.txt. Use with RainDisaggMethod = 102. e.g. MultRainAmongNUpperI(1) = 0.5, MultRainAmongNUpperI(2) = 2.0, MultRainAmongNUpperI(3) = 50.0

\item[{Configuration}] \leavevmode
to fill

\end{description}\end{quote}

\end{fulllineitems}

\index{command line option!DisaggMethodESTM}\index{DisaggMethodESTM!command line option}

\begin{fulllineitems}
\phantomsection\label{\detokenize{input_files/RunControl/Options_related_to_disaggregation_of_input_data:cmdoption-arg-disaggmethodestm}}\pysigline{\sphinxbfcode{\sphinxupquote{DisaggMethodESTM}}\sphinxcode{\sphinxupquote{}}}~\begin{quote}\begin{description}
\item[{Requirement}] \leavevmode
Optional

\item[{Description}] \leavevmode
Specifies how ESTM-related temperatures in the input file are disaggregated to the model time step.

\item[{Configuration}] \leavevmode

\begin{savenotes}\sphinxattablestart
\centering
\begin{tabular}[t]{|\X{20}{100}|\X{80}{100}|}
\hline
\sphinxstyletheadfamily 
Value
&\sphinxstyletheadfamily 
Comments
\\
\hline
1
&\begin{quote}

Linear downscaling of averages.
\end{quote}
\\
\hline
2
&\begin{quote}

Linear downscaling of instantaneous values.
\end{quote}
\\
\hline
\end{tabular}
\par
\sphinxattableend\end{savenotes}

\end{description}\end{quote}

\end{fulllineitems}



\subsection{netCDF related options}
\label{\detokenize{input_files/RunControl/netCDF_related_options:netcdf-related-options}}\label{\detokenize{input_files/RunControl/netCDF_related_options::doc}}\label{\detokenize{input_files/RunControl/netCDF_related_options:id1}}\index{command line option!ncMode}\index{ncMode!command line option}

\begin{fulllineitems}
\phantomsection\label{\detokenize{input_files/RunControl/netCDF_related_options:cmdoption-arg-ncmode}}\pysigline{\sphinxbfcode{\sphinxupquote{ncMode}}\sphinxcode{\sphinxupquote{}}}~\begin{quote}\begin{description}
\item[{Requirement}] \leavevmode
Optional

\item[{Description}] \leavevmode
Determine if the output files should be written in netCDF format.

\item[{Configuration}] \leavevmode

\begin{savenotes}\sphinxattablestart
\centering
\begin{tabular}[t]{|\X{20}{100}|\X{80}{100}|}
\hline
\sphinxstyletheadfamily 
Value
&\sphinxstyletheadfamily 
Comments
\\
\hline
0
&
Output files are kept as plain text files (i.e., .txt).
\\
\hline
1
&
Output files will be written in netCDF format (i.e., .nc).
\\
\hline
\end{tabular}
\par
\sphinxattableend\end{savenotes}

\end{description}\end{quote}

\end{fulllineitems}

\index{command line option!nRow}\index{nRow!command line option}

\begin{fulllineitems}
\phantomsection\label{\detokenize{input_files/RunControl/netCDF_related_options:cmdoption-arg-nrow}}\pysigline{\sphinxbfcode{\sphinxupquote{nRow}}\sphinxcode{\sphinxupquote{}}}~\begin{quote}\begin{description}
\item[{Requirement}] \leavevmode
Optional

\item[{Description}] \leavevmode
Number of rows (e.g., 36) in the output layout (only applicable when ncMode=1).

\item[{Configuration}] \leavevmode
to fill

\end{description}\end{quote}

\end{fulllineitems}

\index{command line option!nCol}\index{nCol!command line option}

\begin{fulllineitems}
\phantomsection\label{\detokenize{input_files/RunControl/netCDF_related_options:cmdoption-arg-ncol}}\pysigline{\sphinxbfcode{\sphinxupquote{nCol}}\sphinxcode{\sphinxupquote{}}}~\begin{quote}\begin{description}
\item[{Requirement}] \leavevmode
Optional

\item[{Description}] \leavevmode
Number of columns (e.g., 47) in the output layout (only applicable when ncMode=1).

\item[{Configuration}] \leavevmode
to fill

\end{description}\end{quote}

\end{fulllineitems}



\section{SUEWS\_SiteInfo.xlsm}
\label{\detokenize{input_files/SUEWS_SiteInfo/SUEWS_SiteInfo::doc}}\label{\detokenize{input_files/SUEWS_SiteInfo/SUEWS_SiteInfo:suews-siteinfo-xlsm}}
The following text files provide SUEWS with information about the study
area.


\subsection{SUEWS\_AnthropogenicHeat.txt}
\label{\detokenize{input_files/SUEWS_SiteInfo/SUEWS_AnthropogenicHeat::doc}}\label{\detokenize{input_files/SUEWS_SiteInfo/SUEWS_AnthropogenicHeat:suews-anthropogenicheat-txt}}\label{\detokenize{input_files/SUEWS_SiteInfo/SUEWS_AnthropogenicHeat:id1}}
SUEWS\_AnthropogenicHeatFlux.txt provides the parameters needed to model
the anthropogenic heat flux using either the method of Järvi et al.
(2011) based on heating and cooling degree days (AnthropHeatMethod = 2
in 4.1 {\hyperref[\detokenize{input_files/RunControl/RunControl:runcontrol-nml}]{\sphinxcrossref{\DUrole{std,std-ref,std,std-ref}{RunControl.nml}}}} (\autopageref*{\detokenize{input_files/RunControl/RunControl:runcontrol-nml}})) or the method of Loridan et
al. (2011) based on air temperature (AnthropHeatMethod = 1 in
{\hyperref[\detokenize{input_files/RunControl/RunControl:runcontrol-nml}]{\sphinxcrossref{\DUrole{std,std-ref,std,std-ref}{RunControl.nml}}}} (\autopageref*{\detokenize{input_files/RunControl/RunControl:runcontrol-nml}})). The sub-daily variation in
anthropogenic heat flux is modelled according to the daily cycles
specified in SUEWS\_Profiles.txt. Alternatively, if available, the
anthropogenic heat flux can be provided in the met forcing file (and set
AnthropHeatMethod = 0 in {\hyperref[\detokenize{input_files/RunControl/RunControl:runcontrol-nml}]{\sphinxcrossref{\DUrole{std,std-ref,std,std-ref}{RunControl.nml}}}} (\autopageref*{\detokenize{input_files/RunControl/RunControl:runcontrol-nml}})), in which
case all columns here except Code and BaseTHDD should be set to ’-999’.


\begin{savenotes}\sphinxattablestart
\centering
\begin{tabulary}{\linewidth}[t]{|T|T|T|T|}
\hline
\sphinxstyletheadfamily 
No.
&\sphinxstyletheadfamily 
Column Name
&\sphinxstyletheadfamily 
Use
&\sphinxstyletheadfamily 
Description
\\
\hline
1
&
\sphinxcode{\sphinxupquote{Code}}
&
{\hyperref[\detokenize{notation:term-19}]{\sphinxtermref{\sphinxcode{\sphinxupquote{L}}}}}
&
Code linking to the AnthropogenicCode column in SUEWS\_SiteSelect.txt . Value of integer is arbitrary but must match code specified in SUEWS\_SiteSelect.txt.
\\
\hline
2
&
\sphinxcode{\sphinxupquote{BaseTHDD}}
&
{\hyperref[\detokenize{notation:term-mu}]{\sphinxtermref{\sphinxcode{\sphinxupquote{MU}}}}}
&
Base temperature for heating degree days {[}°C{]} e.g. Sailor and Vasireddy (2006) {[}39{]}
\\
\hline
3
&
\sphinxcode{\sphinxupquote{QF\_A\_Weekday}}
&
{\hyperref[\detokenize{notation:term-mu}]{\sphinxtermref{\sphinxcode{\sphinxupquote{MU}}}}} {\hyperref[\detokenize{notation:term-o}]{\sphinxtermref{\sphinxcode{\sphinxupquote{O}}}}}
&
Use with AnthropHeatChoice = 2 Example values {[}W m -2 (Cap ha-1) -1 {]} 0.3081 Järvi et al. (2011) {[}1{]}  0.1 Järvi et al. (2014) {[}15{]}
\\
\hline
4
&
\sphinxcode{\sphinxupquote{QF\_B\_Weekday}}
&
{\hyperref[\detokenize{notation:term-mu}]{\sphinxtermref{\sphinxcode{\sphinxupquote{MU}}}}} {\hyperref[\detokenize{notation:term-o}]{\sphinxtermref{\sphinxcode{\sphinxupquote{O}}}}}
&
Use with AnthropHeatMethod = 2 Example values {[}W m -2 K -1 (Cap ha -1 ) -1 {]} 0.0099 Järvi et al. (2011) {[}1{]}  0.0099 Järvi et al. (2014) {[}15{]}
\\
\hline
5
&
\sphinxcode{\sphinxupquote{QF\_C\_Weekday}}
&
{\hyperref[\detokenize{notation:term-mu}]{\sphinxtermref{\sphinxcode{\sphinxupquote{MU}}}}} {\hyperref[\detokenize{notation:term-o}]{\sphinxtermref{\sphinxcode{\sphinxupquote{O}}}}}
&
Use with AnthropHeatMethod = 2 Example values {[}W m -2 K -1 (Cap ha -1 ) -1 {]} 0.0102 Järvi et al. (2011) {[}1{]}  0.0102 Järvi et al. (2014) {[}15{]}
\\
\hline
6
&
\sphinxcode{\sphinxupquote{QF\_A\_Weekend}}
&
{\hyperref[\detokenize{notation:term-mu}]{\sphinxtermref{\sphinxcode{\sphinxupquote{MU}}}}} {\hyperref[\detokenize{notation:term-o}]{\sphinxtermref{\sphinxcode{\sphinxupquote{O}}}}}
&
Use with AnthropHeatMethod = 2 Example values {[}W m -2 (Cap ha -1 ) -1 {]} 0.3081 Järvi et al. (2011) {[}1{]}  0.1 Järvi et al. (2014) {[}15{]}
\\
\hline
7
&
\sphinxcode{\sphinxupquote{QF\_B\_Weekend}}
&
{\hyperref[\detokenize{notation:term-mu}]{\sphinxtermref{\sphinxcode{\sphinxupquote{MU}}}}} {\hyperref[\detokenize{notation:term-o}]{\sphinxtermref{\sphinxcode{\sphinxupquote{O}}}}}
&
Use with AnthropHeatMethod = 2 Example values {[}W m -2 K -1 (Cap ha -1 ) -1 {]} 0.0099 Järvi et al. (2011) {[}1{]}  0.0099 Järvi et al. (2014) {[}15{]}
\\
\hline
8
&
\sphinxcode{\sphinxupquote{QF\_C\_Weekend}}
&
{\hyperref[\detokenize{notation:term-mu}]{\sphinxtermref{\sphinxcode{\sphinxupquote{MU}}}}} {\hyperref[\detokenize{notation:term-o}]{\sphinxtermref{\sphinxcode{\sphinxupquote{O}}}}}
&
Example values {[}W m -2 K -1 (Cap ha -1 ) -1 {]} 0.0102 Järvi et al. (2011) {[}1{]}  0.0102 Järvi et al. (2014) {[}15{]}
\\
\hline
9
&
\sphinxcode{\sphinxupquote{AHMin}}
&
{\hyperref[\detokenize{notation:term-mu}]{\sphinxtermref{\sphinxcode{\sphinxupquote{MU}}}}} {\hyperref[\detokenize{notation:term-o}]{\sphinxtermref{\sphinxcode{\sphinxupquote{O}}}}}
&
Use with AnthropHeatMethod = 1
\\
\hline
10
&
\sphinxcode{\sphinxupquote{AHSlope}}
&
{\hyperref[\detokenize{notation:term-mu}]{\sphinxtermref{\sphinxcode{\sphinxupquote{MU}}}}} {\hyperref[\detokenize{notation:term-o}]{\sphinxtermref{\sphinxcode{\sphinxupquote{O}}}}}
&
Use with AnthropHeatMethod = 1
\\
\hline
11
&
\sphinxcode{\sphinxupquote{TCritic}}
&
{\hyperref[\detokenize{notation:term-mu}]{\sphinxtermref{\sphinxcode{\sphinxupquote{MU}}}}} {\hyperref[\detokenize{notation:term-o}]{\sphinxtermref{\sphinxcode{\sphinxupquote{O}}}}}
&
Use with AnthropHeatMethod = 1
\\
\hline
\end{tabulary}
\par
\sphinxattableend\end{savenotes}


\subsection{SUEWS\_Conductance.txt}
\label{\detokenize{input_files/SUEWS_SiteInfo/SUEWS_Conductance:id1}}\label{\detokenize{input_files/SUEWS_SiteInfo/SUEWS_Conductance::doc}}\label{\detokenize{input_files/SUEWS_SiteInfo/SUEWS_Conductance:suews-conductance-txt}}
SUEWS\_Conductance.txt contains the parameters needed for the Jarvis
(1976) surface conductance model used in the modelling of evaporation in
SUEWS. These values should \sphinxstylestrong{not} be changed independently of each
other. The suggested values below have been derived using datasets for
Los Angeles and Vancouver (see Järvi et al. (2011) \phantomsection\label{\detokenize{input_files/SUEWS_SiteInfo/SUEWS_Conductance:id2}}{\hyperref[\detokenize{references:j11}]{\sphinxcrossref{{[}J11{]}}}} (\autopageref*{\detokenize{references:j11}})) and should be
used with \sphinxcode{\sphinxupquote{gsModel=1}}. An alternative formulation
(\sphinxcode{\sphinxupquote{gsModel=2}}) uses
slightly different functional forms and different coefficients (with
different units).


\begin{savenotes}\sphinxattablestart
\centering
\begin{tabulary}{\linewidth}[t]{|T|T|T|T|}
\hline
\sphinxstyletheadfamily 
No.
&\sphinxstyletheadfamily 
Column Name
&\sphinxstyletheadfamily 
Use
&\sphinxstyletheadfamily 
Description
\\
\hline
1
&
\sphinxcode{\sphinxupquote{Code}}
&
{\hyperref[\detokenize{notation:term-19}]{\sphinxtermref{\sphinxcode{\sphinxupquote{L}}}}}
&
Code linking to the CondCode column in SUEWS\_SiteSelect.txt . Value of integer is arbitrary but must match code specified in SUEWS\_SiteSelect.txt.
\\
\hline
2
&
\sphinxcode{\sphinxupquote{G1}}
&
{\hyperref[\detokenize{notation:term-md}]{\sphinxtermref{\sphinxcode{\sphinxupquote{MD}}}}}
&
Related to maximum surface conductance {[}mm s -1 {]}
\\
\hline
3
&
\sphinxcode{\sphinxupquote{G2}}
&
{\hyperref[\detokenize{notation:term-md}]{\sphinxtermref{\sphinxcode{\sphinxupquote{MD}}}}}
&
Related to Kdown dependence {[}W m -2 {]}
\\
\hline
4
&
\sphinxcode{\sphinxupquote{G3}}
&
{\hyperref[\detokenize{notation:term-md}]{\sphinxtermref{\sphinxcode{\sphinxupquote{MD}}}}}
&
Related to VPD dependence {[}units depend on gsChoice in RunControl.nml {]}
\\
\hline
5
&
\sphinxcode{\sphinxupquote{G4}}
&
{\hyperref[\detokenize{notation:term-md}]{\sphinxtermref{\sphinxcode{\sphinxupquote{MD}}}}}
&
Related to VPD dependence {[}units depend on gsChoice in RunControl.nml {]}
\\
\hline
6
&
\sphinxcode{\sphinxupquote{G5}}
&
{\hyperref[\detokenize{notation:term-md}]{\sphinxtermref{\sphinxcode{\sphinxupquote{MD}}}}}
&
Related to temperature dependence {[}°C{]}
\\
\hline
7
&
\sphinxcode{\sphinxupquote{G6}}
&
{\hyperref[\detokenize{notation:term-md}]{\sphinxtermref{\sphinxcode{\sphinxupquote{MD}}}}}
&
Related to soil moisture dependence {[}mm -1 {]}
\\
\hline
8
&
\sphinxcode{\sphinxupquote{TH}}
&
{\hyperref[\detokenize{notation:term-md}]{\sphinxtermref{\sphinxcode{\sphinxupquote{MD}}}}}
&
Upper air temperature limit {[}°C{]}
\\
\hline
9
&
\sphinxcode{\sphinxupquote{TL}}
&
{\hyperref[\detokenize{notation:term-md}]{\sphinxtermref{\sphinxcode{\sphinxupquote{MD}}}}}
&
Lower air temperature limit {[}°C{]}
\\
\hline
10
&
\sphinxcode{\sphinxupquote{S1}}
&
{\hyperref[\detokenize{notation:term-md}]{\sphinxtermref{\sphinxcode{\sphinxupquote{MD}}}}}
&
Related to soil moisture dependence {[}-{]} These will change in the future to ensure consistency with soil behaviour
\\
\hline
11
&
\sphinxcode{\sphinxupquote{S2}}
&
{\hyperref[\detokenize{notation:term-md}]{\sphinxtermref{\sphinxcode{\sphinxupquote{MD}}}}}
&
Related to soil moisture dependence {[}mm{]} These will change in the future to ensure consistency with soil behaviour
\\
\hline
12
&
\sphinxcode{\sphinxupquote{Kmax}}
&
{\hyperref[\detokenize{notation:term-md}]{\sphinxtermref{\sphinxcode{\sphinxupquote{MD}}}}}
&
Maximum incoming shortwave radiation {[}W m -2 {]}
\\
\hline
13
&
\sphinxcode{\sphinxupquote{gsModel}}
&
{\hyperref[\detokenize{notation:term-md}]{\sphinxtermref{\sphinxcode{\sphinxupquote{MD}}}}}
&
1 = Järvi et al. (2011) {[}1{]} 2 = Ward et al. (2016) {[}2{]} Recommended.
\\
\hline
\end{tabulary}
\par
\sphinxattableend\end{savenotes}


\subsection{SUEWS\_Irrigation.txt}
\label{\detokenize{input_files/SUEWS_SiteInfo/SUEWS_Irrigation:suews-irrigation-txt}}\label{\detokenize{input_files/SUEWS_SiteInfo/SUEWS_Irrigation::doc}}\label{\detokenize{input_files/SUEWS_SiteInfo/SUEWS_Irrigation:id1}}
SUEWS includes a simple model for external water use if observed data
are not available. The model calculates daily water use from the mean
daily air temperature, number of days since rain and fraction of
irrigated area using automatic/manual irrigation. The sub-daily pattern
of water use is modelled according to the daily cycles specified in
{\hyperref[\detokenize{input_files/SUEWS_SiteInfo/SUEWS_Irrigation:SUEWS_Profiles.txt}]{\emph{SUEWS\_Profiles.txt}}} (\autopageref*{\detokenize{input_files/SUEWS_SiteInfo/SUEWS_Irrigation:SUEWS_Profiles.txt}}).

Alternatively, if available, the external water use can be provided in
the met forcing file (and set WaterUseMethod = 1 in
{\hyperref[\detokenize{input_files/SUEWS_SiteInfo/SUEWS_Irrigation:RunControl.nml}]{\emph{RunControl.nml}}} (\autopageref*{\detokenize{input_files/SUEWS_SiteInfo/SUEWS_Irrigation:RunControl.nml}})), in which case all columns here
except Code should be set to ‘-999’.


\begin{savenotes}\sphinxattablestart
\centering
\begin{tabulary}{\linewidth}[t]{|T|T|T|T|}
\hline
\sphinxstyletheadfamily 
No.
&\sphinxstyletheadfamily 
Column Name
&\sphinxstyletheadfamily 
Use
&\sphinxstyletheadfamily 
Description
\\
\hline
1
&
\sphinxcode{\sphinxupquote{Code}}
&
{\hyperref[\detokenize{notation:term-19}]{\sphinxtermref{\sphinxcode{\sphinxupquote{L}}}}}
&
Code linking to {[}{[}\#SUEWS\_SiteSelect.txt\textbar{}SUEWS\_SiteSelect.txt{]} for irrigation modelling (IrrigationCode). Value of integer is arbitrary but must match codes specified in SUEWS\_SiteSelect.txt.
\\
\hline
2
&
\sphinxcode{\sphinxupquote{Ie\_start}}
&
{\hyperref[\detokenize{notation:term-mu}]{\sphinxtermref{\sphinxcode{\sphinxupquote{MU}}}}}
&
Day when irrigation starts {[}DOY{]}
\\
\hline
3
&
\sphinxcode{\sphinxupquote{Ie\_end}}
&
{\hyperref[\detokenize{notation:term-mu}]{\sphinxtermref{\sphinxcode{\sphinxupquote{MU}}}}}
&
Day when irrigation ends {[}DOY{]}
\\
\hline
4
&
\sphinxcode{\sphinxupquote{InternalWaterUse}}
&
{\hyperref[\detokenize{notation:term-mu}]{\sphinxtermref{\sphinxcode{\sphinxupquote{MU}}}}}
&
Internal water use {[}mm h -1 {]}
\\
\hline
5
&
\sphinxcode{\sphinxupquote{Faut}}
&
{\hyperref[\detokenize{notation:term-mu}]{\sphinxtermref{\sphinxcode{\sphinxupquote{MU}}}}}
&
Fraction of irrigated area that is irrigated using automated systems (e.g. sprinklers).
\\
\hline
6
&
\sphinxcode{\sphinxupquote{Ie\_a1}}
&
{\hyperref[\detokenize{notation:term-md}]{\sphinxtermref{\sphinxcode{\sphinxupquote{MD}}}}}
&
Coefficient for automatic irrigation model {[}mm d -1 {]}
\\
\hline
7
&
\sphinxcode{\sphinxupquote{Ie\_a2}}
&
{\hyperref[\detokenize{notation:term-md}]{\sphinxtermref{\sphinxcode{\sphinxupquote{MD}}}}}
&
Coefficient for automatic irrigation model {[}mm d -1 °C -1 {]}
\\
\hline
8
&
\sphinxcode{\sphinxupquote{Ie\_a3}}
&
{\hyperref[\detokenize{notation:term-md}]{\sphinxtermref{\sphinxcode{\sphinxupquote{MD}}}}}
&
Coefficient for automatic irrigation model {[}mm d -2 {]}
\\
\hline
9
&
\sphinxcode{\sphinxupquote{Ie\_m1}}
&
{\hyperref[\detokenize{notation:term-md}]{\sphinxtermref{\sphinxcode{\sphinxupquote{MD}}}}}
&
Coefficient for manual irrigation model {[}mm d -1 {]}
\\
\hline
10
&
\sphinxcode{\sphinxupquote{Ie\_m2}}
&
{\hyperref[\detokenize{notation:term-md}]{\sphinxtermref{\sphinxcode{\sphinxupquote{MD}}}}}
&
Coefficient for manual irrigation model {[}mm d -1 °C -1 {]}
\\
\hline
11
&
\sphinxcode{\sphinxupquote{Ie\_m3}}
&
{\hyperref[\detokenize{notation:term-md}]{\sphinxtermref{\sphinxcode{\sphinxupquote{MD}}}}}
&
Coefficient for manual irrigation model {[}mm d -2 {]}
\\
\hline
12
&
\sphinxcode{\sphinxupquote{DayWat(1)}}
&
{\hyperref[\detokenize{notation:term-mu}]{\sphinxtermref{\sphinxcode{\sphinxupquote{MU}}}}}
&
Irrigation allowed on Sundays {[}1{]}, if not {[}0{]}
\\
\hline
13
&
\sphinxcode{\sphinxupquote{DayWat(2)}}
&
{\hyperref[\detokenize{notation:term-mu}]{\sphinxtermref{\sphinxcode{\sphinxupquote{MU}}}}}
&
Irrigation allowed on Mondays {[}1{]}, if not {[}0{]}
\\
\hline
14
&
\sphinxcode{\sphinxupquote{DayWat(3)}}
&
{\hyperref[\detokenize{notation:term-mu}]{\sphinxtermref{\sphinxcode{\sphinxupquote{MU}}}}}
&
Irrigation allowed on Tuesdays {[}1{]}, if not {[}0{]}
\\
\hline
15
&
\sphinxcode{\sphinxupquote{DayWat(4)}}
&
{\hyperref[\detokenize{notation:term-mu}]{\sphinxtermref{\sphinxcode{\sphinxupquote{MU}}}}}
&
Irrigation allowed on Wednesdays {[}1{]}, if not {[}0{]}
\\
\hline
16
&
\sphinxcode{\sphinxupquote{DayWat(5)}}
&
{\hyperref[\detokenize{notation:term-mu}]{\sphinxtermref{\sphinxcode{\sphinxupquote{MU}}}}}
&
Irrigation allowed on Thursdays {[}1{]}, if not {[}0{]}
\\
\hline
17
&
\sphinxcode{\sphinxupquote{DayWat(6)}}
&
{\hyperref[\detokenize{notation:term-mu}]{\sphinxtermref{\sphinxcode{\sphinxupquote{MU}}}}}
&
Irrigation allowed on Fridays {[}1{]}, if not {[}0{]}
\\
\hline
18
&
\sphinxcode{\sphinxupquote{DayWat(7)}}
&
{\hyperref[\detokenize{notation:term-mu}]{\sphinxtermref{\sphinxcode{\sphinxupquote{MU}}}}}
&
Irrigation allowed on Saturdays {[}1{]}, if not {[}0{]}
\\
\hline
19
&
\sphinxcode{\sphinxupquote{DayWatPer(1)}}
&
{\hyperref[\detokenize{notation:term-mu}]{\sphinxtermref{\sphinxcode{\sphinxupquote{MU}}}}}
&
Fraction of properties using irrigation on Sundays {[}0-1{]}
\\
\hline
20
&
\sphinxcode{\sphinxupquote{DayWatPer(2)}}
&
{\hyperref[\detokenize{notation:term-mu}]{\sphinxtermref{\sphinxcode{\sphinxupquote{MU}}}}}
&
Fraction of properties using irrigation on Mondays {[}0-1{]}
\\
\hline
21
&
\sphinxcode{\sphinxupquote{DayWatPer(3)}}
&
{\hyperref[\detokenize{notation:term-mu}]{\sphinxtermref{\sphinxcode{\sphinxupquote{MU}}}}}
&
Fraction of properties using irrigation on Tuesdays {[}0-1{]}
\\
\hline
22
&
\sphinxcode{\sphinxupquote{DayWatPer(4)}}
&
{\hyperref[\detokenize{notation:term-mu}]{\sphinxtermref{\sphinxcode{\sphinxupquote{MU}}}}}
&
Fraction of properties using irrigation on Wednesdays {[}0-1{]}
\\
\hline
23
&
\sphinxcode{\sphinxupquote{DayWatPer(5)}}
&
{\hyperref[\detokenize{notation:term-mu}]{\sphinxtermref{\sphinxcode{\sphinxupquote{MU}}}}}
&
Fraction of properties using irrigation on Thursdays {[}0-1{]}
\\
\hline
24
&
\sphinxcode{\sphinxupquote{DayWatPer(6)}}
&
{\hyperref[\detokenize{notation:term-mu}]{\sphinxtermref{\sphinxcode{\sphinxupquote{MU}}}}}
&
Fraction of properties using irrigation on Fridays {[}0-1{]}
\\
\hline
25
&
\sphinxcode{\sphinxupquote{DayWatPer(7)}}
&
{\hyperref[\detokenize{notation:term-mu}]{\sphinxtermref{\sphinxcode{\sphinxupquote{MU}}}}}
&
Fraction of properties using irrigation on Saturdays {[}0-1{]}
\\
\hline
\end{tabulary}
\par
\sphinxattableend\end{savenotes}


\subsection{SUEWS\_NonVeg.txt}
\label{\detokenize{input_files/SUEWS_SiteInfo/SUEWS_NonVeg:suews-nonveg-txt}}\label{\detokenize{input_files/SUEWS_SiteInfo/SUEWS_NonVeg::doc}}\label{\detokenize{input_files/SUEWS_SiteInfo/SUEWS_NonVeg:id1}}
SUEWS\_NonVeg.txt specifies the characteristics for the non-vegetated
surface cover types (Paved, Bldgs, BSoil) by linking codes in column 1
of SUEWS\_NonVeg.txt to the codes specified in SUEWS\_SiteSelect.txt
(Code\_Paved, Code\_Bldgs, Code\_BSoil). Each row should correspond to a
particular surface type. For suggestions on how to complete this table,
see: \sphinxhref{http://urban-climate.net/umep/TypicalValues\#Typical\_Values}{Typical
Values}.


\begin{savenotes}\sphinxattablestart
\centering
\begin{tabulary}{\linewidth}[t]{|T|T|T|T|}
\hline
\sphinxstyletheadfamily 
No.
&\sphinxstyletheadfamily 
Column Name
&\sphinxstyletheadfamily 
Use
&\sphinxstyletheadfamily 
Description
\\
\hline
1
&
\sphinxcode{\sphinxupquote{Code}}
&
{\hyperref[\detokenize{notation:term-19}]{\sphinxtermref{\sphinxcode{\sphinxupquote{L}}}}}
&
Code linking to SUEWS\_SiteSelect.txt for paved surfaces (Code\_Paved), buildings (Code\_Bldgs) and bare soil surfaces (Code\_BSoil). Value of integer is arbitrary but must match codes specified in SUEWS\_SiteSelect.txt.
\\
\hline
2
&
\sphinxcode{\sphinxupquote{AlbedoMin}}
&
{\hyperref[\detokenize{notation:term-mu}]{\sphinxtermref{\sphinxcode{\sphinxupquote{MU}}}}}
&
Effective surface albedo (middle of the day value) for wintertime (not including snow). View factors should be taken into account. Not currently used for non-vegetated surfaces \textendash{} set the same as AlbedoMax.
\\
\hline
3
&
\sphinxcode{\sphinxupquote{AlbedoMax}}
&
{\hyperref[\detokenize{notation:term-mu}]{\sphinxtermref{\sphinxcode{\sphinxupquote{MU}}}}}
&
Effective surface albedo (middle of the day value) for summertime. View factors should be taken into account.
\\
\hline
4
&
\sphinxcode{\sphinxupquote{Emissivity}}
&
{\hyperref[\detokenize{notation:term-mu}]{\sphinxtermref{\sphinxcode{\sphinxupquote{MU}}}}}
&
Effective surface emissivity. View factors should be taken into account.
\\
\hline
5
&
\sphinxcode{\sphinxupquote{StorageMin}}
&
{\hyperref[\detokenize{notation:term-md}]{\sphinxtermref{\sphinxcode{\sphinxupquote{MD}}}}}
&
Minimum water storage capacity for upper surfaces (i.e. canopy). Min/max values are to account for seasonal variation (e.g. leaf-on/leaf-off differences for vegetated surfaces). Not currently used for non-vegetated surfaces - set the same as StorageMax. Example values {[}mm{]} 0.48 Paved 0.25 Bldgs 0.8 BSoil
\\
\hline
6
&
\sphinxcode{\sphinxupquote{StorageMax}}
&
{\hyperref[\detokenize{notation:term-md}]{\sphinxtermref{\sphinxcode{\sphinxupquote{MD}}}}}
&
Maximum water storage capacity for upper surfaces (i.e. canopy) Min and max values are to account for seasonal variation (e.g. leaf-on/leaf-off differences for vegetated surfaces). Not currently used for non-vegetated surfaces - set the same as StorageMin. Example values {[}mm{]} 0.48 Paved 0.25 Bldgs 0.8 BSoil
\\
\hline
7
&
\sphinxcode{\sphinxupquote{WetThreshold}}
&
{\hyperref[\detokenize{notation:term-md}]{\sphinxtermref{\sphinxcode{\sphinxupquote{MD}}}}}
&
Depth of water which determines whether evaporation occurs from a partially wet or completely wet surface. Example values {[}mm{]} 0.6 Paved 0.6 Bldgs 1. BSoil
\\
\hline
8
&
\sphinxcode{\sphinxupquote{StateLimit}}
&
{\hyperref[\detokenize{notation:term-md}]{\sphinxtermref{\sphinxcode{\sphinxupquote{MD}}}}}
&
Currently only used for the water surface
\\
\hline
9
&
\sphinxcode{\sphinxupquote{DrainageEq}}
&
{\hyperref[\detokenize{notation:term-md}]{\sphinxtermref{\sphinxcode{\sphinxupquote{MD}}}}}
&
Options 1 Falk and Niemczynowicz (1978) {[}32{]} 2 Halldin et al. (1979) {[}33{]} (Rutter eqn corrected for c=0, see Calder \& Wright (1986) {[}34{]} ) Recommended {[}3{]} for BSoil 3 Falk and Niemczynowicz (1978) {[}32{]} Recommended {[}3{]} for Paved and Bldgs Coefficients are specified in the following two columns.
\\
\hline
10
&
\sphinxcode{\sphinxupquote{DrainageCoef1}}
&
{\hyperref[\detokenize{notation:term-md}]{\sphinxtermref{\sphinxcode{\sphinxupquote{MD}}}}}
&
Example values DrainageEq 10 Coefficient D0 {[}mm h -1 {]} 3 Recommended {[}3{]} for Paved and Bldgs 0.013 Coefficient D0 {[}mm h -1 {]} 2 Recommended {[}3{]} for BSoil
\\
\hline
11
&
\sphinxcode{\sphinxupquote{DrainageCoef2}}
&
{\hyperref[\detokenize{notation:term-md}]{\sphinxtermref{\sphinxcode{\sphinxupquote{MD}}}}}
&
Example values DrainageEq 3 Coefficient b {[}-{]} 3 Recommended {[}3{]} for Paved and Bldgs 1.71 Coefficient b {[}mm -1 {]} 2 Recommended {[}3{]} for BSoil
\\
\hline
12
&
\sphinxcode{\sphinxupquote{SoilTypeCode}}
&
{\hyperref[\detokenize{notation:term-19}]{\sphinxtermref{\sphinxcode{\sphinxupquote{L}}}}}
&
Code for soil characteristics below this surface Provides the link to column 1 of SUEWS\_Soil.txt , which contains the attributes describing sub-surface soil for this surface type. Value of integer is arbitrary but must match code specified in column 1 of SUEWS\_Soil.txt.
\\
\hline
13
&
\sphinxcode{\sphinxupquote{SnowLimPatch}}
&
{\hyperref[\detokenize{notation:term-o}]{\sphinxtermref{\sphinxcode{\sphinxupquote{O}}}}}
&
Not needed if SnowUse = 0 in RunControl.nml . Example values {[}mm{]} 190 Paved Järvi et al. (2014) {[}15{]}  190 Bldgs Järvi et al. (2014) {[}15{]}  190 BSoil Järvi et al. (2014) {[}15{]}
\\
\hline
14
&
\sphinxcode{\sphinxupquote{SnowLimRemove}}
&
{\hyperref[\detokenize{notation:term-o}]{\sphinxtermref{\sphinxcode{\sphinxupquote{O}}}}}
&
Not needed if SnowUse = 0 in RunControl.nml . Currently not implemented for BSoil surface Example values {[}mm{]} 40 Paved Järvi et al. (2014) {[}15{]}  100 Bldgs Järvi et al. (2014) {[}15{]}
\\
\hline
15
&
\sphinxcode{\sphinxupquote{OHMCode\_SummerWet}}
&
{\hyperref[\detokenize{notation:term-19}]{\sphinxtermref{\sphinxcode{\sphinxupquote{L}}}}}
&
Code for OHM coefficients to use for this surface during wet conditions in summer. Links to SUEWS\_OHMCoefficients.txt . Value of integer is arbitrary but must match code specified in column 1 of SUEWS\_OHMCoefficients.txt.
\\
\hline
16
&
\sphinxcode{\sphinxupquote{OHMCode\_SummerDry}}
&
{\hyperref[\detokenize{notation:term-19}]{\sphinxtermref{\sphinxcode{\sphinxupquote{L}}}}}
&
Code for OHM coefficients to use for this surface during dry conditions in summer. Links to SUEWS\_OHMCoefficients.txt . Value of integer is arbitrary but must match code specified in column 1 of SUEWS\_OHMCoefficients.txt.
\\
\hline
17
&
\sphinxcode{\sphinxupquote{OHMCode\_WinterWet}}
&
{\hyperref[\detokenize{notation:term-19}]{\sphinxtermref{\sphinxcode{\sphinxupquote{L}}}}}
&
Code for OHM coefficients to use for this surface during wet conditions in winter. Links to SUEWS\_OHMCoefficients.txt . Value of integer is arbitrary but must match code specified in column 1 of SUEWS\_OHMCoefficients.txt.
\\
\hline
18
&
\sphinxcode{\sphinxupquote{OHMCode\_WinterDry}}
&
{\hyperref[\detokenize{notation:term-19}]{\sphinxtermref{\sphinxcode{\sphinxupquote{L}}}}}
&
Code for OHM coefficients to use for this surface during dry conditions in winter. Links to SUEWS\_OHMCoefficients.txt . Value of integer is arbitrary but must match code specified in column 1 of SUEWS\_OHMCoefficients.txt.
\\
\hline
19
&
\sphinxcode{\sphinxupquote{OHMThresh\_SW}}
&
{\hyperref[\detokenize{notation:term-md}]{\sphinxtermref{\sphinxcode{\sphinxupquote{MD}}}}}
&
Temperature threshold determining whether summer/winter OHM coefficients are applied {[}deg C{]} If 5-day running mean air temperature is greater than or equal to this threshold, OHM coefficients for summertime are applied; otherwise coefficients for wintertime are applied.
\\
\hline
20
&
\sphinxcode{\sphinxupquote{OHMThresh\_WD}}
&
{\hyperref[\detokenize{notation:term-md}]{\sphinxtermref{\sphinxcode{\sphinxupquote{MD}}}}}
&
Soil moisture threshold determining whether wet/dry OHM coefficients are applied {[}-{]} If soil moisture (as a proportion of maximum soil moisture capacity) exceeds this threshold for bare soil and vegetated surfaces, OHM coefficients for wet conditions are applied; otherwise coefficients for dry coefficients are applied. Note that OHM coefficients for wet conditions are applied if the surface is wet. Not actually used for building and paved surfaces (as impervious).
\\
\hline
21
&
\sphinxcode{\sphinxupquote{ESTMCode}}
&
{\hyperref[\detokenize{notation:term-19}]{\sphinxtermref{\sphinxcode{\sphinxupquote{L}}}}}
&
For paved and building surfaces, it is possible to specify multiple codes per grid (3 for paved, 5 for buildings) using SUEWS\_SiteSelect.txt . In this case, set ESTMCode here to zero.
\\
\hline
22
&
\sphinxcode{\sphinxupquote{AnOHM\_Cp}}
&
{\hyperref[\detokenize{notation:term-mu}]{\sphinxtermref{\sphinxcode{\sphinxupquote{MU}}}}}
&
Volumetric heat capacity for this surface to use in AnOHM {[}J m -3 {]}
\\
\hline
23
&
\sphinxcode{\sphinxupquote{AnOHM\_Kk}}
&
{\hyperref[\detokenize{notation:term-mu}]{\sphinxtermref{\sphinxcode{\sphinxupquote{MU}}}}}
&
Thermal conductivity for this surface to use in AnOHM {[}W m K -1 {]}
\\
\hline
24
&
\sphinxcode{\sphinxupquote{AnOHM\_Ch}}
&
{\hyperref[\detokenize{notation:term-mu}]{\sphinxtermref{\sphinxcode{\sphinxupquote{MU}}}}}
&
Bulk transfer coefficient for this surface to use in AnOHM {[}-{]}
\\
\hline
\end{tabulary}
\par
\sphinxattableend\end{savenotes}


\subsection{SUEWS\_OHMCoefficients.txt}
\label{\detokenize{input_files/SUEWS_SiteInfo/SUEWS_OHMCoefficients::doc}}\label{\detokenize{input_files/SUEWS_SiteInfo/SUEWS_OHMCoefficients:suews-ohmcoefficients-txt}}\label{\detokenize{input_files/SUEWS_SiteInfo/SUEWS_OHMCoefficients:id1}}
OHM, the Objective Hysteresis Model (Grimmond et al. 1991) \phantomsection\label{\detokenize{input_files/SUEWS_SiteInfo/SUEWS_OHMCoefficients:id2}}{\hyperref[\detokenize{references:g91ohm}]{\sphinxcrossref{{[}G91OHM{]}}}} (\autopageref*{\detokenize{references:g91ohm}})
calculates the storage heat flux as a function of net all-wave radiation
and surface characteristics.
\begin{itemize}
\item {} 
For each surface, OHM requires three model coefficients (a1, a2, a3).
The three should be selected as a set.

\item {} 
The \sphinxstylestrong{SUEWS\_OHMCoefficients.txt} file provides these coefficients
for each surface type.

\item {} 
A variety of values has been derived for different materials and can
be found in the literature (see:
{[}\sphinxurl{http://urban-climate.net/umep/TypicalValues\#OHM\_Coefficients}\textbar{}
Typical Values{]}).

\item {} 
Coefficients can be changed depending on:

\end{itemize}

:\# surface wetness state (wet/dry) based on the calculated surface
wetness state and soil moisture.

:\# season (summer/winter) based on a 5-day running mean air temperature.
\begin{itemize}
\item {} 
To use the same coefficients irrespective of wet/dry and
summer/winter conditions, use the same code for all four OHM columns
(OHMCode\_SummerWet, OHMCode\_SummerDry, OHMCode\_WinterWet and
OHMCode\_WinterDry).

\end{itemize}

Note, \sphinxstylestrong{AnOHM} does not use the coefficients specified in
SUEWS\_OHMCoefficients.txt but instead requires three parameters to be
specified for each surface type (including snow): heat capacity, thermal
conductivity and bulk transfer coefficient. These are specified in
{\hyperref[\detokenize{input_files/SUEWS_SiteInfo/SUEWS_OHMCoefficients:SUEWS_NonVeg.txt}]{\emph{SUEWS\_NonVeg.txt}}} (\autopageref*{\detokenize{input_files/SUEWS_SiteInfo/SUEWS_OHMCoefficients:SUEWS_NonVeg.txt}}),
{\hyperref[\detokenize{input_files/SUEWS_SiteInfo/SUEWS_OHMCoefficients:SUEWS_Veg.txt}]{\emph{SUEWS\_Veg.txt}}} (\autopageref*{\detokenize{input_files/SUEWS_SiteInfo/SUEWS_OHMCoefficients:SUEWS_Veg.txt}}),
{\hyperref[\detokenize{input_files/SUEWS_SiteInfo/SUEWS_OHMCoefficients:SUEWS_Water.txt}]{\emph{SUEWS\_Water.txt}}} (\autopageref*{\detokenize{input_files/SUEWS_SiteInfo/SUEWS_OHMCoefficients:SUEWS_Water.txt}}) and
{\hyperref[\detokenize{input_files/SUEWS_SiteInfo/SUEWS_OHMCoefficients:SUEWS_Snow.txt}]{\emph{SUEWS\_Snow.txt}}} (\autopageref*{\detokenize{input_files/SUEWS_SiteInfo/SUEWS_OHMCoefficients:SUEWS_Snow.txt}}). No additional files are required
for AnOHM.

\sphinxstylestrong{Note AnOHM is under development in v2017a and should not be used!}


\begin{savenotes}\sphinxattablestart
\centering
\begin{tabulary}{\linewidth}[t]{|T|T|T|T|}
\hline
\sphinxstyletheadfamily 
No.
&\sphinxstyletheadfamily 
Column Name
&\sphinxstyletheadfamily 
Use
&\sphinxstyletheadfamily 
Description
\\
\hline
1
&
\sphinxcode{\sphinxupquote{Code}}
&
{\hyperref[\detokenize{notation:term-19}]{\sphinxtermref{\sphinxcode{\sphinxupquote{L}}}}}
&
Code linking to the OHMCode\_SummerWet, OHMCode\_SummerDry, OHMCode\_WinterWet and OHMCode\_WinterDry columns in SUEWS\_NonVeg.txt, SUEWS\_Veg,txt, SUEWS\_Water.txt and SUEWS\_Snow.txt files. Value of integer is arbitrary but must match code specified in SUEWS\_SiteSelect.txt.
\\
\hline
2
&
\sphinxcode{\sphinxupquote{a1}}
&
{\hyperref[\detokenize{notation:term-mu}]{\sphinxtermref{\sphinxcode{\sphinxupquote{MU}}}}}
&
Coefficient for Q* term {[}-{]}
\\
\hline
3
&
\sphinxcode{\sphinxupquote{a2}}
&
{\hyperref[\detokenize{notation:term-mu}]{\sphinxtermref{\sphinxcode{\sphinxupquote{MU}}}}}
&
Coefficient for dQ*/dt term {[}h{]}
\\
\hline
4
&
\sphinxcode{\sphinxupquote{a3}}
&
{\hyperref[\detokenize{notation:term-mu}]{\sphinxtermref{\sphinxcode{\sphinxupquote{MU}}}}}
&
Constant term {[}W m -2 {]}
\\
\hline
\end{tabulary}
\par
\sphinxattableend\end{savenotes}


\subsection{SUEWS\_Profiles.txt}
\label{\detokenize{input_files/SUEWS_SiteInfo/SUEWS_Profiles:id1}}\label{\detokenize{input_files/SUEWS_SiteInfo/SUEWS_Profiles::doc}}\label{\detokenize{input_files/SUEWS_SiteInfo/SUEWS_Profiles:suews-profiles-txt}}
SUEWS\_Profiles.txt specifies the daily cycle of variables related to
human behaviour (energy use, water use and snow clearing). Different
profiles can be specified for weekdays and weekends. The profiles are
provided at hourly resolution here; the model will then interpolate the
hourly energy and water use profiles to the resolution of the model time
step and normalize the values provided. Thus it does not matter whether
columns 2-25 add up to, say 1, 24, or another number, because the model
will handle this. Currently, the snow clearing profiles are not
interpolated as these are effectively a switch (0 or 1).

If the anthropogenic heat flux and water use are specified in the met
forcing file, the energy and water use profiles are not used.

Profiles are specified for the following
\begin{itemize}
\item {} 
Anthropogenic heat flux (weekday and weekend)

\item {} 
Water use (weekday and weekend; manual and automatic irrigation)

\item {} 
Snow removal (weekday and weekend)

\item {} 
Human activity (weekday and weekend) \sphinxstylestrong{- not used in v2017a}.

\end{itemize}


\begin{savenotes}\sphinxattablestart
\centering
\begin{tabulary}{\linewidth}[t]{|T|T|T|T|}
\hline
\sphinxstyletheadfamily 
No.
&\sphinxstyletheadfamily 
Var
&\sphinxstyletheadfamily 
Use
&\sphinxstyletheadfamily 
Description
\\
\hline
1
&
\sphinxcode{\sphinxupquote{Code}}
&
{\hyperref[\detokenize{notation:term-19}]{\sphinxtermref{\sphinxcode{\sphinxupquote{L}}}}}
&
Code linking to SUEWS\_SiteSelect.txt for snow surfaces (SnowCode). Value of integer is arbitrary but must match code specified in SUEWS\_SiteSelect.txt.
\\
\hline
2
&
2-25
&
{\hyperref[\detokenize{notation:term-mu}]{\sphinxtermref{\sphinxcode{\sphinxupquote{MU}}}}}
&
Multiplier for each hour of the day {[}-{]} for energy and water use. For SnowClearing, set those hours to 1 when snow removal from paved and roof surface is allowed (0 otherwise) if the snow removal limits set in the SUEWS\_Non Veg.txt (SnowLimR emove column) are exceeded.
\\
\hline
\end{tabulary}
\par
\sphinxattableend\end{savenotes}


\subsection{SUEWS\_SiteSelect.txt}
\label{\detokenize{input_files/SUEWS_SiteInfo/SUEWS_SiteSelect::doc}}\label{\detokenize{input_files/SUEWS_SiteInfo/SUEWS_SiteSelect:suews-siteselect-txt}}\label{\detokenize{input_files/SUEWS_SiteInfo/SUEWS_SiteSelect:id1}}
For each year and each grid, site specific surface cover information and
other input parameters is provided to SUEWS by {\hyperref[\detokenize{input_files/SUEWS_SiteInfo/SUEWS_SiteSelect:suews-siteselect-txt}]{\sphinxcrossref{\DUrole{std,std-ref,std,std-ref}{SUEWS\_SiteSelect.txt}}}} (\autopageref*{\detokenize{input_files/SUEWS_SiteInfo/SUEWS_SiteSelect:suews-siteselect-txt}}).
The model currently requires a new row for each year of the model run.
All rows in this file (before the two rows of ‘-9’) will be read by the
model and run. In this file the \sphinxstylestrong{column order is important}. ‘!’ can
be used to indicate comments in the file. Comments are not read by the
programme so they can be used by the user to provide notes for their
interpretation of the contents. This is strongly recommended.


\begin{savenotes}\sphinxatlongtablestart\begin{longtable}{|\X{5}{100}|\X{25}{100}|\X{5}{100}|\X{65}{100}|}
\hline
\sphinxstyletheadfamily 
No.
&\sphinxstyletheadfamily 
Column Name
&\sphinxstyletheadfamily 
Use
&\sphinxstyletheadfamily 
Description
\\
\hline
\endfirsthead

\multicolumn{4}{c}%
{\makebox[0pt]{\sphinxtablecontinued{\tablename\ \thetable{} -- continued from previous page}}}\\
\hline
\sphinxstyletheadfamily 
No.
&\sphinxstyletheadfamily 
Column Name
&\sphinxstyletheadfamily 
Use
&\sphinxstyletheadfamily 
Description
\\
\hline
\endhead

\hline
\multicolumn{4}{r}{\makebox[0pt][r]{\sphinxtablecontinued{Continued on next page}}}\\
\endfoot

\endlastfoot

1
&
\sphinxcode{\sphinxupquote{Grid}}
&
{\hyperref[\detokenize{notation:term-mu}]{\sphinxtermref{\sphinxcode{\sphinxupquote{MU}}}}}
&
Grid numbers do not need to be consecutive and do not need to start at a particular value. Each grid must have a unique grid number. All grids must be present for all years. These grid numbers are referred to in GridConnections (columns 64-79) ( N.B. GridConnections not currently implemented! )
\\
\hline
2
&
\sphinxcode{\sphinxupquote{Year}}
&
{\hyperref[\detokenize{notation:term-mu}]{\sphinxtermref{\sphinxcode{\sphinxupquote{MU}}}}}
&
Year {[}YYYY{]} Years must be continuous. If running multiple years, ensure the rows in SiteSelect.txt are arranged so that all grids for a particular year appear on consecutive lines (rather than grouping all years together for a particular grid).
\\
\hline
3
&
\sphinxcode{\sphinxupquote{StartDLS}}
&
{\hyperref[\detokenize{notation:term-mu}]{\sphinxtermref{\sphinxcode{\sphinxupquote{MU}}}}}
&
Start of the day light savings {[}DOY{]} See section on Day Light Savings .
\\
\hline
4
&
\sphinxcode{\sphinxupquote{EndDLS}}
&
{\hyperref[\detokenize{notation:term-mu}]{\sphinxtermref{\sphinxcode{\sphinxupquote{MU}}}}}
&
End of the day light savings {[}DOY{]} See section on Day Light Savings .
\\
\hline
5
&
\sphinxcode{\sphinxupquote{lat}}
&
{\hyperref[\detokenize{notation:term-mu}]{\sphinxtermref{\sphinxcode{\sphinxupquote{MU}}}}}
&
Use coordinate system WGS84. Positive values are northern hemisphere (negative southern hemisphere). Used in radiation calculations. Note, if the total modelled area is small the latitude and longitude could be the same for each grid but small differences in radiation will not be determined. If you are defining the latitude and longitude differently between grids make certain that you provide enough decimal places.
\\
\hline
6
&
\sphinxcode{\sphinxupquote{lng}}
&
{\hyperref[\detokenize{notation:term-mu}]{\sphinxtermref{\sphinxcode{\sphinxupquote{MU}}}}}
&
Use coordinate system WGS84. For compatibility with GIS, negative values are to the west, positive values are to the east (e.g. Vancouver = -123.12; Shanghai = 121.47) Note this is a change of sign convention between v2016a and v2017a See latitude for more details.
\\
\hline
7
&
\sphinxcode{\sphinxupquote{Timezone}}
&
{\hyperref[\detokenize{notation:term-mu}]{\sphinxtermref{\sphinxcode{\sphinxupquote{MU}}}}}
&
Time zone {[}h{]} for site relative to UTC (east is positive). This should be set according to the times given in the meteorological forcing file(s).
\\
\hline
8
&
\sphinxcode{\sphinxupquote{SurfaceArea}}
&
{\hyperref[\detokenize{notation:term-mu}]{\sphinxtermref{\sphinxcode{\sphinxupquote{MU}}}}}
&
Area of the grid {[}ha{]}.
\\
\hline
9
&
\sphinxcode{\sphinxupquote{Alt}}
&
{\hyperref[\detokenize{notation:term-mu}]{\sphinxtermref{\sphinxcode{\sphinxupquote{MU}}}}}
&
Used for both the radiation and water flow between grids. ( N.B. water flow between grids not currently implemented. )
\\
\hline
10
&
\sphinxcode{\sphinxupquote{z}}
&
{\hyperref[\detokenize{notation:term-mu}]{\sphinxtermref{\sphinxcode{\sphinxupquote{MU}}}}}
&
z must be greater than the displacement height. Forcing data should be representative of the local-scale, i.e. above the height of the roughness elements.
\\
\hline
11
&
\sphinxcode{\sphinxupquote{id}}
&
{\hyperref[\detokenize{notation:term-md}]{\sphinxtermref{\sphinxcode{\sphinxupquote{MD}}}}}
&
Day {[}DOY{]} Not used: set to 1 in this version.
\\
\hline
12
&
\sphinxcode{\sphinxupquote{ih}}
&
{\hyperref[\detokenize{notation:term-md}]{\sphinxtermref{\sphinxcode{\sphinxupquote{MD}}}}}
&
Hour {[}H{]} Not used: set to 0 in this version.
\\
\hline
13
&
\sphinxcode{\sphinxupquote{imin}}
&
{\hyperref[\detokenize{notation:term-md}]{\sphinxtermref{\sphinxcode{\sphinxupquote{MD}}}}}
&
Minute {[}M{]} Not used: set to 0 in this version.
\\
\hline
14
&
\sphinxcode{\sphinxupquote{Fr\_Paved}}
&
{\hyperref[\detokenize{notation:term-mu}]{\sphinxtermref{\sphinxcode{\sphinxupquote{MU}}}}}
&
Columns 14 to 20 must sum to 1 .
\\
\hline
15
&
\sphinxcode{\sphinxupquote{Fr\_Bldgs}}
&
{\hyperref[\detokenize{notation:term-mu}]{\sphinxtermref{\sphinxcode{\sphinxupquote{MU}}}}}
&
Surface cover fraction of buildings {[}-{]}
\\
\hline
16
&
\sphinxcode{\sphinxupquote{Fr\_EveTr}}
&
{\hyperref[\detokenize{notation:term-mu}]{\sphinxtermref{\sphinxcode{\sphinxupquote{MU}}}}}
&
Surface cover fraction of evergreen trees and shrubs {[}-{]}
\\
\hline
17
&
\sphinxcode{\sphinxupquote{Fr\_DecTr}}
&
{\hyperref[\detokenize{notation:term-mu}]{\sphinxtermref{\sphinxcode{\sphinxupquote{MU}}}}}
&
Surface cover fraction of deciduous trees and shrubs {[}-{]}
\\
\hline
18
&
\sphinxcode{\sphinxupquote{Fr\_Grass}}
&
{\hyperref[\detokenize{notation:term-mu}]{\sphinxtermref{\sphinxcode{\sphinxupquote{MU}}}}}
&
Surface cover fraction of grass {[}-{]}
\\
\hline
19
&
\sphinxcode{\sphinxupquote{Fr\_Bsoil}}
&
{\hyperref[\detokenize{notation:term-mu}]{\sphinxtermref{\sphinxcode{\sphinxupquote{MU}}}}}
&
Surface cover fraction of bare soil or unmanaged land {[}-{]}
\\
\hline
20
&
\sphinxcode{\sphinxupquote{Fr\_Water}}
&
{\hyperref[\detokenize{notation:term-mu}]{\sphinxtermref{\sphinxcode{\sphinxupquote{MU}}}}}
&
Surface cover fraction of open water {[}-{]} (e.g. river, lakes, ponds, swimming pools)
\\
\hline
21
&
\sphinxcode{\sphinxupquote{IrrFr\_EveTr}}
&
{\hyperref[\detokenize{notation:term-mu}]{\sphinxtermref{\sphinxcode{\sphinxupquote{MU}}}}}
&
Fraction of evergreen trees that are irrigated {[}-{]} e.g. 50\% of the evergreen trees/shrubs are irrigated
\\
\hline
22
&
\sphinxcode{\sphinxupquote{IrrFr\_DecTr}}
&
{\hyperref[\detokenize{notation:term-mu}]{\sphinxtermref{\sphinxcode{\sphinxupquote{MU}}}}}
&
Fraction of deciduous trees that are irrigated {[}-{]}
\\
\hline
23
&
\sphinxcode{\sphinxupquote{IrrFr\_Grass}}
&
{\hyperref[\detokenize{notation:term-mu}]{\sphinxtermref{\sphinxcode{\sphinxupquote{MU}}}}}
&
Fraction of grass that is irrigated {[}-{]}
\\
\hline
24
&
\sphinxcode{\sphinxupquote{H\_Bldgs}}
&
{\hyperref[\detokenize{notation:term-mu}]{\sphinxtermref{\sphinxcode{\sphinxupquote{MU}}}}}
&
Mean building height {[}m{]}
\\
\hline
25
&
\sphinxcode{\sphinxupquote{H\_EveTr}}
&
{\hyperref[\detokenize{notation:term-mu}]{\sphinxtermref{\sphinxcode{\sphinxupquote{MU}}}}}
&
Mean height of evergreen trees {[}m{]}
\\
\hline
26
&
\sphinxcode{\sphinxupquote{H\_DecTr}}
&
{\hyperref[\detokenize{notation:term-mu}]{\sphinxtermref{\sphinxcode{\sphinxupquote{MU}}}}}
&
Mean height of deciduous trees {[}m{]}
\\
\hline
27
&
\sphinxcode{\sphinxupquote{z0}}
&
{\hyperref[\detokenize{notation:term-o}]{\sphinxtermref{\sphinxcode{\sphinxupquote{O}}}}}
&
Roughness length for momentum {[}m{]} Value supplied here is used if RoughLenMomMethod = 1 in RunControl.nml ; otherwise set to ‘-999’ and a value will be calculated by the model (RoughLenMomMethod = 2, 3).
\\
\hline
28
&
\sphinxcode{\sphinxupquote{zd}}
&
{\hyperref[\detokenize{notation:term-o}]{\sphinxtermref{\sphinxcode{\sphinxupquote{O}}}}}
&
Zero-plane displacement {[}m{]} Value supplied here is used if RoughLenMomMethod = 1 in RunControl.nml ; otherwise set to ‘-999’ and a value will be calculated by the model (RoughLenMomMethod = 2, 3).
\\
\hline
29
&
\sphinxcode{\sphinxupquote{FAI\_Bldgs}}
&
{\hyperref[\detokenize{notation:term-o}]{\sphinxtermref{\sphinxcode{\sphinxupquote{O}}}}}
&
Frontal area index for buildings {[}-{]} Required if RoughLenMomMethod = 3 in RunControl.nml .
\\
\hline
30
&
\sphinxcode{\sphinxupquote{FAI\_EveTr}}
&
{\hyperref[\detokenize{notation:term-o}]{\sphinxtermref{\sphinxcode{\sphinxupquote{O}}}}}
&
Frontal area index for evergreen trees {[}-{]} Required if RoughLenMomMethod = 3 in RunControl.nml .
\\
\hline
31
&
\sphinxcode{\sphinxupquote{FAI\_DecTr}}
&
{\hyperref[\detokenize{notation:term-o}]{\sphinxtermref{\sphinxcode{\sphinxupquote{O}}}}}
&
Frontal area index for deciduous trees {[}-{]} Required if RoughLenMomMethod = 3 in RunControl.nml .
\\
\hline
32
&
\sphinxcode{\sphinxupquote{PopDensDay}}
&
{\hyperref[\detokenize{notation:term-o}]{\sphinxtermref{\sphinxcode{\sphinxupquote{O}}}}}
&
Daytime population density (i.e. workers, tourists) {[}people ha -1 {]} Population density is required if AnthropHeatMethod = 2 in RunControl.nml . The model will use the average of daytime and night-time population densities, unless only one is provided. If daytime population density is unknown, set to -999.
\\
\hline
33
&
\sphinxcode{\sphinxupquote{PopDensNight}}
&
{\hyperref[\detokenize{notation:term-o}]{\sphinxtermref{\sphinxcode{\sphinxupquote{O}}}}}
&
Night-time population density (i.e. residents) {[}people ha -1 {]} Population density is required if AnthropHeatMethod = 2 in RunControl.nml . The model will use the average of daytime and night-time population densities, unless only one is provided. If night-time population density is unknown, set to -999.
\\
\hline
34
&
\sphinxcode{\sphinxupquote{TrafficRate}}
&
{\hyperref[\detokenize{notation:term-o}]{\sphinxtermref{\sphinxcode{\sphinxupquote{O}}}}}
&
Traffic rate {[}veh km m-2 s-1{]} Can be used for CO2 flux calculation. Do not use in v2017a - set to -999
\\
\hline
35
&
\sphinxcode{\sphinxupquote{BuildEnergyUse}}
&
{\hyperref[\detokenize{notation:term-o}]{\sphinxtermref{\sphinxcode{\sphinxupquote{O}}}}}
&
Building energy use {[}W m-2{]} Can be used for CO2 flux calculation. Do not use in v2017a - set to -999
\\
\hline
36
&
\sphinxcode{\sphinxupquote{Code\_Paved}}
&
{\hyperref[\detokenize{notation:term-19}]{\sphinxtermref{\sphinxcode{\sphinxupquote{L}}}}}
&
Code for Paved surface characteristics Provides the link to column 1 of SUEWS\_NonVeg.txt, which contains the attributes describing paved areas in this grid for this year. Value of integer is arbitrary but must match code specified in column 1 of SUEWS\_NonVeg.txt. e.g. 331 means use the characteristics specified in the row of input file SUEWS\_NonVeg.txt which has 331 in column 1 (Code).
\\
\hline
37
&
\sphinxcode{\sphinxupquote{Code\_Bldgs}}
&
{\hyperref[\detokenize{notation:term-19}]{\sphinxtermref{\sphinxcode{\sphinxupquote{L}}}}}
&
Code for Bldgs surface characteristics Provides the link to column 1 of SUEWS\_NonVeg.txt, which contains the attributes describing buildings in this grid for this year. Value of integer is arbitrary but must match code specified in column 1 of SUEWS\_NonVeg.txt.
\\
\hline
38
&
\sphinxcode{\sphinxupquote{Code\_EveTr}}
&
{\hyperref[\detokenize{notation:term-19}]{\sphinxtermref{\sphinxcode{\sphinxupquote{L}}}}}
&
Code for EveTr surface characteristics Provides the link to column 1 of SUEWS\_Veg.txt, which contains the attributes describing evergreen trees and shrubs in this grid for this year. Value of integer is arbitrary but must match code specified in column 1 of SUEWS\_Veg.txt.
\\
\hline
39
&
\sphinxcode{\sphinxupquote{Code\_DecTr}}
&
{\hyperref[\detokenize{notation:term-19}]{\sphinxtermref{\sphinxcode{\sphinxupquote{L}}}}}
&
Code for DecTr surface characteristics Provides the link to column 1 of SUEWS\_Veg.txt, which contains the attributes describing deciduous trees and shrubs in this grid for this year. Value of integer is arbitrary but must match code specified in column 1 of SUEWS\_Veg.txt.
\\
\hline
40
&
\sphinxcode{\sphinxupquote{Code\_Grass}}
&
{\hyperref[\detokenize{notation:term-19}]{\sphinxtermref{\sphinxcode{\sphinxupquote{L}}}}}
&
Code for Grass surface characteristics Provides the link to column 1 of SUEWS\_Veg.txt, which contains the attributes describing grass surfaces in this grid for this year. Value of integer is arbitrary but must match code specified in column 1 of SUEWS\_Veg.txt.
\\
\hline
41
&
\sphinxcode{\sphinxupquote{Code\_Bsoil}}
&
{\hyperref[\detokenize{notation:term-19}]{\sphinxtermref{\sphinxcode{\sphinxupquote{L}}}}}
&
Code for BSoil surface characteristics Provides the link to column 1 of SUEWS\_NonVeg.txt, which contains the attributes describing bare soil in this grid for this year. Value of integer is arbitrary but must match code specified in column 1 of SUEWS\_NonVeg.txt.
\\
\hline
42
&
\sphinxcode{\sphinxupquote{Code\_Water}}
&
{\hyperref[\detokenize{notation:term-19}]{\sphinxtermref{\sphinxcode{\sphinxupquote{L}}}}}
&
Code for Water surface characteristics Provides the link to column 1 of SUEWS\_Water.txt, which contains the attributes describing open water in this grid for this year. Value of integer is arbitrary but must match code specified in column 1 of SUEWS\_Water.txt.
\\
\hline
43
&
\sphinxcode{\sphinxupquote{LUMPS\_DrRate}}
&
{\hyperref[\detokenize{notation:term-md}]{\sphinxtermref{\sphinxcode{\sphinxupquote{MD}}}}}
&
Drainage rate of bucket for LUMPS {[}mm h -1 {]} Used for LUMPS surface wetness control. Default recommended value of 0.25 mm h -1 from Loridan et al. (2011) {[}5{]} .
\\
\hline
44
&
\sphinxcode{\sphinxupquote{LUMPS\_Cover}}
&
{\hyperref[\detokenize{notation:term-md}]{\sphinxtermref{\sphinxcode{\sphinxupquote{MD}}}}}
&
Limit when surface totally covered with water {[}mm{]} Used for LUMPS surface wetness control. Default recommended value of 1 mm from Loridan et al. (2011) {[}5{]} .
\\
\hline
45
&
\sphinxcode{\sphinxupquote{LUMPS\_MaxRes}}
&
{\hyperref[\detokenize{notation:term-md}]{\sphinxtermref{\sphinxcode{\sphinxupquote{MD}}}}}
&
Maximum water bucket reservoir {[}mm{]} Used for LUMPS surface wetness control. Default recommended value of 10 mm from Loridan et al. (2011) {[}5{]} .
\\
\hline
46
&
\sphinxcode{\sphinxupquote{NARP\_Trans}}
&
{\hyperref[\detokenize{notation:term-md}]{\sphinxtermref{\sphinxcode{\sphinxupquote{MD}}}}}
&
Atmospheric transmissivity for NARP {[}-{]} Value must in the range 0-1. Default recommended value of 1.
\\
\hline
47
&
\sphinxcode{\sphinxupquote{CondCode}}
&
{\hyperref[\detokenize{notation:term-19}]{\sphinxtermref{\sphinxcode{\sphinxupquote{L}}}}}
&
Code for surface conductance parameters Provides the link to column 1 of SUEWS\_Conductance.txt, which contains the parameters for the Jarvis (1976) parameterisation of surface conductance. Value of integer is arbitrary but must match code specified in column 1 of SUEWS\_Conductance.txt. e.g. 33 means use the characteristics specified in the row of input file SUEWS\_Conductance.txt which has 33 in column 1 (Code).
\\
\hline
48
&
\sphinxcode{\sphinxupquote{SnowCode}}
&
{\hyperref[\detokenize{notation:term-19}]{\sphinxtermref{\sphinxcode{\sphinxupquote{L}}}}}
&
Code for snow surface characteristics Provides the link to column 1 of SUEWS\_Snow.txt, which contains the attributes describing snow surfaces in this grid for this year. Value of integer is arbitrary but must match code specified in column 1 of SUEWS\_Snow.txt.
\\
\hline
49
&
\sphinxcode{\sphinxupquote{SnowClearingProfWD}}
&
{\hyperref[\detokenize{notation:term-19}]{\sphinxtermref{\sphinxcode{\sphinxupquote{L}}}}}
&
Code for snow clearing profile (weekdays) Provides the link to column 1 of SUEWS\_Profiles.txt. Value of integer is arbitrary but must match code specified in column 1 of SUEWS\_Profiles.txt. e.g. 1 means use the characteristics specified in the row of input file SUEWS\_Profiles.txt which has 1 in column 1 (Code).
\\
\hline
50
&
\sphinxcode{\sphinxupquote{SnowClearingProfWE}}
&
{\hyperref[\detokenize{notation:term-19}]{\sphinxtermref{\sphinxcode{\sphinxupquote{L}}}}}
&
Code for snow clearing profile (weekends) Provides the link to column 1 of SUEWS\_Profiles.txt. Value of integer is arbitrary but must match code specified in column 1 of SUEWS\_Profiles.txt. e.g. 1 means use the characteristics specified in the row of input file SUEWS\_Profiles.txt which has 1 in column 1 (Code). Providing the same code for SnowClearingProfWD and SnowClearingProfWE would link to the same row in SUEWS\_Profiles.txt, i.e. the same profile would be used for weekdays and weekends.
\\
\hline
51
&
\sphinxcode{\sphinxupquote{AnthropogenicCode}}
&
{\hyperref[\detokenize{notation:term-19}]{\sphinxtermref{\sphinxcode{\sphinxupquote{L}}}}}
&
Code for modelling anthropogenic heat flux Provides the link to column 1 of SUEWS\_AnthropogenicHeat.txt, which contains the model coefficients for estimation of the anthropogenic heat flux (used if AnthropHeatChoice = 1, 2 in RunControl.nml ). Value of integer is arbitrary but must match code specified in column 1 of SUEWS\_AnthropogenicHeat.txt.
\\
\hline
52
&
\sphinxcode{\sphinxupquote{EnergyUseProfWD}}
&
{\hyperref[\detokenize{notation:term-19}]{\sphinxtermref{\sphinxcode{\sphinxupquote{L}}}}}
&
Code for energy use profile (weekdays) Provides the link to column 1 of SUEWS\_Profiles.txt. Look the codes Value of integer is arbitrary but must match code specified in column 1 of SUEWS\_Profiles.txt.
\\
\hline
53
&
\sphinxcode{\sphinxupquote{EnergyUseProfWE}}
&
{\hyperref[\detokenize{notation:term-19}]{\sphinxtermref{\sphinxcode{\sphinxupquote{L}}}}}
&
Code for energy use profile (weekends) Provides the link to column 1 of SUEWS\_Profiles.txt. Value of integer is arbitrary but must match code specified in column 1 of SUEWS\_Profiles.txt.
\\
\hline
54
&
\sphinxcode{\sphinxupquote{ActivityProfWD}}
&
{\hyperref[\detokenize{notation:term-19}]{\sphinxtermref{\sphinxcode{\sphinxupquote{L}}}}}
&
Code for human activity profile (weekdays) Provides the link to column 1 of SUEWS\_Profiles.txt. Look the codes Value of integer is arbitrary but must match code specified in column 1 of SUEWS\_Profiles.txt. Used for CO2 flux calculation - not used in v2017a
\\
\hline
55
&
\sphinxcode{\sphinxupquote{ActivityProfWE}}
&
{\hyperref[\detokenize{notation:term-19}]{\sphinxtermref{\sphinxcode{\sphinxupquote{L}}}}}
&
Code for human activity profile (weekends) Provides the link to column 1 of SUEWS\_Profiles.txt. Look the codes Value of integer is arbitrary but must match code specified in column 1 of SUEWS\_Profiles.txt. Used for CO2 flux calculation - not used in v2017a
\\
\hline
56
&
\sphinxcode{\sphinxupquote{IrrigationCode}}
&
{\hyperref[\detokenize{notation:term-19}]{\sphinxtermref{\sphinxcode{\sphinxupquote{L}}}}}
&
Code for modelling irrigation Provides the link to column 1 of SUEWS\_Irrigation.txt, which contains the model coefficients for estimation of the water use (used if WU\_Choice = 0 in RunControl.nml ). Value of integer is arbitrary but must match code specified in column 1 of SUEWS\_Irrigation.txt.
\\
\hline
57
&
\sphinxcode{\sphinxupquote{WaterUseProfManuWD}}
&
{\hyperref[\detokenize{notation:term-19}]{\sphinxtermref{\sphinxcode{\sphinxupquote{L}}}}}
&
Code for water use profile (manual irrigation, weekdays) Provides the link to column 1 of SUEWS\_Profiles.txt. Value of integer is arbitrary but must match code specified in column 1 of SUEWS\_Profiles.txt.
\\
\hline
58
&
\sphinxcode{\sphinxupquote{WaterUseProfManuWE}}
&
{\hyperref[\detokenize{notation:term-19}]{\sphinxtermref{\sphinxcode{\sphinxupquote{L}}}}}
&
Code for water use profile (manual irrigation, weekends) Provides the link to column 1 of SUEWS\_Profiles.txt. Value of integer is arbitrary but must match code specified in column 1 of SUEWS\_Profiles.txt.
\\
\hline
59
&
\sphinxcode{\sphinxupquote{WaterUseProfAutoWD}}
&
{\hyperref[\detokenize{notation:term-19}]{\sphinxtermref{\sphinxcode{\sphinxupquote{L}}}}}
&
Code for water use profile (automatic irrigation, weekdays) Provides the link to column 1 of SUEWS\_Profiles.txt. Value of integer is arbitrary but must match code specified in column 1 of SUEWS\_Profiles.txt.
\\
\hline
60
&
\sphinxcode{\sphinxupquote{WaterUseProfAutoWE}}
&
{\hyperref[\detokenize{notation:term-19}]{\sphinxtermref{\sphinxcode{\sphinxupquote{L}}}}}
&
Code for water use profile (automatic irrigation, weekends) Provides the link to column 1 of SUEWS\_Profiles.txt. Value of integer is arbitrary but must match code specified in column 1 of SUEWS\_Profiles.txt.
\\
\hline
61
&
\sphinxcode{\sphinxupquote{FlowChange}}
&
{\hyperref[\detokenize{notation:term-md}]{\sphinxtermref{\sphinxcode{\sphinxupquote{MD}}}}}
&
Difference in input and output flows for water surface {[}mm h -1 {]} Used to indicate river or stream flow through the grid. Currently not fully tested!
\\
\hline
62
&
\sphinxcode{\sphinxupquote{RunoffToWater}}
&
{\hyperref[\detokenize{notation:term-md}]{\sphinxtermref{\sphinxcode{\sphinxupquote{MD}}}}} {\hyperref[\detokenize{notation:term-mu}]{\sphinxtermref{\sphinxcode{\sphinxupquote{MU}}}}}
&
Fraction of above-ground runoff flowing to water surface during flooding {[}-{]} Value must be in the range 0-1. Fraction of above-ground runoff that can flow to the water surface in the case of flooding.
\\
\hline
63
&
\sphinxcode{\sphinxupquote{PipeCapacity}}
&
{\hyperref[\detokenize{notation:term-md}]{\sphinxtermref{\sphinxcode{\sphinxupquote{MD}}}}} {\hyperref[\detokenize{notation:term-mu}]{\sphinxtermref{\sphinxcode{\sphinxupquote{MU}}}}}
&
Storage capacity of pipes {[}mm{]} Runoff amounting to less than the value specified here is assumed to be removed by pipes.
\\
\hline
64
&
\sphinxcode{\sphinxupquote{GridConnection1of8}}
&
{\hyperref[\detokenize{notation:term-md}]{\sphinxtermref{\sphinxcode{\sphinxupquote{MD}}}}} {\hyperref[\detokenize{notation:term-mu}]{\sphinxtermref{\sphinxcode{\sphinxupquote{MU}}}}}
&
The next 8 pairs of columns specify the water flow between grids. The first column of each pair specifies the grid that the water flows to (from the current grid, column 1); the second column of each pair specifies the fraction of water that flow to that grid. The fraction (i.e. amount) of water transferred may be estimated based on elevation, the length of connecting surface between grids, presence of walls, etc. Water cannot flow from the current grid to the same grid, so the grid number here must be different to the grid number in column 1. Water can flow to a maximum of 8 other grids. If there is no water flow between grids, or a single grid is run, set to 0. See section on Grid Connections
\\
\hline
65
&
\sphinxcode{\sphinxupquote{Fraction1of8}}
&
{\hyperref[\detokenize{notation:term-md}]{\sphinxtermref{\sphinxcode{\sphinxupquote{MD}}}}} {\hyperref[\detokenize{notation:term-mu}]{\sphinxtermref{\sphinxcode{\sphinxupquote{MU}}}}}
&
Fraction of water that can flow to the grid specified in previous column {[}-{]}
\\
\hline
66
&
\sphinxcode{\sphinxupquote{GridConnection2of8}}
&
{\hyperref[\detokenize{notation:term-md}]{\sphinxtermref{\sphinxcode{\sphinxupquote{MD}}}}} {\hyperref[\detokenize{notation:term-mu}]{\sphinxtermref{\sphinxcode{\sphinxupquote{MU}}}}}
&
Number of the grid where water can flow to
\\
\hline
67
&
\sphinxcode{\sphinxupquote{Fraction2of8}}
&
{\hyperref[\detokenize{notation:term-md}]{\sphinxtermref{\sphinxcode{\sphinxupquote{MD}}}}} {\hyperref[\detokenize{notation:term-mu}]{\sphinxtermref{\sphinxcode{\sphinxupquote{MU}}}}}
&
Fraction of water that can flow to the grid specified in previous column {[}-{]}
\\
\hline
68
&
\sphinxcode{\sphinxupquote{GridConnection3of8}}
&
{\hyperref[\detokenize{notation:term-md}]{\sphinxtermref{\sphinxcode{\sphinxupquote{MD}}}}} {\hyperref[\detokenize{notation:term-mu}]{\sphinxtermref{\sphinxcode{\sphinxupquote{MU}}}}}
&
Number of the grid where water can flow to
\\
\hline
69
&
\sphinxcode{\sphinxupquote{Fraction3of8}}
&
{\hyperref[\detokenize{notation:term-md}]{\sphinxtermref{\sphinxcode{\sphinxupquote{MD}}}}} {\hyperref[\detokenize{notation:term-mu}]{\sphinxtermref{\sphinxcode{\sphinxupquote{MU}}}}}
&
Fraction of water that can flow to the grid specified in previous column {[}-{]}
\\
\hline
70
&
\sphinxcode{\sphinxupquote{GridConnection4of8}}
&
{\hyperref[\detokenize{notation:term-md}]{\sphinxtermref{\sphinxcode{\sphinxupquote{MD}}}}} {\hyperref[\detokenize{notation:term-mu}]{\sphinxtermref{\sphinxcode{\sphinxupquote{MU}}}}}
&
Number of the grid where water can flow to
\\
\hline
71
&
\sphinxcode{\sphinxupquote{Fraction4of8}}
&
{\hyperref[\detokenize{notation:term-md}]{\sphinxtermref{\sphinxcode{\sphinxupquote{MD}}}}} {\hyperref[\detokenize{notation:term-mu}]{\sphinxtermref{\sphinxcode{\sphinxupquote{MU}}}}}
&
Fraction of water that can flow to the grid specified in previous column {[}-{]}
\\
\hline
72
&
\sphinxcode{\sphinxupquote{GridConnection5of8}}
&
{\hyperref[\detokenize{notation:term-md}]{\sphinxtermref{\sphinxcode{\sphinxupquote{MD}}}}} {\hyperref[\detokenize{notation:term-mu}]{\sphinxtermref{\sphinxcode{\sphinxupquote{MU}}}}}
&
Number of the grid where water can flow to
\\
\hline
73
&
\sphinxcode{\sphinxupquote{Fraction5of8}}
&
{\hyperref[\detokenize{notation:term-md}]{\sphinxtermref{\sphinxcode{\sphinxupquote{MD}}}}} {\hyperref[\detokenize{notation:term-mu}]{\sphinxtermref{\sphinxcode{\sphinxupquote{MU}}}}}
&
Fraction of water that can flow to the grid specified in previous column {[}-{]}
\\
\hline
74
&
\sphinxcode{\sphinxupquote{GridConnection6of8}}
&
{\hyperref[\detokenize{notation:term-md}]{\sphinxtermref{\sphinxcode{\sphinxupquote{MD}}}}} {\hyperref[\detokenize{notation:term-mu}]{\sphinxtermref{\sphinxcode{\sphinxupquote{MU}}}}}
&
Number of the grid where water can flow to
\\
\hline
75
&
\sphinxcode{\sphinxupquote{Fraction6of8}}
&
{\hyperref[\detokenize{notation:term-md}]{\sphinxtermref{\sphinxcode{\sphinxupquote{MD}}}}} {\hyperref[\detokenize{notation:term-mu}]{\sphinxtermref{\sphinxcode{\sphinxupquote{MU}}}}}
&
Fraction of water that can flow to the grid specified in previous column {[}-{]}
\\
\hline
76
&
\sphinxcode{\sphinxupquote{GridConnection7of8}}
&
{\hyperref[\detokenize{notation:term-md}]{\sphinxtermref{\sphinxcode{\sphinxupquote{MD}}}}} {\hyperref[\detokenize{notation:term-mu}]{\sphinxtermref{\sphinxcode{\sphinxupquote{MU}}}}}
&
Number of the grid where water can flow to
\\
\hline
77
&
\sphinxcode{\sphinxupquote{Fraction7of8}}
&
{\hyperref[\detokenize{notation:term-md}]{\sphinxtermref{\sphinxcode{\sphinxupquote{MD}}}}} {\hyperref[\detokenize{notation:term-mu}]{\sphinxtermref{\sphinxcode{\sphinxupquote{MU}}}}}
&
Fraction of water that can flow to the grid specified in previous column {[}-{]}
\\
\hline
78
&
\sphinxcode{\sphinxupquote{GridConnection8of8}}
&
{\hyperref[\detokenize{notation:term-md}]{\sphinxtermref{\sphinxcode{\sphinxupquote{MD}}}}} {\hyperref[\detokenize{notation:term-mu}]{\sphinxtermref{\sphinxcode{\sphinxupquote{MU}}}}}
&
Number of the grid where water can flow to
\\
\hline
79
&
\sphinxcode{\sphinxupquote{Fraction8of8}}
&
{\hyperref[\detokenize{notation:term-md}]{\sphinxtermref{\sphinxcode{\sphinxupquote{MD}}}}} {\hyperref[\detokenize{notation:term-mu}]{\sphinxtermref{\sphinxcode{\sphinxupquote{MU}}}}}
&
Fraction of water that can flow to the grid specified in previous column {[}-{]}
\\
\hline
80
&
\sphinxcode{\sphinxupquote{WithinGridPavedCode}}
&
{\hyperref[\detokenize{notation:term-19}]{\sphinxtermref{\sphinxcode{\sphinxupquote{L}}}}}
&
Code that links to the fraction of water that flows from Paved surfaces to surfaces in columns 2-10 of SUEWS\_WithinGridWaterDist.txt . Value of integer is arbitrary but must match code specified in column 1 of SUEWS\_WithinGridWaterDist.txt.
\\
\hline
81
&
\sphinxcode{\sphinxupquote{WithinGridBldgsCode}}
&
{\hyperref[\detokenize{notation:term-19}]{\sphinxtermref{\sphinxcode{\sphinxupquote{L}}}}}
&
Code that links to the fraction of water that flows from Bldgs surfaces to surfaces in columns 2-10 of SUEWS\_WithinGridWaterDist.txt. Value of integer is arbitrary but must match code specified in column 1 of SUEWS\_WithinGridWaterDist.txt.
\\
\hline
82
&
\sphinxcode{\sphinxupquote{WithinGridEveTrCode}}
&
{\hyperref[\detokenize{notation:term-19}]{\sphinxtermref{\sphinxcode{\sphinxupquote{L}}}}}
&
Code that links to the fraction of water that flows from EveTr surfaces to surfaces in columns 2-10 of SUEWS\_WithinGridWaterDist.txt. Value of integer is arbitrary but must match code specified in column 1 of SUEWS\_WithinGridWaterDist.txt.
\\
\hline
83
&
\sphinxcode{\sphinxupquote{WithinGridDecTrCode}}
&
{\hyperref[\detokenize{notation:term-19}]{\sphinxtermref{\sphinxcode{\sphinxupquote{L}}}}}
&
Code that links to the fraction of water that flows from DecTr surfaces to surfaces in columns 2-10 of SUEWS\_WithinGridWaterDist.txt. Value of integer is arbitrary but must match code specified in column 1 of SUEWS\_WithinGridWaterDist.txt.
\\
\hline
84
&
\sphinxcode{\sphinxupquote{WithinGridGrassCode}}
&
{\hyperref[\detokenize{notation:term-19}]{\sphinxtermref{\sphinxcode{\sphinxupquote{L}}}}}
&
Code that links to the fraction of water that flows from Grass surfaces to surfaces in columns 2-10 of SUEWS\_WithinGridWaterDist.txt. Value of integer is arbitrary but must match code specified in column 1 of SUEWS\_WithinGridWaterDist.txt.
\\
\hline
85
&
\sphinxcode{\sphinxupquote{WithinGridBSoilCode}}
&
{\hyperref[\detokenize{notation:term-19}]{\sphinxtermref{\sphinxcode{\sphinxupquote{L}}}}}
&
Code that links to the fraction of water that flows from BSoil surfaces to surfaces in columns 2-10 of SUEWS\_WithinGridWaterDist.txt. Value of integer is arbitrary but must match code specified in column 1 of SUEWS\_WithinGridWaterDist.txt.
\\
\hline
86
&
\sphinxcode{\sphinxupquote{WithinGridWaterCode}}
&
{\hyperref[\detokenize{notation:term-19}]{\sphinxtermref{\sphinxcode{\sphinxupquote{L}}}}}
&
Code that links to the fraction of water that flows from Water surfaces to surfaces in columns 2-10 of SUEWS\_WithinGridWaterDist.txt. Value of integer is arbitrary but must match code specified in column 1 of SUEWS\_WithinGridWaterDist.txt.
\\
\hline
87
&
\sphinxcode{\sphinxupquote{AreaWall}}
&
{\hyperref[\detokenize{notation:term-mu}]{\sphinxtermref{\sphinxcode{\sphinxupquote{MU}}}}}
&
Area of wall within grid (needed for ESTM calculation).
\\
\hline
88
&
\sphinxcode{\sphinxupquote{Fr\_ESTMClass\_Paved1}}
&
{\hyperref[\detokenize{notation:term-mu}]{\sphinxtermref{\sphinxcode{\sphinxupquote{MU}}}}}
&
Columns 88-90 must add up to 1
\\
\hline
89
&
\sphinxcode{\sphinxupquote{Fr\_ESTMClass\_Paved2}}
&
{\hyperref[\detokenize{notation:term-mu}]{\sphinxtermref{\sphinxcode{\sphinxupquote{MU}}}}}
&
Columns 88-90 must add up to 1
\\
\hline
90
&
\sphinxcode{\sphinxupquote{Fr\_ESTMClass\_Paved3}}
&
{\hyperref[\detokenize{notation:term-mu}]{\sphinxtermref{\sphinxcode{\sphinxupquote{MU}}}}}
&
Columns 88-90 must add up to 1
\\
\hline
91
&
\sphinxcode{\sphinxupquote{Code\_ESTMClass\_Paved1}}
&
{\hyperref[\detokenize{notation:term-19}]{\sphinxtermref{\sphinxcode{\sphinxupquote{L}}}}}
&
Code linking to SUEWS\_ESTMCoefficients.txt
\\
\hline
92
&
\sphinxcode{\sphinxupquote{Code\_ESTMClass\_Paved2}}
&
{\hyperref[\detokenize{notation:term-19}]{\sphinxtermref{\sphinxcode{\sphinxupquote{L}}}}}
&
Code linking to SUEWS\_ESTMCoefficients.txt
\\
\hline
93
&
\sphinxcode{\sphinxupquote{Code\_ESTMClass\_Paved3}}
&
{\hyperref[\detokenize{notation:term-19}]{\sphinxtermref{\sphinxcode{\sphinxupquote{L}}}}}
&
Code linking to SUEWS\_ESTMCoefficients.txt
\\
\hline
94
&
\sphinxcode{\sphinxupquote{Fr\_ESTMClass\_Bldgs1}}
&
{\hyperref[\detokenize{notation:term-mu}]{\sphinxtermref{\sphinxcode{\sphinxupquote{MU}}}}}
&
Columns 94-98 must add up to 1
\\
\hline
95
&
\sphinxcode{\sphinxupquote{Fr\_ESTMClass\_Bldgs2}}
&
{\hyperref[\detokenize{notation:term-mu}]{\sphinxtermref{\sphinxcode{\sphinxupquote{MU}}}}}
&
Columns 94-98 must add up to 1
\\
\hline
96
&
\sphinxcode{\sphinxupquote{Fr\_ESTMClass\_Bldgs3}}
&
{\hyperref[\detokenize{notation:term-mu}]{\sphinxtermref{\sphinxcode{\sphinxupquote{MU}}}}}
&
Columns 94-98 must add up to 1
\\
\hline
97
&
\sphinxcode{\sphinxupquote{Fr\_ESTMClass\_Bldgs4}}
&
{\hyperref[\detokenize{notation:term-mu}]{\sphinxtermref{\sphinxcode{\sphinxupquote{MU}}}}}
&
Columns 94-98 must add up to 1
\\
\hline
98
&
\sphinxcode{\sphinxupquote{Fr\_ESTMClass\_Bldgs5}}
&
{\hyperref[\detokenize{notation:term-mu}]{\sphinxtermref{\sphinxcode{\sphinxupquote{MU}}}}}
&
Columns 94-98 must add up to 1
\\
\hline
99
&
\sphinxcode{\sphinxupquote{Code\_ESTMClass\_Bldgs1}}
&
{\hyperref[\detokenize{notation:term-19}]{\sphinxtermref{\sphinxcode{\sphinxupquote{L}}}}}
&
Code linking to SUEWS\_ESTMCoefficients.txt
\\
\hline
100
&
\sphinxcode{\sphinxupquote{Code\_ESTMClass\_Bldgs2}}
&
{\hyperref[\detokenize{notation:term-19}]{\sphinxtermref{\sphinxcode{\sphinxupquote{L}}}}}
&
Code linking to SUEWS\_ESTMCoefficients.txt
\\
\hline
101
&
\sphinxcode{\sphinxupquote{Code\_ESTMClass\_Bldgs3}}
&
{\hyperref[\detokenize{notation:term-19}]{\sphinxtermref{\sphinxcode{\sphinxupquote{L}}}}}
&
Code linking to SUEWS\_ESTMCoefficients.txt
\\
\hline
102
&
\sphinxcode{\sphinxupquote{Code\_ESTMClass\_Bldgs4}}
&
{\hyperref[\detokenize{notation:term-19}]{\sphinxtermref{\sphinxcode{\sphinxupquote{L}}}}}
&
Code linking to SUEWS\_ESTMCoefficients.txt
\\
\hline
103
&
\sphinxcode{\sphinxupquote{Code\_ESTMClass\_Bldgs5}}
&
{\hyperref[\detokenize{notation:term-19}]{\sphinxtermref{\sphinxcode{\sphinxupquote{L}}}}}
&
Code linking to SUEWS\_ESTMCoefficients.txt
\\
\hline
\end{longtable}\sphinxatlongtableend\end{savenotes}

see this one: \sphinxcode{\sphinxupquote{a2}}


\subsubsection{Day Light Savings (DLS)}
\label{\detokenize{input_files/SUEWS_SiteInfo/SUEWS_SiteSelect:day-light-savings-dls}}
The dates for DLS normally vary each year and country as they are often
associated with a specific set of Sunday mornings at the beginning of
summer and autumn. Note it is important to remember leap years. You can
check \sphinxurl{http://www.timeanddate.com/time/dst/} for your city.

\begin{sphinxadmonition}{tip}{Tip:}
If DLS does not occur give a start and end day immediately after it.
Make certain the dummy dates are correct for the hemisphere
\begin{itemize}
\item {} 
for northern hemisphere, use: 180 181

\item {} 
for southern hemisphere, use:  365 1

\end{itemize}
\end{sphinxadmonition}
\begin{description}
\item[{Example when running  multiple years (in this case 2008 and 2009 in Canada):}] \leavevmode

\begin{savenotes}\sphinxattablestart
\centering
\begin{tabulary}{\linewidth}[t]{|T|T|T|}
\hline
\sphinxstyletheadfamily 
Year
&\sphinxstyletheadfamily 
start of daylight savings
&\sphinxstyletheadfamily 
end of daylight savings
\\
\hline
2008
&
170
&
240
\\
\hline
2009
&
172
&
242
\\
\hline
\end{tabulary}
\par
\sphinxattableend\end{savenotes}

\end{description}


\subsubsection{Grid Connections (water flow between grids)}
\label{\detokenize{input_files/SUEWS_SiteInfo/SUEWS_SiteSelect:grid-connections-water-flow-between-grids}}
\begin{sphinxadmonition}{caution}{Caution:}\begin{itemize}
\item {} 
not currently implemented

\item {} 
columns 64-79 of {\hyperref[\detokenize{input_files/SUEWS_SiteInfo/SUEWS_SiteSelect:suews-siteselect-txt}]{\sphinxcrossref{\DUrole{std,std-ref,std,std-ref}{SUEWS\_SiteSelect.txt}}}} (\autopageref*{\detokenize{input_files/SUEWS_SiteInfo/SUEWS_SiteSelect:suews-siteselect-txt}}) can be set to zero.

\end{itemize}
\end{sphinxadmonition}

This section gives an example of water flow between grids, calculated
based on the relative elevation of the grids and length of the
connecting surface between adjacent grids. For the square grids in the
figure, water flow is assumed to be zero between diagonally adjacent
grids, as the length of connecting surface linking the grids is very
small. Model grids need not be square or the same size.

The table gives example values for the grid connections part of
{\hyperref[\detokenize{input_files/SUEWS_SiteInfo/SUEWS_SiteSelect:suews-siteselect-txt}]{\sphinxcrossref{\DUrole{std,std-ref,std,std-ref}{SUEWS\_SiteSelect.txt}}}} (\autopageref*{\detokenize{input_files/SUEWS_SiteInfo/SUEWS_SiteSelect:suews-siteselect-txt}}) for the grids shown in
the figure. For each row, only water flowing out of the current grid is
entered (e.g. water flows from 234 to 236 and 237, with a larger
proportion of water flowing to 237 because of the greater length of
connecting surface between 234 and 237 than between 234 and 236. No
water is assumed to flow between 234 and 233 or 235 because there is no
elevation difference between these grids. Grids 234 and 238 are at the
same elevation and only connect at a point, so no water flows between
them. Water enters grid 234 from grids 230, 231 and 232 as these are
more elevated.

\begin{figure}[htbp]
\centering
\capstart

\noindent\sphinxincludegraphics{{GridConnections_1}.jpg}
\caption{Example grid connections showing water flow between grids.}\label{\detokenize{input_files/SUEWS_SiteInfo/SUEWS_SiteSelect:id2}}\end{figure}

\begin{sphinxadmonition}{note}{Note:}
Arrows indicate the water flow in to and out of grid 234,
but note that only only water flowing out of each grid is entered in {\hyperref[\detokenize{input_files/SUEWS_SiteInfo/SUEWS_SiteSelect:suews-siteselect-txt}]{\sphinxcrossref{\DUrole{std,std-ref,std,std-ref}{SUEWS\_SiteSelect.txt}}}} (\autopageref*{\detokenize{input_files/SUEWS_SiteInfo/SUEWS_SiteSelect:suews-siteselect-txt}})
\end{sphinxadmonition}

\begin{figure}[htbp]
\centering
\capstart

\noindent\sphinxincludegraphics{{GridConnections_2_v2}.jpg}
\caption{Example values for the grid connections part of {\hyperref[\detokenize{input_files/SUEWS_SiteInfo/SUEWS_SiteSelect:suews-siteselect-txt}]{\sphinxcrossref{\DUrole{std,std-ref,std,std-ref}{SUEWS\_SiteSelect.txt}}}} (\autopageref*{\detokenize{input_files/SUEWS_SiteInfo/SUEWS_SiteSelect:suews-siteselect-txt}}) for the grids.}\label{\detokenize{input_files/SUEWS_SiteInfo/SUEWS_SiteSelect:id3}}\end{figure}


\subsection{SUEWS\_Snow.txt}
\label{\detokenize{input_files/SUEWS_SiteInfo/SUEWS_Snow:suews-snow-txt}}\label{\detokenize{input_files/SUEWS_SiteInfo/SUEWS_Snow::doc}}\label{\detokenize{input_files/SUEWS_SiteInfo/SUEWS_Snow:id1}}
SUEWS\_Snow.txt specifies the characteristics for snow surfaces when
{\hyperref[\detokenize{input_files/RunControl/Model_run_options:cmdoption-arg-snowuse}]{\sphinxcrossref{\sphinxcode{\sphinxupquote{SnowUse=1}}}}} (\autopageref*{\detokenize{input_files/RunControl/Model_run_options:cmdoption-arg-snowuse}}) in {\hyperref[\detokenize{input_files/RunControl/RunControl:runcontrol-nml}]{\sphinxcrossref{\DUrole{std,std-ref,std,std-ref}{RunControl.nml}}}} (\autopageref*{\detokenize{input_files/RunControl/RunControl:runcontrol-nml}}). If the snow part of
the model is not run, fill this table with ‘-999’ except for the first
(Code) column and set {\hyperref[\detokenize{input_files/RunControl/Model_run_options:cmdoption-arg-snowuse}]{\sphinxcrossref{\sphinxcode{\sphinxupquote{SnowUse=0}}}}} (\autopageref*{\detokenize{input_files/RunControl/Model_run_options:cmdoption-arg-snowuse}}) in {\hyperref[\detokenize{input_files/RunControl/RunControl:runcontrol-nml}]{\sphinxcrossref{\DUrole{std,std-ref,std,std-ref}{RunControl.nml}}}} (\autopageref*{\detokenize{input_files/RunControl/RunControl:runcontrol-nml}}).
For a detailed description of the variables, see Järvi et al.
(2014) \phantomsection\label{\detokenize{input_files/SUEWS_SiteInfo/SUEWS_Snow:id2}}{\hyperref[\detokenize{references:leena2014}]{\sphinxcrossref{{[}Leena2014{]}}}} (\autopageref*{\detokenize{references:leena2014}}).

\begin{sphinxadmonition}{warning}{Warning:}
In the current release {\hyperref[\detokenize{input_files/RunControl/Model_run_options:cmdoption-arg-snowuse}]{\sphinxcrossref{\sphinxcode{\sphinxupquote{SnowUse}}}}} (\autopageref*{\detokenize{input_files/RunControl/Model_run_options:cmdoption-arg-snowuse}}) should be set to 0.
\end{sphinxadmonition}


\begin{savenotes}\sphinxattablestart
\centering
\begin{tabular}[t]{|\X{5}{100}|\X{25}{100}|\X{5}{100}|\X{65}{100}|}
\hline
\sphinxstyletheadfamily 
No.
&\sphinxstyletheadfamily 
Column Name
&\sphinxstyletheadfamily 
Use
&\sphinxstyletheadfamily 
Description
\\
\hline
1
&
\sphinxcode{\sphinxupquote{Code}}
&
{\hyperref[\detokenize{notation:term-19}]{\sphinxtermref{\sphinxcode{\sphinxupquote{L}}}}}
&
Code linking to SUEWS\_SiteSelect.txt for snow surfaces (SnowCode). Value of integer is arbitrary but must match code specified in SUEWS\_SiteSelect.txt.
\\
\hline
2
&
\sphinxcode{\sphinxupquote{RadMeltFactor}}
&
{\hyperref[\detokenize{notation:term-mu}]{\sphinxtermref{\sphinxcode{\sphinxupquote{MU}}}}}
&
Hourly radiation melt factor of snow {[}mm W -1 h -1 {]}
\\
\hline
3
&
\sphinxcode{\sphinxupquote{TempMeltFactor}}
&
{\hyperref[\detokenize{notation:term-mu}]{\sphinxtermref{\sphinxcode{\sphinxupquote{MU}}}}}
&
Hourly temperature melt factor of snow {[}mm °C -1 h -1 {]} (In previous model version, this parameter was 0.12)
\\
\hline
4
&
\sphinxcode{\sphinxupquote{AlbedoMin}}
&
{\hyperref[\detokenize{notation:term-mu}]{\sphinxtermref{\sphinxcode{\sphinxupquote{MU}}}}}
&
Example values {[}-{]} 0.18 Järvi et al. (2014) {[}15{]}
\\
\hline
5
&
\sphinxcode{\sphinxupquote{AlbedoMax}}
&
{\hyperref[\detokenize{notation:term-mu}]{\sphinxtermref{\sphinxcode{\sphinxupquote{MU}}}}}
&
Example values {[}-{]} 0.85 Järvi et al. (2014) {[}15{]}
\\
\hline
6
&
\sphinxcode{\sphinxupquote{Emissivity}}
&
{\hyperref[\detokenize{notation:term-mu}]{\sphinxtermref{\sphinxcode{\sphinxupquote{MU}}}}}
&
Effective surface emissivity. View factors should be taken into account Example values {[}-{]} 0.99 Järvi et al. (2014) {[}15{]}
\\
\hline
7
&
\sphinxcode{\sphinxupquote{tau\_a}}
&
{\hyperref[\detokenize{notation:term-md}]{\sphinxtermref{\sphinxcode{\sphinxupquote{MD}}}}}
&
Time constant for snow albedo aging in cold snow {[}-{]}
\\
\hline
8
&
\sphinxcode{\sphinxupquote{tau\_f}}
&
{\hyperref[\detokenize{notation:term-md}]{\sphinxtermref{\sphinxcode{\sphinxupquote{MD}}}}}
&
Time constant for snow albedo aging in melting snow {[}-{]}
\\
\hline
9
&
\sphinxcode{\sphinxupquote{PrecipiLimAlb}}
&
{\hyperref[\detokenize{notation:term-md}]{\sphinxtermref{\sphinxcode{\sphinxupquote{MD}}}}}
&
Limit for hourly precipitation when the ground is fully covered with snow. Then snow albedo is reset to AlbedoMax {[}mm{]}
\\
\hline
10
&
\sphinxcode{\sphinxupquote{snowDensMin}}
&
{\hyperref[\detokenize{notation:term-md}]{\sphinxtermref{\sphinxcode{\sphinxupquote{MD}}}}}
&
Fresh snow density {[}kg m -3 {]}
\\
\hline
11
&
\sphinxcode{\sphinxupquote{snowDensMax}}
&
{\hyperref[\detokenize{notation:term-md}]{\sphinxtermref{\sphinxcode{\sphinxupquote{MD}}}}}
&
Maximum snow density {[}kg m -3 {]}
\\
\hline
12
&
\sphinxcode{\sphinxupquote{tau\_r}}
&
{\hyperref[\detokenize{notation:term-md}]{\sphinxtermref{\sphinxcode{\sphinxupquote{MD}}}}}
&
Time constant for snow density ageing {[}-{]}
\\
\hline
13
&
\sphinxcode{\sphinxupquote{CRWMin}}
&
{\hyperref[\detokenize{notation:term-md}]{\sphinxtermref{\sphinxcode{\sphinxupquote{MD}}}}}
&
Minimum water holding capacity of snow {[}mm{]}
\\
\hline
14
&
\sphinxcode{\sphinxupquote{CRWMax}}
&
{\hyperref[\detokenize{notation:term-md}]{\sphinxtermref{\sphinxcode{\sphinxupquote{MD}}}}}
&
Maximum water holding capacity of snow {[}mm{]}
\\
\hline
15
&
\sphinxcode{\sphinxupquote{PrecipLimSnow}}
&
{\hyperref[\detokenize{notation:term-md}]{\sphinxtermref{\sphinxcode{\sphinxupquote{MD}}}}}
&
Auer (1974) {[}38{]}
\\
\hline
16
&
\sphinxcode{\sphinxupquote{OHMCode\_SummerWet}}
&
{\hyperref[\detokenize{notation:term-19}]{\sphinxtermref{\sphinxcode{\sphinxupquote{L}}}}}
&
Code for OHM coefficients to use for this surface during wet conditions in summer. Links to SUEWS\_OHMCoefficients.txt . Value of integer is arbitrary but must match code specified in column 1 of SUEWS\_OHMCoefficients.txt.
\\
\hline
17
&
\sphinxcode{\sphinxupquote{OHMCode\_SummerDry}}
&
{\hyperref[\detokenize{notation:term-19}]{\sphinxtermref{\sphinxcode{\sphinxupquote{L}}}}}
&
Code for OHM coefficients to use for this surface during dry conditions in summer. Links to SUEWS\_OHMCoefficients.txt . Value of integer is arbitrary but must match code specified in column 1 of SUEWS\_OHMCoefficients.txt.
\\
\hline
18
&
\sphinxcode{\sphinxupquote{OHMCode\_WinterWet}}
&
{\hyperref[\detokenize{notation:term-19}]{\sphinxtermref{\sphinxcode{\sphinxupquote{L}}}}}
&
Code for OHM coefficients to use for this surface during wet conditions in winter. Links to SUEWS\_OHMCoefficients.txt . Value of integer is arbitrary but must match code specified in column 1 of SUEWS\_OHMCoefficients.txt.
\\
\hline
19
&
\sphinxcode{\sphinxupquote{OHMCode\_WinterDry}}
&
{\hyperref[\detokenize{notation:term-19}]{\sphinxtermref{\sphinxcode{\sphinxupquote{L}}}}}
&
Code for OHM coefficients to use for this surface during dry conditions in winter. Links to SUEWS\_OHMCoefficients.txt . Value of integer is arbitrary but must match code specified in column 1 of SUEWS\_OHMCoefficients.txt.
\\
\hline
20
&
\sphinxcode{\sphinxupquote{OHMThresh\_SW}}
&
{\hyperref[\detokenize{notation:term-md}]{\sphinxtermref{\sphinxcode{\sphinxupquote{MD}}}}}
&
Temperature threshold determining whether summer/winter OHM coefficients are applied {[}deg C{]} If 5-day running mean air temperature is greater than or equal to this threshold, OHM coefficients for summertime are applied; otherwise coefficients for wintertime are applied. Not actually used for Snow surface as winter wet conditions always assumed.
\\
\hline
21
&
\sphinxcode{\sphinxupquote{OHMThresh\_WD}}
&
{\hyperref[\detokenize{notation:term-md}]{\sphinxtermref{\sphinxcode{\sphinxupquote{MD}}}}}
&
Soil moisture threshold determining whether wet/dry OHM coefficients are applied {[}-{]} If soil moisture (as a proportion of maximum soil moisture capacity) exceeds this threshold for bare soil and vegetated surfaces, OHM coefficients for wet conditions are applied; otherwise coefficients for dry coefficients are applied. Note that OHM coefficients for wet conditions are applied if the surface is wet. Not actually used for Snow surface as winter wet conditions always assumed.
\\
\hline
22
&
\sphinxcode{\sphinxupquote{ESTMCode}}
&
{\hyperref[\detokenize{notation:term-19}]{\sphinxtermref{\sphinxcode{\sphinxupquote{L}}}}}
&
For paved and building surfaces, it is possible to specify multiple codes per grid (3 for paved, 5 for buildings) using SUEWS\_SiteSelect.txt . In this case, set ESTM code here to zero.
\\
\hline
23
&
\sphinxcode{\sphinxupquote{AnOHM\_Cp}}
&
{\hyperref[\detokenize{notation:term-mu}]{\sphinxtermref{\sphinxcode{\sphinxupquote{MU}}}}}
&
Volumetric heat capacity for this surface to use in AnOHM {[}J m -3 {]}
\\
\hline
24
&
\sphinxcode{\sphinxupquote{AnOHM\_Kk}}
&
{\hyperref[\detokenize{notation:term-mu}]{\sphinxtermref{\sphinxcode{\sphinxupquote{MU}}}}}
&
Thermal conductivity for this surface to use in AnOHM {[}W m K -1 {]}
\\
\hline
25
&
\sphinxcode{\sphinxupquote{AnOHM\_Ch}}
&
{\hyperref[\detokenize{notation:term-mu}]{\sphinxtermref{\sphinxcode{\sphinxupquote{MU}}}}}
&
Bulk transfer coefficient for this surface to use in AnOHM {[}-{]}
\\
\hline
\end{tabular}
\par
\sphinxattableend\end{savenotes}


\subsection{SUEWS\_Soil.txt}
\label{\detokenize{input_files/SUEWS_SiteInfo/SUEWS_Soil:suews-soil-txt}}\label{\detokenize{input_files/SUEWS_SiteInfo/SUEWS_Soil::doc}}\label{\detokenize{input_files/SUEWS_SiteInfo/SUEWS_Soil:id1}}
SUEWS\_Soil.txt specifies the characteristics of the sub-surface soil
below each of the non-water surface types (Paved, Bldgs, EveTr, DecTr,
Grass, BSoil). The model does not have a soi store below the water
surfaces. Note that these sub-surface soil stores are different to the
bare soil/unmamnaged surface cover type. Each of the non-water surface
types need to link to soil characteristics specified here. If the soil
characteristics are assumed to be the same for all surface types, use a
single code value to link the characteristics here with the SoilTypeCode
columns in {\hyperref[\detokenize{input_files/SUEWS_SiteInfo/SUEWS_NonVeg:suews-nonveg-txt}]{\sphinxcrossref{\DUrole{std,std-ref,std,std-ref}{SUEWS\_NonVeg.txt}}}} (\autopageref*{\detokenize{input_files/SUEWS_SiteInfo/SUEWS_NonVeg:suews-nonveg-txt}}) and {\hyperref[\detokenize{input_files/SUEWS_SiteInfo/SUEWS_Veg:suews-veg-txt}]{\sphinxcrossref{\DUrole{std,std-ref,std,std-ref}{SUEWS\_Veg.txt}}}} (\autopageref*{\detokenize{input_files/SUEWS_SiteInfo/SUEWS_Veg:suews-veg-txt}}).

Soil moisture can either be provided using observational data in the met
forcing file (smd\_choice = 1 or 2 in
{\hyperref[\detokenize{input_files/RunControl/RunControl:runcontrol-nml}]{\sphinxcrossref{\DUrole{std,std-ref,std,std-ref}{RunControl.nml}}}} (\autopageref*{\detokenize{input_files/RunControl/RunControl:runcontrol-nml}})) and providing some metadata
information here (OBS columns), or modelled by SUEWS (smd\_choice = 0
in {\hyperref[\detokenize{input_files/RunControl/RunControl:runcontrol-nml}]{\sphinxcrossref{\DUrole{std,std-ref,std,std-ref}{RunControl.nml}}}} (\autopageref*{\detokenize{input_files/RunControl/RunControl:runcontrol-nml}})). \sphinxstylestrong{- Note, the option to use
observational data is not operational in the current release!}


\begin{savenotes}\sphinxattablestart
\centering
\begin{tabulary}{\linewidth}[t]{|T|T|T|T|}
\hline
\sphinxstyletheadfamily 
No.
&\sphinxstyletheadfamily 
Column Name
&\sphinxstyletheadfamily 
Use
&\sphinxstyletheadfamily 
Description
\\
\hline
1
&
\sphinxcode{\sphinxupquote{Code}}
&
{\hyperref[\detokenize{notation:term-19}]{\sphinxtermref{\sphinxcode{\sphinxupquote{L}}}}}
&
Code linking to the SoilTypeCode column in SUEWS\_NonVeg.txt (for Paved, Bldgs and BSoil surfaces) and SUEWS\_Veg.txt (for EveTr, DecTr and Grass surfaces). Value of integer is arbitrary but must match code specified in SUEWS\_SiteSelect.txt.
\\
\hline
2
&
\sphinxcode{\sphinxupquote{SoilDepth}}
&
{\hyperref[\detokenize{notation:term-md}]{\sphinxtermref{\sphinxcode{\sphinxupquote{MD}}}}}
&
Depth of sub-surface soil store {[}mm{]} i.e. the depth of soil beneath the surface
\\
\hline
3
&
\sphinxcode{\sphinxupquote{SoilStoreCap}}
&
{\hyperref[\detokenize{notation:term-md}]{\sphinxtermref{\sphinxcode{\sphinxupquote{MD}}}}}
&
SoilStoreCap must not be greater than SoilDepth.
\\
\hline
4
&
\sphinxcode{\sphinxupquote{SatHydraulicCond}}
&
{\hyperref[\detokenize{notation:term-md}]{\sphinxtermref{\sphinxcode{\sphinxupquote{MD}}}}}
&
Hydraulic conductivity for saturated soil {[}mm s -1 {]}
\\
\hline
5
&
\sphinxcode{\sphinxupquote{SoilDensity}}
&
{\hyperref[\detokenize{notation:term-md}]{\sphinxtermref{\sphinxcode{\sphinxupquote{MD}}}}}
&
Soil density {[}kg m -3 {]}
\\
\hline
6
&
\sphinxcode{\sphinxupquote{InfiltrationRate}}
&
{\hyperref[\detokenize{notation:term-o}]{\sphinxtermref{\sphinxcode{\sphinxupquote{O}}}}}
&
Not currently used
\\
\hline
7
&
\sphinxcode{\sphinxupquote{OBS\_SMDepth}}
&
{\hyperref[\detokenize{notation:term-o}]{\sphinxtermref{\sphinxcode{\sphinxupquote{O}}}}}
&
Use only if soil moisture is observed and provided in the met forcing file and smd\_choice = 1 or 2. Use of observed soil moisture not currently tested
\\
\hline
8
&
\sphinxcode{\sphinxupquote{OBS\_SMCap}}
&
{\hyperref[\detokenize{notation:term-o}]{\sphinxtermref{\sphinxcode{\sphinxupquote{O}}}}}
&
Use only if soil moisture is observed and provided in the met forcing file and smd\_choice = 1 or 2. Use of observed soil moisture not currently tested
\\
\hline
9
&
\sphinxcode{\sphinxupquote{OBS\_SoilNotRocks}}
&
{\hyperref[\detokenize{notation:term-o}]{\sphinxtermref{\sphinxcode{\sphinxupquote{O}}}}}
&
Use only if soil moisture is observed and provided in the met forcing file and smd\_choice = 1 or 2. Use of observed soil moisture not currently tested
\\
\hline
\end{tabulary}
\par
\sphinxattableend\end{savenotes}


\subsection{SUEWS\_Veg.txt}
\label{\detokenize{input_files/SUEWS_SiteInfo/SUEWS_Veg:suews-veg-txt}}\label{\detokenize{input_files/SUEWS_SiteInfo/SUEWS_Veg::doc}}\label{\detokenize{input_files/SUEWS_SiteInfo/SUEWS_Veg:id1}}
SUEWS\_Veg.txt specifies the characteristics for the vegetated surface
cover types (EveTr, DecTr, Grass) by linking codes in column 1 of
SUEWS\_Veg.txt to the codes specified in
{\hyperref[\detokenize{input_files/SUEWS_SiteInfo/SUEWS_Veg:SUEWS_SiteSelect.txt}]{\emph{SUEWS\_SiteSelect.txt}}} (\autopageref*{\detokenize{input_files/SUEWS_SiteInfo/SUEWS_Veg:SUEWS_SiteSelect.txt}}) (Code\_EveTr,
Code\_DecTr, Code\_Grass). Each row should correspond to a particular
surface type. For suggestions on how to complete this table, see:
\sphinxhref{http://urban-climate.net/umep/TypicalValues\#Typical\_Values}{Typical
Values}.


\begin{savenotes}\sphinxatlongtablestart\begin{longtable}{|\X{5}{100}|\X{25}{100}|\X{5}{100}|\X{65}{100}|}
\hline
\sphinxstyletheadfamily 
No.
&\sphinxstyletheadfamily 
Column Name
&\sphinxstyletheadfamily 
Use
&\sphinxstyletheadfamily 
Description
\\
\hline
\endfirsthead

\multicolumn{4}{c}%
{\makebox[0pt]{\sphinxtablecontinued{\tablename\ \thetable{} -- continued from previous page}}}\\
\hline
\sphinxstyletheadfamily 
No.
&\sphinxstyletheadfamily 
Column Name
&\sphinxstyletheadfamily 
Use
&\sphinxstyletheadfamily 
Description
\\
\hline
\endhead

\hline
\multicolumn{4}{r}{\makebox[0pt][r]{\sphinxtablecontinued{Continued on next page}}}\\
\endfoot

\endlastfoot

1
&
\sphinxcode{\sphinxupquote{Code}}
&
{\hyperref[\detokenize{notation:term-19}]{\sphinxtermref{\sphinxcode{\sphinxupquote{L}}}}}
&
Code linking to SUEWS\_SiteSelect.txt for evergreen trees and shrubs (Code\_EveTr), deciduous trees and shrubs (Code\_DecTr) and grass surfaces (Code\_Grass). Value of integer is arbitrary but must match codes specified in SUEWS\_SiteSelect.txt.
\\
\hline
2
&
\sphinxcode{\sphinxupquote{AlbedoMin}}
&
{\hyperref[\detokenize{notation:term-mu}]{\sphinxtermref{\sphinxcode{\sphinxupquote{MU}}}}}
&
Effective surface albedo (middle of the day value) for wintertime (not including snow), leaf-off. View factors should be taken into account. Example values {[}-{]} 0.1 EveTr Oke (1987) {[}35{]}  0.18 DecTr Oke (1987) {[}35{]}  0.21 Grass Oke (1987) {[}35{]}
\\
\hline
3
&
\sphinxcode{\sphinxupquote{AlbedoMax}}
&
{\hyperref[\detokenize{notation:term-mu}]{\sphinxtermref{\sphinxcode{\sphinxupquote{MU}}}}}
&
Effective surface albedo (middle of the day value) for summertime, full leaf-on. View factors should be taken into account. Example values {[}-{]} 0.1 EveTr Oke (1987) {[}35{]}  0.18 DecTr Oke (1987) {[}35{]}  0.21 Grass Oke (1987) {[}35{]}
\\
\hline
4
&
\sphinxcode{\sphinxupquote{Emissivity}}
&
{\hyperref[\detokenize{notation:term-mu}]{\sphinxtermref{\sphinxcode{\sphinxupquote{MU}}}}}
&
Effective surface emissivity. View factors should be taken into account. Example values {[}-{]} 0.98 EveTr Oke (1987) {[}35{]}  0.98 DecTr Oke (1987) {[}35{]}  0.93 Grass Oke (1987) {[}35{]}
\\
\hline
5
&
\sphinxcode{\sphinxupquote{StorageMin}}
&
{\hyperref[\detokenize{notation:term-md}]{\sphinxtermref{\sphinxcode{\sphinxupquote{MD}}}}}
&
Minimum water storage capacity for upper surfaces (i.e. canopy). Min/max values are to account for seasonal variation (e.g. leaf-off/leaf-on differences for vegetated surfaces). Example values {[}mm{]} 1.3 EveTr Breuer et al. (2003) {[}36{]}  0.3 DecTr Breuer et al. (2003) {[}36{]}  1.9 Grass Breuer et al. (2003) {[}36{]}
\\
\hline
6
&
\sphinxcode{\sphinxupquote{StorageMax}}
&
{\hyperref[\detokenize{notation:term-md}]{\sphinxtermref{\sphinxcode{\sphinxupquote{MD}}}}}
&
Maximum water storage capacity for upper surfaces (i.e. canopy) Min/max values are to account for seasonal variation (e.g. leaf-off/leaf-on differences for vegetated surfaces) Only used for DecTr surfaces - set EveTr and Grass values the same as StorageMin. Example values {[}mm{]} 1.3 EveTr Breuer et al. (2003) {[}36{]}  0.8 DecTr Breuer et al. (2003) {[}36{]}  1.9 Grass Breuer et al. (2003) {[}36{]}
\\
\hline
7
&
\sphinxcode{\sphinxupquote{WetThreshold}}
&
{\hyperref[\detokenize{notation:term-md}]{\sphinxtermref{\sphinxcode{\sphinxupquote{MD}}}}}
&
Depth of water which determines whether evaporation occurs from a partially wet or completely wet surface. Example values {[}mm{]} 1.8 EveTr 1. DecTr 2. Grass
\\
\hline
8
&
\sphinxcode{\sphinxupquote{StateLimit}}
&
{\hyperref[\detokenize{notation:term-md}]{\sphinxtermref{\sphinxcode{\sphinxupquote{MD}}}}}
&
Currently only used for the water surface
\\
\hline
9
&
\sphinxcode{\sphinxupquote{DrainageEq}}
&
{\hyperref[\detokenize{notation:term-md}]{\sphinxtermref{\sphinxcode{\sphinxupquote{MD}}}}}
&
Options 1 Falk and Niemczynowicz (1978) {[}32{]} 2 Halldin et al. (1979) {[}33{]} (Rutter eqn corrected for c=0, see Calder \& Wright (1986) {[}34{]} ) Recommended {[}3{]} for EveTr, DecTr, Grass (unirrigated) 3 Falk and Niemczynowicz (1978) {[}32{]} Recommended {[}3{]} for Grass (irrigated) Coefficients are specified in the following two columns.
\\
\hline
10
&
\sphinxcode{\sphinxupquote{DrainageCoef1}}
&
{\hyperref[\detokenize{notation:term-md}]{\sphinxtermref{\sphinxcode{\sphinxupquote{MD}}}}}
&
Example values DrainageEq 10 Coefficient D0 {[}mm h -1 {]} 3 Recommended {[}3{]} for Grass (irrigated) 0.013 Coefficient D0 {[}mm h -1 {]} 2 Recommended {[}3{]} for EveTr, DecTr, Grass (unirrigated)
\\
\hline
11
&
\sphinxcode{\sphinxupquote{DrainageCoef2}}
&
{\hyperref[\detokenize{notation:term-md}]{\sphinxtermref{\sphinxcode{\sphinxupquote{MD}}}}}
&
Example values DrainageEq 3 Coefficient b {[}-{]} 3 Recommended {[}3{]} for Grass (irrigated) 1.71 Coefficient b {[}mm -1 {]} 2 Recommended {[}3{]} for EveTr, DecTr, Grass (unirrigated)
\\
\hline
12
&
\sphinxcode{\sphinxupquote{SoilTypeCode}}
&
{\hyperref[\detokenize{notation:term-19}]{\sphinxtermref{\sphinxcode{\sphinxupquote{L}}}}}
&
Code for soil characteristics below this surface Provides the link to column 1 of SUEWS\_Soil.txt , which contains the attributes describing sub-surface soil for this surface type. Value of integer is arbitrary but must match code specified in column 1 of SUEWS\_Soil.txt.
\\
\hline
13
&
\sphinxcode{\sphinxupquote{SnowLimPatch}}
&
{\hyperref[\detokenize{notation:term-o}]{\sphinxtermref{\sphinxcode{\sphinxupquote{O}}}}}
&
Limit of snow water equivalent when the surface surface is fully covered with snow. Not needed if SnowUse = 0 in RunControl.nml . Example values {[}mm{]} 190 EveTr Järvi et al. (2014) {[}15{]}  190 DecTr Järvi et al. (2014) {[}15{]}  190 Grass Järvi et al. (2014) {[}15{]}
\\
\hline
14
&
\sphinxcode{\sphinxupquote{BaseT}}
&
{\hyperref[\detokenize{notation:term-mu}]{\sphinxtermref{\sphinxcode{\sphinxupquote{MU}}}}}
&
See section 2.2 Järvi et al. (2011); Appendix A Järvi et al. (2014). Example values {[}°C{]} 5 EveTr Järvi et al. (2011) {[}1{]}  5 DecTr Järvi et al. (2011) {[}1{]}  5 Grass Järvi et al. (2011) {[}1{]}
\\
\hline
15
&
\sphinxcode{\sphinxupquote{BaseTe}}
&
{\hyperref[\detokenize{notation:term-mu}]{\sphinxtermref{\sphinxcode{\sphinxupquote{MU}}}}}
&
See section 2.2 Järvi et al. (2011) {[}1{]} ; Appendix A Järvi et al. (2014) {[}15{]} . Example values {[}°C{]} 10 EveTr Järvi et al. (2011) {[}1{]}  10 DecTr Järvi et al. (2011) {[}1{]}  10 Grass Järvi et al. (2011) {[}1{]}
\\
\hline
16
&
\sphinxcode{\sphinxupquote{GDDFull}}
&
{\hyperref[\detokenize{notation:term-mu}]{\sphinxtermref{\sphinxcode{\sphinxupquote{MU}}}}}
&
This should be checked carefully for your study area using modelled LAI from the DailyState output file compared to known behaviour in the study area. See section 2.2 Järvi et al. (2011) {[}1{]} ; Appendix A Järvi et al. (2014) {[}15{]} for more details. Example values {[}°C{]} 300 EveTr Järvi et al. (2011) {[}1{]}  300 DecTr Järvi et al. (2011) {[}1{]}  300 Grass Järvi et al. (2011) {[}1{]}
\\
\hline
17
&
\sphinxcode{\sphinxupquote{SDDFull}}
&
{\hyperref[\detokenize{notation:term-mu}]{\sphinxtermref{\sphinxcode{\sphinxupquote{MU}}}}}
&
This should be checked carefully for your study area using modelled LAI from the DailyState output file compared to known behaviour in the study area. See section 2.2 Järvi et al. (2011) {[}1{]} ; Appendix A Järvi et al. (2014) {[}15{]} for more details. Example values {[}°C{]} -450 EveTr Järvi et al. (2011) {[}1{]}  -450 DecTr Järvi et al. (2011) {[}1{]}  -450 Grass Järvi et al. (2011) {[}1{]}
\\
\hline
18
&
\sphinxcode{\sphinxupquote{LAIMin}}
&
{\hyperref[\detokenize{notation:term-md}]{\sphinxtermref{\sphinxcode{\sphinxupquote{MD}}}}}
&
leaf-off wintertime value Example values {[}m -2 m -2 {]} 4. EveTr Järvi et al. (2011) {[}1{]}  1. DecTr Järvi et al. (2011) {[}1{]}  1.6 Grass Grimmond and Oke (1991) {[}3{]} and references therein
\\
\hline
19
&
\sphinxcode{\sphinxupquote{LAIMax}}
&
{\hyperref[\detokenize{notation:term-md}]{\sphinxtermref{\sphinxcode{\sphinxupquote{MD}}}}}
&
full leaf-on summertime value Example values {[}m -2 m -2 {]} 5.1 EveTr Breuer et al. (2003) {[}36{]}  5.5 DecTr Breuer et al. (2003) {[}36{]}  5.9 Grass Breuer et al. (2003) {[}36{]}
\\
\hline
20
&
\sphinxcode{\sphinxupquote{PorosityMin}}
&
{\hyperref[\detokenize{notation:term-md}]{\sphinxtermref{\sphinxcode{\sphinxupquote{MD}}}}}
&
leaf-off wintertime value Used only for DecTr (can affect roughness calculation)
\\
\hline
21
&
\sphinxcode{\sphinxupquote{PorosityMax}}
&
{\hyperref[\detokenize{notation:term-md}]{\sphinxtermref{\sphinxcode{\sphinxupquote{MD}}}}}
&
full leaf-on summertime value Used only for DecTr (can affect roughness calculation)
\\
\hline
22
&
\sphinxcode{\sphinxupquote{MaxConductance}}
&
{\hyperref[\detokenize{notation:term-md}]{\sphinxtermref{\sphinxcode{\sphinxupquote{MD}}}}}
&
Example values {[}mm s -1 {]} 7.4 EveTr Järvi et al. (2011) {[}1{]}  11.7 DecTr Järvi et al. (2011) {[}1{]}  33.1 Grass (unirrigated) Järvi et al. (2011) {[}1{]}  40. Grass (irrigated) Järvi et al. (2011) {[}1{]}
\\
\hline
23
&
\sphinxcode{\sphinxupquote{LAIEq}}
&
{\hyperref[\detokenize{notation:term-md}]{\sphinxtermref{\sphinxcode{\sphinxupquote{MD}}}}}
&
Options 0 Järvi et al. (2011) {[}1{]}  1 Järvi et al. (2014) {[}15{]}  Coefficients are specified in the following four columns. N.B. North and South hemispheres are treated slightly differently.
\\
\hline
24
&
\sphinxcode{\sphinxupquote{LeafGrowthPower1}}
&
{\hyperref[\detokenize{notation:term-md}]{\sphinxtermref{\sphinxcode{\sphinxupquote{MD}}}}}
&
Example values LAIEq 0.03 Järvi et al. (2011) {[}1{]} 0 0.04 Järvi et al. (2014) {[}15{]} 1
\\
\hline
25
&
\sphinxcode{\sphinxupquote{LeafGrowthPower2}}
&
{\hyperref[\detokenize{notation:term-md}]{\sphinxtermref{\sphinxcode{\sphinxupquote{MD}}}}}
&
Example values {[}°C -1 {]} LAIEq 0.0005 Järvi et al. (2011) {[}1{]} 0 0.001 Järvi et al. (2014) {[}15{]} 1
\\
\hline
26
&
\sphinxcode{\sphinxupquote{LeafOffPower1}}
&
{\hyperref[\detokenize{notation:term-md}]{\sphinxtermref{\sphinxcode{\sphinxupquote{MD}}}}}
&
Example values LAIEq 0.03 Järvi et al. (2011) {[}1{]} 0 -1.5 Järvi et al. (2014) {[}15{]} 1
\\
\hline
27
&
\sphinxcode{\sphinxupquote{LeafOffPower2}}
&
{\hyperref[\detokenize{notation:term-md}]{\sphinxtermref{\sphinxcode{\sphinxupquote{MD}}}}}
&
Example values {[}°C -1 {]} LAIEq 0.0005 Järvi et al. (2011) {[}1{]} 0 0.0015 Järvi et al. (2014) {[}15{]} 1
\\
\hline
28
&
\sphinxcode{\sphinxupquote{OHMCode\_SummerWet}}
&
{\hyperref[\detokenize{notation:term-19}]{\sphinxtermref{\sphinxcode{\sphinxupquote{L}}}}}
&
Code for OHM coefficients to use for this surface during wet conditions in summer. Links to SUEWS\_OHMCoefficients.txt . Value of integer is arbitrary but must match code specified in column 1 of SUEWS\_OHMCoefficients.txt.
\\
\hline
29
&
\sphinxcode{\sphinxupquote{OHMCode\_SummerDry}}
&
{\hyperref[\detokenize{notation:term-19}]{\sphinxtermref{\sphinxcode{\sphinxupquote{L}}}}}
&
Code for OHM coefficients to use for this surface during dry conditions in summer. Links to SUEWS\_OHMCoefficients.txt . Value of integer is arbitrary but must match code specified in column 1 of SUEWS\_OHMCoefficients.txt.
\\
\hline
30
&
\sphinxcode{\sphinxupquote{OHMCode\_WinterWet}}
&
{\hyperref[\detokenize{notation:term-19}]{\sphinxtermref{\sphinxcode{\sphinxupquote{L}}}}}
&
Code for OHM coefficients to use for this surface during wet conditions in winter. Links to SUEWS\_OHMCoefficients.txt . Value of integer is arbitrary but must match code specified in column 1 of SUEWS\_OHMCoefficients.txt.
\\
\hline
31
&
\sphinxcode{\sphinxupquote{OHMCode\_WinterDry}}
&
{\hyperref[\detokenize{notation:term-19}]{\sphinxtermref{\sphinxcode{\sphinxupquote{L}}}}}
&
Code for OHM coefficients to use for this surface during dry conditions in winter. Links to SUEWS\_OHMCoefficients.txt . Value of integer is arbitrary but must match code specified in column 1 of SUEWS\_OHMCoefficients.txt.
\\
\hline
32
&
\sphinxcode{\sphinxupquote{OHMThresh\_SW}}
&
{\hyperref[\detokenize{notation:term-md}]{\sphinxtermref{\sphinxcode{\sphinxupquote{MD}}}}}
&
Temperature threshold determining whether summer/winter OHM coefficients are applied {[}deg C{]} If 5-day running mean air temperature is greater than or equal to this threshold, OHM coefficients for summertime are applied; otherwise coefficients for wintertime are applied.
\\
\hline
33
&
\sphinxcode{\sphinxupquote{OHMThresh\_WD}}
&
{\hyperref[\detokenize{notation:term-md}]{\sphinxtermref{\sphinxcode{\sphinxupquote{MD}}}}}
&
Soil moisture threshold determining whether wet/dry OHM coefficients are applied {[}-{]} If soil moisture (as a proportion of maximum soil moisture capacity) exceeds this threshold for bare soil and vegetated surfaces, OHM coefficients for wet conditions are applied; otherwise coefficients for dry coefficients are applied. Note that OHM coefficients for wet conditions are applied if the surface is wet.
\\
\hline
34
&
\sphinxcode{\sphinxupquote{ESTMCode}}
&
{\hyperref[\detokenize{notation:term-19}]{\sphinxtermref{\sphinxcode{\sphinxupquote{L}}}}}
&
Code for ESTM coefficients to use for this surface. Links to SUEWS\_ESTMCoefficients.txt . Value of integer is arbitrary but must match code specified in column 1 of SUEWS\_ESTMCoefficients.txt.
\\
\hline
35
&
\sphinxcode{\sphinxupquote{AnOHM\_Cp}}
&
{\hyperref[\detokenize{notation:term-mu}]{\sphinxtermref{\sphinxcode{\sphinxupquote{MU}}}}}
&
Volumetric heat capacity for this surface to use in AnOHM {[}J m -3 {]}
\\
\hline
36
&
\sphinxcode{\sphinxupquote{AnOHM\_Kk}}
&
{\hyperref[\detokenize{notation:term-mu}]{\sphinxtermref{\sphinxcode{\sphinxupquote{MU}}}}}
&
Thermal conductivity for this surface to use in AnOHM {[}W m K -1 {]}
\\
\hline
37
&
\sphinxcode{\sphinxupquote{AnOHM\_Ch}}
&
{\hyperref[\detokenize{notation:term-mu}]{\sphinxtermref{\sphinxcode{\sphinxupquote{MU}}}}}
&
Bulk transfer coefficient for this surface to use in AnOHM {[}-{]}
\\
\hline
\end{longtable}\sphinxatlongtableend\end{savenotes}


\subsection{SUEWS\_Water.txt}
\label{\detokenize{input_files/SUEWS_SiteInfo/SUEWS_Water:suews-water-txt}}\label{\detokenize{input_files/SUEWS_SiteInfo/SUEWS_Water::doc}}\label{\detokenize{input_files/SUEWS_SiteInfo/SUEWS_Water:id1}}
SUEWS\_Water.txt specifies the characteristics for the water surface
cover type by linking codes in column 1 of SUEWS\_Water.txt to the codes
specified in SUEWS\_SiteSelect.txt (Code\_Water).


\begin{savenotes}\sphinxattablestart
\centering
\begin{tabulary}{\linewidth}[t]{|T|T|T|T|}
\hline
\sphinxstyletheadfamily 
No.
&\sphinxstyletheadfamily 
Column Name
&\sphinxstyletheadfamily 
Use
&\sphinxstyletheadfamily 
Description
\\
\hline
1
&
\sphinxcode{\sphinxupquote{Code}}
&
{\hyperref[\detokenize{notation:term-19}]{\sphinxtermref{\sphinxcode{\sphinxupquote{L}}}}}
&
Code linking to SUEWS\_SiteSelect.txt for water surfaces (Code\_Water). Value of integer is arbitrary but must match code specified in SUEWS\_SiteSelect.txt.
\\
\hline
2
&
\sphinxcode{\sphinxupquote{AlbedoMin}}
&
{\hyperref[\detokenize{notation:term-mu}]{\sphinxtermref{\sphinxcode{\sphinxupquote{MU}}}}}
&
View factors should be taken into account. Not currently used for water surface - set same as AlbedoMax.
\\
\hline
3
&
\sphinxcode{\sphinxupquote{AlbedoMax}}
&
{\hyperref[\detokenize{notation:term-mu}]{\sphinxtermref{\sphinxcode{\sphinxupquote{MU}}}}}
&
Effective albedo of the water surface. View factors should be taken into account. Example values {[}-{]} 0.1 Water Oke (1987) {[}35{]}
\\
\hline
4
&
\sphinxcode{\sphinxupquote{Emissivity}}
&
{\hyperref[\detokenize{notation:term-mu}]{\sphinxtermref{\sphinxcode{\sphinxupquote{MU}}}}}
&
Effective surface emissivity. View factors should be taken into account Example values {[}-{]} 0.95 Water Oke (1987) {[}35{]}
\\
\hline
5
&
\sphinxcode{\sphinxupquote{StorageMin}}
&
{\hyperref[\detokenize{notation:term-md}]{\sphinxtermref{\sphinxcode{\sphinxupquote{MD}}}}}
&
Minimum water storage capacity for upper surfaces (i.e. canopy). Min/max values are to account for seasonal variation - not used for water surfaces. Example values {[}mm{]} 0.5 Water
\\
\hline
6
&
\sphinxcode{\sphinxupquote{StorageMax}}
&
{\hyperref[\detokenize{notation:term-md}]{\sphinxtermref{\sphinxcode{\sphinxupquote{MD}}}}}
&
Maximum water storage capacity for upper surfaces (i.e. canopy) Min and max values are to account for seasonal variation - not used for water surfaces so set same as StorageMin.
\\
\hline
7
&
\sphinxcode{\sphinxupquote{WetThreshold}}
&
{\hyperref[\detokenize{notation:term-md}]{\sphinxtermref{\sphinxcode{\sphinxupquote{MD}}}}}
&
Depth of water which determines whether evaporation occurs from a partially wet or completely wet surface. Example values {[}mm{]} 0.5 Water
\\
\hline
8
&
\sphinxcode{\sphinxupquote{StateLimit}}
&
{\hyperref[\detokenize{notation:term-mu}]{\sphinxtermref{\sphinxcode{\sphinxupquote{MU}}}}}
&
Surface state cannot exceed this value. Set to a large value (e.g. 20000 mm = 20 m) if the water body is substantial (lake, river, etc) or a small value (e.g. 10 mm) if water bodies are very shallow (e.g. fountains). WaterDepth (column 9) must not exceed this value.
\\
\hline
9
&
\sphinxcode{\sphinxupquote{WaterDepth}}
&
{\hyperref[\detokenize{notation:term-mu}]{\sphinxtermref{\sphinxcode{\sphinxupquote{MU}}}}}
&
Set to a large value (e.g. 20000 mm = 20 m) if the water body is substantial (lake, river, etc) or a small value (e.g. 10 mm) if water bodies are very shallow (e.g. fountains). This value must not exceed StateLimit (column 8).
\\
\hline
10
&
\sphinxcode{\sphinxupquote{DrainageEq}}
&
{\hyperref[\detokenize{notation:term-md}]{\sphinxtermref{\sphinxcode{\sphinxupquote{MD}}}}}
&
Not currently used for water surface.
\\
\hline
11
&
\sphinxcode{\sphinxupquote{DrainageCoef1}}
&
{\hyperref[\detokenize{notation:term-md}]{\sphinxtermref{\sphinxcode{\sphinxupquote{MD}}}}}
&
Not currently used for water surface
\\
\hline
12
&
\sphinxcode{\sphinxupquote{DrainageCoef2}}
&
{\hyperref[\detokenize{notation:term-md}]{\sphinxtermref{\sphinxcode{\sphinxupquote{MD}}}}}
&
Not currently used for water surface
\\
\hline
13
&
\sphinxcode{\sphinxupquote{OHMCode\_SummerWet}}
&
{\hyperref[\detokenize{notation:term-19}]{\sphinxtermref{\sphinxcode{\sphinxupquote{L}}}}}
&
Code for OHM coefficients to use for this surface during wet conditions in summer. Links to SUEWS\_OHMCoefficients.txt . Value of integer is arbitrary but must match code specified in column 1 of SUEWS\_OHMCoefficients.txt.
\\
\hline
14
&
\sphinxcode{\sphinxupquote{OHMCode\_SummerDry}}
&
{\hyperref[\detokenize{notation:term-19}]{\sphinxtermref{\sphinxcode{\sphinxupquote{L}}}}}
&
Code for OHM coefficients to use for this surface during dry conditions in summer. Links to SUEWS\_OHMCoefficients.txt . Value of integer is arbitrary but must match code specified in column 1 of SUEWS\_OHMCoefficients.txt.
\\
\hline
15
&
\sphinxcode{\sphinxupquote{OHMCode\_WinterWet}}
&
{\hyperref[\detokenize{notation:term-19}]{\sphinxtermref{\sphinxcode{\sphinxupquote{L}}}}}
&
Code for OHM coefficients to use for this surface during wet conditions in winter. Links to SUEWS\_OHMCoefficients.txt . Value of integer is arbitrary but must match code specified in column 1 of SUEWS\_OHMCoefficients.txt.
\\
\hline
16
&
\sphinxcode{\sphinxupquote{OHMCode\_WinterDry}}
&
{\hyperref[\detokenize{notation:term-19}]{\sphinxtermref{\sphinxcode{\sphinxupquote{L}}}}}
&
Code for OHM coefficients to use for this surface during dry conditions in winter. Links to SUEWS\_OHMCoefficients.txt . Value of integer is arbitrary but must match code specified in column 1 of SUEWS\_OHMCoefficients.txt.
\\
\hline
17
&
\sphinxcode{\sphinxupquote{OHMThresh\_SW}}
&
{\hyperref[\detokenize{notation:term-md}]{\sphinxtermref{\sphinxcode{\sphinxupquote{MD}}}}}
&
Temperature threshold determining whether summer/winter OHM coefficients are applied {[}deg C{]} If 5-day running mean air temperature is greater than or equal to this threshold, OHM coefficients for summertime are applied; otherwise coefficients for wintertime are applied.
\\
\hline
18
&
\sphinxcode{\sphinxupquote{OHMThresh\_WD}}
&
{\hyperref[\detokenize{notation:term-md}]{\sphinxtermref{\sphinxcode{\sphinxupquote{MD}}}}}
&
Soil moisture threshold determining whether wet/dry OHM coefficients are applied {[}-{]} If soil moisture (as a proportion of maximum soil moisture capacity) exceeds this threshold for bare soil and vegetated surfaces, OHM coefficients for wet conditions are applied; otherwise coefficients for dry coefficients are applied. Note that OHM coefficients for wet conditions are applied if the surface is wet. Not actually used for water surface (as no soil surface beneath).
\\
\hline
19
&
\sphinxcode{\sphinxupquote{ESTMCode}}
&
{\hyperref[\detokenize{notation:term-19}]{\sphinxtermref{\sphinxcode{\sphinxupquote{L}}}}}
&
Code for ESTM coefficients to use for this surface. Links to SUEWS\_ESTMCoefficients.txt . Value of integer is arbitrary but must match code specified in column 1 of SUEWS\_ESTMCoefficients.txt.
\\
\hline
20
&
\sphinxcode{\sphinxupquote{AnOHM\_Cp}}
&
{\hyperref[\detokenize{notation:term-mu}]{\sphinxtermref{\sphinxcode{\sphinxupquote{MU}}}}}
&
Volumetric heat capacity for this surface to use in AnOHM {[}J m -3 {]}
\\
\hline
21
&
\sphinxcode{\sphinxupquote{AnOHM\_Kk}}
&
{\hyperref[\detokenize{notation:term-mu}]{\sphinxtermref{\sphinxcode{\sphinxupquote{MU}}}}}
&
Thermal conductivity for this surface to use in AnOHM {[}W m K -1 {]}
\\
\hline
22
&
\sphinxcode{\sphinxupquote{AnOHM\_Ch}}
&
{\hyperref[\detokenize{notation:term-mu}]{\sphinxtermref{\sphinxcode{\sphinxupquote{MU}}}}}
&
Bulk transfer coefficient for this surface to use in AnOHM {[}-{]}
\\
\hline
\end{tabulary}
\par
\sphinxattableend\end{savenotes}


\subsection{SUEWS\_WithinGridWaterDist.txt}
\label{\detokenize{input_files/SUEWS_SiteInfo/SUEWS_WithinGridWaterDist:suews-withingridwaterdist-txt}}\label{\detokenize{input_files/SUEWS_SiteInfo/SUEWS_WithinGridWaterDist::doc}}\label{\detokenize{input_files/SUEWS_SiteInfo/SUEWS_WithinGridWaterDist:id1}}
SUEWS\_WithinGridWaterDist.txt specifies the movement of water between
surfaces within a grid/area. It allows impervious connectivity to be
taken into account.

Each row corresponds to a surface type (linked by the Code in column 1
to the {\hyperref[\detokenize{input_files/SUEWS_SiteInfo/SUEWS_SiteSelect:suews-siteselect-txt}]{\sphinxcrossref{\DUrole{std,std-ref,std,std-ref}{SUEWS\_SiteSelect.txt}}}} (\autopageref*{\detokenize{input_files/SUEWS_SiteInfo/SUEWS_SiteSelect:suews-siteselect-txt}}) columns:
WithinGridPavedCode, WithinGridBldgsCode, …, WithinGridWaterCode). Each
column contains the fraction of water flowing from the surface type to
each of the other surface types or to runoff or the sub-surface soil
store.

Note:
\begin{itemize}
\item {} 
The sum of each row (excluding the Code) must equal 1.

\item {} 
Water cannot flow from one surface to that same surface, so the
diagonal elements should be zero.

\item {} 
The row corresponding to the water surface should be zero, as there
is currently no flow permitted from the water surface to other
surfaces by the model.

\item {} 
Currently water \sphinxstylestrong{cannot} go to both runoff and soil store (i.e. it
must go to one or the other \textendash{} runoff for impervious surfaces;
soilstore for pervious surfaces).

\end{itemize}

In the table below, for example,
\begin{itemize}
\item {} 
all flow from paved surfaces goes to runoff;

\item {} 
90\% of flow from buildings goes to runoff, with small amounts going
to other surfaces (mostly paved surfaces as buildings are often
surrounded by paved areas);

\item {} 
all flow from vegetated and bare soil areas goes into the sub-surface
soil store;

\item {} 
the row corresponding to water contains zeros (as it is currently not
used).

\end{itemize}


\begin{savenotes}\sphinxattablestart
\centering
\begin{tabulary}{\linewidth}[t]{|T|T|T|T|}
\hline
\sphinxstyletheadfamily 
No.
&\sphinxstyletheadfamily 
Column Name
&\sphinxstyletheadfamily 
Use
&\sphinxstyletheadfamily 
Description
\\
\hline
1
&
\sphinxcode{\sphinxupquote{ToPaved}}
&
{\hyperref[\detokenize{notation:term-mu}]{\sphinxtermref{\sphinxcode{\sphinxupquote{MU}}}}}
&
Fraction of water going to {\hyperref[\detokenize{notation:term-paved}]{\sphinxtermref{\sphinxcode{\sphinxupquote{Paved}}}}}
\\
\hline
2
&
\sphinxcode{\sphinxupquote{ToBldgs}}
&
{\hyperref[\detokenize{notation:term-mu}]{\sphinxtermref{\sphinxcode{\sphinxupquote{MU}}}}}
&
Fraction of water going to {\hyperref[\detokenize{notation:term-bldgs}]{\sphinxtermref{\sphinxcode{\sphinxupquote{Bldgs}}}}}
\\
\hline
3
&
\sphinxcode{\sphinxupquote{ToEveTr}}
&
{\hyperref[\detokenize{notation:term-mu}]{\sphinxtermref{\sphinxcode{\sphinxupquote{MU}}}}}
&
Fraction of water going to {\hyperref[\detokenize{notation:term-evetr}]{\sphinxtermref{\sphinxcode{\sphinxupquote{EveTr}}}}}
\\
\hline
4
&
\sphinxcode{\sphinxupquote{ToDecTr}}
&
{\hyperref[\detokenize{notation:term-mu}]{\sphinxtermref{\sphinxcode{\sphinxupquote{MU}}}}}
&
Fraction of water going to {\hyperref[\detokenize{notation:term-dectr}]{\sphinxtermref{\sphinxcode{\sphinxupquote{DecTr}}}}}
\\
\hline
5
&
\sphinxcode{\sphinxupquote{ToGrass}}
&
{\hyperref[\detokenize{notation:term-mu}]{\sphinxtermref{\sphinxcode{\sphinxupquote{MU}}}}}
&
Fraction of water going to {\hyperref[\detokenize{notation:term-grass}]{\sphinxtermref{\sphinxcode{\sphinxupquote{Grass}}}}}
\\
\hline
6
&
\sphinxcode{\sphinxupquote{ToBSoil}}
&
{\hyperref[\detokenize{notation:term-mu}]{\sphinxtermref{\sphinxcode{\sphinxupquote{MU}}}}}
&
Fraction of water going to {\hyperref[\detokenize{notation:term-bsoil}]{\sphinxtermref{\sphinxcode{\sphinxupquote{BSoil}}}}}
\\
\hline
7
&
\sphinxcode{\sphinxupquote{ToWater}}
&
{\hyperref[\detokenize{notation:term-mu}]{\sphinxtermref{\sphinxcode{\sphinxupquote{MU}}}}}
&
Fraction of water going to {\hyperref[\detokenize{notation:term-water}]{\sphinxtermref{\sphinxcode{\sphinxupquote{Water}}}}}
\\
\hline
8
&
\sphinxcode{\sphinxupquote{ToRunoff}}
&
{\hyperref[\detokenize{notation:term-mu}]{\sphinxtermref{\sphinxcode{\sphinxupquote{MU}}}}}
&
Fraction of water going to \sphinxcode{\sphinxupquote{Runoff}}
\\
\hline
9
&
\sphinxcode{\sphinxupquote{ToSoilStore}}
&
{\hyperref[\detokenize{notation:term-mu}]{\sphinxtermref{\sphinxcode{\sphinxupquote{MU}}}}}
&
Fraction of water going to \sphinxcode{\sphinxupquote{SoilStore}}
\\
\hline
\end{tabulary}
\par
\sphinxattableend\end{savenotes}

These text files are stored as worksheets in
\sphinxstylestrong{SUEWS\_SiteInfo.xlsm} and can be either edited using Excel and then
generated using the macro, or edited directly (see {\hyperref[\detokenize{input_files/SUEWS_SiteInfo/SUEWS_SiteInfo:Data_Entry}]{\emph{Data
Entry}}} (\autopageref*{\detokenize{input_files/SUEWS_SiteInfo/SUEWS_SiteInfo:Data_Entry}})). Please note this file is subject to possible
changes from version to version due to new features, modifications, etc.
Please be aware of using the correct copy of this worksheet that are
always shipped with the SUEWS public release.


\begin{savenotes}\sphinxattablestart
\centering
\begin{tabulary}{\linewidth}[t]{|T|T|}
\hline
\sphinxstyletheadfamily 
Use
&\sphinxstyletheadfamily 
Column
\\
\hline
MU
&
Parameters which must be supplied
and must be specific for the
site/grid being run.
\\
\hline
MD
&
Parameters which must be supplied
and must be specific for the
site/grid being run (but default
values may be ok if these values
are not known specifically for
the site).
\\
\hline
O
&
Parameters that are optional,
depending on the model settings
in RunControl. Set any parameters
that are not used/not known to
‘-999’.
\\
\hline
L
&
Codes that are used to link
between the input files. These
codes are required but their
values are completely arbitrary,
providing that they link the
input files in the correct way.
The user should choose these
codes, bearing in mind that the
codes they match up with in
column 1 of the corresponding
input file must be unique within
that file. Codes must be
integers. Note that the codes
must match up with column 1 of
the corresponding input file,
even if those parameters are not
used (in which case set all
columns except column 1 to ‘-999’
in the corresponding input file),
otherwise the model run will
fail.
\\
\hline
\end{tabulary}
\par
\sphinxattableend\end{savenotes}


\section{Initial Conditions file}
\label{\detokenize{input_files/Initial_Conditions/Initial_Conditions::doc}}\label{\detokenize{input_files/Initial_Conditions/Initial_Conditions:initial-conditions}}\label{\detokenize{input_files/Initial_Conditions/Initial_Conditions:initial-conditions-file}}
To start the model, information about the conditions at the start of the
run is required. This information is provided in initial conditions
file. One file can be specified for each grid
({\hyperref[\detokenize{input_files/RunControl/File_related_options:cmdoption-arg-multipleinitfiles}]{\sphinxcrossref{\sphinxcode{\sphinxupquote{MultipleInitFiles=1}}}}} (\autopageref*{\detokenize{input_files/RunControl/File_related_options:cmdoption-arg-multipleinitfiles}}) in
{\hyperref[\detokenize{input_files/RunControl/RunControl:runcontrol-nml}]{\sphinxcrossref{\DUrole{std,std-ref}{RunControl.nml}}}} (\autopageref*{\detokenize{input_files/RunControl/RunControl:runcontrol-nml}}), filename includes grid number) or,
alternatively, a single file can be specified for all grids
(MultipleInitFiles=0 in {\hyperref[\detokenize{input_files/RunControl/RunControl:runcontrol-nml}]{\sphinxcrossref{\DUrole{std,std-ref}{RunControl.nml}}}} (\autopageref*{\detokenize{input_files/RunControl/RunControl:runcontrol-nml}}), no grid
number in the filename). After that, a new
InitialConditionsSSss\_YYYY.nml file will be written for each grid for
the following years. It is recommended that you look at these files
(written to the input directory) to check the status of various surfaces
at the end or the run. This may help you get more realistic starting
values if you are uncertain what they should be. Note this file will be
created for each year for multiyear runs for each grid. If the run
finishes before the end of the year the InitialConditions file is still
written and the file name is appended with ‘\_EndofRun’.

A sample file of \sphinxstylestrong{InitialConditionsSSss\_YYYY.nml} looks like

\fvset{hllines={, ,}}%
\begin{sphinxVerbatim}[commandchars=\\\{\}]
\PYG{o}{\PYGZam{}}\PYG{n}{InitialConditions}
\PYG{n}{LeavesOutInitially}\PYG{o}{=}\PYG{l+m+mi}{0}
\PYG{n}{SoilstorePavedState}\PYG{o}{=}\PYG{l+m+mi}{150}
\PYG{n}{SoilstoreBldgsState}\PYG{o}{=}\PYG{l+m+mi}{150}
\PYG{n}{SoilstoreEveTrstate}\PYG{o}{=}\PYG{l+m+mi}{150}
\PYG{n}{SoilstoreDecTrState}\PYG{o}{=}\PYG{l+m+mi}{150}
\PYG{n}{SoilstoreGrassState}\PYG{o}{=}\PYG{l+m+mi}{150}
\PYG{n}{SoilstoreBSoilState}\PYG{o}{=}\PYG{l+m+mi}{150}
\PYG{n}{BoInit}\PYG{o}{=}\PYG{l+m+mi}{10}
\PYG{o}{/}
\end{sphinxVerbatim}

The two most important pieces of information in the initial conditions
file is the soil moisture and state of vegetation at the start of the
run. This is the minimal information required; other information can be
provided if known, otherwise SUEWS will make an estimate of initial
conditions.

The parameters and their setting instructions are provided through the links below:

\begin{sphinxadmonition}{note}{Note:}
Variables can be in any order
\end{sphinxadmonition}
\begin{itemize}
\item {} 
{\hyperref[\detokenize{input_files/Initial_Conditions/Soil_moisture_states:soil-moisture-states}]{\sphinxcrossref{\DUrole{std,std-ref}{Soil moisture states}}}} (\autopageref*{\detokenize{input_files/Initial_Conditions/Soil_moisture_states:soil-moisture-states}})
\begin{itemize}\setlength{\itemsep}{0pt}\setlength{\parskip}{0pt}
\item {} 
{\hyperref[\detokenize{input_files/Initial_Conditions/Soil_moisture_states:cmdoption-arg-soilstorepavedstate}]{\sphinxcrossref{\sphinxcode{\sphinxupquote{SoilstorePavedState}}}}} (\autopageref*{\detokenize{input_files/Initial_Conditions/Soil_moisture_states:cmdoption-arg-soilstorepavedstate}})

\item {} 
{\hyperref[\detokenize{input_files/Initial_Conditions/Soil_moisture_states:cmdoption-arg-soilstorebldgsstate}]{\sphinxcrossref{\sphinxcode{\sphinxupquote{SoilstoreBldgsState}}}}} (\autopageref*{\detokenize{input_files/Initial_Conditions/Soil_moisture_states:cmdoption-arg-soilstorebldgsstate}})

\item {} 
{\hyperref[\detokenize{input_files/Initial_Conditions/Soil_moisture_states:cmdoption-arg-soilstoreevetrstate}]{\sphinxcrossref{\sphinxcode{\sphinxupquote{SoilstoreEveTrState}}}}} (\autopageref*{\detokenize{input_files/Initial_Conditions/Soil_moisture_states:cmdoption-arg-soilstoreevetrstate}})

\item {} 
{\hyperref[\detokenize{input_files/Initial_Conditions/Soil_moisture_states:cmdoption-arg-soilstoredectrstate}]{\sphinxcrossref{\sphinxcode{\sphinxupquote{SoilstoreDecTrState}}}}} (\autopageref*{\detokenize{input_files/Initial_Conditions/Soil_moisture_states:cmdoption-arg-soilstoredectrstate}})

\item {} 
{\hyperref[\detokenize{input_files/Initial_Conditions/Soil_moisture_states:cmdoption-arg-soilstoregrassstate}]{\sphinxcrossref{\sphinxcode{\sphinxupquote{SoilstoreGrassState}}}}} (\autopageref*{\detokenize{input_files/Initial_Conditions/Soil_moisture_states:cmdoption-arg-soilstoregrassstate}})

\item {} 
{\hyperref[\detokenize{input_files/Initial_Conditions/Soil_moisture_states:cmdoption-arg-soilstorebsoilstate}]{\sphinxcrossref{\sphinxcode{\sphinxupquote{SoilstoreBSoilState}}}}} (\autopageref*{\detokenize{input_files/Initial_Conditions/Soil_moisture_states:cmdoption-arg-soilstorebsoilstate}})

\end{itemize}

\item {} 
{\hyperref[\detokenize{input_files/Initial_Conditions/Vegetation_parameters:vegetation-parameters}]{\sphinxcrossref{\DUrole{std,std-ref}{Vegetation parameters}}}} (\autopageref*{\detokenize{input_files/Initial_Conditions/Vegetation_parameters:vegetation-parameters}})
\begin{itemize}\setlength{\itemsep}{0pt}\setlength{\parskip}{0pt}
\item {} 
{\hyperref[\detokenize{input_files/Initial_Conditions/Vegetation_parameters:cmdoption-arg-leavesoutintially}]{\sphinxcrossref{\sphinxcode{\sphinxupquote{LeavesOutIntially}}}}} (\autopageref*{\detokenize{input_files/Initial_Conditions/Vegetation_parameters:cmdoption-arg-leavesoutintially}})

\item {} 
{\hyperref[\detokenize{input_files/Initial_Conditions/Vegetation_parameters:cmdoption-arg-gdd-1-0}]{\sphinxcrossref{\sphinxcode{\sphinxupquote{GDD\_1\_0}}}}} (\autopageref*{\detokenize{input_files/Initial_Conditions/Vegetation_parameters:cmdoption-arg-gdd-1-0}})

\item {} 
{\hyperref[\detokenize{input_files/Initial_Conditions/Vegetation_parameters:cmdoption-arg-gdd-2-0}]{\sphinxcrossref{\sphinxcode{\sphinxupquote{GDD\_2\_0}}}}} (\autopageref*{\detokenize{input_files/Initial_Conditions/Vegetation_parameters:cmdoption-arg-gdd-2-0}})

\item {} 
{\hyperref[\detokenize{input_files/Initial_Conditions/Vegetation_parameters:cmdoption-arg-laiinitialevetr}]{\sphinxcrossref{\sphinxcode{\sphinxupquote{LAIinitialEveTr}}}}} (\autopageref*{\detokenize{input_files/Initial_Conditions/Vegetation_parameters:cmdoption-arg-laiinitialevetr}})

\item {} 
{\hyperref[\detokenize{input_files/Initial_Conditions/Vegetation_parameters:cmdoption-arg-laiinitialdectr}]{\sphinxcrossref{\sphinxcode{\sphinxupquote{LAIinitialDecTr}}}}} (\autopageref*{\detokenize{input_files/Initial_Conditions/Vegetation_parameters:cmdoption-arg-laiinitialdectr}})

\item {} 
{\hyperref[\detokenize{input_files/Initial_Conditions/Vegetation_parameters:cmdoption-arg-laiinitialgrass}]{\sphinxcrossref{\sphinxcode{\sphinxupquote{LAIinitialGrass}}}}} (\autopageref*{\detokenize{input_files/Initial_Conditions/Vegetation_parameters:cmdoption-arg-laiinitialgrass}})

\item {} 
{\hyperref[\detokenize{input_files/Initial_Conditions/Vegetation_parameters:cmdoption-arg-albevetr0}]{\sphinxcrossref{\sphinxcode{\sphinxupquote{albEveTr0}}}}} (\autopageref*{\detokenize{input_files/Initial_Conditions/Vegetation_parameters:cmdoption-arg-albevetr0}})

\item {} 
{\hyperref[\detokenize{input_files/Initial_Conditions/Vegetation_parameters:cmdoption-arg-albdectr0}]{\sphinxcrossref{\sphinxcode{\sphinxupquote{albDecTr0}}}}} (\autopageref*{\detokenize{input_files/Initial_Conditions/Vegetation_parameters:cmdoption-arg-albdectr0}})

\item {} 
{\hyperref[\detokenize{input_files/Initial_Conditions/Vegetation_parameters:cmdoption-arg-albgrass0}]{\sphinxcrossref{\sphinxcode{\sphinxupquote{albGrass0}}}}} (\autopageref*{\detokenize{input_files/Initial_Conditions/Vegetation_parameters:cmdoption-arg-albgrass0}})

\item {} 
{\hyperref[\detokenize{input_files/Initial_Conditions/Vegetation_parameters:cmdoption-arg-decidcap0}]{\sphinxcrossref{\sphinxcode{\sphinxupquote{decidCap0}}}}} (\autopageref*{\detokenize{input_files/Initial_Conditions/Vegetation_parameters:cmdoption-arg-decidcap0}})

\item {} 
{\hyperref[\detokenize{input_files/Initial_Conditions/Vegetation_parameters:cmdoption-arg-porosity0}]{\sphinxcrossref{\sphinxcode{\sphinxupquote{porosity0}}}}} (\autopageref*{\detokenize{input_files/Initial_Conditions/Vegetation_parameters:cmdoption-arg-porosity0}})

\end{itemize}

\item {} 
{\hyperref[\detokenize{input_files/Initial_Conditions/Recent_meteorology:recent-meteorology}]{\sphinxcrossref{\DUrole{std,std-ref}{Recent meteorology}}}} (\autopageref*{\detokenize{input_files/Initial_Conditions/Recent_meteorology:recent-meteorology}})
\begin{itemize}\setlength{\itemsep}{0pt}\setlength{\parskip}{0pt}
\item {} 
{\hyperref[\detokenize{input_files/Initial_Conditions/Recent_meteorology:cmdoption-arg-dayssincerain}]{\sphinxcrossref{\sphinxcode{\sphinxupquote{DaysSinceRain}}}}} (\autopageref*{\detokenize{input_files/Initial_Conditions/Recent_meteorology:cmdoption-arg-dayssincerain}})

\item {} 
{\hyperref[\detokenize{input_files/Initial_Conditions/Recent_meteorology:cmdoption-arg-temp-c0}]{\sphinxcrossref{\sphinxcode{\sphinxupquote{Temp\_C0}}}}} (\autopageref*{\detokenize{input_files/Initial_Conditions/Recent_meteorology:cmdoption-arg-temp-c0}})

\end{itemize}

\item {} 
{\hyperref[\detokenize{input_files/Initial_Conditions/Above_Ground_State:above-ground-state}]{\sphinxcrossref{\DUrole{std,std-ref}{Above Ground State}}}} (\autopageref*{\detokenize{input_files/Initial_Conditions/Above_Ground_State:above-ground-state}})
\begin{itemize}\setlength{\itemsep}{0pt}\setlength{\parskip}{0pt}
\item {} 
{\hyperref[\detokenize{input_files/Initial_Conditions/Above_Ground_State:cmdoption-arg-pavedstate}]{\sphinxcrossref{\sphinxcode{\sphinxupquote{PavedState}}}}} (\autopageref*{\detokenize{input_files/Initial_Conditions/Above_Ground_State:cmdoption-arg-pavedstate}})

\item {} 
{\hyperref[\detokenize{input_files/Initial_Conditions/Above_Ground_State:cmdoption-arg-bldgsstate}]{\sphinxcrossref{\sphinxcode{\sphinxupquote{BldgsState}}}}} (\autopageref*{\detokenize{input_files/Initial_Conditions/Above_Ground_State:cmdoption-arg-bldgsstate}})

\item {} 
{\hyperref[\detokenize{input_files/Initial_Conditions/Above_Ground_State:cmdoption-arg-evetrstate}]{\sphinxcrossref{\sphinxcode{\sphinxupquote{EveTrState}}}}} (\autopageref*{\detokenize{input_files/Initial_Conditions/Above_Ground_State:cmdoption-arg-evetrstate}})

\item {} 
{\hyperref[\detokenize{input_files/Initial_Conditions/Above_Ground_State:cmdoption-arg-dectrstate}]{\sphinxcrossref{\sphinxcode{\sphinxupquote{DecTrState}}}}} (\autopageref*{\detokenize{input_files/Initial_Conditions/Above_Ground_State:cmdoption-arg-dectrstate}})

\item {} 
{\hyperref[\detokenize{input_files/Initial_Conditions/Above_Ground_State:cmdoption-arg-grassstate}]{\sphinxcrossref{\sphinxcode{\sphinxupquote{GrassState}}}}} (\autopageref*{\detokenize{input_files/Initial_Conditions/Above_Ground_State:cmdoption-arg-grassstate}})

\item {} 
{\hyperref[\detokenize{input_files/Initial_Conditions/Above_Ground_State:cmdoption-arg-bsoilstate}]{\sphinxcrossref{\sphinxcode{\sphinxupquote{BSoilState}}}}} (\autopageref*{\detokenize{input_files/Initial_Conditions/Above_Ground_State:cmdoption-arg-bsoilstate}})

\item {} 
{\hyperref[\detokenize{input_files/Initial_Conditions/Above_Ground_State:cmdoption-arg-waterstate}]{\sphinxcrossref{\sphinxcode{\sphinxupquote{WaterState}}}}} (\autopageref*{\detokenize{input_files/Initial_Conditions/Above_Ground_State:cmdoption-arg-waterstate}})

\end{itemize}

\item {} 
{\hyperref[\detokenize{input_files/Initial_Conditions/Snow_related_parameters:snow-related-parameters}]{\sphinxcrossref{\DUrole{std,std-ref}{Snow related parameters}}}} (\autopageref*{\detokenize{input_files/Initial_Conditions/Snow_related_parameters:snow-related-parameters}})
\begin{itemize}\setlength{\itemsep}{0pt}\setlength{\parskip}{0pt}
\item {} 
{\hyperref[\detokenize{input_files/Initial_Conditions/Snow_related_parameters:cmdoption-arg-snowintially}]{\sphinxcrossref{\sphinxcode{\sphinxupquote{SnowIntially}}}}} (\autopageref*{\detokenize{input_files/Initial_Conditions/Snow_related_parameters:cmdoption-arg-snowintially}})

\item {} 
{\hyperref[\detokenize{input_files/Initial_Conditions/Snow_related_parameters:cmdoption-arg-snowwaterpavedstate}]{\sphinxcrossref{\sphinxcode{\sphinxupquote{SnowWaterPavedState}}}}} (\autopageref*{\detokenize{input_files/Initial_Conditions/Snow_related_parameters:cmdoption-arg-snowwaterpavedstate}})

\item {} 
{\hyperref[\detokenize{input_files/Initial_Conditions/Snow_related_parameters:cmdoption-arg-snowwaterbldgsstate}]{\sphinxcrossref{\sphinxcode{\sphinxupquote{SnowWaterBldgsState}}}}} (\autopageref*{\detokenize{input_files/Initial_Conditions/Snow_related_parameters:cmdoption-arg-snowwaterbldgsstate}})

\item {} 
{\hyperref[\detokenize{input_files/Initial_Conditions/Snow_related_parameters:cmdoption-arg-snowwaterevetrstate}]{\sphinxcrossref{\sphinxcode{\sphinxupquote{SnowWaterEveTrState}}}}} (\autopageref*{\detokenize{input_files/Initial_Conditions/Snow_related_parameters:cmdoption-arg-snowwaterevetrstate}})

\item {} 
{\hyperref[\detokenize{input_files/Initial_Conditions/Snow_related_parameters:cmdoption-arg-snowwaterdectrstate}]{\sphinxcrossref{\sphinxcode{\sphinxupquote{SnowWaterDecTrState}}}}} (\autopageref*{\detokenize{input_files/Initial_Conditions/Snow_related_parameters:cmdoption-arg-snowwaterdectrstate}})

\item {} 
{\hyperref[\detokenize{input_files/Initial_Conditions/Snow_related_parameters:cmdoption-arg-snowwatergrassstate}]{\sphinxcrossref{\sphinxcode{\sphinxupquote{SnowWaterGrassState}}}}} (\autopageref*{\detokenize{input_files/Initial_Conditions/Snow_related_parameters:cmdoption-arg-snowwatergrassstate}})

\item {} 
{\hyperref[\detokenize{input_files/Initial_Conditions/Snow_related_parameters:cmdoption-arg-snowwaterbsoilstate}]{\sphinxcrossref{\sphinxcode{\sphinxupquote{SnowWaterBSoilState}}}}} (\autopageref*{\detokenize{input_files/Initial_Conditions/Snow_related_parameters:cmdoption-arg-snowwaterbsoilstate}})

\item {} 
{\hyperref[\detokenize{input_files/Initial_Conditions/Snow_related_parameters:cmdoption-arg-snowwaterwaterstate}]{\sphinxcrossref{\sphinxcode{\sphinxupquote{SnowWaterWaterState}}}}} (\autopageref*{\detokenize{input_files/Initial_Conditions/Snow_related_parameters:cmdoption-arg-snowwaterwaterstate}})

\item {} 
{\hyperref[\detokenize{input_files/Initial_Conditions/Snow_related_parameters:cmdoption-arg-snowpackpaved}]{\sphinxcrossref{\sphinxcode{\sphinxupquote{SnowPackPaved}}}}} (\autopageref*{\detokenize{input_files/Initial_Conditions/Snow_related_parameters:cmdoption-arg-snowpackpaved}})

\item {} 
{\hyperref[\detokenize{input_files/Initial_Conditions/Snow_related_parameters:cmdoption-arg-snowpackbldgs}]{\sphinxcrossref{\sphinxcode{\sphinxupquote{SnowPackBldgs}}}}} (\autopageref*{\detokenize{input_files/Initial_Conditions/Snow_related_parameters:cmdoption-arg-snowpackbldgs}})

\item {} 
{\hyperref[\detokenize{input_files/Initial_Conditions/Snow_related_parameters:cmdoption-arg-snowpackevetr}]{\sphinxcrossref{\sphinxcode{\sphinxupquote{SnowPackEveTr}}}}} (\autopageref*{\detokenize{input_files/Initial_Conditions/Snow_related_parameters:cmdoption-arg-snowpackevetr}})

\item {} 
{\hyperref[\detokenize{input_files/Initial_Conditions/Snow_related_parameters:cmdoption-arg-snowpackdectr}]{\sphinxcrossref{\sphinxcode{\sphinxupquote{SnowPackDecTr}}}}} (\autopageref*{\detokenize{input_files/Initial_Conditions/Snow_related_parameters:cmdoption-arg-snowpackdectr}})

\item {} 
{\hyperref[\detokenize{input_files/Initial_Conditions/Snow_related_parameters:cmdoption-arg-snowpackgrass}]{\sphinxcrossref{\sphinxcode{\sphinxupquote{SnowPackGrass}}}}} (\autopageref*{\detokenize{input_files/Initial_Conditions/Snow_related_parameters:cmdoption-arg-snowpackgrass}})

\item {} 
{\hyperref[\detokenize{input_files/Initial_Conditions/Snow_related_parameters:cmdoption-arg-snowpackbsoil}]{\sphinxcrossref{\sphinxcode{\sphinxupquote{SnowPackBSoil}}}}} (\autopageref*{\detokenize{input_files/Initial_Conditions/Snow_related_parameters:cmdoption-arg-snowpackbsoil}})

\item {} 
{\hyperref[\detokenize{input_files/Initial_Conditions/Snow_related_parameters:cmdoption-arg-snowpackwater}]{\sphinxcrossref{\sphinxcode{\sphinxupquote{SnowPackWater}}}}} (\autopageref*{\detokenize{input_files/Initial_Conditions/Snow_related_parameters:cmdoption-arg-snowpackwater}})

\item {} 
{\hyperref[\detokenize{input_files/Initial_Conditions/Snow_related_parameters:cmdoption-arg-snowfracpaved}]{\sphinxcrossref{\sphinxcode{\sphinxupquote{SnowFracPaved}}}}} (\autopageref*{\detokenize{input_files/Initial_Conditions/Snow_related_parameters:cmdoption-arg-snowfracpaved}})

\item {} 
{\hyperref[\detokenize{input_files/Initial_Conditions/Snow_related_parameters:cmdoption-arg-snowfracbldgs}]{\sphinxcrossref{\sphinxcode{\sphinxupquote{SnowFracBldgs}}}}} (\autopageref*{\detokenize{input_files/Initial_Conditions/Snow_related_parameters:cmdoption-arg-snowfracbldgs}})

\item {} 
{\hyperref[\detokenize{input_files/Initial_Conditions/Snow_related_parameters:cmdoption-arg-snowfracevetr}]{\sphinxcrossref{\sphinxcode{\sphinxupquote{SnowFracEveTr}}}}} (\autopageref*{\detokenize{input_files/Initial_Conditions/Snow_related_parameters:cmdoption-arg-snowfracevetr}})

\item {} 
{\hyperref[\detokenize{input_files/Initial_Conditions/Snow_related_parameters:cmdoption-arg-snowfracdectr}]{\sphinxcrossref{\sphinxcode{\sphinxupquote{SnowFracDecTr}}}}} (\autopageref*{\detokenize{input_files/Initial_Conditions/Snow_related_parameters:cmdoption-arg-snowfracdectr}})

\item {} 
{\hyperref[\detokenize{input_files/Initial_Conditions/Snow_related_parameters:cmdoption-arg-snowfracgras}]{\sphinxcrossref{\sphinxcode{\sphinxupquote{SnowFracGras}}}}} (\autopageref*{\detokenize{input_files/Initial_Conditions/Snow_related_parameters:cmdoption-arg-snowfracgras}})

\item {} 
{\hyperref[\detokenize{input_files/Initial_Conditions/Snow_related_parameters:cmdoption-arg-snowfracbsoil}]{\sphinxcrossref{\sphinxcode{\sphinxupquote{SnowFracBSoil}}}}} (\autopageref*{\detokenize{input_files/Initial_Conditions/Snow_related_parameters:cmdoption-arg-snowfracbsoil}})

\item {} 
{\hyperref[\detokenize{input_files/Initial_Conditions/Snow_related_parameters:cmdoption-arg-snowfracwater}]{\sphinxcrossref{\sphinxcode{\sphinxupquote{SnowFracWater}}}}} (\autopageref*{\detokenize{input_files/Initial_Conditions/Snow_related_parameters:cmdoption-arg-snowfracwater}})

\item {} 
{\hyperref[\detokenize{input_files/Initial_Conditions/Snow_related_parameters:cmdoption-arg-snowdenspaved}]{\sphinxcrossref{\sphinxcode{\sphinxupquote{SnowDensPaved}}}}} (\autopageref*{\detokenize{input_files/Initial_Conditions/Snow_related_parameters:cmdoption-arg-snowdenspaved}})

\item {} 
{\hyperref[\detokenize{input_files/Initial_Conditions/Snow_related_parameters:cmdoption-arg-snowdensbldgs}]{\sphinxcrossref{\sphinxcode{\sphinxupquote{SnowDensBldgs}}}}} (\autopageref*{\detokenize{input_files/Initial_Conditions/Snow_related_parameters:cmdoption-arg-snowdensbldgs}})

\item {} 
{\hyperref[\detokenize{input_files/Initial_Conditions/Snow_related_parameters:cmdoption-arg-snowdensevetr}]{\sphinxcrossref{\sphinxcode{\sphinxupquote{SnowDensEveTr}}}}} (\autopageref*{\detokenize{input_files/Initial_Conditions/Snow_related_parameters:cmdoption-arg-snowdensevetr}})

\item {} 
{\hyperref[\detokenize{input_files/Initial_Conditions/Snow_related_parameters:cmdoption-arg-snowdensdectr}]{\sphinxcrossref{\sphinxcode{\sphinxupquote{SnowDensDecTr}}}}} (\autopageref*{\detokenize{input_files/Initial_Conditions/Snow_related_parameters:cmdoption-arg-snowdensdectr}})

\item {} 
{\hyperref[\detokenize{input_files/Initial_Conditions/Snow_related_parameters:cmdoption-arg-snowdensgrass}]{\sphinxcrossref{\sphinxcode{\sphinxupquote{SnowDensGrass}}}}} (\autopageref*{\detokenize{input_files/Initial_Conditions/Snow_related_parameters:cmdoption-arg-snowdensgrass}})

\item {} 
{\hyperref[\detokenize{input_files/Initial_Conditions/Snow_related_parameters:cmdoption-arg-snowdensbsoil}]{\sphinxcrossref{\sphinxcode{\sphinxupquote{SnowDensBSoil}}}}} (\autopageref*{\detokenize{input_files/Initial_Conditions/Snow_related_parameters:cmdoption-arg-snowdensbsoil}})

\item {} 
{\hyperref[\detokenize{input_files/Initial_Conditions/Snow_related_parameters:cmdoption-arg-snowdenswater}]{\sphinxcrossref{\sphinxcode{\sphinxupquote{SnowDensWater}}}}} (\autopageref*{\detokenize{input_files/Initial_Conditions/Snow_related_parameters:cmdoption-arg-snowdenswater}})

\end{itemize}

\end{itemize}


\subsection{Soil moisture states}
\label{\detokenize{input_files/Initial_Conditions/Soil_moisture_states:soil-moisture-states}}\label{\detokenize{input_files/Initial_Conditions/Soil_moisture_states::doc}}\label{\detokenize{input_files/Initial_Conditions/Soil_moisture_states:id1}}\index{command line option!SoilstorePavedState}\index{SoilstorePavedState!command line option}

\begin{fulllineitems}
\phantomsection\label{\detokenize{input_files/Initial_Conditions/Soil_moisture_states:cmdoption-arg-soilstorepavedstate}}\pysigline{\sphinxbfcode{\sphinxupquote{SoilstorePavedState}}\sphinxcode{\sphinxupquote{}}}~\begin{quote}\begin{description}
\item[{Requirement}] \leavevmode
Required

\item[{Description}] \leavevmode
For maximum values, see the used soil code in SUEWS\_Soil.txt

\item[{Configuration}] \leavevmode
to fill

\end{description}\end{quote}

\end{fulllineitems}

\index{command line option!SoilstoreBldgsState}\index{SoilstoreBldgsState!command line option}

\begin{fulllineitems}
\phantomsection\label{\detokenize{input_files/Initial_Conditions/Soil_moisture_states:cmdoption-arg-soilstorebldgsstate}}\pysigline{\sphinxbfcode{\sphinxupquote{SoilstoreBldgsState}}\sphinxcode{\sphinxupquote{}}}~\begin{quote}\begin{description}
\item[{Requirement}] \leavevmode
Required

\item[{Description}] \leavevmode
For maximum values, see the used soil code in SUEWS\_Soil.txt

\item[{Configuration}] \leavevmode
to fill

\end{description}\end{quote}

\end{fulllineitems}

\index{command line option!SoilstoreEveTrState}\index{SoilstoreEveTrState!command line option}

\begin{fulllineitems}
\phantomsection\label{\detokenize{input_files/Initial_Conditions/Soil_moisture_states:cmdoption-arg-soilstoreevetrstate}}\pysigline{\sphinxbfcode{\sphinxupquote{SoilstoreEveTrState}}\sphinxcode{\sphinxupquote{}}}~\begin{quote}\begin{description}
\item[{Requirement}] \leavevmode
Required

\item[{Description}] \leavevmode
For maximum values, see the used soil code in SUEWS\_Soil.txt

\item[{Configuration}] \leavevmode
to fill

\end{description}\end{quote}

\end{fulllineitems}

\index{command line option!SoilstoreDecTrState}\index{SoilstoreDecTrState!command line option}

\begin{fulllineitems}
\phantomsection\label{\detokenize{input_files/Initial_Conditions/Soil_moisture_states:cmdoption-arg-soilstoredectrstate}}\pysigline{\sphinxbfcode{\sphinxupquote{SoilstoreDecTrState}}\sphinxcode{\sphinxupquote{}}}~\begin{quote}\begin{description}
\item[{Requirement}] \leavevmode
Required

\item[{Description}] \leavevmode
For maximum values, see the used soil code in SUEWS\_Soil.txt

\item[{Configuration}] \leavevmode
to fill

\end{description}\end{quote}

\end{fulllineitems}

\index{command line option!SoilstoreGrassState}\index{SoilstoreGrassState!command line option}

\begin{fulllineitems}
\phantomsection\label{\detokenize{input_files/Initial_Conditions/Soil_moisture_states:cmdoption-arg-soilstoregrassstate}}\pysigline{\sphinxbfcode{\sphinxupquote{SoilstoreGrassState}}\sphinxcode{\sphinxupquote{}}}~\begin{quote}\begin{description}
\item[{Requirement}] \leavevmode
Required

\item[{Description}] \leavevmode
For maximum values, see the used soil code in SUEWS\_Soil.txt

\item[{Configuration}] \leavevmode
to fill

\end{description}\end{quote}

\end{fulllineitems}

\index{command line option!SoilstoreBSoilState}\index{SoilstoreBSoilState!command line option}

\begin{fulllineitems}
\phantomsection\label{\detokenize{input_files/Initial_Conditions/Soil_moisture_states:cmdoption-arg-soilstorebsoilstate}}\pysigline{\sphinxbfcode{\sphinxupquote{SoilstoreBSoilState}}\sphinxcode{\sphinxupquote{}}}~\begin{quote}\begin{description}
\item[{Requirement}] \leavevmode
Required

\item[{Description}] \leavevmode
For maximum values, see the used soil code in SUEWS\_Soil.txt

\item[{Configuration}] \leavevmode
to fill

\end{description}\end{quote}

\end{fulllineitems}



\subsection{Vegetation parameters}
\label{\detokenize{input_files/Initial_Conditions/Vegetation_parameters:vegetation-parameters}}\label{\detokenize{input_files/Initial_Conditions/Vegetation_parameters::doc}}\label{\detokenize{input_files/Initial_Conditions/Vegetation_parameters:id1}}\index{command line option!LeavesOutIntially}\index{LeavesOutIntially!command line option}

\begin{fulllineitems}
\phantomsection\label{\detokenize{input_files/Initial_Conditions/Vegetation_parameters:cmdoption-arg-leavesoutintially}}\pysigline{\sphinxbfcode{\sphinxupquote{LeavesOutIntially}}\sphinxcode{\sphinxupquote{}}}~\begin{quote}\begin{description}
\item[{Requirement}] \leavevmode
Optional

\item[{Description}] \leavevmode
If the model run starts in winter when trees are bare, set LeavesOutIntially = 0 and the vegetation parameters will be set accordingly based on the values set in SUEWS\_SiteInfo.xlsm. If the model run starts in summer when leaves are fully out, set LeavesOutIntially = 1 and the vegetation parameters will be set accordingly based on the values set in SUEWS\_SiteInfo.xlsm. Not LeavesOutInitially can only be set to 0, 1 or -999 (fractional values cannot be used to indicate partial leaf-out). The value of LeavesOutInitially overrides any values provided for the individual vegetation parameters. To prevent LeavesOutInitially from setting the initial conditions, either omit it from the namelist or set to -999. If values are provided individually, they should be consistent the information provided in SUEWS\_Veg.txt and the time of year. If values are provided individually, values for all required surfaces must be provided (i.e. specifying only albGrass0 but not albDecTr0 nor albEveTr0 is not permitted).

\item[{Configuration}] \leavevmode
to fill

\end{description}\end{quote}

\end{fulllineitems}

\index{command line option!GDD\_1\_0}\index{GDD\_1\_0!command line option}

\begin{fulllineitems}
\phantomsection\label{\detokenize{input_files/Initial_Conditions/Vegetation_parameters:cmdoption-arg-gdd-1-0}}\pysigline{\sphinxbfcode{\sphinxupquote{GDD\_1\_0}}\sphinxcode{\sphinxupquote{}}}~\begin{quote}\begin{description}
\item[{Requirement}] \leavevmode
Optional

\item[{Description}] \leavevmode
Cannot be negative. If leaves are already full, then this should be the same as GDDFull in SUEWS\_Veg.txt. If winter , set to 0. It is important that the vegetation characteristics are set correctly (i.e. for the start of the run in summer/winter).

\item[{Configuration}] \leavevmode
to fill

\end{description}\end{quote}

\end{fulllineitems}

\index{command line option!GDD\_2\_0}\index{GDD\_2\_0!command line option}

\begin{fulllineitems}
\phantomsection\label{\detokenize{input_files/Initial_Conditions/Vegetation_parameters:cmdoption-arg-gdd-2-0}}\pysigline{\sphinxbfcode{\sphinxupquote{GDD\_2\_0}}\sphinxcode{\sphinxupquote{}}}~\begin{quote}\begin{description}
\item[{Requirement}] \leavevmode
Optional

\item[{Description}] \leavevmode
Cannot be positive If the leaves are full but in early/mid summer then set to 0. If late summer or autumn , this should be a negative value. If leaves are off , then use the values of SDDFull in SUEWS\_Veg.txt to guide your minimum value. It is important that the vegetation characteristics are set correctly (i.e. for the start of the run in summer/winter).

\item[{Configuration}] \leavevmode
to fill

\end{description}\end{quote}

\end{fulllineitems}

\index{command line option!LAIinitialEveTr}\index{LAIinitialEveTr!command line option}

\begin{fulllineitems}
\phantomsection\label{\detokenize{input_files/Initial_Conditions/Vegetation_parameters:cmdoption-arg-laiinitialevetr}}\pysigline{\sphinxbfcode{\sphinxupquote{LAIinitialEveTr}}\sphinxcode{\sphinxupquote{}}}~\begin{quote}\begin{description}
\item[{Requirement}] \leavevmode
Optional

\item[{Description}] \leavevmode
Initial LAI for evergreen trees. The recommended values can be found from SUEWS\_Veg.txt

\item[{Configuration}] \leavevmode
to fill

\end{description}\end{quote}

\end{fulllineitems}

\index{command line option!LAIinitialDecTr}\index{LAIinitialDecTr!command line option}

\begin{fulllineitems}
\phantomsection\label{\detokenize{input_files/Initial_Conditions/Vegetation_parameters:cmdoption-arg-laiinitialdectr}}\pysigline{\sphinxbfcode{\sphinxupquote{LAIinitialDecTr}}\sphinxcode{\sphinxupquote{}}}~\begin{quote}\begin{description}
\item[{Requirement}] \leavevmode
Optional

\item[{Description}] \leavevmode
Initial LAI for deciduous trees. The recommended values can be found from SUEWS\_Veg.txt

\item[{Configuration}] \leavevmode
to fill

\end{description}\end{quote}

\end{fulllineitems}

\index{command line option!LAIinitialGrass}\index{LAIinitialGrass!command line option}

\begin{fulllineitems}
\phantomsection\label{\detokenize{input_files/Initial_Conditions/Vegetation_parameters:cmdoption-arg-laiinitialgrass}}\pysigline{\sphinxbfcode{\sphinxupquote{LAIinitialGrass}}\sphinxcode{\sphinxupquote{}}}~\begin{quote}\begin{description}
\item[{Requirement}] \leavevmode
Optional

\item[{Description}] \leavevmode
Initial LAI for irrigated grass. The recommended values can be found from SUEWS\_Veg.txt

\item[{Configuration}] \leavevmode
to fill

\end{description}\end{quote}

\end{fulllineitems}

\index{command line option!albEveTr0}\index{albEveTr0!command line option}

\begin{fulllineitems}
\phantomsection\label{\detokenize{input_files/Initial_Conditions/Vegetation_parameters:cmdoption-arg-albevetr0}}\pysigline{\sphinxbfcode{\sphinxupquote{albEveTr0}}\sphinxcode{\sphinxupquote{}}}~\begin{quote}\begin{description}
\item[{Requirement}] \leavevmode
Optional

\item[{Description}] \leavevmode
Albedo of evergreen surface on day 0 of run

\item[{Configuration}] \leavevmode
to fill

\end{description}\end{quote}

\end{fulllineitems}

\index{command line option!albDecTr0}\index{albDecTr0!command line option}

\begin{fulllineitems}
\phantomsection\label{\detokenize{input_files/Initial_Conditions/Vegetation_parameters:cmdoption-arg-albdectr0}}\pysigline{\sphinxbfcode{\sphinxupquote{albDecTr0}}\sphinxcode{\sphinxupquote{}}}~\begin{quote}\begin{description}
\item[{Requirement}] \leavevmode
Optional

\item[{Description}] \leavevmode
Albedo of deciduous surface on day 0 of run

\item[{Configuration}] \leavevmode
to fill

\end{description}\end{quote}

\end{fulllineitems}

\index{command line option!albGrass0}\index{albGrass0!command line option}

\begin{fulllineitems}
\phantomsection\label{\detokenize{input_files/Initial_Conditions/Vegetation_parameters:cmdoption-arg-albgrass0}}\pysigline{\sphinxbfcode{\sphinxupquote{albGrass0}}\sphinxcode{\sphinxupquote{}}}~\begin{quote}\begin{description}
\item[{Requirement}] \leavevmode
Optional

\item[{Description}] \leavevmode
Albedo of grass surface on day 0 of run

\item[{Configuration}] \leavevmode
to fill

\end{description}\end{quote}

\end{fulllineitems}

\index{command line option!decidCap0}\index{decidCap0!command line option}

\begin{fulllineitems}
\phantomsection\label{\detokenize{input_files/Initial_Conditions/Vegetation_parameters:cmdoption-arg-decidcap0}}\pysigline{\sphinxbfcode{\sphinxupquote{decidCap0}}\sphinxcode{\sphinxupquote{}}}~\begin{quote}\begin{description}
\item[{Requirement}] \leavevmode
Optional

\item[{Description}] \leavevmode
Deciduous storage capacity on day 0 of run.

\item[{Configuration}] \leavevmode
to fill

\end{description}\end{quote}

\end{fulllineitems}

\index{command line option!porosity0}\index{porosity0!command line option}

\begin{fulllineitems}
\phantomsection\label{\detokenize{input_files/Initial_Conditions/Vegetation_parameters:cmdoption-arg-porosity0}}\pysigline{\sphinxbfcode{\sphinxupquote{porosity0}}\sphinxcode{\sphinxupquote{}}}~\begin{quote}\begin{description}
\item[{Requirement}] \leavevmode
Optional

\item[{Description}] \leavevmode
Porosity of deciduous vegetation on day 0 of run. This varies between 0.2 (leaf-on) and 0.6 (leaf-off).

\item[{Configuration}] \leavevmode
to fill

\end{description}\end{quote}

\end{fulllineitems}



\subsection{Recent meteorology}
\label{\detokenize{input_files/Initial_Conditions/Recent_meteorology:recent-meteorology}}\label{\detokenize{input_files/Initial_Conditions/Recent_meteorology::doc}}\label{\detokenize{input_files/Initial_Conditions/Recent_meteorology:id1}}\index{command line option!DaysSinceRain}\index{DaysSinceRain!command line option}

\begin{fulllineitems}
\phantomsection\label{\detokenize{input_files/Initial_Conditions/Recent_meteorology:cmdoption-arg-dayssincerain}}\pysigline{\sphinxbfcode{\sphinxupquote{DaysSinceRain}}\sphinxcode{\sphinxupquote{}}}~\begin{quote}\begin{description}
\item[{Requirement}] \leavevmode
Optional

\item[{Description}] \leavevmode
Important to use correct value if starting in summer season If starting when external water use is not occurring it will be reset with the first rain so can just be set to 0. If unknown, SUEWS sets to zero by default. Used to model irrigation.

\item[{Configuration}] \leavevmode
to fill

\end{description}\end{quote}

\end{fulllineitems}

\index{command line option!Temp\_C0}\index{Temp\_C0!command line option}

\begin{fulllineitems}
\phantomsection\label{\detokenize{input_files/Initial_Conditions/Recent_meteorology:cmdoption-arg-temp-c0}}\pysigline{\sphinxbfcode{\sphinxupquote{Temp\_C0}}\sphinxcode{\sphinxupquote{}}}~\begin{quote}\begin{description}
\item[{Requirement}] \leavevmode
Optional

\item[{Description}] \leavevmode
If unknown, SUEWS uses the mean temperature for the first day of the run.

\item[{Configuration}] \leavevmode
to fill

\end{description}\end{quote}

\end{fulllineitems}



\subsection{Above Ground State}
\label{\detokenize{input_files/Initial_Conditions/Above_Ground_State:above-ground-state}}\label{\detokenize{input_files/Initial_Conditions/Above_Ground_State::doc}}\label{\detokenize{input_files/Initial_Conditions/Above_Ground_State:id1}}\index{command line option!PavedState}\index{PavedState!command line option}

\begin{fulllineitems}
\phantomsection\label{\detokenize{input_files/Initial_Conditions/Above_Ground_State:cmdoption-arg-pavedstate}}\pysigline{\sphinxbfcode{\sphinxupquote{PavedState}}\sphinxcode{\sphinxupquote{}}}~\begin{quote}\begin{description}
\item[{Requirement}] \leavevmode
Optional

\item[{Description}] \leavevmode
If unknown, model assumes dry surfaces (acceptable as rainfall or irrigation will update these states quickly).

\item[{Configuration}] \leavevmode
to fill

\end{description}\end{quote}

\end{fulllineitems}

\index{command line option!BldgsState}\index{BldgsState!command line option}

\begin{fulllineitems}
\phantomsection\label{\detokenize{input_files/Initial_Conditions/Above_Ground_State:cmdoption-arg-bldgsstate}}\pysigline{\sphinxbfcode{\sphinxupquote{BldgsState}}\sphinxcode{\sphinxupquote{}}}~\begin{quote}\begin{description}
\item[{Requirement}] \leavevmode
Optional

\item[{Description}] \leavevmode
If unknown, model assumes dry surfaces (acceptable as rainfall or irrigation will update these states quickly).

\item[{Configuration}] \leavevmode
to fill

\end{description}\end{quote}

\end{fulllineitems}

\index{command line option!EveTrState}\index{EveTrState!command line option}

\begin{fulllineitems}
\phantomsection\label{\detokenize{input_files/Initial_Conditions/Above_Ground_State:cmdoption-arg-evetrstate}}\pysigline{\sphinxbfcode{\sphinxupquote{EveTrState}}\sphinxcode{\sphinxupquote{}}}~\begin{quote}\begin{description}
\item[{Requirement}] \leavevmode
Optional

\item[{Description}] \leavevmode
If unknown, model assumes dry surfaces (acceptable as rainfall or irrigation will update these states quickly).

\item[{Configuration}] \leavevmode
to fill

\end{description}\end{quote}

\end{fulllineitems}

\index{command line option!DecTrState}\index{DecTrState!command line option}

\begin{fulllineitems}
\phantomsection\label{\detokenize{input_files/Initial_Conditions/Above_Ground_State:cmdoption-arg-dectrstate}}\pysigline{\sphinxbfcode{\sphinxupquote{DecTrState}}\sphinxcode{\sphinxupquote{}}}~\begin{quote}\begin{description}
\item[{Requirement}] \leavevmode
Optional

\item[{Description}] \leavevmode
If unknown, model assumes dry surfaces (acceptable as rainfall or irrigation will update these states quickly).

\item[{Configuration}] \leavevmode
to fill

\end{description}\end{quote}

\end{fulllineitems}

\index{command line option!GrassState}\index{GrassState!command line option}

\begin{fulllineitems}
\phantomsection\label{\detokenize{input_files/Initial_Conditions/Above_Ground_State:cmdoption-arg-grassstate}}\pysigline{\sphinxbfcode{\sphinxupquote{GrassState}}\sphinxcode{\sphinxupquote{}}}~\begin{quote}\begin{description}
\item[{Requirement}] \leavevmode
Optional

\item[{Description}] \leavevmode
If unknown, model assumes dry surfaces (acceptable as rainfall or irrigation will update these states quickly).

\item[{Configuration}] \leavevmode
to fill

\end{description}\end{quote}

\end{fulllineitems}

\index{command line option!BSoilState}\index{BSoilState!command line option}

\begin{fulllineitems}
\phantomsection\label{\detokenize{input_files/Initial_Conditions/Above_Ground_State:cmdoption-arg-bsoilstate}}\pysigline{\sphinxbfcode{\sphinxupquote{BSoilState}}\sphinxcode{\sphinxupquote{}}}~\begin{quote}\begin{description}
\item[{Requirement}] \leavevmode
Optional

\item[{Description}] \leavevmode
If unknown, model assumes dry surfaces (acceptable as rainfall or irrigation will update these states quickly).

\item[{Configuration}] \leavevmode
to fill

\end{description}\end{quote}

\end{fulllineitems}

\index{command line option!WaterState}\index{WaterState!command line option}

\begin{fulllineitems}
\phantomsection\label{\detokenize{input_files/Initial_Conditions/Above_Ground_State:cmdoption-arg-waterstate}}\pysigline{\sphinxbfcode{\sphinxupquote{WaterState}}\sphinxcode{\sphinxupquote{}}}~\begin{quote}\begin{description}
\item[{Requirement}] \leavevmode
Optional

\item[{Description}] \leavevmode
For a large water body (e.g. river, sea, lake) set WaterState to a large value, e.g. 20000 mm; for small water bodies (e.g. ponds, fountains) set WaterState to smaller value, e.g. 1000 mm. This value must not exceed StateLimit specified in SUEWS\_Water.txt . If unknown, model uses value of WaterDepth specified in SUEWS\_Water.txt .

\item[{Configuration}] \leavevmode
to fill

\end{description}\end{quote}

\end{fulllineitems}



\subsection{Snow related parameters}
\label{\detokenize{input_files/Initial_Conditions/Snow_related_parameters:snow-related-parameters}}\label{\detokenize{input_files/Initial_Conditions/Snow_related_parameters::doc}}\label{\detokenize{input_files/Initial_Conditions/Snow_related_parameters:id1}}\index{command line option!SnowIntially}\index{SnowIntially!command line option}

\begin{fulllineitems}
\phantomsection\label{\detokenize{input_files/Initial_Conditions/Snow_related_parameters:cmdoption-arg-snowintially}}\pysigline{\sphinxbfcode{\sphinxupquote{SnowIntially}}\sphinxcode{\sphinxupquote{}}}~\begin{quote}\begin{description}
\item[{Requirement}] \leavevmode
Optional

\item[{Description}] \leavevmode
If the model run starts when there is no snow on the ground, set SnowIntially = 0 and the snow-related parameters will be set accordingly. If the model run starts when there is snow on the ground, the following snow-related parameters must be set appropriately. The value of SnowInitially overrides any values provided for the individual snow-related parameters. To prevent SnowInitially from setting the initial conditions, either omit it from the namelist or set to -999. If values are provided individually, they should be consistent the information provided in SUEWS\_Snow.txt .

\item[{Configuration}] \leavevmode
to fill

\end{description}\end{quote}

\end{fulllineitems}

\index{command line option!SnowWaterPavedState}\index{SnowWaterPavedState!command line option}

\begin{fulllineitems}
\phantomsection\label{\detokenize{input_files/Initial_Conditions/Snow_related_parameters:cmdoption-arg-snowwaterpavedstate}}\pysigline{\sphinxbfcode{\sphinxupquote{SnowWaterPavedState}}\sphinxcode{\sphinxupquote{}}}~\begin{quote}\begin{description}
\item[{Requirement}] \leavevmode
Optional

\item[{Description}] \leavevmode
Initial amount of liquid water in the snow on paved surfaces.

\item[{Configuration}] \leavevmode
to fill

\end{description}\end{quote}

\end{fulllineitems}

\index{command line option!SnowWaterBldgsState}\index{SnowWaterBldgsState!command line option}

\begin{fulllineitems}
\phantomsection\label{\detokenize{input_files/Initial_Conditions/Snow_related_parameters:cmdoption-arg-snowwaterbldgsstate}}\pysigline{\sphinxbfcode{\sphinxupquote{SnowWaterBldgsState}}\sphinxcode{\sphinxupquote{}}}~\begin{quote}\begin{description}
\item[{Requirement}] \leavevmode
Optional

\item[{Description}] \leavevmode
Initial amount of liquid water in the snow on buildings

\item[{Configuration}] \leavevmode
to fill

\end{description}\end{quote}

\end{fulllineitems}

\index{command line option!SnowWaterEveTrState}\index{SnowWaterEveTrState!command line option}

\begin{fulllineitems}
\phantomsection\label{\detokenize{input_files/Initial_Conditions/Snow_related_parameters:cmdoption-arg-snowwaterevetrstate}}\pysigline{\sphinxbfcode{\sphinxupquote{SnowWaterEveTrState}}\sphinxcode{\sphinxupquote{}}}~\begin{quote}\begin{description}
\item[{Requirement}] \leavevmode
Optional

\item[{Description}] \leavevmode
Initial amount of liquid water in the snow on evergreen trees

\item[{Configuration}] \leavevmode
to fill

\end{description}\end{quote}

\end{fulllineitems}

\index{command line option!SnowWaterDecTrState}\index{SnowWaterDecTrState!command line option}

\begin{fulllineitems}
\phantomsection\label{\detokenize{input_files/Initial_Conditions/Snow_related_parameters:cmdoption-arg-snowwaterdectrstate}}\pysigline{\sphinxbfcode{\sphinxupquote{SnowWaterDecTrState}}\sphinxcode{\sphinxupquote{}}}~\begin{quote}\begin{description}
\item[{Requirement}] \leavevmode
Optional

\item[{Description}] \leavevmode
Initial amount of liquid water in the snow on deciduous trees

\item[{Configuration}] \leavevmode
to fill

\end{description}\end{quote}

\end{fulllineitems}

\index{command line option!SnowWaterGrassState}\index{SnowWaterGrassState!command line option}

\begin{fulllineitems}
\phantomsection\label{\detokenize{input_files/Initial_Conditions/Snow_related_parameters:cmdoption-arg-snowwatergrassstate}}\pysigline{\sphinxbfcode{\sphinxupquote{SnowWaterGrassState}}\sphinxcode{\sphinxupquote{}}}~\begin{quote}\begin{description}
\item[{Requirement}] \leavevmode
Optional

\item[{Description}] \leavevmode
Initial amount of liquid water in the snow on grass surfaces

\item[{Configuration}] \leavevmode
to fill

\end{description}\end{quote}

\end{fulllineitems}

\index{command line option!SnowWaterBSoilState}\index{SnowWaterBSoilState!command line option}

\begin{fulllineitems}
\phantomsection\label{\detokenize{input_files/Initial_Conditions/Snow_related_parameters:cmdoption-arg-snowwaterbsoilstate}}\pysigline{\sphinxbfcode{\sphinxupquote{SnowWaterBSoilState}}\sphinxcode{\sphinxupquote{}}}~\begin{quote}\begin{description}
\item[{Requirement}] \leavevmode
Optional

\item[{Description}] \leavevmode
Initial amount of liquid water in the snow on bare soil surfaces

\item[{Configuration}] \leavevmode
to fill

\end{description}\end{quote}

\end{fulllineitems}

\index{command line option!SnowWaterWaterState}\index{SnowWaterWaterState!command line option}

\begin{fulllineitems}
\phantomsection\label{\detokenize{input_files/Initial_Conditions/Snow_related_parameters:cmdoption-arg-snowwaterwaterstate}}\pysigline{\sphinxbfcode{\sphinxupquote{SnowWaterWaterState}}\sphinxcode{\sphinxupquote{}}}~\begin{quote}\begin{description}
\item[{Requirement}] \leavevmode
Optional

\item[{Description}] \leavevmode
Initial amount of liquid water in the snow in water

\item[{Configuration}] \leavevmode
to fill

\end{description}\end{quote}

\end{fulllineitems}

\index{command line option!SnowPackPaved}\index{SnowPackPaved!command line option}

\begin{fulllineitems}
\phantomsection\label{\detokenize{input_files/Initial_Conditions/Snow_related_parameters:cmdoption-arg-snowpackpaved}}\pysigline{\sphinxbfcode{\sphinxupquote{SnowPackPaved}}\sphinxcode{\sphinxupquote{}}}~\begin{quote}\begin{description}
\item[{Requirement}] \leavevmode
Optional

\item[{Description}] \leavevmode
Initial snow water equivalent if the snow on paved surfaces

\item[{Configuration}] \leavevmode
to fill

\end{description}\end{quote}

\end{fulllineitems}

\index{command line option!SnowPackBldgs}\index{SnowPackBldgs!command line option}

\begin{fulllineitems}
\phantomsection\label{\detokenize{input_files/Initial_Conditions/Snow_related_parameters:cmdoption-arg-snowpackbldgs}}\pysigline{\sphinxbfcode{\sphinxupquote{SnowPackBldgs}}\sphinxcode{\sphinxupquote{}}}~\begin{quote}\begin{description}
\item[{Requirement}] \leavevmode
Optional

\item[{Description}] \leavevmode
Initial snow water equivalent if the snow on buildings

\item[{Configuration}] \leavevmode
to fill

\end{description}\end{quote}

\end{fulllineitems}

\index{command line option!SnowPackEveTr}\index{SnowPackEveTr!command line option}

\begin{fulllineitems}
\phantomsection\label{\detokenize{input_files/Initial_Conditions/Snow_related_parameters:cmdoption-arg-snowpackevetr}}\pysigline{\sphinxbfcode{\sphinxupquote{SnowPackEveTr}}\sphinxcode{\sphinxupquote{}}}~\begin{quote}\begin{description}
\item[{Requirement}] \leavevmode
Optional

\item[{Description}] \leavevmode
Initial snow water equivalent if the snow on evergreen trees

\item[{Configuration}] \leavevmode
to fill

\end{description}\end{quote}

\end{fulllineitems}

\index{command line option!SnowPackDecTr}\index{SnowPackDecTr!command line option}

\begin{fulllineitems}
\phantomsection\label{\detokenize{input_files/Initial_Conditions/Snow_related_parameters:cmdoption-arg-snowpackdectr}}\pysigline{\sphinxbfcode{\sphinxupquote{SnowPackDecTr}}\sphinxcode{\sphinxupquote{}}}~\begin{quote}\begin{description}
\item[{Requirement}] \leavevmode
Optional

\item[{Description}] \leavevmode
Initial snow water equivalent if the snow on deciduous trees

\item[{Configuration}] \leavevmode
to fill

\end{description}\end{quote}

\end{fulllineitems}

\index{command line option!SnowPackGrass}\index{SnowPackGrass!command line option}

\begin{fulllineitems}
\phantomsection\label{\detokenize{input_files/Initial_Conditions/Snow_related_parameters:cmdoption-arg-snowpackgrass}}\pysigline{\sphinxbfcode{\sphinxupquote{SnowPackGrass}}\sphinxcode{\sphinxupquote{}}}~\begin{quote}\begin{description}
\item[{Requirement}] \leavevmode
Optional

\item[{Description}] \leavevmode
Initial snow water equivalent if the snow on grass surfaces

\item[{Configuration}] \leavevmode
to fill

\end{description}\end{quote}

\end{fulllineitems}

\index{command line option!SnowPackBSoil}\index{SnowPackBSoil!command line option}

\begin{fulllineitems}
\phantomsection\label{\detokenize{input_files/Initial_Conditions/Snow_related_parameters:cmdoption-arg-snowpackbsoil}}\pysigline{\sphinxbfcode{\sphinxupquote{SnowPackBSoil}}\sphinxcode{\sphinxupquote{}}}~\begin{quote}\begin{description}
\item[{Requirement}] \leavevmode
Optional

\item[{Description}] \leavevmode
Initial snow water equivalent if the snow on bare soil surfaces

\item[{Configuration}] \leavevmode
to fill

\end{description}\end{quote}

\end{fulllineitems}

\index{command line option!SnowPackWater}\index{SnowPackWater!command line option}

\begin{fulllineitems}
\phantomsection\label{\detokenize{input_files/Initial_Conditions/Snow_related_parameters:cmdoption-arg-snowpackwater}}\pysigline{\sphinxbfcode{\sphinxupquote{SnowPackWater}}\sphinxcode{\sphinxupquote{}}}~\begin{quote}\begin{description}
\item[{Requirement}] \leavevmode
Optional

\item[{Description}] \leavevmode
Initial snow water equivalent if the snow on water

\item[{Configuration}] \leavevmode
to fill

\end{description}\end{quote}

\end{fulllineitems}

\index{command line option!SnowFracPaved}\index{SnowFracPaved!command line option}

\begin{fulllineitems}
\phantomsection\label{\detokenize{input_files/Initial_Conditions/Snow_related_parameters:cmdoption-arg-snowfracpaved}}\pysigline{\sphinxbfcode{\sphinxupquote{SnowFracPaved}}\sphinxcode{\sphinxupquote{}}}~\begin{quote}\begin{description}
\item[{Requirement}] \leavevmode
Optional

\item[{Description}] \leavevmode
Initial plan area fraction of snow on paved surfaces

\item[{Configuration}] \leavevmode
to fill

\end{description}\end{quote}

\end{fulllineitems}

\index{command line option!SnowFracBldgs}\index{SnowFracBldgs!command line option}

\begin{fulllineitems}
\phantomsection\label{\detokenize{input_files/Initial_Conditions/Snow_related_parameters:cmdoption-arg-snowfracbldgs}}\pysigline{\sphinxbfcode{\sphinxupquote{SnowFracBldgs}}\sphinxcode{\sphinxupquote{}}}~\begin{quote}\begin{description}
\item[{Requirement}] \leavevmode
Optional

\item[{Description}] \leavevmode
Initial plan area fraction of snow on buildings

\item[{Configuration}] \leavevmode
to fill

\end{description}\end{quote}

\end{fulllineitems}

\index{command line option!SnowFracEveTr}\index{SnowFracEveTr!command line option}

\begin{fulllineitems}
\phantomsection\label{\detokenize{input_files/Initial_Conditions/Snow_related_parameters:cmdoption-arg-snowfracevetr}}\pysigline{\sphinxbfcode{\sphinxupquote{SnowFracEveTr}}\sphinxcode{\sphinxupquote{}}}~\begin{quote}\begin{description}
\item[{Requirement}] \leavevmode
Optional

\item[{Description}] \leavevmode
Initial plan area fraction of snow on evergreen trees

\item[{Configuration}] \leavevmode
to fill

\end{description}\end{quote}

\end{fulllineitems}

\index{command line option!SnowFracDecTr}\index{SnowFracDecTr!command line option}

\begin{fulllineitems}
\phantomsection\label{\detokenize{input_files/Initial_Conditions/Snow_related_parameters:cmdoption-arg-snowfracdectr}}\pysigline{\sphinxbfcode{\sphinxupquote{SnowFracDecTr}}\sphinxcode{\sphinxupquote{}}}~\begin{quote}\begin{description}
\item[{Requirement}] \leavevmode
Optional

\item[{Description}] \leavevmode
Initial plan area fraction of snow on deciduous trees

\item[{Configuration}] \leavevmode
to fill

\end{description}\end{quote}

\end{fulllineitems}

\index{command line option!SnowFracGras}\index{SnowFracGras!command line option}

\begin{fulllineitems}
\phantomsection\label{\detokenize{input_files/Initial_Conditions/Snow_related_parameters:cmdoption-arg-snowfracgras}}\pysigline{\sphinxbfcode{\sphinxupquote{SnowFracGras}}\sphinxcode{\sphinxupquote{}}}~\begin{quote}\begin{description}
\item[{Requirement}] \leavevmode
Optional

\item[{Description}] \leavevmode
Initial plan area fraction of snow on grass surfaces

\item[{Configuration}] \leavevmode
to fill

\end{description}\end{quote}

\end{fulllineitems}

\index{command line option!SnowFracBSoil}\index{SnowFracBSoil!command line option}

\begin{fulllineitems}
\phantomsection\label{\detokenize{input_files/Initial_Conditions/Snow_related_parameters:cmdoption-arg-snowfracbsoil}}\pysigline{\sphinxbfcode{\sphinxupquote{SnowFracBSoil}}\sphinxcode{\sphinxupquote{}}}~\begin{quote}\begin{description}
\item[{Requirement}] \leavevmode
Optional

\item[{Description}] \leavevmode
Initial plan area fraction of snow on bare soil surfaces

\item[{Configuration}] \leavevmode
to fill

\end{description}\end{quote}

\end{fulllineitems}

\index{command line option!SnowFracWater}\index{SnowFracWater!command line option}

\begin{fulllineitems}
\phantomsection\label{\detokenize{input_files/Initial_Conditions/Snow_related_parameters:cmdoption-arg-snowfracwater}}\pysigline{\sphinxbfcode{\sphinxupquote{SnowFracWater}}\sphinxcode{\sphinxupquote{}}}~\begin{quote}\begin{description}
\item[{Requirement}] \leavevmode
Optional

\item[{Description}] \leavevmode
Initial plan area fraction of snow on water

\item[{Configuration}] \leavevmode
to fill

\end{description}\end{quote}

\end{fulllineitems}

\index{command line option!SnowDensPaved}\index{SnowDensPaved!command line option}

\begin{fulllineitems}
\phantomsection\label{\detokenize{input_files/Initial_Conditions/Snow_related_parameters:cmdoption-arg-snowdenspaved}}\pysigline{\sphinxbfcode{\sphinxupquote{SnowDensPaved}}\sphinxcode{\sphinxupquote{}}}~\begin{quote}\begin{description}
\item[{Requirement}] \leavevmode
Optional

\item[{Description}] \leavevmode
Initial snow density on paved surfaces

\item[{Configuration}] \leavevmode
to fill

\end{description}\end{quote}

\end{fulllineitems}

\index{command line option!SnowDensBldgs}\index{SnowDensBldgs!command line option}

\begin{fulllineitems}
\phantomsection\label{\detokenize{input_files/Initial_Conditions/Snow_related_parameters:cmdoption-arg-snowdensbldgs}}\pysigline{\sphinxbfcode{\sphinxupquote{SnowDensBldgs}}\sphinxcode{\sphinxupquote{}}}~\begin{quote}\begin{description}
\item[{Requirement}] \leavevmode
Optional

\item[{Description}] \leavevmode
Initial snow density on buildings

\item[{Configuration}] \leavevmode
to fill

\end{description}\end{quote}

\end{fulllineitems}

\index{command line option!SnowDensEveTr}\index{SnowDensEveTr!command line option}

\begin{fulllineitems}
\phantomsection\label{\detokenize{input_files/Initial_Conditions/Snow_related_parameters:cmdoption-arg-snowdensevetr}}\pysigline{\sphinxbfcode{\sphinxupquote{SnowDensEveTr}}\sphinxcode{\sphinxupquote{}}}~\begin{quote}\begin{description}
\item[{Requirement}] \leavevmode
Optional

\item[{Description}] \leavevmode
Initial snow density on evergreen trees

\item[{Configuration}] \leavevmode
to fill

\end{description}\end{quote}

\end{fulllineitems}

\index{command line option!SnowDensDecTr}\index{SnowDensDecTr!command line option}

\begin{fulllineitems}
\phantomsection\label{\detokenize{input_files/Initial_Conditions/Snow_related_parameters:cmdoption-arg-snowdensdectr}}\pysigline{\sphinxbfcode{\sphinxupquote{SnowDensDecTr}}\sphinxcode{\sphinxupquote{}}}~\begin{quote}\begin{description}
\item[{Requirement}] \leavevmode
Optional

\item[{Description}] \leavevmode
Initial snow density on deciduous trees

\item[{Configuration}] \leavevmode
to fill

\end{description}\end{quote}

\end{fulllineitems}

\index{command line option!SnowDensGrass}\index{SnowDensGrass!command line option}

\begin{fulllineitems}
\phantomsection\label{\detokenize{input_files/Initial_Conditions/Snow_related_parameters:cmdoption-arg-snowdensgrass}}\pysigline{\sphinxbfcode{\sphinxupquote{SnowDensGrass}}\sphinxcode{\sphinxupquote{}}}~\begin{quote}\begin{description}
\item[{Requirement}] \leavevmode
Optional

\item[{Description}] \leavevmode
Initial snow density on grass surfaces

\item[{Configuration}] \leavevmode
to fill

\end{description}\end{quote}

\end{fulllineitems}

\index{command line option!SnowDensBSoil}\index{SnowDensBSoil!command line option}

\begin{fulllineitems}
\phantomsection\label{\detokenize{input_files/Initial_Conditions/Snow_related_parameters:cmdoption-arg-snowdensbsoil}}\pysigline{\sphinxbfcode{\sphinxupquote{SnowDensBSoil}}\sphinxcode{\sphinxupquote{}}}~\begin{quote}\begin{description}
\item[{Requirement}] \leavevmode
Optional

\item[{Description}] \leavevmode
Initial snow density on bare soil surfaces

\item[{Configuration}] \leavevmode
to fill

\end{description}\end{quote}

\end{fulllineitems}

\index{command line option!SnowDensWater}\index{SnowDensWater!command line option}

\begin{fulllineitems}
\phantomsection\label{\detokenize{input_files/Initial_Conditions/Snow_related_parameters:cmdoption-arg-snowdenswater}}\pysigline{\sphinxbfcode{\sphinxupquote{SnowDensWater}}\sphinxcode{\sphinxupquote{}}}~\begin{quote}\begin{description}
\item[{Requirement}] \leavevmode
Optional

\item[{Description}] \leavevmode
Initial snow density on water

\item[{Configuration}] \leavevmode
to fill

\end{description}\end{quote}

\end{fulllineitems}



\section{Meteorological Input File}
\label{\detokenize{input_files/met_input::doc}}\label{\detokenize{input_files/met_input:meteorological-input-file}}
SUEWS is designed to run using commonly measured meteorological
variables.
\begin{itemize}
\item {} 
Required inputs must be continuous \textendash{} i.e. \sphinxstylestrong{gap fill} any missing
data.

\item {} 
The table below gives the required (R) and optional (O) additional
input variables.

\item {} 
If an optional input variable is not available or will not be used by
the model, enter ‘-999.0’ for this column.

\item {} 
Since v2017a forcing files no longer need to end with two rows
containing ‘-9’ in the first column.

\item {} 
One single meteorological file can be used for all grids
(\sphinxstylestrong{MultipleMetFiles=0} in {\hyperref[\detokenize{input_files/met_input:RunControl.nml}]{\emph{RunControl.nml}}} (\autopageref*{\detokenize{input_files/met_input:RunControl.nml}}), no
grid number in file name) if appropriate for the study area, or

\item {} 
separate met files can be used for each grid if data are available
(\sphinxstylestrong{MultipleMetFiles=1} in {\hyperref[\detokenize{input_files/met_input:RunControl.nml}]{\emph{RunControl.nml}}} (\autopageref*{\detokenize{input_files/met_input:RunControl.nml}}),
filename includes grid number).

\item {} 
The meteorological forcing file names should be appended with the
temporal resolution in minutes (SS\_YYYY\_data\_tt.txt, or
SSss\_YYYY\_data\_tt.txt for multiple grids).

\item {} 
Separate met forcing files should be provided for each year.

\item {} 
Files do not need to start/end at the start/end of the year, but they
must contain a whole number of days.

\item {} 
The meteorological input file should match the information given in
{\hyperref[\detokenize{input_files/met_input:SUEWS_SiteSelect.txt}]{\emph{SUEWS\_SiteSelect.txt}}} (\autopageref*{\detokenize{input_files/met_input:SUEWS_SiteSelect.txt}}).

\item {} 
If a \sphinxstyleemphasis{partial year} is used that specific year must be given in
SUEWS\_SiteSelect.txt.

\item {} 
If \sphinxstyleemphasis{multiple years} are used, all years should be included in
SUEWS\_SiteSelect.txt.

\item {} 
If a \sphinxstyleemphasis{whole year} (e.g. 2011) is intended to be modelled using and
hourly resolution dataset, the number of lines in the met data file
should be 8760 and begin and end with:

\fvset{hllines={, ,}}%
\begin{sphinxVerbatim}[commandchars=\\\{\}]
iy     id  it  imin
2011   1   1   0 …
…
2012   1   0   0 …
\end{sphinxVerbatim}

\end{itemize}


\subsection{SSss\_YYYY\_data\_tt.txt}
\label{\detokenize{input_files/met_input:ssss-yyyy-data-tt-txt}}\label{\detokenize{input_files/met_input:id1}}
Main meteorological data file.


\begin{savenotes}\sphinxattablestart
\centering
\begin{tabulary}{\linewidth}[t]{|T|T|T|T|}
\hline
\sphinxstyletheadfamily 
No.
&\sphinxstyletheadfamily 
Use
&\sphinxstyletheadfamily 
Column name
&\sphinxstyletheadfamily 
Description
\\
\hline
1
&
R
&
iy
&
Year {[}YYYY{]}
\\
\hline
2
&
R
&
id
&
Day of year
{[}DOY{]}
\\
\hline
3
&
R
&
it
&
Hour {[}H{]}
\\
\hline
4
&
R
&
imin
&
Minute {[}M{]}
\\
\hline
5
&
O
&
qn
&
Net
all-wave
radiation
{[}W
m\textasciicircum{}-2 {]}
-  Required
if
\sphinxstylestrong{NetRad
iationMetho
d}
= 1.
\\
\hline
6
&
O
&
qh
&
Sensible
heat flux
{[}W
m\textasciicircum{}-2
{]}
\\
\hline
7
&
O
&
qe
&
Latent heat
flux {[}W
m\textasciicircum{}-2
{]}
\\
\hline
8
&
O
&
qs
&
Storage
heat flux
{[}W
m\textasciicircum{}-2
{]}
\\
\hline
9
&
O
&
qf
&
Anthropogen
ic
heat flux
{[}W
m\textasciicircum{}-2
{]}
\\
\hline
10
&
R
&
U
&
Wind speed
{[}m
s\textasciicircum{}-1
{]}
*Height of
the wind
speed
measurement
(z) is
needed in
{[}{[}\#SUEWS\_Si
teSelect.tx
t{]}
\\
\hline
11
&
R
&
RH
&
Relative
Humidity
{[}\%{]}
\\
\hline
12
&
R
&
Tair
&
Air
temperature
{[}°C{]}
\\
\hline
13
&
R
&
pres
&
Barometric
pressure
{[}kPa{]}
\\
\hline
14
&
R
&
rain
&
Rainfall
{[}mm{]}
\\
\hline
15
&
R
&
kdown
&
Incoming
shortwave
radiation
{[}W
m\textasciicircum{}-2
{]}
-  Must be \textgreater{} 0 W
m\textasciicircum{}-2.
\\
\hline
16
&
O
&
snow
&
Snow {[}mm{]}
-  Required
if
\sphinxstylestrong{SnowUs
e}
= 1
\\
\hline
17
&
O
&
ldown
&
Incoming
longwave
radiation
{[}W
m\textasciicircum{}-2{]}
\\
\hline
18
&
O
&
fcld
&
Cloud
fraction
{[}tenths{]}
\\
\hline
19
&
O
&
Wuh
&
External
water use
{[}m\textasciicircum{}-3 {]}
\\
\hline
20
&
O
&
xsmd
&
Observed
soil
moisture
{[}m\textasciicircum{}-3 m\textasciicircum{}-3
{]}
or {[}kg
kg\textasciicircum{}-1
{]}
\\
\hline
21
&
O
&
lai
&
Observed
leaf area
index
{[}m\textasciicircum{}-2
m\textasciicircum{}-2 {]}
\\
\hline
22
&
O
&
kdiff
&
Diffuse
radiation
{[}W
m\textasciicircum{}-2{]}
-  Recommended
if
\sphinxstylestrong{SOLWEIGUse}
= 1
\\
\hline
23
&
O
&
kdir
&
Direct
radiation
{[}W
m\textasciicircum{}-2
{]}
-  Recommended
if
\sphinxstylestrong{SOLWEIGUse}
= 1
\\
\hline
24
&
O
&
wdir
&
Wind
direction
{[}°{]}
-  Currently
not
implemented
\\
\hline
\end{tabulary}
\par
\sphinxattableend\end{savenotes}


\section{CBL input files}
\label{\detokenize{input_files/CBL_input/CBL_input::doc}}\label{\detokenize{input_files/CBL_input/CBL_input:cbl-input-files}}
Main references for this part of the model: Onomura et al. (2015) \phantomsection\label{\detokenize{input_files/CBL_input/CBL_input:id1}}{\hyperref[\detokenize{references:shiho2015}]{\sphinxcrossref{{[}Shiho2015{]}}}} (\autopageref*{\detokenize{references:shiho2015}})
and Cleugh and Grimmond (2001) \phantomsection\label{\detokenize{input_files/CBL_input/CBL_input:id2}}{\hyperref[\detokenize{references:cg2001}]{\sphinxcrossref{{[}CG2001{]}}}} (\autopageref*{\detokenize{references:cg2001}}).

If CBL slab model is used ({\hyperref[\detokenize{input_files/RunControl/Model_run_options:cmdoption-arg-cbluse}]{\sphinxcrossref{\sphinxcode{\sphinxupquote{CBLuse = 1}}}}} (\autopageref*{\detokenize{input_files/RunControl/Model_run_options:cmdoption-arg-cbluse}}) in
{\hyperref[\detokenize{input_files/RunControl/RunControl:runcontrol}]{\sphinxcrossref{\DUrole{std,std-ref}{RunControl.nml}}}} (\autopageref*{\detokenize{input_files/RunControl/RunControl:runcontrol}})) the following files are needed.


\begin{savenotes}\sphinxattablestart
\centering
\begin{tabular}[t]{|\X{50}{100}|\X{50}{100}|}
\hline
\sphinxstyletheadfamily 
Filename
&\sphinxstyletheadfamily 
Purpose
\\
\hline
{\hyperref[\detokenize{input_files/CBL_input/CBL_input:cbl-initial-data-txt}]{\sphinxcrossref{\DUrole{std,std-ref,std,std-ref}{CBL\_initial\_data.txt}}}} (\autopageref*{\detokenize{input_files/CBL_input/CBL_input:cbl-initial-data-txt}})
&
Gives initial data every morning
* when CBL slab model starts running.
* filename must match the InitialData\_FileName in CBLInput.nml
* fixed formats.
\\
\hline
{\hyperref[\detokenize{input_files/CBL_input/CBL_input:cblinput-nml}]{\sphinxcrossref{\DUrole{std,std-ref,std,std-ref}{CBLInput.nml}}}} (\autopageref*{\detokenize{input_files/CBL_input/CBL_input:cblinput-nml}})
&
Specifies run options, parameters and input file names.
* Can be in any order
\\
\hline
\end{tabular}
\par
\sphinxattableend\end{savenotes}


\subsection{CBL\_initial\_data.txt}
\label{\detokenize{input_files/CBL_input/CBL_input:cbl-initial-data-txt}}\label{\detokenize{input_files/CBL_input/CBL_input:id3}}
This file should give initial data every morning when CBL slab model
starts running. The file name should match the InitialData\_FileName in
CBLInput.nml.

Definitions and example file of initial values prepared for Sacramento.


\begin{savenotes}\sphinxattablestart
\centering
\begin{tabular}[t]{|\X{33}{99}|\X{33}{99}|\X{33}{99}|}
\hline
\sphinxstyletheadfamily 
No.
&\sphinxstyletheadfamily 
Column name
&\sphinxstyletheadfamily 
Description
\\
\hline
1
&
id
&
Day of year {[}DOY{]}
\\
\hline
2
&
zi0
&
initial convective  boundary layer height (m)
\\
\hline
3
&
gamt\_Km
&
vertical gradient of potential temperature (K m $^{\text{-1}}$) strength of the inversion
\\
\hline
4
&
gamq\_gkgm
&
vertical gradient of specific humidity (g kg$^{\text{-1}}$ m$^{\text{-1}}$)
\\
\hline
5
&
Theta+\_K
&
potential temperature at the top of CBL (K)
\\
\hline
6
&
q+\_gkg
&
specific humidity at the top of CBL (g kg$^{\text{-1}}$)
\\
\hline
7
&
Theta\_K
&
potential temperature in CBL (K)
\\
\hline
8
&
q\_gkg
&
specific humidiy in CBL (g kg$^{\text{-1}}$)
\\
\hline
\end{tabular}
\par
\sphinxattableend\end{savenotes}
\begin{itemize}
\item {} 
gamt\_Km and gamq\_gkgm written to two significant figures are required
for the model performance in appropriate ranges \phantomsection\label{\detokenize{input_files/CBL_input/CBL_input:id4}}{\hyperref[\detokenize{references:shiho2015}]{\sphinxcrossref{{[}Shiho2015{]}}}} (\autopageref*{\detokenize{references:shiho2015}}).

\end{itemize}


\begin{savenotes}\sphinxattablestart
\centering
\begin{tabular}[t]{|\X{12}{96}|\X{12}{96}|\X{12}{96}|\X{12}{96}|\X{12}{96}|\X{12}{96}|\X{12}{96}|\X{12}{96}|}
\hline
\sphinxstyletheadfamily 
id
&\sphinxstyletheadfamily 
zi0
&\sphinxstyletheadfamily 
gamt\_Km
&\sphinxstyletheadfamily 
gamq\_gkgm
&\sphinxstyletheadfamily 
Theta+\_K
&\sphinxstyletheadfamily 
q+\_gkg
&\sphinxstyletheadfamily 
theta\_K
&\sphinxstyletheadfamily 
q\_gkg
\\
\hline
234
&
188
&
0.0032
&
0.00082
&
290.4
&
9.6
&
288.7
&
8.3
\\
\hline
235
&
197
&
0.0089
&
0.089
&
290.2
&
8.4
&
288.3
&
8.7
\\
\hline
︙
&
︙
&
︙
&
︙
&
︙
&
︙
&
︙
&
︙
\\
\hline
︙
&
︙
&
︙
&
︙
&
︙
&
︙
&
︙
&
︙
\\
\hline&&&&&&&\\
\hline
\end{tabular}
\par
\sphinxattableend\end{savenotes}


\subsection{CBLInput.nml}
\label{\detokenize{input_files/CBL_input/CBL_input:id5}}\label{\detokenize{input_files/CBL_input/CBL_input:cblinput-nml}}\begin{quote}

sample file of \sphinxstylestrong{CBLInput.nml} looks like
\end{quote}

\fvset{hllines={, ,}}%
\begin{sphinxVerbatim}[commandchars=\\\{\}]
\PYGZam{}CBLInput
EntrainmentType=1       ! 1.Tennekes and Driedonks(1981), 2.McNaughton and Springgs(1986), 3.Rayner and Watson(1991),4.Tennekes(1973),
QH\PYGZus{}choice=1             ! 1.suews  2.lumps 3.obs
CO2\PYGZus{}included=0
cblday(236)=1
cblday(258)=1
cblday(259)=1
cblday(260)=1
cblday(285)=1
cblday(297)=1
wsb=\PYGZhy{}0.01  
InitialData\PYGZus{}use=1
InitialDataFileName=\PYGZsq{}CBLinputfiles/CBL\PYGZus{}initial\PYGZus{}data.txt\PYGZsq{}
sondeflag=0
FileSonde(234)=\PYGZsq{}CBLinputfiles\PYGZbs{}Sonde\PYGZus{}Sc\PYGZus{}1991\PYGZus{}0822\PYGZus{}0650.txt\PYGZsq{}
FileSonde(235)=\PYGZsq{}CBLinputfiles\PYGZbs{}Sonde\PYGZus{}Sc\PYGZus{}1991\PYGZus{}0823\PYGZus{}0715.txt\PYGZsq{}
FileSonde(236)=\PYGZsq{}CBLinputfiles\PYGZbs{}Sonde\PYGZus{}Sc\PYGZus{}1991\PYGZus{}0824\PYGZus{}0647.txt\PYGZsq{}
FileSonde(238)=\PYGZsq{}CBLinputfiles\PYGZbs{}Sonde\PYGZus{}Sc\PYGZus{}1991\PYGZus{}0826\PYGZus{}0642.txt\PYGZsq{}
FileSonde(239)=\PYGZsq{}CBLinputfiles\PYGZbs{}Sonde\PYGZus{}Sc\PYGZus{}1991\PYGZus{}0827\PYGZus{}0640.txt\PYGZsq{}
FileSonde(240)=\PYGZsq{}CBLinputfiles\PYGZbs{}Sonde\PYGZus{}Sc\PYGZus{}1991\PYGZus{}0828\PYGZus{}0640.txt\PYGZsq{}
/
\end{sphinxVerbatim}

\begin{sphinxadmonition}{note}{Note:}
The file contents can be in any order.
\end{sphinxadmonition}

The parameters and their setting instructions
are provided through {\hyperref[\detokenize{input_files/CBL_input/CBLinput:cblinput}]{\sphinxcrossref{\DUrole{std,std-ref}{the links below}}}} (\autopageref*{\detokenize{input_files/CBL_input/CBLinput:cblinput}}):
\begin{quote}
\begin{itemize}\setlength{\itemsep}{0pt}\setlength{\parskip}{0pt}
\item {} 
{\hyperref[\detokenize{input_files/CBL_input/CBLinput:cmdoption-arg-entrainmenttype}]{\sphinxcrossref{\sphinxcode{\sphinxupquote{EntrainmentType}}}}} (\autopageref*{\detokenize{input_files/CBL_input/CBLinput:cmdoption-arg-entrainmenttype}})

\item {} 
{\hyperref[\detokenize{input_files/CBL_input/CBLinput:cmdoption-arg-qh-choice}]{\sphinxcrossref{\sphinxcode{\sphinxupquote{QH\_Choice}}}}} (\autopageref*{\detokenize{input_files/CBL_input/CBLinput:cmdoption-arg-qh-choice}})

\item {} 
{\hyperref[\detokenize{input_files/CBL_input/CBLinput:cmdoption-arg-initialdata-use}]{\sphinxcrossref{\sphinxcode{\sphinxupquote{InitialData\_use}}}}} (\autopageref*{\detokenize{input_files/CBL_input/CBLinput:cmdoption-arg-initialdata-use}})

\item {} 
{\hyperref[\detokenize{input_files/CBL_input/CBLinput:cmdoption-arg-sondeflag}]{\sphinxcrossref{\sphinxcode{\sphinxupquote{Sondeflag}}}}} (\autopageref*{\detokenize{input_files/CBL_input/CBLinput:cmdoption-arg-sondeflag}})

\item {} 
{\hyperref[\detokenize{input_files/CBL_input/CBLinput:cmdoption-arg-cblday-id}]{\sphinxcrossref{\sphinxcode{\sphinxupquote{CBLday(id)}}}}} (\autopageref*{\detokenize{input_files/CBL_input/CBLinput:cmdoption-arg-cblday-id}})

\item {} 
{\hyperref[\detokenize{input_files/CBL_input/CBLinput:cmdoption-arg-co2-included}]{\sphinxcrossref{\sphinxcode{\sphinxupquote{CO2\_included}}}}} (\autopageref*{\detokenize{input_files/CBL_input/CBLinput:cmdoption-arg-co2-included}})

\item {} 
{\hyperref[\detokenize{input_files/CBL_input/CBLinput:cmdoption-arg-filesonde-id}]{\sphinxcrossref{\sphinxcode{\sphinxupquote{FileSonde(id)}}}}} (\autopageref*{\detokenize{input_files/CBL_input/CBLinput:cmdoption-arg-filesonde-id}})

\item {} 
{\hyperref[\detokenize{input_files/CBL_input/CBLinput:cmdoption-arg-initialdatafilename}]{\sphinxcrossref{\sphinxcode{\sphinxupquote{InitialDataFileName}}}}} (\autopageref*{\detokenize{input_files/CBL_input/CBLinput:cmdoption-arg-initialdatafilename}})

\item {} 
{\hyperref[\detokenize{input_files/CBL_input/CBLinput:cmdoption-arg-wsb}]{\sphinxcrossref{\sphinxcode{\sphinxupquote{Wsb}}}}} (\autopageref*{\detokenize{input_files/CBL_input/CBLinput:cmdoption-arg-wsb}})

\end{itemize}
\end{quote}


\subsubsection{CBLinput}
\label{\detokenize{input_files/CBL_input/CBLinput:id1}}\label{\detokenize{input_files/CBL_input/CBLinput::doc}}\label{\detokenize{input_files/CBL_input/CBLinput:cblinput}}\index{command line option!EntrainmentType}\index{EntrainmentType!command line option}

\begin{fulllineitems}
\phantomsection\label{\detokenize{input_files/CBL_input/CBLinput:cmdoption-arg-entrainmenttype}}\pysigline{\sphinxbfcode{\sphinxupquote{EntrainmentType}}\sphinxcode{\sphinxupquote{}}}~\begin{quote}\begin{description}
\item[{Requirement}] \leavevmode
Required

\item[{Description}] \leavevmode
Determines entrainment scheme. See Cleugh and Grimmond 2000 {[}16{]} for details.

\item[{Configuration}] \leavevmode

\begin{savenotes}\sphinxattablestart
\centering
\begin{tabular}[t]{|\X{20}{100}|\X{80}{100}|}
\hline
\sphinxstyletheadfamily 
Value
&\sphinxstyletheadfamily 
Comments
\\
\hline
1
&
Tennekes and Driedonks (1981) - Recommended
\\
\hline
2
&
McNaughton and Springs (1986)
\\
\hline
3
&
Rayner and Watson (1991)
\\
\hline
4
&
Tennekes (1973)
\\
\hline
\end{tabular}
\par
\sphinxattableend\end{savenotes}

\end{description}\end{quote}

\end{fulllineitems}

\index{command line option!QH\_Choice}\index{QH\_Choice!command line option}

\begin{fulllineitems}
\phantomsection\label{\detokenize{input_files/CBL_input/CBLinput:cmdoption-arg-qh-choice}}\pysigline{\sphinxbfcode{\sphinxupquote{QH\_Choice}}\sphinxcode{\sphinxupquote{}}}~\begin{quote}\begin{description}
\item[{Requirement}] \leavevmode
Required

\item[{Description}] \leavevmode
Determines QH used for CBL model.

\item[{Configuration}] \leavevmode

\begin{savenotes}\sphinxattablestart
\centering
\begin{tabular}[t]{|\X{20}{100}|\X{80}{100}|}
\hline
\sphinxstyletheadfamily 
Value
&\sphinxstyletheadfamily 
Comments
\\
\hline
1
&
QH modelled by SUEWS
\\
\hline
2
&
QH modelled by LUMPS
\\
\hline
3
&
Observed QH values are used from the meteorological input file
\\
\hline
\end{tabular}
\par
\sphinxattableend\end{savenotes}

\end{description}\end{quote}

\end{fulllineitems}

\index{command line option!InitialData\_use}\index{InitialData\_use!command line option}

\begin{fulllineitems}
\phantomsection\label{\detokenize{input_files/CBL_input/CBLinput:cmdoption-arg-initialdata-use}}\pysigline{\sphinxbfcode{\sphinxupquote{InitialData\_use}}\sphinxcode{\sphinxupquote{}}}~\begin{quote}\begin{description}
\item[{Requirement}] \leavevmode
Required

\item[{Description}] \leavevmode
Determines initial values (see CBL\_Initial\_data.txt)

\item[{Configuration}] \leavevmode

\begin{savenotes}\sphinxattablestart
\centering
\begin{tabular}[t]{|\X{20}{100}|\X{80}{100}|}
\hline
\sphinxstyletheadfamily 
Value
&\sphinxstyletheadfamily 
Comments
\\
\hline
0
&
All initial values are calculated. (Not available in current release.)
\\
\hline
1
&
Take zi0, gamt\_Km and gamq\_gkgm from input data file. Theta+\_K, q+\_gkg, Theta\_K and q\_gkg are calculated using Temp\_C, avrh and Pres\_kPa in meteorological input file.
\\
\hline
2
&
Take all initial values from input data file (see CBL\_Initial\_data.txt).
\\
\hline
\end{tabular}
\par
\sphinxattableend\end{savenotes}

\end{description}\end{quote}

\end{fulllineitems}

\index{command line option!Sondeflag}\index{Sondeflag!command line option}

\begin{fulllineitems}
\phantomsection\label{\detokenize{input_files/CBL_input/CBLinput:cmdoption-arg-sondeflag}}\pysigline{\sphinxbfcode{\sphinxupquote{Sondeflag}}\sphinxcode{\sphinxupquote{}}}~\begin{quote}\begin{description}
\item[{Requirement}] \leavevmode
Required

\item[{Description}] \leavevmode
to fill

\item[{Configuration}] \leavevmode

\begin{savenotes}\sphinxattablestart
\centering
\begin{tabular}[t]{|\X{20}{100}|\X{80}{100}|}
\hline
\sphinxstyletheadfamily 
Value
&\sphinxstyletheadfamily 
Comments
\\
\hline
0
&
Does not read radiosonde vertical profile data - recommended
\\
\hline
1
&
Reads radiosonde vertical profile data
\\
\hline
\end{tabular}
\par
\sphinxattableend\end{savenotes}

\end{description}\end{quote}

\end{fulllineitems}

\index{command line option!CBLday(id)}\index{CBLday(id)!command line option}

\begin{fulllineitems}
\phantomsection\label{\detokenize{input_files/CBL_input/CBLinput:cmdoption-arg-cblday-id}}\pysigline{\sphinxbfcode{\sphinxupquote{CBLday(id)}}\sphinxcode{\sphinxupquote{}}}~\begin{quote}\begin{description}
\item[{Requirement}] \leavevmode
Required

\item[{Description}] \leavevmode
Set CBLday(id) = 1 If CBL model is set to run for DOY 175\textendash{}177, CBLday(175) = 1, CBLday(176) = 1, CBLday(177) = 1

\item[{Configuration}] \leavevmode
to fill

\end{description}\end{quote}

\end{fulllineitems}

\index{command line option!CO2\_included}\index{CO2\_included!command line option}

\begin{fulllineitems}
\phantomsection\label{\detokenize{input_files/CBL_input/CBLinput:cmdoption-arg-co2-included}}\pysigline{\sphinxbfcode{\sphinxupquote{CO2\_included}}\sphinxcode{\sphinxupquote{}}}~\begin{quote}\begin{description}
\item[{Requirement}] \leavevmode
Required

\item[{Description}] \leavevmode
Set to zero in current version

\item[{Configuration}] \leavevmode
to fill

\end{description}\end{quote}

\end{fulllineitems}

\index{command line option!FileSonde(id)}\index{FileSonde(id)!command line option}

\begin{fulllineitems}
\phantomsection\label{\detokenize{input_files/CBL_input/CBLinput:cmdoption-arg-filesonde-id}}\pysigline{\sphinxbfcode{\sphinxupquote{FileSonde(id)}}\sphinxcode{\sphinxupquote{}}}~\begin{quote}\begin{description}
\item[{Requirement}] \leavevmode
Required

\item[{Description}] \leavevmode
If Sondeflag=1, write the file name including the path from site directory e.g. FileSonde(id)= ‘CBLinputfilesXXX.txt’, XXX is an arbitrary name.

\item[{Configuration}] \leavevmode
to fill

\end{description}\end{quote}

\end{fulllineitems}

\index{command line option!InitialDataFileName}\index{InitialDataFileName!command line option}

\begin{fulllineitems}
\phantomsection\label{\detokenize{input_files/CBL_input/CBLinput:cmdoption-arg-initialdatafilename}}\pysigline{\sphinxbfcode{\sphinxupquote{InitialDataFileName}}\sphinxcode{\sphinxupquote{}}}~\begin{quote}\begin{description}
\item[{Requirement}] \leavevmode
Required

\item[{Description}] \leavevmode
If InitialData\_use ≥ 1, write the file name including the path from site directory e.g. InitialDataFileName=’CBLinputfilesCBL\_initial\_data.txt’

\item[{Configuration}] \leavevmode
to fill

\end{description}\end{quote}

\end{fulllineitems}

\index{command line option!Wsb}\index{Wsb!command line option}

\begin{fulllineitems}
\phantomsection\label{\detokenize{input_files/CBL_input/CBLinput:cmdoption-arg-wsb}}\pysigline{\sphinxbfcode{\sphinxupquote{Wsb}}\sphinxcode{\sphinxupquote{}}}~\begin{quote}\begin{description}
\item[{Requirement}] \leavevmode
Required

\item[{Description}] \leavevmode
Subsidence velocity (m s -1 ) in eq. 1 and 2 of Onomura et al. (2015) {[}17{]} . (-0.01 m s -1 recommended)

\item[{Configuration}] \leavevmode
to fill

\end{description}\end{quote}

\end{fulllineitems}



\section{ESTM-related files}
\label{\detokenize{input_files/ESTM_related_files/ESTM_related_files:estm-related-files}}\label{\detokenize{input_files/ESTM_related_files/ESTM_related_files::doc}}

\subsection{SUEWS\_ESTMCoefficients.txt}
\label{\detokenize{input_files/ESTM_related_files/ESTM_related_files:suews-estmcoefficients-txt}}\label{\detokenize{input_files/ESTM_related_files/ESTM_related_files:id1}}
\sphinxstylestrong{Note ESTM is under development in v2017a and should not be used!}

The Element Surface Temperature Method (ESTM) (Offerle et al., 2005)
calculates the net storage heat flux from surface temperatures. In the
method the three-dimensional urban volume is reduced to four 1-d
elements (i.e. building roofs, walls, and internal mass and ground
(road, vegetation, etc)). The storage heat flux is calculated from the
heat conduction through the different elements. For the inside surfaces
of the roof and walls, and both surfaces for the internal mass
(ceilings/floors, internal walls), the surface temperature of the
element is determined by setting the conductive heat transfer out of (in
to) the surface equal to the radiative and convective heat losses
(gains). Each element (roof, wall, internal element and ground) can have
maximum five layers and each layer has three parameters tied to it:
thickness (x), thermal conductivity (k), volumetric heat capacity
(rhoCp).

If ESTM is used (QSchoice=4), the files
{\hyperref[\detokenize{input_files/ESTM_related_files/ESTM_related_files:suews-estmcoefficients-txt}]{\sphinxcrossref{\DUrole{std,std-ref,std,std-ref}{SUEWS\_ESTMCoefficients.txt}}}} (\autopageref*{\detokenize{input_files/ESTM_related_files/ESTM_related_files:suews-estmcoefficients-txt}}),
\sphinxcode{\sphinxupquote{ESTMinput.nml}} and
{\hyperref[\detokenize{input_files/ESTM_related_files/ESTM_related_files:ssss-yyyy-estm-ts-data-tt-txt}]{\sphinxcrossref{\DUrole{std,std-ref,std,std-ref}{SSss\_YYYY\_ESTM\_Ts\_data\_tt.txt}}}} (\autopageref*{\detokenize{input_files/ESTM_related_files/ESTM_related_files:ssss-yyyy-estm-ts-data-tt-txt}}) should be
prepared.

SUEWS\_ESTMCoefficients.txt contains the parameters for the layers of
each of the elements (roofs, wall, ground, internal mass).
\begin{itemize}
\item {} 
If less than five layers are used, the parameters for unused layers
should be set to -999.

\item {} 
The ESTM coefficients with the prefix \sphinxstyleemphasis{Surf\_} must be specified for
each surface type (plus snow) but the \sphinxstyleemphasis{Wall\_} and \sphinxstyleemphasis{Internal\_}
variables apply to the building surfaces only.

\item {} 
For each grid, one set of ESTM coefficients must be specified for
each surface type; for paved and building surfaces it is possible to
specify up to three and five sets of coefficients per grid (e.g. to
represent different building materials) using the relevant columns in
{\hyperref[\detokenize{input_files/SUEWS_SiteInfo/SUEWS_SiteSelect:suews-siteselect-txt}]{\sphinxcrossref{\DUrole{std,std-ref,std,std-ref}{SUEWS\_SiteSelect.txt}}}} (\autopageref*{\detokenize{input_files/SUEWS_SiteInfo/SUEWS_SiteSelect:suews-siteselect-txt}}). For the model to
use these columns in site select, the ESTMCode column in
{\hyperref[\detokenize{input_files/SUEWS_SiteInfo/SUEWS_NonVeg:suews-nonveg-txt}]{\sphinxcrossref{\DUrole{std,std-ref,std,std-ref}{SUEWS\_NonVeg.txt}}}} (\autopageref*{\detokenize{input_files/SUEWS_SiteInfo/SUEWS_NonVeg:suews-nonveg-txt}}) should be set to zero.

\end{itemize}

\sphinxstylestrong{Note ESTM is under development in v2017a and should not be used!}

The following input files are required if ESTM is used to calculate the
storage heat flux.


\subsection{ESTMinput.nml}
\label{\detokenize{input_files/ESTM_related_files/ESTM_related_files:estminput-nml}}
ESTMinput.nml specifies the model settings and default values.

A sample file of \sphinxstylestrong{ESTMinput.nml} looks like

\fvset{hllines={, ,}}%
\begin{sphinxVerbatim}[commandchars=\\\{\}]
\PYGZam{}ESTMinput
TsurfChoice= 0
evolveTibld= 0      ! !!!!!FO!!!!! 0 originally
ibldCHmod  = 0
LBC\PYGZus{}soil   = 13.00             !!FO!! 4, 8 or 17 degC \PYGZhy{} could be set as the annual mean air temperature (12.8 degC for London)
THEAT\PYGZus{}ON   = 18.
THEAT\PYGZus{}OFF  = 22.
THEAT\PYGZus{}FIX  = 19.
/
\end{sphinxVerbatim}

\begin{sphinxadmonition}{note}{Note:}
The file contents can be in any order.
\end{sphinxadmonition}

The parameters and their setting instructions
are provided through {\hyperref[\detokenize{input_files/ESTM_related_files/ESTMinput:estminput}]{\sphinxcrossref{\DUrole{std,std-ref}{the links below}}}} (\autopageref*{\detokenize{input_files/ESTM_related_files/ESTMinput:estminput}}):
\begin{itemize}\setlength{\itemsep}{0pt}\setlength{\parskip}{0pt}
\item {} 
{\hyperref[\detokenize{input_files/ESTM_related_files/ESTMinput:cmdoption-arg-tsurfchoice}]{\sphinxcrossref{\sphinxcode{\sphinxupquote{TsurfChoice}}}}} (\autopageref*{\detokenize{input_files/ESTM_related_files/ESTMinput:cmdoption-arg-tsurfchoice}})

\item {} 
{\hyperref[\detokenize{input_files/ESTM_related_files/ESTMinput:cmdoption-arg-evolvetibld}]{\sphinxcrossref{\sphinxcode{\sphinxupquote{evolveTibld}}}}} (\autopageref*{\detokenize{input_files/ESTM_related_files/ESTMinput:cmdoption-arg-evolvetibld}})

\item {} 
{\hyperref[\detokenize{input_files/ESTM_related_files/ESTMinput:cmdoption-arg-ibldchmod}]{\sphinxcrossref{\sphinxcode{\sphinxupquote{IbldCHmod}}}}} (\autopageref*{\detokenize{input_files/ESTM_related_files/ESTMinput:cmdoption-arg-ibldchmod}})

\item {} 
{\hyperref[\detokenize{input_files/ESTM_related_files/ESTMinput:cmdoption-arg-lbc-soil}]{\sphinxcrossref{\sphinxcode{\sphinxupquote{LBC\_soil}}}}} (\autopageref*{\detokenize{input_files/ESTM_related_files/ESTMinput:cmdoption-arg-lbc-soil}})

\item {} 
{\hyperref[\detokenize{input_files/ESTM_related_files/ESTMinput:cmdoption-arg-theat-fix}]{\sphinxcrossref{\sphinxcode{\sphinxupquote{Theat\_fix}}}}} (\autopageref*{\detokenize{input_files/ESTM_related_files/ESTMinput:cmdoption-arg-theat-fix}})

\item {} 
{\hyperref[\detokenize{input_files/ESTM_related_files/ESTMinput:cmdoption-arg-theat-off}]{\sphinxcrossref{\sphinxcode{\sphinxupquote{Theat\_off}}}}} (\autopageref*{\detokenize{input_files/ESTM_related_files/ESTMinput:cmdoption-arg-theat-off}})

\item {} 
{\hyperref[\detokenize{input_files/ESTM_related_files/ESTMinput:cmdoption-arg-theat-on}]{\sphinxcrossref{\sphinxcode{\sphinxupquote{Theat\_on}}}}} (\autopageref*{\detokenize{input_files/ESTM_related_files/ESTMinput:cmdoption-arg-theat-on}})

\end{itemize}


\subsubsection{ESTMinput}
\label{\detokenize{input_files/ESTM_related_files/ESTMinput::doc}}\label{\detokenize{input_files/ESTM_related_files/ESTMinput:estminput}}\label{\detokenize{input_files/ESTM_related_files/ESTMinput:id1}}\index{command line option!TsurfChoice}\index{TsurfChoice!command line option}

\begin{fulllineitems}
\phantomsection\label{\detokenize{input_files/ESTM_related_files/ESTMinput:cmdoption-arg-tsurfchoice}}\pysigline{\sphinxbfcode{\sphinxupquote{TsurfChoice}}\sphinxcode{\sphinxupquote{}}}~\begin{quote}\begin{description}
\item[{Requirement}] \leavevmode
Required

\item[{Description}] \leavevmode
Source of surface temperature data used.

\item[{Configuration}] \leavevmode

\begin{savenotes}\sphinxattablestart
\centering
\begin{tabular}[t]{|\X{20}{100}|\X{80}{100}|}
\hline
\sphinxstyletheadfamily 
Value
&\sphinxstyletheadfamily 
Comments
\\
\hline
0
&
{\color{red}\bfseries{}*}Tsurf in SSss\_YYYY\_ESTM\_Ts\_data\_tt.txt used for all surface elements.
\\
\hline
1
&
Input surface temperature are different for ground, roof and wall.
\\
\hline
2
&
Wall surface temperature is different for four directions.
\\
\hline
\end{tabular}
\par
\sphinxattableend\end{savenotes}

\end{description}\end{quote}

\end{fulllineitems}

\index{command line option!evolveTibld}\index{evolveTibld!command line option}

\begin{fulllineitems}
\phantomsection\label{\detokenize{input_files/ESTM_related_files/ESTMinput:cmdoption-arg-evolvetibld}}\pysigline{\sphinxbfcode{\sphinxupquote{evolveTibld}}\sphinxcode{\sphinxupquote{}}}~\begin{quote}\begin{description}
\item[{Requirement}] \leavevmode
Required

\item[{Description}] \leavevmode
Source of internal building temperature (Tibld)

\item[{Configuration}] \leavevmode

\begin{savenotes}\sphinxattablestart
\centering
\begin{tabular}[t]{|\X{20}{100}|\X{80}{100}|}
\hline
\sphinxstyletheadfamily 
Value
&\sphinxstyletheadfamily 
Comments
\\
\hline
0
&
{\color{red}\bfseries{}*}Tiair in SSss\_YYYY\_ESTM\_Ts\_data\_tt.txt used.
\\
\hline
1
&
{\color{red}\bfseries{}*}Tibld calculated considering the effect of anthropogenic heat from HVAC
\\
\hline
2
&
{\color{red}\bfseries{}*}Tibld calculated without considering the influence of HVAC.
\\
\hline
\end{tabular}
\par
\sphinxattableend\end{savenotes}

\end{description}\end{quote}

\end{fulllineitems}

\index{command line option!IbldCHmod}\index{IbldCHmod!command line option}

\begin{fulllineitems}
\phantomsection\label{\detokenize{input_files/ESTM_related_files/ESTMinput:cmdoption-arg-ibldchmod}}\pysigline{\sphinxbfcode{\sphinxupquote{IbldCHmod}}\sphinxcode{\sphinxupquote{}}}~\begin{quote}\begin{description}
\item[{Requirement}] \leavevmode
Required

\item[{Description}] \leavevmode
Method to calculate internal convective heat exchange coefficients (CH) for internal building, wall and roof if evolveTibld is 1 or 2.

\item[{Configuration}] \leavevmode

\begin{savenotes}\sphinxattablestart
\centering
\begin{tabular}[t]{|\X{20}{100}|\X{80}{100}|}
\hline
\sphinxstyletheadfamily 
Value
&\sphinxstyletheadfamily 
Comments
\\
\hline
0
&
CHs are read from SUEWS\_ESTMcoefficients.txt.
\\
\hline
1
&
CHs are calculated based on ASHRAE (2001)
\\
\hline
2
&
CHs are calculated based on Awbi (1998).
\\
\hline
\end{tabular}
\par
\sphinxattableend\end{savenotes}

\end{description}\end{quote}

\end{fulllineitems}

\index{command line option!LBC\_soil}\index{LBC\_soil!command line option}

\begin{fulllineitems}
\phantomsection\label{\detokenize{input_files/ESTM_related_files/ESTMinput:cmdoption-arg-lbc-soil}}\pysigline{\sphinxbfcode{\sphinxupquote{LBC\_soil}}\sphinxcode{\sphinxupquote{}}}~\begin{quote}\begin{description}
\item[{Requirement}] \leavevmode
Required

\item[{Description}] \leavevmode
Soil temperature at lowest boundary condition {[}˚C{]}

\item[{Configuration}] \leavevmode
to fill

\end{description}\end{quote}

\end{fulllineitems}

\index{command line option!Theat\_fix}\index{Theat\_fix!command line option}

\begin{fulllineitems}
\phantomsection\label{\detokenize{input_files/ESTM_related_files/ESTMinput:cmdoption-arg-theat-fix}}\pysigline{\sphinxbfcode{\sphinxupquote{Theat\_fix}}\sphinxcode{\sphinxupquote{}}}~\begin{quote}\begin{description}
\item[{Requirement}] \leavevmode
Required

\item[{Description}] \leavevmode
Ideal internal building temperature {[}˚C{]}

\item[{Configuration}] \leavevmode
to fill

\end{description}\end{quote}

\end{fulllineitems}

\index{command line option!Theat\_off}\index{Theat\_off!command line option}

\begin{fulllineitems}
\phantomsection\label{\detokenize{input_files/ESTM_related_files/ESTMinput:cmdoption-arg-theat-off}}\pysigline{\sphinxbfcode{\sphinxupquote{Theat\_off}}\sphinxcode{\sphinxupquote{}}}~\begin{quote}\begin{description}
\item[{Requirement}] \leavevmode
Required

\item[{Description}] \leavevmode
Temperature at which heat control is turned off (used when evolveTibld=1) {[}˚C{]}

\item[{Configuration}] \leavevmode
to fill

\end{description}\end{quote}

\end{fulllineitems}

\index{command line option!Theat\_on}\index{Theat\_on!command line option}

\begin{fulllineitems}
\phantomsection\label{\detokenize{input_files/ESTM_related_files/ESTMinput:cmdoption-arg-theat-on}}\pysigline{\sphinxbfcode{\sphinxupquote{Theat\_on}}\sphinxcode{\sphinxupquote{}}}~\begin{quote}\begin{description}
\item[{Requirement}] \leavevmode
Required

\item[{Description}] \leavevmode
Temperature at which heat control is turned on (used when evolveTibld =1) {[}˚C{]}

\item[{Configuration}] \leavevmode
to fill

\end{description}\end{quote}

\end{fulllineitems}



\subsection{SSss\_YYYY\_ESTM\_Ts\_data\_tt.txt}
\label{\detokenize{input_files/ESTM_related_files/ESTM_related_files:id2}}\label{\detokenize{input_files/ESTM_related_files/ESTM_related_files:ssss-yyyy-estm-ts-data-tt-txt}}
{\hyperref[\detokenize{input_files/ESTM_related_files/ESTM_related_files:ssss-yyyy-estm-ts-data-tt-txt}]{\sphinxcrossref{\DUrole{std,std-ref,std,std-ref}{SSss\_YYYY\_ESTM\_Ts\_data\_tt.txt}}}} (\autopageref*{\detokenize{input_files/ESTM_related_files/ESTM_related_files:ssss-yyyy-estm-ts-data-tt-txt}}) contains a time-series of input surface
temperature for roof, wall, ground and internal elements.


\begin{savenotes}\sphinxattablestart
\centering
\begin{tabulary}{\linewidth}[t]{|T|T|T|T|}
\hline
\sphinxstyletheadfamily 
No.
&\sphinxstyletheadfamily 
Column Name
&\sphinxstyletheadfamily 
Use
&\sphinxstyletheadfamily 
Description
\\
\hline
1
&
\sphinxcode{\sphinxupquote{iy}}
&
{\hyperref[\detokenize{notation:term-mu}]{\sphinxtermref{\sphinxcode{\sphinxupquote{MU}}}}}
&
Year {[}YYYY{]}
\\
\hline
2
&
\sphinxcode{\sphinxupquote{id}}
&
{\hyperref[\detokenize{notation:term-mu}]{\sphinxtermref{\sphinxcode{\sphinxupquote{MU}}}}}
&
Day of year {[}DOY{]}
\\
\hline
3
&
\sphinxcode{\sphinxupquote{it}}
&
{\hyperref[\detokenize{notation:term-mu}]{\sphinxtermref{\sphinxcode{\sphinxupquote{MU}}}}}
&
Hour {[}H{]}
\\
\hline
4
&
\sphinxcode{\sphinxupquote{imin}}
&
{\hyperref[\detokenize{notation:term-mu}]{\sphinxtermref{\sphinxcode{\sphinxupquote{MU}}}}}
&
Minute {[}M{]}
\\
\hline
5
&
\sphinxcode{\sphinxupquote{Tiair}}
&
{\hyperref[\detokenize{notation:term-mu}]{\sphinxtermref{\sphinxcode{\sphinxupquote{MU}}}}}
&
Indoor air temperature {[}˚C{]}
\\
\hline
6
&
\sphinxcode{\sphinxupquote{Tsurf}}
&
{\hyperref[\detokenize{notation:term-mu}]{\sphinxtermref{\sphinxcode{\sphinxupquote{MU}}}}}
&
Bulk surface temperature {[}˚C{]} (used when TsurfCoice = 0)
\\
\hline
7
&
\sphinxcode{\sphinxupquote{Troof}}
&
{\hyperref[\detokenize{notation:term-mu}]{\sphinxtermref{\sphinxcode{\sphinxupquote{MU}}}}}
&
Roof surface temperature {[}˚C{]} (used when TsurfChoice = 1 or 2)
\\
\hline
8
&
\sphinxcode{\sphinxupquote{Troad}}
&
{\hyperref[\detokenize{notation:term-mu}]{\sphinxtermref{\sphinxcode{\sphinxupquote{MU}}}}}
&
Ground surface temperature {[}˚C{]} (used when TsurfChoice = 1 or 2)
\\
\hline
9
&
\sphinxcode{\sphinxupquote{Twall}}
&
{\hyperref[\detokenize{notation:term-mu}]{\sphinxtermref{\sphinxcode{\sphinxupquote{MU}}}}}
&
Wall surface temperature {[}˚C{]} (used when TsurfChoice = 1)
\\
\hline
10
&
\sphinxcode{\sphinxupquote{Twall\_n}}
&
{\hyperref[\detokenize{notation:term-mu}]{\sphinxtermref{\sphinxcode{\sphinxupquote{MU}}}}}
&
North-facing wall surface temperature {[}˚C{]} (used when TsurfChoice = 2)
\\
\hline
11
&
\sphinxcode{\sphinxupquote{Twall\_e}}
&
{\hyperref[\detokenize{notation:term-mu}]{\sphinxtermref{\sphinxcode{\sphinxupquote{MU}}}}}
&
East-facing wall surface temperature {[}˚C{]} (used when TsurfChoice = 2)
\\
\hline
12
&
\sphinxcode{\sphinxupquote{Twall\_s}}
&
{\hyperref[\detokenize{notation:term-mu}]{\sphinxtermref{\sphinxcode{\sphinxupquote{MU}}}}}
&
South-facing wall surface temperature {[}˚C{]} (used when TsurfChoice = 2)
\\
\hline
13
&
\sphinxcode{\sphinxupquote{Twall\_w}}
&
{\hyperref[\detokenize{notation:term-mu}]{\sphinxtermref{\sphinxcode{\sphinxupquote{MU}}}}}
&
West-facing wall surface temperature {[}˚C{]} (used when TsurfChoice = 2)
\\
\hline
\end{tabulary}
\par
\sphinxattableend\end{savenotes}


\section{SOLWEIG input files}
\label{\detokenize{input_files/SOLWEIG_input:solweig-input-files}}\label{\detokenize{input_files/SOLWEIG_input::doc}}
If the SOLWEIG model option is used (SOLWEIGout=1), spatial data and a
SOLWEIGInput.nml file need to be prepared. The Digital Surface Models
(DSMs) as well as derivatives originating from DSMs, e.g. Sky View
Factors (SVF) must have the same spatial resolution and extent. Since
SOLWEIG is a 2D model it will considerably increase computation time and
should be used with care.

Description of choices in SOLWEIGinput\_file.nml file. The file can be in
any order.


\begin{savenotes}\sphinxatlongtablestart\begin{longtable}{|*{3}{\X{1}{3}|}}
\hline
\sphinxstyletheadfamily 
Name
&\sphinxstyletheadfamily 
Units
&\sphinxstyletheadfamily 
Description
\\
\hline
\endfirsthead

\multicolumn{3}{c}%
{\makebox[0pt]{\sphinxtablecontinued{\tablename\ \thetable{} -- continued from previous page}}}\\
\hline
\sphinxstyletheadfamily 
Name
&\sphinxstyletheadfamily 
Units
&\sphinxstyletheadfamily 
Description
\\
\hline
\endhead

\hline
\multicolumn{3}{r}{\makebox[0pt][r]{\sphinxtablecontinued{Continued on next page}}}\\
\endfoot

\endlastfoot
&&\\
\hline
Posture
&\begin{itemize}
\item {} 
\end{itemize}
&
Determines the
posture of a human
for which the radiant
fluxes should be
considered
\\
\hline
1
&
Standing (default)
&\\
\hline
2
&
Sitting
&\\
\hline
absL
&\begin{itemize}
\item {} 
\end{itemize}
&
Absorption
coefficient of
longwave radiation of
a person.
-  Recommended value:
0.97
\\
\hline
absK
&\begin{itemize}
\item {} 
\end{itemize}
&
Absorption
coefficient of
shortwave radiation
of a person.
-  Recommended value:
0.70
\\
\hline
heightgravity
&
m
&
Centre of gravity for
a person.
-  Recommended value
for a standing
man: 1.1 m
\\
\hline
usevegdem
&\begin{itemize}
\item {} 
\end{itemize}
&
Vegetation scheme
\\
\hline
1
&
Vegetation scheme is
active (Lindberg and
Grimmond 2011 \phantomsection\label{\detokenize{input_files/SOLWEIG_input:id1}}{\hyperref[\detokenize{references:fl2011}]{\sphinxcrossref{{[}FL2011{]}}}} (\autopageref*{\detokenize{references:fl2011}}))
&\\
\hline
2
&
No vegetation scheme
used
&\\
\hline
DSMPath
&\begin{itemize}
\item {} 
\end{itemize}
&
Path to Digital
Surface Models (DSM).
\\
\hline
DSMname
&\begin{itemize}
\item {} 
\end{itemize}
&
Ground and Building
DSM
\\
\hline
CDSMname
&\begin{itemize}
\item {} 
\end{itemize}
&
Vegetation canopy DSM
\\
\hline
TDSMname
&\begin{itemize}
\item {} 
\end{itemize}
&
Vegetation trunk zone
DSM
\\
\hline
TransMin
&\begin{itemize}
\item {} 
\end{itemize}
&
Tranmissivity of K
through deciduous
vegetation (leaf on)
-  Recommended value:
0.02 (Konarska et
al. 2014 \phantomsection\label{\detokenize{input_files/SOLWEIG_input:id2}}{\hyperref[\detokenize{references:ko14}]{\sphinxcrossref{{[}Ko14{]}}}} (\autopageref*{\detokenize{references:ko14}}))
\\
\hline
TransMax
&\begin{itemize}
\item {} 
\end{itemize}
&
Tranmissivity of K
through deciduous
vegetation (leaf off)
-  Recommended value:
0.50 (Konarska et
al. 2014 \phantomsection\label{\detokenize{input_files/SOLWEIG_input:id3}}{\hyperref[\detokenize{references:ko14}]{\sphinxcrossref{{[}Ko14{]}}}} (\autopageref*{\detokenize{references:ko14}}))
\\
\hline
SVFPath
&\begin{itemize}
\item {} 
\end{itemize}
&
Path to SVFs matrices
(See Lindberg and
Grimmond
(2011) \phantomsection\label{\detokenize{input_files/SOLWEIG_input:id4}}{\hyperref[\detokenize{references:fl2011}]{\sphinxcrossref{{[}FL2011{]}}}} (\autopageref*{\detokenize{references:fl2011}}) for
details)
\\
\hline
SVFSuffix
&\begin{itemize}
\item {} 
\end{itemize}
&
Suffix used (if any)
\\
\hline
BuildingName
&\begin{itemize}
\item {} 
\end{itemize}
&
Boolean matrix for
locations of building
pixels
\\
\hline
row
&\begin{itemize}
\item {} 
\end{itemize}
&
X coordinate for
point of interest.
Here all variables
from the model will
written to
SOLWEIGpoiOUT.txt
\\
\hline
col
&\begin{itemize}
\item {} 
\end{itemize}
&
Y coordinate for
point of interest.
Here all variables
from the model will
written to
SOLWEIGpoiOUT.txt
\\
\hline
onlyglobal
&\begin{itemize}
\item {} 
\end{itemize}
&
Global radiation
\\
\hline
0
&
Diffuse and direct
shortwave radiation
taken from met
forcing file.
&\\
\hline
1
&
Diffuse and direct
shortwave radiation
calculated from
Reindl et al.
(1990) \phantomsection\label{\detokenize{input_files/SOLWEIG_input:id5}}{\hyperref[\detokenize{references:re90}]{\sphinxcrossref{{[}Re90{]}}}} (\autopageref*{\detokenize{references:re90}})
&\\
\hline
SOLWEIGpoi\_out
&\begin{itemize}
\item {} 
\end{itemize}
&
Write output
variables at point of
interest (see below)
\\
\hline
0
&
No POI output
&\\
\hline
Tmrt\_out
&\begin{itemize}
\item {} 
\end{itemize}
&\\
\hline
0
&
No grid output
&\\
\hline
1
&
Write grid to file
(saves as ERSI Ascii
grid)
&\\
\hline
Lup2d\_out
&\begin{itemize}
\item {} 
\end{itemize}
&\\
\hline
0
&
No grid output
&\\
\hline
1
&
Write grid to file
(saves as ERSI Ascii
grid)
&\\
\hline
Ldown2d\_out
&\begin{itemize}
\item {} 
\end{itemize}
&\\
\hline
0
&
No grid output
&\\
\hline
1
&
Write grid to file
(saves as ERSI Ascii
grid)
&\\
\hline
Kup2d\_out
&\begin{itemize}
\item {} 
\end{itemize}
&\\
\hline
0
&
No grid output
&\\
\hline
1
&
Write grid to file
(saves as ERSI Ascii
grid)
&\\
\hline
Kdown2d\_out
&\begin{itemize}
\item {} 
\end{itemize}
&\\
\hline
0
&
No grid output
&\\
\hline
1
&
Write grid to file
(saves as ERSI Ascii
grid)
&\\
\hline
GVF\_out
&\begin{itemize}
\item {} 
\end{itemize}
&\\
\hline
0
&
No grid output
&\\
\hline
1
&
Write grid to file
(saves as ERSI Ascii
grid)
&\\
\hline
SOLWEIG\_ldown
&\begin{itemize}
\item {} 
\end{itemize}
&\\
\hline
0
&
Not active (use SUEWS
to estimate Ldown
above canyon)
&\\
\hline
1
&
Use SOLWEIG to
estimate Ldown above
canyon
&\\
\hline
OutInterval
&
min
&
Output interval. Set
to 60 in current
version.
\\
\hline
RunForGrid
&\begin{itemize}
\item {} 
\end{itemize}
&
Grid for which
SOLWEIG should be
run.
\\
\hline
-999
&
All grids (use with
care)
&\\
\hline&&\\
\hline
\end{longtable}\sphinxatlongtableend\end{savenotes}


\chapter{Output files}
\label{\detokenize{output_files/output_files:output-files}}\label{\detokenize{output_files/output_files::doc}}\label{\detokenize{output_files/output_files:id1}}

\section{Runtime diagnostic information}
\label{\detokenize{output_files/output_files:runtime-diagnostic-information}}

\subsection{Error messages: problems.txt}
\label{\detokenize{output_files/output_files:error-messages-problems-txt}}
see this {\hyperref[\detokenize{output_files/output_files:output-files}]{\sphinxcrossref{\DUrole{std,std-ref}{Output files}}}} (\autopageref*{\detokenize{output_files/output_files:output-files}})

If there are problems running the program serious error messages will be
written to problems.txt.
\begin{itemize}
\item {} 
Serious problems will usually cause the program to stop after writing
the error message. If this is the case, the last line of problems.txt
will contain a non-zero number (the error code).

\item {} 
If the program runs successfully, problems.txt file ends with:

\fvset{hllines={, ,}}%
\begin{sphinxVerbatim}[commandchars=\\\{\}]
\PYG{n}{Run} \PYG{n}{completed}\PYG{o}{.}
\PYG{l+m+mi}{0}
\end{sphinxVerbatim}

\end{itemize}

SUEWS has a large number of error messages included to try to capture
common errors to help the user determine what the problem is. If you
encounter an error that does not provide an error message please capture
the details so we can hopefully provide better error messages in future.

See {\hyperref[\detokenize{output_files/output_files:Troubleshooting}]{\emph{Troubleshooting}}} (\autopageref*{\detokenize{output_files/output_files:Troubleshooting}}) section for help solving
problems. If the file paths are not correct the program will return an
error when run (see {\hyperref[\detokenize{output_files/output_files:Preparing_to_run_the_model}]{\emph{Preparing to run the
model}}} (\autopageref*{\detokenize{output_files/output_files:Preparing_to_run_the_model}})).


\subsection{Warning messages: warnings.txt}
\label{\detokenize{output_files/output_files:warning-messages-warnings-txt}}\begin{itemize}
\item {} 
If the program encounters a more minor issue it will not stop but a
warning may be written to warnings.txt. It is advisable to check the
warnings to ensure there is not a more serious problem.

\item {} 
The warnings.txt file can be large (over several GBs) given warning
messages are written out during a large scale simulation, you can use
\sphinxcode{\sphinxupquote{tail}}/\sphinxcode{\sphinxupquote{head}} to view the ending/starting part without opening
the whole file on Unix-like systems (Linux/mac OS), which may slow
down your system.

\item {} 
To prevent warnings.txt from being written, set \sphinxstylestrong{SuppressWarnings}
to 1 in {\hyperref[\detokenize{output_files/output_files:RunControl.nml}]{\emph{RunControl.nml}}} (\autopageref*{\detokenize{output_files/output_files:RunControl.nml}}).

\item {} 
Warning messages are usually written with a grid number, timestamp
and error count. If the problem occurs in the initial stages (i.e.
before grid numbers and timestamps are assigned, these are printed as
00000).

\end{itemize}


\subsection{Summary of model parameters: SS\_FileChoices.txt}
\label{\detokenize{output_files/output_files:summary-of-model-parameters-ss-filechoices-txt}}
For each run, the model parameters specified in the input files are
written out to the file SS\_FileChoices.txt.


\section{Model output files}
\label{\detokenize{output_files/output_files:model-output-files}}

\subsection{SSss\_YYYY\_TT.txt}
\label{\detokenize{output_files/output_files:ssss-yyyy-tt-txt}}
SUEWS produces the main output file (SSss\_YYYY\_tt.txt) with time
resolution (TT min) set by \sphinxstylestrong{ResolutionFilesOut} in
{\hyperref[\detokenize{output_files/output_files:RunControl}]{\emph{RunControl}}} (\autopageref*{\detokenize{output_files/output_files:RunControl}}).

Before these main data files are written out, SUEWS provides a summary
of the column names, units and variables included in the file
Ss\_YYYY\_TT\_OutputFormat.txt (one file per run).

The variables included in the main output file are determined according
to {\hyperref[\detokenize{input_files/RunControl/File_related_options:cmdoption-arg-writeoutoption}]{\sphinxcrossref{\sphinxcode{\sphinxupquote{WriteOutOption}}}}} (\autopageref*{\detokenize{input_files/RunControl/File_related_options:cmdoption-arg-writeoutoption}}) set in {\hyperref[\detokenize{input_files/RunControl/RunControl:runcontrol-nml}]{\sphinxcrossref{\DUrole{std,std-ref}{RunControl.nml}}}} (\autopageref*{\detokenize{input_files/RunControl/RunControl:runcontrol-nml}}).


\begin{savenotes}\sphinxatlongtablestart\begin{longtable}{|l|l|l|l|}
\hline
\sphinxstyletheadfamily 
Column
&\sphinxstyletheadfamily 
Name
&\sphinxstyletheadfamily 
WriteOutOption
&\sphinxstyletheadfamily 
Description
\\
\hline
\endfirsthead

\multicolumn{4}{c}%
{\makebox[0pt]{\sphinxtablecontinued{\tablename\ \thetable{} -- continued from previous page}}}\\
\hline
\sphinxstyletheadfamily 
Column
&\sphinxstyletheadfamily 
Name
&\sphinxstyletheadfamily 
WriteOutOption
&\sphinxstyletheadfamily 
Description
\\
\hline
\endhead

\hline
\multicolumn{4}{r}{\makebox[0pt][r]{\sphinxtablecontinued{Continued on next page}}}\\
\endfoot

\endlastfoot

1
&
Year
&
0,1,2
&
Year {[}YYYY{]}
\\
\hline
2
&
DOY
&
0,1,2
&
Day of year {[}DOY{]}
\\
\hline
3
&
Hour
&
0,1,2
&
Hour {[}H{]}
\\
\hline
4
&
Min
&
0,1,2
&
Minute {[}M{]}
\\
\hline
5
&
Dectime
&
0,1,2
&
Decimal time {[}-{]}
\\
\hline
6
&
Kdown
&
0,1,2
&
Incoming shortwave radiation {[}W m -2 {]}
\\
\hline
7
&
Kup
&
0,1,2
&
Outgoing shortwave radiation {[}W m -2 {]}
\\
\hline
8
&
Ldown
&
0,1,2
&
Incoming longwave radiation {[}W m -2 {]}
\\
\hline
9
&
Lup
&
0,1,2
&
Outgoing longwave radiation {[}W m -2 {]}
\\
\hline
10
&
Tsurf
&
0,1,2
&
Bulk surface temperature {[}°C{]}
\\
\hline
11
&
QN
&
0,1,2
&
Net all-wave radiation {[}W m -2 {]}
\\
\hline
12
&
QF
&
0,1,2
&
Anthropogenic heat flux {[}W m -2 {]}
\\
\hline
13
&
QS
&
0,1,2
&
Storage heat flux {[}W m -2 {]}
\\
\hline
14
&
QH
&
0,1,2
&
Sensible heat flux (calculated using SUEWS) {[}W m -2 {]}
\\
\hline
15
&
QE
&
0,1,2
&
Latent heat flux (calculated using SUEWS) {[}W m -2 {]}
\\
\hline
16
&
QHlumps
&
0,1
&
Sensible heat flux (calculated using LUMPS) {[}W m -2 {]}
\\
\hline
17
&
QElumps
&
0,1
&
Latent heat flux (calculated using LUMPS) {[}W m -2 {]}
\\
\hline
18
&
QHresis
&
0,1
&
Sensible heat flux (calculated using resistance method) {[}W m -2 {]} Do not use in v2017b!
\\
\hline
19
&
Rain
&
0,1,2
&
Rain {[}mm{]}
\\
\hline
20
&
Irr
&
0,1,2
&
Irrigation {[}mm{]}
\\
\hline
21
&
Evap
&
0,1,2
&
Evaporation {[}mm{]}
\\
\hline
22
&
RO
&
0,1,2
&
Runoff {[}mm{]}
\\
\hline
23
&
TotCh
&
0,1,2
&
Change in surface and soil moisture stores {[}mm{]}
\\
\hline
24
&
SurfCh
&
0,1,2
&
Change in surface moisture store {[}mm{]}
\\
\hline
25
&
State
&
0,1,2
&
Surface wetness state {[}mm{]}
\\
\hline
26
&
NWtrState
&
0,1,2
&
Surface wetness state (for non-water surfaces) {[}mm{]}
\\
\hline
27
&
Drainage
&
0,1,2
&
Drainage {[}mm{]}
\\
\hline
28
&
SMD
&
0,1,2
&
Soil moisture deficit {[}mm{]}
\\
\hline
29
&
FlowCh
&
0,1
&
Additional flow into water body {[}mm{]}
\\
\hline
30
&
AddWater
&
0,1
&
Additional water flow received from other grids {[}mm{]}
\\
\hline
31
&
ROSoil
&
0,1
&
Runoff to soil (sub-surface) {[}mm{]}
\\
\hline
32
&
ROPipe
&
0,1
&
Runoff to pipes {[}mm{]}
\\
\hline
33
&
ROImp
&
0,1
&
Above ground runoff over impervious surfaces {[}mm{]}
\\
\hline
34
&
ROVeg
&
0,1
&
Above ground runoff over vegetated surfaces {[}mm{]}
\\
\hline
35
&
ROWater
&
0,1
&
Runoff for water body {[}mm{]}
\\
\hline
36
&
WUInt
&
0,1
&
Internal water use {[}mm{]}
\\
\hline
37
&
WUEveTr
&
0,1
&
Water use for irrigation of evergreen trees {[}mm{]}
\\
\hline
38
&
WUDecTr
&
0,1
&
Water use for irrigation of deciduous trees {[}mm{]}
\\
\hline
39
&
WUGrass
&
0,1
&
Water use for irrigation of grass {[}mm{]}
\\
\hline
40
&
SMDPaved
&
0,1
&
Soil moisture deficit for paved surface {[}mm{]}
\\
\hline
41
&
SMDBldgs
&
0,1
&
Soil moisture deficit for building surface {[}mm{]}
\\
\hline
42
&
SMDEveTr
&
0,1
&
Soil moisture deficit for evergreen surface {[}mm{]}
\\
\hline
43
&
SMDDecTr
&
0,1
&
Soil moisture deficit for deciduous surface {[}mm{]}
\\
\hline
44
&
SMDGrass
&
0,1
&
Soil moisture deficit for grass surface {[}mm{]}
\\
\hline
45
&
SMDBSoil
&
0,1
&
Soil moisture deficit for bare soil surface {[}mm{]}
\\
\hline
46
&
StPaved
&
0,1
&
Surface wetness state for paved surface {[}mm{]}
\\
\hline
47
&
StBldgs
&
0,1
&
Surface wetness state for building surface {[}mm{]}
\\
\hline
48
&
StEveTr
&
0,1
&
Surface wetness state for evergreen tree surface {[}mm{]}
\\
\hline
49
&
StDecTr
&
0,1
&
Surface wetness state for deciduous tree surface {[}mm{]}
\\
\hline
50
&
StGrass
&
0,1
&
Surface wetness state for grass surface {[}mm{]}
\\
\hline
51
&
StBSoil
&
0,1
&
Surface wetness state for bare soil surface {[}mm{]}
\\
\hline
52
&
StWater
&
0,1
&
Surface wetness state for water surface {[}mm{]}
\\
\hline
53
&
Zenith
&
0,1,2
&
Solar zenith angle {[}°{]}
\\
\hline
54
&
Azimuth
&
0,1,2
&
Solar azimuth angle {[}°{]}
\\
\hline
55
&
AlbBulk
&
0,1,2
&
Bulk albedo {[}-{]}
\\
\hline
56
&
Fcld
&
0,1,2
&
Cloud fraction {[}-{]}
\\
\hline
57
&
LAI
&
0,1,2
&
Leaf area index {[}m 2 m -2 {]}
\\
\hline
58
&
z0m
&
0,1
&
Roughness length for momentum {[}m{]}
\\
\hline
59
&
zdm
&
0,1
&
Zero-plane displacement height {[}m{]}
\\
\hline
60
&
ustar
&
0,1,2
&
Friction velocity {[}m s -1 {]}
\\
\hline
61
&
Lob
&
0,1,2
&
Obukhov length {[}m{]}
\\
\hline
62
&
ra
&
0,1
&
Aerodynamic resistance {[}s m -1 {]}
\\
\hline
63
&
rs
&
0,1
&
Surface resistance {[}s m -1 {]}
\\
\hline
64
&
Fc
&
0,1,2
&
CO2 flux {[}umol m -2 s -1 {]} Do not use in v2017b!
\\
\hline
65
&
FcPhoto
&
0,1
&
CO2 flux from photosynthesis {[}umol m -2 s -1 {]} Do not use in v2017b!
\\
\hline
66
&
FcRespi
&
0,1
&
CO2 flux from respiration {[}umol m -2 s -1 {]} Do not use in v2017b!
\\
\hline
67
&
FcMetab
&
0,1
&
CO2 flux from metabolism {[}umol m -2 s -1 {]} Do not use in v2017b!
\\
\hline
68
&
FcTraff
&
0,1
&
CO2 flux from traffic {[}umol m -2 s -1 {]} Do not use in v2017b!
\\
\hline
69
&
FcBuild
&
0,1
&
CO2 flux from buildings {[}umol m -2 s -1 {]} Do not use in v2017b!
\\
\hline
70
&
QNSnowFr
&
1
&
Net all-wave radiation for snow-free area {[}W m -2 {]}
\\
\hline
71
&
QNSnow
&
1
&
Net all-wave radiation for snow area {[}W m -2 {]}
\\
\hline
72
&
AlbSnow
&
1
&
Snow albedo {[}-{]}
\\
\hline
73
&
QM
&
1
&
Snow-related heat exchange {[}W m -2 {]}
\\
\hline
74
&
QMFreeze
&
1
&
Internal energy change {[}W m -2 {]}
\\
\hline
75
&
QMRain
&
1
&
Heat released by rain on snow {[}W m -2 {]}
\\
\hline
76
&
SWE
&
1
&
Snow water equivalent {[}mm{]}
\\
\hline
77
&
MeltWater
&
1
&
Meltwater {[}mm{]}
\\
\hline
78
&
MeltWStore
&
1
&
Meltwater store {[}mm{]}
\\
\hline
79
&
SnowCh
&
1
&
Change in snow pack {[}mm{]}
\\
\hline
80
&
SnowRPaved
&
1
&
Snow removed from paved surface {[}mm{]}
\\
\hline
81
&
SnowRBldgs
&
1
&
Snow removed from building surface {[}mm{]}
\\
\hline
82
&
T2
&
0,1,2
&
Air temperature at 2 m agl {[}°C{]}
\\
\hline
83
&
Q2
&
0,1,2
&
Air specific humidity at 2 m agl {[}g kg -1 {]}
\\
\hline
84
&
U10
&
0,1,2
&
Wind speed at 10 m agl {[}m s -1 {]}
\\
\hline
\end{longtable}\sphinxatlongtableend\end{savenotes}


\subsection{SSss\_YYYY\_nn\_TT.nc}
\label{\detokenize{output_files/output_files:ssss-yyyy-nn-tt-nc}}
UEWS can also produce the main output file in netCDF format by setting ncMode=1 (set in {\hyperref[\detokenize{output_files/output_files:RunControl}]{\emph{RunControl}}} (\autopageref*{\detokenize{output_files/output_files:RunControl}})).

As the date and time information is incorporated in the netCDF output as
separate dimension, the first five variables in the normal text output
file (in .txt) are not included in the netCDF output but other variables
are all kept.

N.B., considering the file size limit by the classic netCDF format, the
output frequency is determined automatically by the internal SUEWS
program setting to avoid the oversize problem in the netCDF files.


\subsection{SSss\_DailyState.txt}
\label{\detokenize{output_files/output_files:ssss-dailystate-txt}}
Contains information about the state of the surface and soil and
vegetation parameters at a time resolution of one day. One file is
written for each grid so it may contain multiple years.


\begin{savenotes}\sphinxatlongtablestart\begin{longtable}{|l|l|l|}
\hline
\sphinxstyletheadfamily 
Column
&\sphinxstyletheadfamily 
Name
&\sphinxstyletheadfamily 
Description
\\
\hline
\endfirsthead

\multicolumn{3}{c}%
{\makebox[0pt]{\sphinxtablecontinued{\tablename\ \thetable{} -- continued from previous page}}}\\
\hline
\sphinxstyletheadfamily 
Column
&\sphinxstyletheadfamily 
Name
&\sphinxstyletheadfamily 
Description
\\
\hline
\endhead

\hline
\multicolumn{3}{r}{\makebox[0pt][r]{\sphinxtablecontinued{Continued on next page}}}\\
\endfoot

\endlastfoot

1
&
iy
&
Year {[}YYYY{]}
\\
\hline
2
&
id
&
Day of year {[}DOY{]}
\\
\hline
3
&
HDD1\_h
&
Heating degree days {[}°C{]}
\\
\hline
4
&
HDD2\_c
&
Cooling degree days {[}°C{]}
\\
\hline
5
&
HDD3\_Tmean
&
Average daily air temperature {[}°C{]}
\\
\hline
6
&
HDT4\_T5d
&
5-day running-mean air temperature {[}°C{]}
\\
\hline
7
&
P/day
&
Daily total precipitation {[}mm{]}
\\
\hline
8
&
DaysSR
&
Days since rain {[}days{]}
\\
\hline
9
&
GDD1\_g
&
Growing degree days for leaf growth {[}°C{]}
\\
\hline
10
&
GDD2\_s
&
Growing degree days for senescence {[}°C{]}
\\
\hline
11
&
GDD3\_Tmin
&
Daily minimum temperature {[}°C{]}
\\
\hline
12
&
GDD4\_Tmax
&
Daily maximum temperature {[}°C{]}
\\
\hline
13
&
GDD5\_DayLHrs
&
Day length {[}h{]}
\\
\hline
14
&
LAI\_EveTr
&
Leaf area index of evergreen trees {[}m -2 m -2 {]}
\\
\hline
15
&
LAI\_DecTr
&
Leaf area index of deciduous trees {[}m -2 m -2 {]}
\\
\hline
16
&
LAI\_Grass
&
Leaf area index of grass {[}m -2 m -2 {]}
\\
\hline
17
&
DecidCap
&
Moisture storage capacity of deciduous trees {[}mm{]}
\\
\hline
18
&
Porosity
&
Porosity of deciduous trees {[}-{]}
\\
\hline
19
&
AlbEveTr
&
Albedo of evergreen trees {[}-{]}
\\
\hline
20
&
AlbDecTr
&
Albedo of deciduous trees {[}-{]}
\\
\hline
21
&
AlbGrass
&
Albedo of grass {[}-{]}
\\
\hline
22
&
WU\_EveTr(1)
&
Total water use for evergreen trees {[}mm{]}
\\
\hline
23
&
WU\_EveTr(2)
&
Automatic water use for evergreen trees {[}mm{]}
\\
\hline
24
&
WU\_EveTr(3)
&
Manual water use for evergreen trees {[}mm{]}
\\
\hline
25
&
WU\_DecTr(1)
&
Total water use for deciduous trees {[}mm{]}
\\
\hline
26
&
WU\_DecTr(2)
&
Automatic water use for deciduous trees {[}mm{]}
\\
\hline
27
&
WU\_DecTr(3)
&
Manual water use for deciduous trees {[}mm{]}
\\
\hline
28
&
WU\_Grass(1)
&
Total water use for grass {[}mm{]}
\\
\hline
29
&
WU\_Grass(2)
&
Automatic water use for grass {[}mm{]}
\\
\hline
30
&
WU\_Grass(3)
&
Manual water use for grass {[}mm{]}
\\
\hline
31
&
deltaLAI
&
Change in leaf area index (normalised 0-1) {[}-{]}
\\
\hline
32
&
LAIlumps
&
Leaf area index used in LUMPS (normalised 0-1) {[}-{]}
\\
\hline
33
&
AlbSnow
&
Snow albedo {[}-{]}
\\
\hline
34
&
DensSnow\_Paved
&
Snow density - paved surface {[}kg m -3 {]}
\\
\hline
35
&
DensSnow\_Bldgs
&
Snow density - building surface {[}kg m -3 {]}
\\
\hline
36
&
DensSnow\_EveTr
&
Snow density - evergreen surface {[}kg m -3 {]}
\\
\hline
37
&
DensSnow\_DecTr
&
Snow density - deciduous surface {[}kg m -3 {]}
\\
\hline
38
&
DensSnow\_Grass
&
Snow density - grass surface {[}kg m -3 {]}
\\
\hline
39
&
DensSnow\_BSoil
&
Snow density - bare soil surface {[}kg m -3 {]}
\\
\hline
40
&
DensSnow\_Water
&
Snow density - water surface {[}kg m -3 {]}
\\
\hline
\end{longtable}\sphinxatlongtableend\end{savenotes}


\subsection{InitialConditionsSSss\_YYYY.nml}
\label{\detokenize{output_files/output_files:initialconditionsssss-yyyy-nml}}\label{\detokenize{output_files/output_files:initialconditionsssss-yyyy-nml-1}}
At the end of the model run (or the end of each year in the model run) a
new InitialConditions file is written out (to the input folder) for each
grid, see \sphinxcode{\sphinxupquote{InitialConditionsSSss\_YYYY.nml}}


\subsection{SSss\_YYYY\_snow\_TT.txt}
\label{\detokenize{output_files/output_files:ssss-yyyy-snow-tt-txt}}
SUEWS produces a separate output file for snow (when snowUse = 1 in
RunControl.nml) with details for each surface type.

File format of SSss\_YYYY\_snow\_60.txt


\begin{savenotes}\sphinxatlongtablestart\begin{longtable}{|l|l|l|}
\hline
\sphinxstyletheadfamily 
Column
&\sphinxstyletheadfamily 
Name
&\sphinxstyletheadfamily 
Description
\\
\hline
\endfirsthead

\multicolumn{3}{c}%
{\makebox[0pt]{\sphinxtablecontinued{\tablename\ \thetable{} -- continued from previous page}}}\\
\hline
\sphinxstyletheadfamily 
Column
&\sphinxstyletheadfamily 
Name
&\sphinxstyletheadfamily 
Description
\\
\hline
\endhead

\hline
\multicolumn{3}{r}{\makebox[0pt][r]{\sphinxtablecontinued{Continued on next page}}}\\
\endfoot

\endlastfoot

1
&
iy
&
Year {[}YYYY{]}
\\
\hline
2
&
id
&
Day of year {[}DOY{]}
\\
\hline
3
&
it
&
Hour {[}H{]}
\\
\hline
4
&
imin
&
Minute {[}M{]}
\\
\hline
5
&
dectime
&
Decimal time {[}-{]}
\\
\hline
6
&
SWE\_Paved
&
Snow water equivalent \textendash{} paved surface {[}mm{]}
\\
\hline
7
&
SWE\_Bldgs
&
Snow water equivalent \textendash{} building surface {[}mm{]}
\\
\hline
8
&
SWE\_EveTr
&
Snow water equivalent \textendash{} evergreen surface {[}mm{]}
\\
\hline
9
&
SWE\_DecTr
&
Snow water equivalent \textendash{} deciduous surface {[}mm{]}
\\
\hline
10
&
SWE\_Grass
&
Snow water equivalent \textendash{} grass surface {[}mm{]}
\\
\hline
11
&
SWE\_BSoil
&
Snow water equivalent \textendash{} bare soil surface {[}mm{]}
\\
\hline
12
&
SWE\_Water
&
Snow water equivalent \textendash{} water surface {[}mm{]}
\\
\hline
13
&
Mw\_Paved
&
Meltwater \textendash{} paved surface {[}mm h -1 {]}
\\
\hline
14
&
Mw\_Bldgs
&
Meltwater \textendash{} building surface {[}mm h -1 {]}
\\
\hline
15
&
Mw\_EveTr
&
Meltwater \textendash{} evergreen surface {[}mm h -1 {]}
\\
\hline
16
&
Mw\_DecTr
&
Meltwater \textendash{} deciduous surface {[}mm h -1 {]}
\\
\hline
17
&
Mw\_Grass
&
Meltwater \textendash{} grass surface {[}mm h -1 1{]}
\\
\hline
18
&
Mw\_BSoil
&
Meltwater \textendash{} bare soil surface {[}mm h -1 {]}
\\
\hline
19
&
Mw\_Water
&
Meltwater \textendash{} water surface {[}mm h -1 {]}
\\
\hline
20
&
Qm\_Paved
&
Snowmelt-related heat \textendash{} paved surface {[}W m -2 {]}
\\
\hline
21
&
Qm\_Bldgs
&
Snowmelt-related heat \textendash{} building surface {[}W m -2 {]}
\\
\hline
22
&
Qm\_EveTr
&
Snowmelt-related heat \textendash{} evergreen surface {[}W m -2 {]}
\\
\hline
23
&
Qm\_DecTr
&
Snowmelt-related heat \textendash{} deciduous surface {[}W m -2 {]}
\\
\hline
24
&
Qm\_Grass
&
Snowmelt-related heat \textendash{} grass surface {[}W m -2 {]}
\\
\hline
25
&
Qm\_BSoil
&
Snowmelt-related heat \textendash{} bare soil surface {[}W m -2 {]}
\\
\hline
26
&
Qm\_Water
&
Snowmelt-related heat \textendash{} water surface {[}W m -2 {]}
\\
\hline
27
&
Qa\_Paved
&
Advective heat \textendash{} paved surface {[}W m -2 {]}
\\
\hline
28
&
Qa\_Bldgs
&
Advective heat \textendash{} building surface {[}W m -2 {]}
\\
\hline
29
&
Qa\_EveTr
&
Advective heat \textendash{} evergreen surface {[}W m -2 {]}
\\
\hline
30
&
Qa\_DecTr
&
Advective heat \textendash{} deciduous surface {[}W m -2 {]}
\\
\hline
31
&
Qa\_Grass
&
Advective heat \textendash{} grass surface {[}W m -2 {]}
\\
\hline
32
&
Qa\_BSoil
&
Advective heat \textendash{} bare soil surface {[}W m -2 {]}
\\
\hline
33
&
Qa\_Water
&
Advective heat \textendash{} water surface {[}W m -2 {]}
\\
\hline
34
&
QmFr\_Paved
&
Heat related to freezing of surface store \textendash{} paved surface {[}W m -2 {]}
\\
\hline
35
&
QmFr\_Bldgs
&
Heat related to freezing of surface store \textendash{} building surface {[}W m -2 {]}
\\
\hline
36
&
QmFr\_EveTr
&
Heat related to freezing of surface store \textendash{} evergreen surface {[}W m -2 {]}
\\
\hline
37
&
QmFr\_DecTr
&
Heat related to freezing of surface store \textendash{} deciduous surface {[}W m -2 {]}
\\
\hline
38
&
QmFr\_Grass
&
Heat related to freezing of surface store \textendash{} grass surface {[}W m -2 {]}
\\
\hline
39
&
QmFr\_BSoil
&
Heat related to freezing of surface store \textendash{} bare soil surface {[}W m -2 {]}
\\
\hline
40
&
QmFr\_Water
&
Heat related to freezing of surface store \textendash{} water {[}W m -2 {]}
\\
\hline
41
&
fr\_Paved
&
Fraction of snow \textendash{} paved surface {[}-{]}
\\
\hline
42
&
fr\_Bldgs
&
Fraction of snow \textendash{} building surface {[}-{]}
\\
\hline
43
&
fr\_EveTr
&
Fraction of snow \textendash{} evergreen surface {[}-{]}
\\
\hline
44
&
fr\_DecTr
&
Fraction of snow \textendash{} deciduous surface {[}-{]}
\\
\hline
45
&
fr\_Grass
&
Fraction of snow \textendash{} grass surface {[}-{]}
\\
\hline
46
&
Fr\_BSoil
&
Fraction of snow \textendash{} bare soil surface {[}-{]}
\\
\hline
47
&
RainSn\_Paved
&
Rain on snow \textendash{} paved surface {[}mm{]}
\\
\hline
48
&
RainSn\_Bdgs
&
Rain on snow \textendash{} building surface {[}mm{]}
\\
\hline
49
&
RainSn\_EveTr
&
Rain on snow \textendash{} evergreen surface {[}mm{]}
\\
\hline
50
&
RainSn\_DecTr
&
Rain on snow \textendash{} deciduous surface {[}mm{]}
\\
\hline
51
&
RainSn\_Grass
&
Rain on snow \textendash{} grass surface {[}mm{]}
\\
\hline
52
&
RainSn\_BSoil
&
Rain on snow \textendash{} bare soil surface {[}mm{]}
\\
\hline
53
&
RainSn\_Water
&
Rain on snow \textendash{} water surface {[}mm{]}
\\
\hline
54
&
qn\_PavedSnow
&
Net all-wave radiation \textendash{} paved surface {[}W m -2 {]}
\\
\hline
55
&
qn\_BldgsSnow
&
Net all-wave radiation \textendash{} building surface {[}W m -2 {]}
\\
\hline
56
&
qn\_EveTrSnow
&
Net all-wave radiation \textendash{} evergreen surface {[}W m -2 {]}
\\
\hline
57
&
qn\_DecTrSnow
&
Net all-wave radiation \textendash{} deciduous surface {[}W m -2 {]}
\\
\hline
58
&
qn\_GrassSnow
&
Net all-wave radiation \textendash{} grass surface {[}W m -2 {]}
\\
\hline
59
&
qn\_BSoilSnow
&
Net all-wave radiation \textendash{} bare soil surface {[}W m -2 {]}
\\
\hline
60
&
qn\_WaterSnow
&
Net all-wave radiation \textendash{} water surface {[}W m -2 {]}
\\
\hline
61
&
kup\_PavedSnow
&
Reflected shortwave radiation \textendash{} paved surface {[}W m -2 {]}
\\
\hline
62
&
kup\_BldgsSnow
&
Reflected shortwave radiation \textendash{} building surface {[}W m -2 {]}
\\
\hline
63
&
kup\_EveTrSnow
&
Reflected shortwave radiation \textendash{} evergreen surface {[}W m -2 {]}
\\
\hline
64
&
kup\_DecTrSnow
&
Reflected shortwave radiation \textendash{} deciduous surface {[}W m -2 {]}
\\
\hline
65
&
kup\_GrassSnow
&
Reflected shortwave radiation \textendash{} grass surface {[}W m -2 {]}
\\
\hline
66
&
kup\_BSoilSnow
&
Reflected shortwave radiation \textendash{} bare soil surface {[}W m -2 {]}
\\
\hline
67
&
kup\_WaterSnow
&
Reflected shortwave radiation \textendash{} water surface {[}W m -2 {]}
\\
\hline
68
&
frMelt\_Paved
&
Amount of freezing melt water \textendash{} paved surface {[}mm{]}
\\
\hline
69
&
frMelt\_Bldgs
&
Amount of freezing melt water \textendash{} building surface {[}mm{]}
\\
\hline
70
&
frMelt\_EveTr
&
Amount of freezing melt water \textendash{} evergreen surface {[}mm{]}
\\
\hline
71
&
frMelt\_DecTr
&
Amount of freezing melt water \textendash{} deciduous surface {[}mm{]}
\\
\hline
72
&
frMelt\_Grass
&
Amount of freezing melt water \textendash{} grass surface {[}mm{]}
\\
\hline
73
&
frMelt\_BSoil
&
Amount of freezing melt water \textendash{} bare soil surface {[}mm{]}
\\
\hline
74
&
frMelt\_Water
&
Amount of freezing melt water \textendash{} water surface {[}mm{]}
\\
\hline
75
&
MwStore\_Paved
&
Melt water store \textendash{} paved surface {[}mm{]}
\\
\hline
76
&
MwStore\_Bldgs
&
Melt water store \textendash{} building surface {[}mm{]}
\\
\hline
77
&
MwStore\_EveTt
&
Melt water store \textendash{} evergreen surface {[}mm{]}
\\
\hline
78
&
MwStore\_DecTr
&
Melt water store \textendash{} deciduous surface {[}mm{]}
\\
\hline
79
&
MwStore\_Grass
&
Melt water store \textendash{} grass surface {[}mm{]}
\\
\hline
80
&
MwStore\_BSoil
&
Melt water store \textendash{} bare soil surface {[}mm{]}
\\
\hline
81
&
MwStore\_Water
&
Melt water store \textendash{} water surface {[}mm{]}
\\
\hline
82
&
DensSnow\_Paved
&
Snow density \textendash{} paved surface {[}kg m -3 {]}
\\
\hline
83
&
DensSnow\_Bldgs
&
Snow density \textendash{} building surface {[}kg m -3 {]}
\\
\hline
84
&
DensSnow\_EveTr
&
Snow density \textendash{} evergreen surface {[}kg m -3 {]}
\\
\hline
85
&
DensSnow\_DecTr
&
Snow density \textendash{} deciduous surface {[}kg m -3 {]}
\\
\hline
86
&
DensSnow\_Grass
&
Snow density \textendash{} grass surface {[}kg m -3 {]}
\\
\hline
87
&
DensSnow\_BSoil
&
Snow density \textendash{} bare soil surface {[}kg m -3 {]}
\\
\hline
88
&
DensSnow\_Water
&
Snow density \textendash{} water surface {[}kg m -3 {]}
\\
\hline
89
&
Sd\_Paved
&
Snow depth \textendash{} paved surface {[}mm{]}
\\
\hline
90
&
Sd\_Bldgs
&
Snow depth \textendash{} building surface {[}mm{]}
\\
\hline
91
&
Sd\_EveTr
&
Snow depth \textendash{} evergreen surface {[}mm{]}
\\
\hline
92
&
Sd\_DecTr
&
Snow depth \textendash{} deciduous surface {[}mm{]}
\\
\hline
93
&
Sd\_Grass
&
Snow depth \textendash{} grass surface {[}mm{]}
\\
\hline
94
&
Sd\_BSoil
&
Snow depth \textendash{} bare soil surface {[}mm{]}
\\
\hline
95
&
Sd\_Water
&
Snow depth \textendash{} water surface {[}mm{]}
\\
\hline
96
&
Tsnow\_Paved
&
Snow surface temperature \textendash{} paved surface {[}°C{]}
\\
\hline
97
&
Tsnow\_Bldgs
&
Snow surface temperature \textendash{} building surface {[}°C{]}
\\
\hline
98
&
Tsnow\_EveTr
&
Snow surface temperature \textendash{} evergreen surface {[}°C{]}
\\
\hline
99
&
Tsnow\_DecTr
&
Snow surface temperature \textendash{} deciduous surface {[}°C{]}
\\
\hline
100
&
Tsnow\_Grass
&
Snow surface temperature \textendash{} grass surface {[}°C{]}
\\
\hline
101
&
Tsnow\_BSoil
&
Snow surface temperature \textendash{} bare soil surface {[}°C{]}
\\
\hline
102
&
Tsnow\_Water
&
Snow surface temperature \textendash{} water surface {[}°C{]}
\\
\hline
\end{longtable}\sphinxatlongtableend\end{savenotes}


\subsection{SSss\_YYYY\_BL.txt}
\label{\detokenize{output_files/output_files:ssss-yyyy-bl-txt}}
Meteorological variables modelled by CBL portion of the model are output
in to this file created for each day with time step (see section CBL
Input).


\begin{savenotes}\sphinxattablestart
\centering
\begin{tabulary}{\linewidth}[t]{|T|T|T|T|}
\hline
\sphinxstyletheadfamily 
Column
&\sphinxstyletheadfamily 
Name
&\sphinxstyletheadfamily 
Description
&\sphinxstyletheadfamily 
Units
\\
\hline
1
&
iy
&
Year {[}YYYY{]}
&\\
\hline
2
&
id
&
Day of year {[}DoY{]}
&\\
\hline
3
&
it
&
Hour {[}H{]}
&\\
\hline
4
&
imin
&
Minute {[}M{]}
&\\
\hline
5
&
dectime
&
Decimal time {[}-{]}
&\\
\hline
6
&
zi
&
Convectibe boundary layer height
&
m
\\
\hline
7
&
Theta
&
Potential temperature in the inertial sublayer
&
K
\\
\hline
8
&
Q
&
Specific humidity in the inertial sublayer
&
g kg -1
\\
\hline
9
&
theta+
&
Potential temperature just above the CBL
&
K
\\
\hline
10
&
q+
&
Specific humidity just above the CBL
&
g kg -1
\\
\hline
11
&
Temp\_C
&
Air temperature
&
°C
\\
\hline
12
&
RH
&
Relative humidity
&
\%
\\
\hline
13
&
QH\_use
&
Sensible heat flux used for calculation
&
W m -2
\\
\hline
14
&
QE\_use
&
Latent heat flux used for calculation
&
W m -2
\\
\hline
15
&
Press\_hPa
&
Pressure used for calculation
&
hPa
\\
\hline
16
&
avu1
&
Wind speed used for calculation
&
m s -1
\\
\hline
17
&
ustar
&
Friction velocity used for calculation
&
m s -1
\\
\hline
18
&
avdens
&
Air density used for calculation
&
kg m -3
\\
\hline
19
&
lv\_J\_kg
&
Latent heat of vaporization used for calculation
&
J kg -1
\\
\hline
20
&
avcp
&
Specific heat capacity used for calculation
&
J kg -1 K -1
\\
\hline
21
&
gamt
&
Vertical gradient of potential temperature
&
K m -1
\\
\hline
22
&
gamq
&
Vertical gradient of specific humidity
&
kg kg -1 m -1
\\
\hline
\end{tabulary}
\par
\sphinxattableend\end{savenotes}


\subsection{SOLWEIGpoiOut.txt}
\label{\detokenize{output_files/output_files:solweigpoiout-txt}}
Calculated variables from POI, point of interest (row, col) stated in
SOLWEIGinput.nml.

SOLWEIG model output file format: SOLWEIGpoiOUT.txt


\begin{savenotes}\sphinxattablestart
\centering
\begin{tabulary}{\linewidth}[t]{|T|T|T|T|}
\hline
\sphinxstyletheadfamily 
Column
&\sphinxstyletheadfamily 
Name
&\sphinxstyletheadfamily 
Description
&\sphinxstyletheadfamily 
Units
\\
\hline
1
&
id
&
Day of year
&\\
\hline
2
&
dectime
&
Decimal time
&\\
\hline
3
&
azimuth
&
Azimuth angle of the Sun
&
°
\\
\hline
4
&
altitude
&
Altitude angle of the Sun
&
°
\\
\hline
5
&
GlobalRad
&
Input Kdn
&
W m -2
\\
\hline
6
&
DiffuseRad
&
Diffuse shortwave radiation
&
W m -2
\\
\hline
7
&
DirectRad
&
Direct shortwave radiation
&
W m -2
\\
\hline
8
&
Kdown2d
&
Incoming shortwave radiation at POI
&
W m -2
\\
\hline
9
&
Kup2d
&
Outgoing shortwave radiation at POI
&
W m -2
\\
\hline
10
&
Ksouth
&
Shortwave radiation from south at POI
&
W m -2
\\
\hline
11
&
Kwest
&
Shortwave radiation from west at POI
&
W m -2
\\
\hline
12
&
Knorth
&
Shortwave radiation from north at POI
&
W m -2
\\
\hline
13
&
Keast
&
Shortwave radiation from east at POI
&
W m -2
\\
\hline
14
&
Ldown2d
&
Incoming longwave radiation at POI
&
W m -2
\\
\hline
15
&
Lup2d
&
Outgoing longwave radiation at POI
&
W m -2
\\
\hline
16
&
Lsouth
&
Longwave radiation from south at POI
&
W m -2
\\
\hline
17
&
Lwest
&
Longwave radiation from west at POI
&
W m -2
\\
\hline
18
&
Lnorth
&
Longwave radiation from north at POI
&
W m -2
\\
\hline
19
&
Least
&
Longwave radiation from east at POI
&
W m -2
\\
\hline
20
&
Tmrt
&
Mean Radiant Temperature
&
°C
\\
\hline
21
&
I0
&
theoretical value of maximum incoming solar radiation
&
W m -2
\\
\hline
22
&
CI
&
clearness index for Ldown (Lindberg et al. 2008)
&\\
\hline
23
&
gvf
&
Ground view factor (Lindberg and Grimmond 2011)
&\\
\hline
24
&
shadow
&
Shadow value (0= shadow, 1 = sun)
&\\
\hline
25
&
svf
&
Sky View Factor from ground and buildings
&\\
\hline
26
&
svfbuveg
&
Sky View Factor from ground, buildings and vegetation
&\\
\hline
27
&
Ta
&
Air temperature
&
°C
\\
\hline
28
&
Tg
&
Surface temperature
&
°C
\\
\hline
\end{tabulary}
\par
\sphinxattableend\end{savenotes}


\subsection{SSss\_YYYY\_ESTM\_TT.txt}
\label{\detokenize{output_files/output_files:ssss-yyyy-estm-tt-txt}}
If the ESTM model option is run, the following output file is created.
\sphinxstylestrong{Note: First time steps of storage output could give NaN values during
the initial converging phase.}

ESTM output file format


\begin{savenotes}\sphinxatlongtablestart\begin{longtable}{|l|l|l|l|}
\hline
\sphinxstyletheadfamily 
Column
&\sphinxstyletheadfamily 
Name
&\sphinxstyletheadfamily 
Description
&\sphinxstyletheadfamily 
Units
\\
\hline
\endfirsthead

\multicolumn{4}{c}%
{\makebox[0pt]{\sphinxtablecontinued{\tablename\ \thetable{} -- continued from previous page}}}\\
\hline
\sphinxstyletheadfamily 
Column
&\sphinxstyletheadfamily 
Name
&\sphinxstyletheadfamily 
Description
&\sphinxstyletheadfamily 
Units
\\
\hline
\endhead

\hline
\multicolumn{4}{r}{\makebox[0pt][r]{\sphinxtablecontinued{Continued on next page}}}\\
\endfoot

\endlastfoot

1
&
iy
&
Year
&\\
\hline
2
&
id
&
Day of year
&\\
\hline
3
&
it
&
Hour
&\\
\hline
4
&
imin
&
Minute
&\\
\hline
5
&
dectime
&
Decimal time
&\\
\hline
6
&
QSnet
&
Net storage heat flux (QSwall+QSground+QS)
&
W m -2
\\
\hline
7
&
QSair
&
Storage heat flux into air
&
W m -2
\\
\hline
8
&
QSwall
&
Storage heat flux into wall
&
W m -2
\\
\hline
9
&
QSroof
&
Storage heat flux into roof
&
W m -2
\\
\hline
10
&
QSground
&
Storage heat flux into ground
&
W m -2
\\
\hline
11
&
QSibld
&
Storage heat flux into internal elements in buildling
&
W m -2
\\
\hline
12
&
Twall1
&
Temperature in the first layer of wall (outer-most)
&
K
\\
\hline
13
&
Twall2
&
Temperature in the first layer of wall
&
K
\\
\hline
14
&
Twall3
&
Temperature in the first layer of wall
&
K
\\
\hline
15
&
Twall4
&
Temperature in the first layer of wall
&
K
\\
\hline
16
&
Twall5
&
Temperature in the first layer of wall (inner-most)
&
K
\\
\hline
17
&
Troof1
&
Temperature in the first layer of roof (outer-most)
&
K
\\
\hline
18
&
Troof2
&
Temperature in the first layer of roof
&
K
\\
\hline
19
&
Troof3
&
Temperature in the first layer of roof
&
K
\\
\hline
20
&
Troof4
&
Temperature in the first layer of roof
&
K
\\
\hline
21
&
Troof5
&
Temperature in the first layer of ground (inner-most)
&
K
\\
\hline
22
&
Tground1
&
Temperature in the first layer of ground (outer-most)
&
K
\\
\hline
23
&
Tground2
&
Temperature in the first layer of ground
&
K
\\
\hline
24
&
Tground3
&
Temperature in the first layer of ground
&
K
\\
\hline
25
&
Tground4
&
Temperature in the first layer of ground
&
K
\\
\hline
26
&
Tground5
&
Temperature in the first layer of ground (inner-most)
&
K
\\
\hline
27
&
Tibld1
&
Temperature in the first layer of internal elements
&
K
\\
\hline
28
&
Tibld2
&
Temperature in the first layer of internal elements
&
K
\\
\hline
29
&
Tibld3
&
Temperature in the first layer of internal elements
&
K
\\
\hline
30
&
Tibld4
&
Temperature in the first layer of internal elements
&
K
\\
\hline
31
&
Tibld5
&
Temperature in the first layer of internal elements
&
K
\\
\hline
32
&
Tabld
&
Air temperature in buildings
&
K
\\
\hline
\end{longtable}\sphinxatlongtableend\end{savenotes}


\chapter{Troubleshooting}
\label{\detokenize{troubleshooting::doc}}\label{\detokenize{troubleshooting:troubleshooting}}

\section{How to create a directory?}
\label{\detokenize{troubleshooting:how-to-create-a-directory}}\begin{quote}

please search the web using this phrase if you do not know how to
create a folder or directory
\end{quote}


\section{How to unzip a file}
\label{\detokenize{troubleshooting:how-to-unzip-a-file}}\begin{quote}

please search the web using this phrase if you do not know how to
unzip a file
\end{quote}


\section{A text editor}
\label{\detokenize{troubleshooting:a-text-editor}}\begin{quote}

is a program to edit plain text files. If you search on the web
using the phrase ‘text editor’ you will find numerous programs.
These include for example, NotePad, EditPad, Text Pad etc
\end{quote}


\section{Command prompt}
\label{\detokenize{troubleshooting:command-prompt}}\begin{quote}

From Start select run \textendash{}type cmd \textendash{} this will open a window. Change
directory to the location of where you stored your files. The
following website may be helpful if you do not know what a command
prompt is: \sphinxurl{http://dosprompt.info/}
\end{quote}


\section{Day of year {[}DOY{]}}
\label{\detokenize{troubleshooting:day-of-year-doy}}\begin{quote}

January 1st is day 1, February 1st is day 32. If you search on the
web using the phrase ‘day of year calendar’ you will find tables
that allow rapid conversions. Remember that after February 28th DOY
will be different between leap years and non-leap years.
\end{quote}


\section{ESTM output}
\label{\detokenize{troubleshooting:estm-output}}
First time steps of storage output could give NaN values during the
initial converging phase.


\section{First things to Check if the program seems to have problems}
\label{\detokenize{troubleshooting:first-things-to-check-if-the-program-seems-to-have-problems}}\begin{itemize}
\item {} 
Check the problems.txt file.

\item {} 
Check file options \textendash{} in RunControl.nml.

\item {} 
Look in the output directory for the SS\_FileChoices.txt. This allows
you to check all options that were used in the run. You may want to
compare it with the original version supplied with the model.

\item {} 
Note there can not be missing time steps in the data. If you need
help with this you may want to checkout
\sphinxhref{http://urban-climate.net/umep/UMEP}{UMEP}

\end{itemize}


\subsection{A pop-up saying “file path not found”}
\label{\detokenize{troubleshooting:a-pop-up-saying-file-path-not-found}}
This means the program cannot find the file paths defined in
RunControl.nml file. Possible solutions:
\begin{itemize}
\item {} 
Check that you have created the folder that you specified in
RunControl.nml.

\item {} 
Check does the output directory exist?

\item {} 
Check that you have a single or double quotes around the
FileInputPath, FileOutputPath and FileCode

\end{itemize}

====“\%sat\_vap\_press.f temp=0.0000 pressure dectime”==== Temperature is
zero in the calculation of water vapour pressure parameterization.
\begin{itemize}
\item {} 
You don’t need to worry if the temperature should be (is) 0°C.

\item {} 
If it should not be 0°C this suggests that there is a problem with
the data.

\end{itemize}


\subsection{\%T changed to fit limits}
\label{\detokenize{troubleshooting:t-changed-to-fit-limits}}\begin{itemize}
\item {} 
{[}TL =0.1{]}/ {[}TL =39.9{]} You may want to change the coefficients for
surface resistance. If you have data from these temperatures, we
would happily determine them.

\end{itemize}


\subsection{\%Iteration loop stopped for too stable conditions.}
\label{\detokenize{troubleshooting:iteration-loop-stopped-for-too-stable-conditions}}\begin{itemize}
\item {} 
{[}zL{]}/{[}USTAR{]} This warning indicates that the atmospheric stability
gets above 2. In these conditions \sphinxhref{http://glossary.ametsoc.org/wiki/Monin-obukhov\_similarity\_theory}{MO
theory}
is not necessarily valid. The iteration loop to calculate the
\sphinxhref{http://glossary.ametsoc.org/wiki/Obukhov\_length}{Obukhov length}
and \sphinxhref{http://glossary.ametsoc.org/wiki/Friction\_velocity}{friction
velocity} is
stopped so that stability does not get too high values. This is
something you do not need to worry as it does not mean wrong input
data.

\end{itemize}


\subsection{“Reference to undefined variable, array element or function result”}
\label{\detokenize{troubleshooting:reference-to-undefined-variable-array-element-or-function-result}}\begin{itemize}
\item {} 
Parameter(s) missing from input files.

\end{itemize}

See also the error messages provided in problems.txt and warnings.txt


\subsection{Email list}
\label{\detokenize{troubleshooting:email-list}}\begin{itemize}
\item {} 
SUEWS email list

\end{itemize}

\sphinxurl{https://www.lists.reading.ac.uk/mailman/listinfo/met-suews}
\begin{itemize}
\item {} 
UMEP email list

\end{itemize}

\sphinxurl{https://www.lists.reading.ac.uk/mailman/listinfo/met-umep}


\chapter{Acknowledgements}
\label{\detokenize{acknowledgement:acknowledgements}}\label{\detokenize{acknowledgement::doc}}\begin{itemize}
\item {} 
People who have contributed to the development of SUEWS (plus
co-authors of papers):

\item {} 
Current contributors:
\begin{itemize}
\item {} 
Prof C.S.B. Grimmond (University of Reading; previously Indiana
University, King’s College London, UK); Dr Leena Järvi (University
of Helsinki, Finland); Dr Helen Ward (University of Reading), Dr
Fredrik Lindberg (Göteborg University, Sweden), Dr Andy Gabey
(Reading), Dr Ting SUN (Reading), Dr Jie PENG (SIMS), Dr Natalie
Theeuwes (Reading),

\end{itemize}

\item {} 
Past Contributors:
\begin{itemize}
\item {} 
Dr Brian Offerle (Indiana University), Dr Thomas Loridan (King’s
College London),Dr Shiho Onomura (Göteborg University, Sweden)

\end{itemize}

\item {} 
Users who have brought things to our attention to improve this manual
and the model:
\begin{itemize}
\item {} 
Dr Andy Coutts (Monash University, Australia), Kerry Nice (Monash
University, Australia), Shiho Onomura (Göteborg University,
Sweden), Dr Stefan Smith (University of Reading, UK), Dr Helen
Ward (King’s College London, UK; University of Reading, UK); Duick
Young (King’s College London), Dr Ning Zhang (Nanjing University,
China)

\end{itemize}

\item {} 
Funding to support development:
\begin{itemize}
\item {} 
National Science Foundation (USA, BCS-0095284, ATM-0710631), EU
Framework 7 BRIDGE (211345), EUf7 emBRACE; UK Met Office; NERC
ClearfLo, NERC/Belmont TRUC, Newton/Met Office CSSP-China, H2020
UrbanFluxes, EPSRC LoHCool

\end{itemize}

\end{itemize}


\chapter{Notation}
\label{\detokenize{notation::doc}}\label{\detokenize{notation:notation}}\begin{description}
\item[{\sphinxstyleemphasis{\(\lambda\)F}\index{\(\lambda\)F|textbf}}] \leavevmode\phantomsection\label{\detokenize{notation:term-f}}
frontal area  index

\item[{\(\Delta\)QS*\index{\(\Delta\)QS*|textbf}}] \leavevmode\phantomsection\label{\detokenize{notation:term-qs}}
storage heat flux

\item[{BLUEWS\index{BLUEWS|textbf}}] \leavevmode\phantomsection\label{\detokenize{notation:term-bluews}}
Boundary Layer  part of SUEWS
\begin{quote}

\begin{figure}[htbp]
\centering
\capstart

\noindent\sphinxincludegraphics{{Bluews_1}.jpg}
\caption{Relation between BLUEWS and SUEWS}\label{\detokenize{notation:id10}}\end{figure}
\end{quote}

\item[{Bldgs\index{Bldgs|textbf}}] \leavevmode\phantomsection\label{\detokenize{notation:term-bldgs}}
Building  surface

\item[{CBL\index{CBL|textbf}}] \leavevmode\phantomsection\label{\detokenize{notation:term-cbl}}
Convective  boundary layer

\item[{DEM\index{DEM|textbf}}] \leavevmode\phantomsection\label{\detokenize{notation:term-dem}}
Digital   Elevation Model

\item[{DSM\index{DSM|textbf}}] \leavevmode\phantomsection\label{\detokenize{notation:term-dsm}}
Digital surface  model

\item[{DTM\index{DTM|textbf}}] \leavevmode\phantomsection\label{\detokenize{notation:term-dtm}}
Digital Terrain Model

\item[{DecTr\index{DecTr|textbf}}] \leavevmode\phantomsection\label{\detokenize{notation:term-dectr}}
deciduous trees and shrubs

\item[{EveTr\index{EveTr|textbf}}] \leavevmode\phantomsection\label{\detokenize{notation:term-evetr}}
Evergreen trees and shrubs

\item[{ESTM\index{ESTM|textbf}}] \leavevmode\phantomsection\label{\detokenize{notation:term-estm}}
Element Surface Temperature Method (Offerle et al.,2005 \phantomsection\label{\detokenize{notation:id1}}{\hyperref[\detokenize{references:oaf2005}]{\sphinxcrossref{{[}Oaf2005{]}}}} (\autopageref*{\detokenize{references:oaf2005}}))

\item[{Grass\index{Grass|textbf}}] \leavevmode\phantomsection\label{\detokenize{notation:term-grass}}
Grass surface

\item[{BSoil\index{BSoil|textbf}}] \leavevmode\phantomsection\label{\detokenize{notation:term-bsoil}}
Unmanaged land and/or bare soil

\item[{L↓\index{L↓|textbf}}] \leavevmode\phantomsection\label{\detokenize{notation:term-l}}
incoming longwave radiation

\item[{LAI\index{LAI|textbf}}] \leavevmode\phantomsection\label{\detokenize{notation:term-lai}}
Leaf area index

\item[{LUMPS\index{LUMPS|textbf}}] \leavevmode\phantomsection\label{\detokenize{notation:term-lumps}}
Local-scale Urban Meteorological Parameterization Scheme
(Loridan   et al. 2011 \phantomsection\label{\detokenize{notation:id2}}{\hyperref[\detokenize{references:l2011}]{\sphinxcrossref{{[}L2011{]}}}} (\autopageref*{\detokenize{references:l2011}}))

\item[{MU\index{MU|textbf}}] \leavevmode\phantomsection\label{\detokenize{notation:term-mu}}
Parameters which must be supplied and must be specific for the site/grid being run.

\item[{MD\index{MD|textbf}}] \leavevmode\phantomsection\label{\detokenize{notation:term-md}}
Parameters which must be supplied and must be specific for the site/grid being run (but default values may be ok if these values are not known specifically for the site).

\item[{O\index{O|textbf}}] \leavevmode\phantomsection\label{\detokenize{notation:term-o}}
Parameters that are optional, depending on the model settings in RunControl. Set any parameters that are not used/not known to ‘-999’.

\item[{L\index{L|textbf}}] \leavevmode\phantomsection\label{\detokenize{notation:term-19}}
Codes that are used to link between the input files. These codes are required but their values are completely arbitrary, providing that they link the input files in the correct way. The user should choose these codes, bearing in mind that the codes they match up with in column 1 of the corresponding input file must be unique within that file. Codes must be integers. Note that the codes must match up with column 1 of the corresponding input file, even if those parameters are not used (in which case set all columns except column 1 to ‘-999’ in the corresponding input file), otherwise the model run will fail.

\item[{NARP\index{NARP|textbf}}] \leavevmode\phantomsection\label{\detokenize{notation:term-narp}}
Net All-wave  Radiation   Parameterization (Offerle et al. 2003 \phantomsection\label{\detokenize{notation:id3}}{\hyperref[\detokenize{references:o2003}]{\sphinxcrossref{{[}O2003{]}}}} (\autopageref*{\detokenize{references:o2003}}), Loridan et al. 2011 \phantomsection\label{\detokenize{notation:id4}}{\hyperref[\detokenize{references:l2011}]{\sphinxcrossref{{[}L2011{]}}}} (\autopageref*{\detokenize{references:l2011}}))

\item[{OHM\index{OHM|textbf}}] \leavevmode\phantomsection\label{\detokenize{notation:term-ohm}}
Objective Hysteresis Model (Grimmond et al. 1991 \phantomsection\label{\detokenize{notation:id5}}{\hyperref[\detokenize{references:g91ohm}]{\sphinxcrossref{{[}G91OHM{]}}}} (\autopageref*{\detokenize{references:g91ohm}}), Grimmond \& Oke 1999a \phantomsection\label{\detokenize{notation:id6}}{\hyperref[\detokenize{references:go99qs}]{\sphinxcrossref{{[}GO99QS{]}}}} (\autopageref*{\detokenize{references:go99qs}}), 2002 \phantomsection\label{\detokenize{notation:id7}}{\hyperref[\detokenize{references:go2002}]{\sphinxcrossref{{[}GO2002{]}}}} (\autopageref*{\detokenize{references:go2002}}))

\item[{Paved\index{Paved|textbf}}] \leavevmode\phantomsection\label{\detokenize{notation:term-paved}}
Paved surface

\item[{Q$^{\text{*}}$\index{Qstar|textbf}}] \leavevmode\phantomsection\label{\detokenize{notation:term-qstar}}
net all-wave radiation

\item[{QE\index{QE|textbf}}] \leavevmode\phantomsection\label{\detokenize{notation:term-qe}}
latent heat flux

\item[{QF\index{QF|textbf}}] \leavevmode\phantomsection\label{\detokenize{notation:term-qf}}
anthropogenic  heat flux

\item[{QH\index{QH|textbf}}] \leavevmode\phantomsection\label{\detokenize{notation:term-qh}}
sensible heat  flux

\item[{SOLWEIG\index{SOLWEIG|textbf}}] \leavevmode\phantomsection\label{\detokenize{notation:term-solweig}}
The solar and longwave environmental irradiance geometry model
(Lindberg et al. 2008 \phantomsection\label{\detokenize{notation:id8}}{\hyperref[\detokenize{references:fl2008}]{\sphinxcrossref{{[}FL2008{]}}}} (\autopageref*{\detokenize{references:fl2008}}),   Lindberg and Grimmond 2011 \phantomsection\label{\detokenize{notation:id9}}{\hyperref[\detokenize{references:fl2011}]{\sphinxcrossref{{[}FL2011{]}}}} (\autopageref*{\detokenize{references:fl2011}}))

\item[{SVF\index{SVF|textbf}}] \leavevmode\phantomsection\label{\detokenize{notation:term-svf}}
Sky view factor

\item[{theta\index{theta|textbf}}] \leavevmode\phantomsection\label{\detokenize{notation:term-theta}}
potential  temperature

\item[{tt\index{tt|textbf}}] \leavevmode\phantomsection\label{\detokenize{notation:term-tt}}
time step of data

\item[{UMEP\index{UMEP|textbf}}] \leavevmode\phantomsection\label{\detokenize{notation:term-umep}}
\sphinxhref{http://urban-climate.net/umep/UMEP}{Urban Multi-scale Environmental Predictor}

\item[{Water\index{Water|textbf}}] \leavevmode\phantomsection\label{\detokenize{notation:term-water}}
Water surface

\item[{zi\index{zi|textbf}}] \leavevmode\phantomsection\label{\detokenize{notation:term-zi}}
Convective boundary layer height

\end{description}


\chapter{Development, Suggestions and Support}
\label{\detokenize{development::doc}}\label{\detokenize{development:urban-multi-scale-environmental-predictor}}\label{\detokenize{development:development-suggestions-and-support}}\begin{enumerate}
\item {} 
{[}\sphinxurl{http://urban-climate.net/umep/DevelopmentGuidelines\#Coding\_Guidelines}\textbar{}
Coding Guidelines{]}

\item {} 
Recommendations, Errors, Help/Updates - please join our email list
\begin{enumerate}
\item {} 
\sphinxhref{https://www.lists.reading.ac.uk/mailman/listinfo/met-suews}{www.lists.reading.ac.uk/mailman/listinfo/met-suews}

\item {} 
As UMEP has a number of tools to support SUEWS you may want to
join that list also
\sphinxhref{https://www.lists.reading.ac.uk/mailman/listinfo/met-umep}{www.lists.reading.ac.uk/mailman/listinfo/met-umep}

\end{enumerate}

\end{enumerate}


\chapter{Version History}
\label{\detokenize{version-history:version-history}}\label{\detokenize{version-history::doc}}\label{\detokenize{version-history:id1}}

\section{New in SUEWS Version 2018a}
\label{\detokenize{version-history:new-latest}}\label{\detokenize{version-history:new-in-suews-version-2018a}}
see {\hyperref[\detokenize{version-history:version-history}]{\sphinxcrossref{\DUrole{std,std-ref}{Version History}}}} (\autopageref*{\detokenize{version-history:version-history}}).


\section{New in SUEWS Version 2017b (released 2 August 2017)}
\label{\detokenize{version-history:new-in-suews-version-2017b-released-2-august-2017}}
\sphinxhref{:File:SUEWS\_V2017b\_Manual.pdf}{PDF Manual for v2017b}
\begin{enumerate}
\item {} 
Surface-level diagnostics: T2 (air temperature at 2 m agl), Q2 (air
specific humidity at 2 m agl) and U10 (wind speed at 10 m agl) added
as default output.

\item {} 
Output in netCDF format. Please note this feature is \sphinxstylestrong{NOT} enabled
in the public release due to the dependency of netCDF library.
Assistance in enabling this feature may be requested to the
development team via \sphinxhref{https://www.lists.reading.ac.uk/mailman/listinfo/met-suews}{SUEWS mail
list}.

\item {} 
Edits to the manual.

\item {} 
New capabilities being developed, including two new options for
calculating storage heat flux (AnOHM, ESTM) and modelling of carbon
dioxide fluxes. These are currently under development and \sphinxstylestrong{should
not be used} in v2017b.

\item {} 
Known issues
\begin{enumerate}
\item {} 
BLUEWS parameters need to be checked

\item {} 
Observed soil moisture can not be used as an input

\item {} 
Wind direction is not currently downscaled so non -999 values will
cause an error.

\end{enumerate}

\end{enumerate}


\section{New in SUEWS Version 2017a (Feb 2017)}
\label{\detokenize{version-history:new-in-suews-version-2017a-feb-2017}}\begin{enumerate}
\item {} 
Changes to input file formats (including RunControl.nml and
InitialConditions files) to facilitate setting up and running the
model. Met forcing files no longer need two rows of -9 at the end to
indicate the end of the file.

\item {} 
Changes to output file formats (now option to write out only a subset
of variables, rather than all variables).

\item {} 
SUEWS can now disaggregate forcing files to the model time-step and
aggregate output at the model time-step to lower resolution. This
removes the need for the python wrapper used with previous versions.

\item {} 
InitialConditions format and requirements changed. A single file can
now be provided for multiple grids. SUEWS will approximate most (but
not all) of the required initial conditions if values are unknown.
(However, if detailed information about the initial conditions is
known, this can still be provided to and used by SUEWS.)

\item {} 
Leaf area index calculations now use parameters provided for each
vegetated surface (previously only the deciduous tree LAI development
parameters were applied to all vegetated surfaces).

\item {} 
For compatibility with GIS, \sphinxstylestrong{the sign convention for longitude has
been changed}. Now negative values are to the west, positive values
are to the east. Note this appears to have been incorrectly coded in
previous versions (but may not necessarily have been problematic).

\item {} 
Storage heat flux calculation adapted for shorter (sub-hourly) model
time-step: hysteresis calculation now based on running means over the
previous hour.

\item {} 
Improved error handling, including separate files for serious errors
(problems.txt) and less critical issues (warnings.txt).

\item {} 
Edits to the manual.

\item {} 
New capabilities being developed, including two new options for
calculating storage heat flux (AnOHM, ESTM) and modelling of carbon
dioxide fluxes. These are currently under development and \sphinxstylestrong{should
not be used} in v2017a.

\end{enumerate}


\section{New in SUEWS Version 2016a (released 21 June 2016)}
\label{\detokenize{version-history:new-in-suews-version-2016a-released-21-june-2016}}
\sphinxhref{:File:SUEWS\_V2016a\_Manual.pdf}{PDF Manual for v2016a}
\begin{enumerate}
\item {} 
Major changes to the input file formats to facilitate the running of
multiple grids and multiple years. Surface characteristics are
provided in SiteSelect.txt and other input files are cross-referenced
via codes or profile types.

\item {} 
The surface types have been altered:
\begin{itemize}
\item {} 
Previously, grass surfaces were entered separately as irrigated
grass and unirrigated grass surfaces, whilst the ‘unmanaged’ land
cover fraction was assumed by the model to behave as unirrigated
grass. There is now a single surface type for grass (total for
irrigated plus unirrigated) and a new bare soil surface type.

\item {} 
The proportion of irrigated vegetation must now be specified for
grass, evergreen trees and deciduous trees individually.

\end{itemize}

\item {} 
The entire model now runs at a time step specified by the user. Note
that 5 min is strongly recommended. (Previously only the water
balance calculations were done at 5 min with the energy balance
calculations at 60 min).

\item {} 
Surface conductance now depends on the soil moisture under the
vegetated surfaces only (rather than the total soil moisture for the
whole study area as previously).

\item {} 
Albedo of evergreen trees and grass surfaces can now change with leaf
area index as was previously possible for deciduous trees only.

\item {} 
New suggestions in Troubleshooting section.

\item {} 
Edits to the manual.

\item {} 
CBL model included.

\item {} 
SUEWS has been incorporated into
\sphinxhref{http://urban-climate.net/umep/UMEP}{UMEP}

\end{enumerate}


\section{New in SUEWS Version 2014b (released 8 October 2014)}
\label{\detokenize{version-history:new-in-suews-version-2014b-released-8-october-2014}}
\sphinxhref{http://www.met.rdg.ac.uk/micromet/documents/SUEWS\_Manual.pdf}{V2014
manual}
These affect the run configuration if previously run with older versions
of the model:
\begin{enumerate}
\item {} 
New input of three additional columns in the Meteorological input
file (diffusive and direct solar radiation, and wind direction)

\item {} 
Change of input variables in InitialConditions.nml file. Note we now
refer to CT as ET (ie. Evergreen trees rather than coniferous trees)

\item {} 
In GridConnectionsYYYY.txt, the site names should now be without the
underscore (e.g “Sm” and not “{\color{red}\bfseries{}Sm\_}”)

\end{enumerate}

Other issues:
\begin{enumerate}
\item {} 
Number of grid areas that can be modelled (for one grid, one year
120; for one grid two years 80)

\item {} 
Comment about Time interval of input data

\item {} 
Bug fix: Column headers corrected in 5 min file

\item {} 
Bug fix: Surface state 60 min file - corrected to give the last 5 min
of the hour (rather than cumulating through the hour)

\item {} 
Bug fix: units in the Horizontal soil water transfer

\item {} 
ErrorHints: More have been added to the problems.txt file.

\item {} 
Manual: new section on running the model appropriately

\item {} 
Manual: notation table updated

\item {} 
Possibility to add snow accumulation and melt: new paper

\end{enumerate}

Järvi L, Grimmond CSB, Taka M, Nordbo A, Setälä H, and Strachan IB 2014:
Development of the Surface Urban Energy and Water balance Scheme (SUEWS)
for cold climate cities, Geosci. Model Dev. 7, 1691-1711,
doi:10.5194/gmd-7-1691-2014.


\section{New in SUEWS Version 2014a.1 (released 26 February 2014)}
\label{\detokenize{version-history:new-in-suews-version-2014a-1-released-26-february-2014}}\begin{enumerate}
\item {} 
Please see the large number of changes made in the 2014a release.

\item {} 
This is a minor change to address installing the software.

\item {} 
Minor updates to the manual

\end{enumerate}


\section{New in SUEWS Version 2014a (released 21 February 2014)}
\label{\detokenize{version-history:new-in-suews-version-2014a-released-21-february-2014}}\begin{enumerate}
\item {} 
Bug fix: External irrigation is calculated as combined from automatic
and manual irrigation and during precipitation events the manual
irrigation is reduced to 60\% of the calculated values. In previous
version of the model, the irrigation was in all cases taken 60\% of
the calculated value, but now this has been fixed.

\item {} 
In previous versions of the model, irrigation was only allowed on the
irrigated grass surface type. Now, irrigation is also allowed on
evergreen and deciduous trees/shrubs surfaces. These are not however
treated as separate surfaces, but the amount of irrigation is evenly
distributed to the whole surface type in the modelled area. The
amount of water is calculated using same equation as for grass
surface (equation 5 in Järvi et al. 2011), and the fraction of
irrigated trees/shrubs (relative to the area of tree/shrubs surface)
is set in the gis file (See Table 4.11: SSss\_YYYY.gis)

\item {} 
In the current version of the model, the user is able to adjust the
leaf-on and leaf-off lengths in the FunctionalTypes. nml file. In
addition, user can choose whether to use temperature dependent
functions or combination of temperature and day length (advised to be
used at high-latitudes)

\item {} 
In the gis-file, there is a new variable Alt that is the area
altitude above sea level. If not known exactly use an approximate
value.

\item {} 
Snow removal profile has been added to the
HourlyProfileSSss\_YYYY.txt. Not yet used!

\item {} 
Model time interval has been changed from minutes to seconds.
Preferred interval is 3600 seconds (1 hour)

\item {} 
Manual correction: input variable Soil moisture said soil moisture
deficit in the manual \textendash{} word removed

\item {} 
Multiple compiled versions of SUEWS released. There are now users in
Apple, Linux and Windows environments. So we will now release
compiled versions for more operating systems (section 3).

\item {} 
There are some changes in the output file columns so please, check
the respective table of each used output file.

\item {} 
Bug fix: with very small amount of vegetation in an area \textendash{} impacted
Phenology for LUMPS

\end{enumerate}


\section{New in SUEWS Version 2013a}
\label{\detokenize{version-history:new-in-suews-version-2013a}}\begin{enumerate}
\item {} 
Radiation selection bug fixed

\item {} 
Aerodynamic resistance \textendash{} when very low - no longer reverts to neutral
(which caused a large jump) \textendash{} but stays low

\item {} 
Irrigation day of week fixed

\item {} 
New error messages

\item {} 
min file \textendash{} now includes a decimal time column \textendash{} see Section 5.4 \textendash{}
Table 5.3

\end{enumerate}


\section{New in SUEWS Version 2012b}
\label{\detokenize{version-history:new-in-suews-version-2012b}}\begin{enumerate}
\item {} 
Error message generated if all the data are not available for the
surface resistance calculations

\item {} 
Error message generated if wind data are below zero plane
displacement height.

\item {} 
All error messages now written to ‘Problem.txt’ rather than embedded
in an ErrorFile. Note some errors will be written and the program
will continue others will stop the program.

\item {} 
Default variables removed (see below). Model will stop if any data
are problematic. File should be checked to ensure that reasonable
data are being used. If an error occurs when there should not be one
let us know as it may mean we have made the limits too restrictive.

\end{enumerate}

Contents no longer used File defaultFcld=0.1 defaultPres=1013
defaultRH=50 defaultT=10 defaultU=3 RunControl.nml
\begin{itemize}
\item {} 
Just delete lines from file

\item {} 
Values you had were likely different from these example value shown
here

\end{itemize}


\section{New in SUEWS Version 2012a}
\label{\detokenize{version-history:new-in-suews-version-2012a}}\begin{enumerate}
\item {} 
Improved error messages when an error is encountered. Error message
will generally be written to the screen and to the file
‘problems.txt’

\item {} 
Format of all input files have changed.

\item {} 
New excel spreadsheet and R programme to help prepare required data
files. (Not required)

\item {} 
Format of coef flux (OHM) input files have changed.
\begin{itemize}
\item {} 
This allows for clearer identification for users of the
coefficients that are actually to be used

\item {} 
This requires an additional file with coefficients. These do not
need to be adjusted but new coefficients can be added. We would
appreciate receiving additional coefficients so they can be
included in future releases \textendash{} Please email Sue.

\end{itemize}

\item {} 
Storage heat flux (OHM) coefficients can be changed by
\begin{itemize}
\item {} 
time of year (summer, winter)

\item {} 
surface wetness state

\end{itemize}

\item {} 
New files are written: DailyState.txt
\begin{itemize}
\item {} 
Provides the status of variables that are updated on a daily or
basis or a snapshot at the end of each day.

\end{itemize}

\item {} 
Surface Types
\begin{itemize}
\item {} 
Clarification of surface types has been made. See GIS and OHM
related files

\end{itemize}

\end{enumerate}


\section{New in SUEWS Version2011b}
\label{\detokenize{version-history:new-in-suews-version2011b}}\begin{enumerate}
\item {} 
Storage heat flux (\(\Delta\)Qs) and anthropogenic heat flux (QF) can be set
to be 0 W m$^{\text{-2}}$

\item {} 
Calculation of hydraulic conductivity in soil has been improved and
HydraulicConduct in SUEWSInput.nml is replaced with name
SatHydraulicConduct

\item {} 
Following removed from HeaderInput.nml
\begin{itemize}
\item {} 
HydraulicConduct

\item {} 
GrassFractionIrrigated

\item {} 
PavedFractionIrrigated

\item {} 
TreeFractionIrrigated

\end{itemize}

\end{enumerate}

The lower three are now determined from the water use behaviour used in
SUEWS
\begin{enumerate}
\item {} 
Following added to HeaderInput.nml
\begin{itemize}
\item {} 
SatHydraulicConduct

\item {} 
defaultQf

\item {} 
defaultQs

\end{itemize}

\item {} 
If \(\Delta\)Qs and QF are not calculated in the model but are given as an
input, the missing data is replaced with the default values.

\item {} 
Added to SAHP input file
\begin{itemize}
\item {} 
AHDIUPRF \textendash{} diurnal profile used if AnthropHeatChoice = 1

\end{itemize}

\end{enumerate}

V2012a this became obsolete OHM file (SSss\_YYYY.ohm)


\chapter{Differences between SUEWS, LUMPS and FRAISE}
\label{\detokenize{differences-suews-lumps-fraise::doc}}\label{\detokenize{differences-suews-lumps-fraise:differences-between-suews-lumps-and-fraise}}
The largest difference between LUMPS and SUEWS is that the latter
simulates the urban water balance in detail while LUMPS takes a simpler
approach for the sensible and latent heat fluxes and the water balance
(“water bucket”). The calculation of evaporation/latent heat in SUEWS is
more biophysically based. Due to its simplicity, LUMPS requires less
parameters in order to run. SUEWS gives turbulent heat fluxes calculated
with both models as an output. \sphinxstylestrong{The model can run LUMPS alone without
running SUEWS (Table 4.1 \textendash{} SuewsStatus).}

Similarities and differences between LUMPS and SUEWS.


\begin{savenotes}\sphinxattablestart
\centering
\begin{tabulary}{\linewidth}[t]{|T|T|T|}
\hline
\sphinxstyletheadfamily &\sphinxstyletheadfamily 
LUMPS
&\sphinxstyletheadfamily 
SUEWS
\\
\hline
Net all-wave
radiation (Q*)
&
Input or NARP
&
Input or NARP
\\
\hline
Storage heat flux
(\(\Delta\)QS)
&
Input or from OHM
&
Input or from OHM
\\
\hline
Anthropogenic heat
flux (QF)
&
Input or calculated
&
Input or calculated
\\
\hline
Latent heat (QE)
&
DeBruin and Holtslag
(1982)
&
Penman-Monteith
equation2
\\
\hline
Sensible heat flux
(QH)
&
DeBruin and Holtslag
(1982)
&
Residual from
available energy
minus QE
\\
\hline
Water balance
&
No water balance
included
&
Running water balance
of canopy and water
balance of soil
\\
\hline
Soil moisture
&
Not considered
&
Modelled
\\
\hline
Surface wetness
&
Simple water bucket
model
&
Running water balance
\\
\hline
Irrigation
&
Only fraction of
surface area that is
irrigated
&
Input or calculated
with a simple model
\\
\hline
Surface cover
&
buildings, paved,
vegetation
&
buildings, paved,
coniferous and
deciduous
trees/shrubs,
irrigated and
unirrigated grass
\\
\hline&&\\
\hline
\end{tabulary}
\par
\sphinxattableend\end{savenotes}


\section{FRAISE Flux Ratio \textendash{} Active Index Surface Exchange}
\label{\detokenize{differences-suews-lumps-fraise:fraise-flux-ratio-active-index-surface-exchange}}
FRAISE provides an estimate of mean midday (\(\pm\)3 h around solar noon)
energy partitioning from information on the surface characteristics and
estimates of the mean midday incoming radiative energy and anthropogenic
heat release. Please refer to Loridan and Grimmond (2012) \phantomsection\label{\detokenize{differences-suews-lumps-fraise:id1}}{\hyperref[\detokenize{references:lg2012}]{\sphinxcrossref{{[}LG2012{]}}}} (\autopageref*{\detokenize{references:lg2012}}) for
further details.


\begin{savenotes}\sphinxattablestart
\centering
\begin{tabulary}{\linewidth}[t]{|T|T|T|T|}
\hline
\sphinxstyletheadfamily 
Topic
&\sphinxstyletheadfamily 
FRAISE
&\sphinxstyletheadfamily 
LUMPS
&\sphinxstyletheadfamily 
SUEWS
\\
\hline
\sphinxstylestrong{Complexity}
&
Simplest:
FRAISE
&&
More complex:
SUEWS
\\
\hline
\sphinxstylestrong{Software
provided:}
&
R code
&
Windows exe
(written in
Fortran)
&
Windows exe
(written in
Fortran) -
other versions
available
\\
\hline
Applicable
period:
&
Midday (within
3 h of solar
noon)
&
hourly
&
5
min-hourly-annu
al
\\
\hline
Unique
features:
&
calculates
active surface
\textendash{} and fluxes
&
radiation and
energy balances
&
radiation,
energy and
water balance
(includes
LUMPS)
\\
\hline&&&\\
\hline
\end{tabulary}
\par
\sphinxattableend\end{savenotes}

\cleardoublepage
\begingroup
\renewcommand\chapter[1]{\endgroup}
\phantomsection


\chapter{References}
\label{\detokenize{references:references}}\label{\detokenize{references:refs}}\label{\detokenize{references::doc}}
\begin{sphinxthebibliography}{Leena2014}
\bibitem[J11]{\detokenize{J11}}{\phantomsection\label{\detokenize{references:j11}} 
Järvi L, Grimmond CSB \& Christen A (2011) The Surface Urban Energy
and Water Balance Scheme (SUEWS): Evaluation in Los Angeles and
Vancouver. J. Hydrol. 411, 219-237.
}
\bibitem[W16]{\detokenize{W16}}{\phantomsection\label{\detokenize{references:w16}} 
Ward HC, Kotthaus S, Järvi L and Grimmond CSB 2016: Surface Urban
Energy and Water Balance Scheme (SUEWS): development and evaluation
at two UK sites. Urban Climate. 18, 1-32 \sphinxhref{https://doi.org/10.1016/j.uclim.2016.05.001}{doi:
10.1016/j.uclim.2016.05.001}
}
\bibitem[G91]{\detokenize{G91}}{\phantomsection\label{\detokenize{references:g91}} 
Grimmond CSB \& Oke TR (1991) An Evaporation-Interception Model for
Urban Areas. Water Resour. Res. 27, 1739-1755.
}
\bibitem[O2003]{\detokenize{O2003}}{\phantomsection\label{\detokenize{references:o2003}} 
Offerle B, Grimmond CSB \& Oke TR (2003) Parameterization of Net
All-Wave Radiation for Urban Areas. J. Appl. Meteorol. 42, 1157-1173.
}
\bibitem[L2011]{\detokenize{L2011}}{\phantomsection\label{\detokenize{references:l2011}} 
Loridan T, CSB Grimmond, BD Offerle, DT Young, T Smith, L Järvi, F
Lindberg (2011) Local-Scale Urban Meteorological Parameterization
Scheme (LUMPS): longwave radiation parameterization \& seasonality
related developments. Journal of Applied Meteorology \& Climatology
50, 185-202, doi: 10.1175/2010JAMC2474.1
}
\bibitem[lucy]{\detokenize{lucy}}{\phantomsection\label{\detokenize{references:lucy}} 
Allen L, F Lindberg, CSB Grimmond (2011) Global to city scale model
for anthropogenic heat flux, International Journal of Climatology,
31, 1990-2005.
}
\bibitem[lucy2]{\detokenize{lucy2}}{\phantomsection\label{\detokenize{references:lucy2}} 
Lindberg F, Grimmond CSB, Nithiandamdan Y, Kotthaus S, Allen L (2013)
Impact of city changes and weather on anthropogenic heat flux in
Europe 1995\textendash{}2015, Urban Climate,4,1-13
\sphinxhref{http://dx.doi.org/10.1016/j.uclim.2013.03.002}{paper}
}
\bibitem[I11]{\detokenize{I11}}{\phantomsection\label{\detokenize{references:i11}} 
Iamarino M, Beevers S \& Grimmond CSB (2011) High-resolution (space,
time) anthropogenic heat emissions: London 1970-2025. International
J. of Climatology. 32, 1754-1767.
}
\bibitem[G91OHM]{\detokenize{G91OHM}}{\phantomsection\label{\detokenize{references:g91ohm}} 
Grimmond CSB, Cleugh HA \& Oke TR (1991) An objective urban heat
storage model and its comparison with other schemes. Atmos. Env. 25B,
311-174.
}
\bibitem[GO99QS]{\detokenize{GO99QS}}{\phantomsection\label{\detokenize{references:go99qs}} 
Grimmond CSB \& Oke TR (1999a) Heat storage in urban areas:
Local-scale observations and evaluation of a simple model. J. Appl.
Meteorol. 38, 922-940.
}
\bibitem[GO2002]{\detokenize{GO2002}}{\phantomsection\label{\detokenize{references:go2002}} 
Grimmond CSB \& Oke TR (2002) Turbulent Heat Fluxes in Urban Areas:
Observations and a Local-Scale Urban Meteorological Parameterization
Scheme (LUMPS) J. Appl. Meteorol. 41, 792-810.
}
\bibitem[AnOHM17]{\detokenize{AnOHM17}}{\phantomsection\label{\detokenize{references:anohm17}} 
Sun T, Wang ZH, Oechel W \& Grimmond CSB (2017) The Analytical
Objective Hysteresis Model (AnOHM v1.0): Methodology to Determine
Bulk Storage Heat Flux Coefficients. Geosci. Model Dev. Discuss. doi:
10.5194/gmd-2016-300.
}
\bibitem[Oaf2005]{\detokenize{Oaf2005}}{\phantomsection\label{\detokenize{references:oaf2005}} 
Offerle B, CSB Grimmond, K Fortuniak (2005) Heat storage \&
anthropogenic heat flux in relation to the energy balance of a
central European city center. International J. of Climatology. 25:
1405\textendash{}1419 doi: 10.1002/joc.1198
}
\bibitem[G86]{\detokenize{G86}}{\phantomsection\label{\detokenize{references:g86}} 
Grimmond CSB, Oke TR and Steyn DG (1986) Urban water-balance 1. A
model for daily totals. Water Resour Res 22: 1397-1403.
}
\bibitem[Leena2014]{\detokenize{Leena2014}}{\phantomsection\label{\detokenize{references:leena2014}} 
Järvi L, Grimmond CSB, Taka M, Nordbo A, Setälä H \& Strachan IB
(2014) Development of the Surface Urban Energy and Water balance
Scheme (SUEWS) for cold climate cities, Geosci. Model Dev. 7,
1691-1711, doi:10.5194/gmd-7-1691-2014.
}
\bibitem[CG2001]{\detokenize{CG2001}}{\phantomsection\label{\detokenize{references:cg2001}} 
Cleugh HA \& Grimmond CSB (2001) Modelling regional scale surface
energy exchanges and CBL growth in a heterogeneous, urban-rural
landscape. Bound.-Layer Meteor. 98, 1-31.
}
\bibitem[Shiho2015]{\detokenize{Shiho2015}}{\phantomsection\label{\detokenize{references:shiho2015}} 
Onomura S, Grimmond CSB, Lindberg F, Holmer B \& Thorsson S (2015)
Meteorological forcing data for urban outdoor thermal comfort models
from a coupled convective boundary layer and surface energy balance
scheme Urban Climate,11, 1-23 doi:10.1016/j.uclim.2014.11.001
}
\bibitem[FL2008]{\detokenize{FL2008}}{\phantomsection\label{\detokenize{references:fl2008}} 
Lindberg F, Holmer B \& Thorsson S (2008) SOLWEIG 1.0 \textendash{} Modelling
spatial variations of 3D radiant fluxes and mean radiant temperature
in complex urban settings. International Journal of Biometeorology
52, 697\textendash{}713.
}
\bibitem[FL2011]{\detokenize{FL2011}}{\phantomsection\label{\detokenize{references:fl2011}} 
Lindberg F \& Grimmond C (2011) The influence of vegetation and
building morphology on shadow patterns and mean radiant temperature
in urban areas: model development and evaluation. Theoretical and
Applied Climatology 105:3, 311-323.
}
\bibitem[Ko17]{\detokenize{Ko17}}{\phantomsection\label{\detokenize{references:ko17}} 
Kokkonen TV, Grimmond CSB, Räty O, Ward HC, Christen A, Oke TR,
Kotthaus S \& Järvi L (in review) Sensitivity of Surface Urban Energy
and Water Balance Scheme (SUEWS) to downscaling of reanalysis forcing
data.
}
\bibitem[Best2014]{\detokenize{Best2014}}{\phantomsection\label{\detokenize{references:best2014}} 
Best MJ \& Grimmond CSB (2014) Importance of initial state and
atmospheric conditions for urban land surface models’ performance.
Urban Climate 10: 387-406. doi: 10.1016/j.uclim.2013.10.006.
}
\bibitem[D74]{\detokenize{D74}}{\phantomsection\label{\detokenize{references:d74}} 
Dyer AJ (1974) A review of flux-profile relationships. Boundary-Layer
Meteorol. 7, 363-372.
}
\bibitem[H88]{\detokenize{H88}}{\phantomsection\label{\detokenize{references:h88}} 
Högström U (1988) Non-dimensional wind and temperature profiles in
the atmospheric surface layer: A re-evaluation. Boundary-Layer
Meteorol. 42, 55\textendash{}78.
}
\bibitem[VUH85]{\detokenize{VUH85}}{\phantomsection\label{\detokenize{references:vuh85}} 
Van Ulden AP \& Holtslag AAM (1985) Estimation of atmospheric boundary
layer parameters for boundary layer applications. J. Clim. Appl.
Meteorol. 24, 1196-1207.
}
\bibitem[CNstab]{\detokenize{CNstab}}{\phantomsection\label{\detokenize{references:cnstab}} 
Campbell GS \& Norman JM (1998) Introduction to Environmental
Biophysics. Springer Science, US.
}
\bibitem[B71]{\detokenize{B71}}{\phantomsection\label{\detokenize{references:b71}} 
Businger JA, Wyngaard JC, Izumi Y \& Bradley EF (1971) Flux-Profile
Relationships in the Atmospheric Surface Layer. J. Atmos. Sci., 28,
181\textendash{}189.
}
\bibitem[Ka09]{\detokenize{Ka09}}{\phantomsection\label{\detokenize{references:ka09}} 
Kawai T, Ridwan MK \& Kanda M (2009) Evaluation of the simple urban
energy balance model using selected data from 1-yr flux observations
at two cities. J. Appl. Meteorol. Clim. 48, 693-715.
}
\bibitem[VG00]{\detokenize{VG00}}{\phantomsection\label{\detokenize{references:vg00}} 
Voogt JA \& Grimmond CSB (2000) Modeling surface sensible heat flux
using surface radiative temperatures in a simple urban terrain. J.
Appl. Meteorol. 39, 1679-1699.
}
\bibitem[Ka07]{\detokenize{Ka07}}{\phantomsection\label{\detokenize{references:ka07}} 
Kanda M, Kanega M, Kawai T, Moriwaki R \& Sugawara H (2007). Roughness
lengths for momentum and heat derived from outdoor urban scale
models. J. Appl. Meteorol. Clim. 46, 1067-1079.
}
\bibitem[GO99]{\detokenize{GO99}}{\phantomsection\label{\detokenize{references:go99}} 
Grimmond CSB \& Oke TR (1999) Aerodynamic properties of urban areas
derived from analysis of surface form. J. Appl. Meteorol. 38,
1262-1292.
}
\bibitem[Mc98]{\detokenize{Mc98}}{\phantomsection\label{\detokenize{references:mc98}} 
MacDonald RW, Griffiths RF \& Hall DJ (1998) An improved method for
estimation of surface roughness of obstacle arrays. Atmos. Env. 32,
1857-1864.
}
\bibitem[FN78]{\detokenize{FN78}}{\phantomsection\label{\detokenize{references:fn78}} 
Falk J \& Niemczynowicz J, (1978) Characteristics of the above ground
runoff in sewered catchments, in Urban Storm Drainage, edited by
Helliwell PR, Pentech, London
}
\bibitem[Ha79]{\detokenize{Ha79}}{\phantomsection\label{\detokenize{references:ha79}} 
Halldin S, Grip H \& Perttu K. (1979) Model for energy exchange of a
pine forest canopy. In: Halldin S (Ed.), Comparison of Forest Water
and Energy Exchange Models. International Society of Ecological
Modeling
}
\bibitem[CW86]{\detokenize{CW86}}{\phantomsection\label{\detokenize{references:cw86}} 
Calder IR and Wright IR (1986) Gamma Ray Attenuation Studies of
Interception From Sitka Spruce: Some Evidence for an Additional
Transport Mechanism. Water Resour. Res., 22(3), 409\textendash{}417.
}
\bibitem[Ok87]{\detokenize{Ok87}}{\phantomsection\label{\detokenize{references:ok87}} 
Oke TR (1987) Boundary Layer Climates. Routledge, London, UK
}
\bibitem[Br03]{\detokenize{Br03}}{\phantomsection\label{\detokenize{references:br03}} 
Breuer L, Eckhardt K and Frede H-G (2003) Plant parameter values for
models in temperate climates. Ecol. Model. 169, 237-293.
}
\bibitem[Ja76]{\detokenize{Ja76}}{\phantomsection\label{\detokenize{references:ja76}} 
Jarvis PG (1976) The interpretation of the variations in leaf water
potential and stomatal conductance found in canopies in the field.
Philos. Trans. R. Soc. London, Ser. B., 273, 593-610.
}
\bibitem[Au74]{\detokenize{Au74}}{\phantomsection\label{\detokenize{references:au74}} 
Auer AH (1974) The rain versus snow threshold temperatures.
Weatherwise, 27, 67.
}
\bibitem[SV06]{\detokenize{SV06}}{\phantomsection\label{\detokenize{references:sv06}} 
Sailor DJ and Vasireddy C (2006) Correcting aggregate energy
consumption data account for variability in local weather. Environ.
Modell. Softw. 21, 733-738.
}
\bibitem[Ko14]{\detokenize{Ko14}}{\phantomsection\label{\detokenize{references:ko14}} 
Konarska J, Lindberg F, Larsson A, Thorsson S and Holmer B (2014)
Transmissivity of solar radiation through crowns of single urban
trees—application for outdoor thermal comfort modelling. Theor Appl
Climatol 117:363\textendash{}376.
}
\bibitem[Re90]{\detokenize{Re90}}{\phantomsection\label{\detokenize{references:re90}} 
Reindl DT, Beckman WA and Duffie JA (1990) Diffuse fraction
correlation. Sol Energy 45:1\textendash{}7.
}
\bibitem[LG2012]{\detokenize{LG2012}}{\phantomsection\label{\detokenize{references:lg2012}} 
Loridan T and Grimmond CSB (2012) Characterization of energy flux
partitioning in urban environments: links with surface seasonal
properties. J. of Applied Meteorology and Climatology 51,219-241 doi:
10.1175/JAMC-D-11-038.1
}
\end{sphinxthebibliography}



\renewcommand{\indexname}{Index}
\printindex
\end{document}